	%%%%%%%%%%%%%%%%%
% This is an example CV created using altacv.cls (v1.1.2, 1 February 2017) written by
% LianTze Lim (liantze@gmail.com), based on the 
% Cv created by BusinessInsider at http://www.businessinsider.my/a-sample-resume-for-marissa-mayer-2016-7/?r=US&IR=T
% 
%% It may be distributed and/or modified under the
%% conditions of the LaTeX Project Public License, either version 1.3
%% of this license or (at your option) any later version.
%% The latest version of this license is in
%%    http://www.latex-project.org/lppl.txt
%% and version 1.3 or later is part of all distributions of LaTeX
%% version 2003/12/01 or later.
%%%%%%%%%%%%%%%%

%% If you want to use \orcid or the
%% academicons icons, add "academicons"
%% to the \documentclass options. 
%% Then compile with XeLaTeX or LuaLaTeX.
% \documentclass[10pt,a4paper,academicons]{altacv}
\documentclass[10pt,a4paper]{altacv}

%% AltaCV uses the fontawesome and academicon fonts
%% and packages. 
%% See texdoc.net/pkg/fontawecome and http://texdoc.net/pkg/academicons for full list of symbols.
%% When using the "academicons" option,
%% Compile with LuaLaTeX for best results. If you
%% want to use XeLaTeX, you may need to install
%% Academicons.ttf in your operating system's font %% folder.


% Change the page layout if you need to
\geometry{left=1cm,right=9cm,marginparwidth=6.8cm,marginparsep=1.2cm,top=1cm,bottom=1cm}

% Change the font if you want to.

% If using pdflatex:
\usepackage[utf8]{inputenc}
\usepackage[T1]{fontenc}
\usepackage[default]{lato}
\usepackage{hyperref}

% If using xelatex or lualatex:
% \setmainfont{Lato}

% Change the colours if you want to
\definecolor{VividPurple}{HTML}{1B1B1B}
\definecolor{SlateGrey}{HTML}{464646}
\definecolor{LightGrey}{HTML}{444444}
\colorlet{heading}{VividPurple}
\colorlet{accent}{VividPurple}
\colorlet{emphasis}{SlateGrey}
\colorlet{body}{LightGrey}

\hypersetup{
colorlinks,
linkcolor=VividPurple,
urlcolor=VividPurple
}

% Change the bullets for itemize and rating marker
% for \cvskill if you want to
\renewcommand{\itemmarker}{{\small\textbullet}}
\renewcommand{\ratingmarker}{\faCircle}

%% sample.bib contains your publications
\addbibresource{sample.bib}

\begin{document}
\name{Himanshu Shekhar}
\tagline{Software Developer}
\title{Himanshu Shekhar's Resume}
% Cropped to square from https://en.wikipedia.org/wiki/Marissa_Mayer#/media/File:Marissa_Mayer_May_2014_(cropped).jpg, CC-BY 2.0
%I don't need photo :P
%\photo{2.5cm}{mmayer-wikipedia-cc-by-2_0}
\personalinfo{%
  % Not all of these are required!
  % You can add your own with \printinfo{symbol}{detail}
  \normalsize\email{himanshushekharb16@gmail.com}
%   \phone{000-00-0000}
%  \mailaddress{Address, Street, 00000 County}
% \location{Sunnyvale, CA}
  \normalsize\homepage{https://himanshub16.github.io}
  \normalsize\github{himanshub16}
  \normalsize\twitter{himanshub16}
  \normalsize\linkedin{himanshub16}
}

%% Make the header extend all the way to the right, if you want. Extend the right margin by 8cm (=6.8cm marginparwidth + 1.2cm marginparsep)
\begin{adjustwidth}{}{-8cm}
\makecvheader
\end{adjustwidth}

%% Provide the file name containing the sidebar contents as an optional parameter to \cvsection.
%% You can always just use \marginpar{...} if you do
%% not need to align the top of the contents to any
%% \cvsection title in the "main" bar.
\cvsection[page1sidebar]{experience}

\cvproject
	{\href{https://socialcops.com}{SocialCops}}
    {Backend Development Intern}
    {May 2017 - July 2017}
\begin{itemize}
	\item {Created product toolbox to provide API level access over database}
    \item {Created reporting tool to automate reports for support tasks}
\end{itemize}

\divider

\cvproject 
	{\href{https://geekhaven.iiita.ac.in}{Technical Society, IIIT-A}}
    {Software Development Coordinator}
    {April 2017 - Present}
\begin{itemize}
	\item{Mentor students to get started with software development}
	\item{Developed alumni portal to manage profiles and events}
	\item{Organize workshops/talks/events to promote hacker culture}
\end{itemize}

\cvsection{projects}

%% do not include other file, a blank page is made
%% copy the projects from projects.tex

\cvproject
    {\href{https://github.com/himanshub16/ProxyMan/}{ProxyMan}}
    {Bash}
    {2016 (Maintained)}
\begin{itemize}
	\item {Tool to set up system-wide proxy configuration for Linux}
    \item {Supports other developer tools like apt, git, npm too}
    \item {150+ stars on GitHub}
\end{itemize}

\divider

\cvproject
    {\href{https://github.com/himanshub16/dexter/}{Dexter}}
    {JavaScript}
    {Feb 2018}
\begin{itemize}
    \item {Winner at Hack36, MNNIT Allahabad}
	\item {Lets you code with your voice}
    \item {Have boring tasks done fast in text editor with a voice command}
\end{itemize}

\divider

\cvproject
    {Expedite.ai}
    {Python, Django, IBM Watson, NLP}
    {Oct 2017}
\begin{itemize}
	\item {Winner at annual intra-college hackathon}
    \item {Bridging customer issues on social media to support desk}
    \item {NLP classifier categorizes issues and helps the right team track it}
\end{itemize}

\divider

\cvproject
	{\href{https://github.com/himanshub16/21Lane/}{21Lane}}
    {Python, FTP, PyQt5}
    {May 2016 - Dec 2016}  
\begin{itemize}
	\item {Cross-platform peer-to-peer file sharing based on FTP}
    \item {Removes dependency of central hub as in DC++}
    \item {Deployed at IIIT-Allahabad to compensate the need of similar service}
\end{itemize}

\divider

\cvproject
    {\href{https://github.com/himanshub16/moodmonk}{MoodMonk}}
    {NodeJS, Analytics, IBM Watson}
    {March 2017}
\begin{itemize}
	\item {Won 6th prize at Hack in The North}
    \item {Lets you speak your feelings, and analyse your mood from the transcript generated}
    \item {Generates insights about your mood, and maps your stress levels for the last week}
\end{itemize}

\divider

% \cvproject 
% 	{\href{https://github.com/himanshub16/backbone}{Backbone}}
%     {Django, SQL, SocketIO}
%     {August 2017}
% \begin{itemize}
%     \item {Database as a Service with real-time notifications/triggers on custom events}
% 	\item {Created at CodeNCounter2.0 hackathong final round at Naggaro}
% \end{itemize}
% \cvproject
%     {\href{https://github.com/himanshub16/IPScanner}{IPScanner}}
%     {Python, Networking}
%     {April 2016}
% \begin{itemize}
% 	\item {Checks for IP addresses on your local network with functional Internet connection}
%     \item {Useful when Internet usage is limited to certain IPs on a local network}
% \end{itemize}

% \divider

% \cvproject
%     {\href{https://github.com/himanshub16/python-resolver}{Python Resolver}}
%     {Python}
%     {Feb. 2016}
% \begin{itemize}
% 	\item {Generates list of modules used by analysing the source code of a project}
%     \item {Installs missing modules if necessary}
% \end{itemize}

% \divider

% \cvevent{Product Engineer}{Google}{23 June 1999 -- 2001}{Palo Alto, CA}

% \begin{itemize}
% \item Joined the company as employe \#20 and female employee \#1
% \item Developed targeted advertisement in order to use user's search queries and show them related ads
% \end{itemize}

% \cvsection{A Day of My Life}

% Adapted from @Jake's answer from http://tex.stackexchange.com/a/82729/226
% \wheelchart{outer radius}{inner radius}{
% comma-separated list of value/text width/color/detail}
% \wheelchart{1.5cm}{0.5cm}{%
%   10/13em/accent!30/Sleeping \& dreaming about work, 
%   25/9em/accent!60/Public resolving issues with Yahoo!\ investors,
%   5/12em/accent!10/New York \& San Francisco Ballet Jawbone board member, 
%   20/12em/accent!40/Spending time with family,
%   5/8em/accent!20/Business development for Yahoo!\ after the Verizon acquisition,
%   30/9em/accent/Showing Yahoo!\ employees that their work has meaning,
%   5/8em/accent!20/Baking cupcakes
% }

\clearpage


%% If the NEXT page doesn't start with a \cvsection but you'd
%% still like to add a sidebar, then use this command on THIS
%% page to add it. The optional argument lets you pull up the 
%% sidebar a bit so that it looks aligned with the top of the
%% main column.
% \addnextpagesidebar[-1ex]{page3sidebar}


\end{document}
