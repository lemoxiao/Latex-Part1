\documentclass[8pt,oneside,final]{article}

% set all margins to 0 and set business card size
\usepackage[paperwidth=2in,paperheight=3.5in,margin=0cm,noheadfoot]{geometry}
\setlength{\baselineskip}{0cm}
\setlength{\topskip}{0pt}

\usepackage[utf8]{inputenc}
\usepackage[french]{babel}
\usepackage{parskip}         % remove paragraph indents
\usepackage[T1]{fontenc}     % load external fonts
\usepackage{tikz}            % drawing
\usepackage{fontawesome}     % icon font
\usepackage{xcolor}          % more colour options
\usepackage{graphics}        % load images
\usepackage[nolinks]{qrcode} % create QR codes

% local path for my logo
\usepackage{graphicx}
  	  	\graphicspath{{./fig/}} %fig path
%Define CV Variables
\newcommand{\cvfirstname}{David}
\newcommand{\cvlastname}{Beauchemin}
\newcommand{\cvtitle}{B. Sc. Actuariat}

\newcommand{\cvemail}{david.beauchemin.5@ulaval.ca}
\newcommand{\cvphonenumber}{(514) 250-3616}
\newcommand{\github}{davebulaval}

% load and configure tikz libraries
\usetikzlibrary{matrix,calc,positioning}

% define some lengths for internal spacing
\newlength{\seplinewidth}    \setlength{\seplinewidth}{2cm}
\newlength{\seplineheight}   \setlength{\seplineheight}{1pt}
\newlength{\seplinedistance} \setlength{\seplinedistance}{0.3cm}

% colour options
\definecolor{white}{RGB}{255,255,255}
\definecolor{seplinecolour}{RGB}{0,0,153}  % blue
\definecolor{iconcolour}{RGB}{90, 94, 107}     % dark
\definecolor{textcolour}{RGB}{90, 94, 107}   % dark
\definecolor{jobtitlecolour}{RGB}{90, 94, 107} % light dark

% define some lengths for internal spacing
\newlength{\qrheight}  \setlength{\qrheight}{1in}
\newlength{\edgemargin} \setlength{\edgemargin}{0.2in}
\newlength{\logowidth}  \setlength{\logowidth}{0.5in}

%Command to produce a round shaped corner framebox with defined colors
\newcommand{\mybox}[1]{
	\tikz
	\node[rectangle,rounded corners=1mm,draw=iconcolour,anchor=text,fill=iconcolour,inner sep=3pt,minimum height=0.1cm]
		{\bfseries\textcolor{white}{#1}};}

%Command for printing white words
\newcommand\printww[1]{
	\textcolor{white}{
	\foreach \x in #1{\x \- \,}}}


% change global colour
\makeatletter
\newcommand{\globalcolor}[1]{%
  \color{#1}\global\let\default@color\current@color
}
\makeatother
\AtBeginDocument{\globalcolor{textcolour}}

\begin{document}
  \thispagestyle{empty}
  \vspace*{\fill}
  \tiny
  \begin{center}
    \begin{tikzpicture}
      % name
      \matrix[every node/.style={anchor=center,font=\huge},anchor=center] (name) {
        \node{\textbf{\cvfirstname}}; \\
        \node{\textbf{\cvlastname}}; \\
        \node{\color{jobtitlecolour}\normalsize\textit{\cvtitle}}; \\
        \node{\includegraphics[width=.25\textwidth]{LogoDB.png}}; \\
      };
      % sep line 1
      \node[below=\seplinedistance of name] (hl1) {};
      \draw[line width=\seplineheight,color=seplinecolour] (hl1)++(-\seplinewidth/2,0) -- ++(\seplinewidth,0);
      % contact info
      \matrix [below=\seplinedistance of hl1,%
               column 1/.style={anchor=center,color=iconcolour},%
               column 2/.style={anchor=west}] (contact){
        \node{\faAt}; &\node{\cvemail};\\
        \node{\faPhone}; &\node{\cvphonenumber}; \\
        \node{\faGithub}; &\node{\github}; \\
        %\node{\faLinkedin}; &\node{\cvlinkedin}; \\
      };
			% sep line 2
      \node[below=\seplinedistance of contact] (hl2) {};
      \draw[line width=\seplineheight,color=seplinecolour] (hl2)++(-\seplinewidth/2,0) -- ++(\seplinewidth,0);
      % interests
      \matrix [below=\seplinedistance of hl2,
         every node/.style={anchor=center,font=\LARGE}]
         (interests) {
        \node{\faCode}; & \node{\faBicycle}; &
        \node{\faLineChart}; & \node{\faMap}; &
        \node{\faDatabase}; \\
      }; 
  		% sep line 3
      \node[below=\seplinedistance of interests] (hl3) {};
      \draw[line width=\seplineheight,color=seplinecolour] (hl3)++(-\seplinewidth/2,0) -- ++(\seplinewidth,0);
      % interests
      \matrix [below=\seplinedistance of hl3,
         every node/.style={anchor=center,font=\tiny}]
         (interests) {
         \node{\mybox{Actuarial risk}}; &  \node{\mybox{Coding}};\\
      };     
      
    \end{tikzpicture}
  \end{center}
  \vspace*{\fill}
  \clearpage
  
\end{document}
