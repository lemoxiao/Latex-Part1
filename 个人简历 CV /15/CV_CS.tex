% !TEX TS-program = luatex
% Awesome Source CV LaTeX Template
%
% This template is originally maintained by Christophe ROGER from:
% https://github.com/darwiin/awesome-neue-latex-cv
%
% This version is based on the modification from:
% https://github.com/innerTide/awesome-neue-latex-cv-extended
% Author:
% Zhihao Zhou
%
%

%roman numeral
\makeatletter
\newcommand{\rmnum}[1]{\romannumeral #1}
\newcommand{\Rmnum}[1]{\expandafter\@slowromancap\romannumeral #1@}
\makeatother

\documentclass[localFont, alternative]{awesome-source-cv}

% Name should be {<Given name>} {<Family name>}.
% The display order is automatically reversed with Chinese support is enabled.

\name{Zhihao}{Zhou}
\photo{2.6cm}{figs/photo}
\tagline{M.Sc. of Electrical Engineering}
\socialinfo{
	\smartphone{+31(0)651655432}
	\email{zzh2616@hotmail.com}\\
	\linkedin{zhihao-zhou-cn1992}\\
  %   \linkedin{yuefeng-wu-cn1116}
	% \github{innerTide}\\
	\address{Burgwal 21, 2611 GE, Delft, the Netherlands}\\
	\infos{Born on \textbf{Mar. 10, 1992} in \textbf{Shandong, China}}\\
}
%------------------------------------------
\begin{document}

% \makecvheader
\makecvheaderwithoutphoto
%--------------------SECTIONS-----------------------------------
\begin{rubric}{Education}

\entry*[2009 -- 2013]%
	\textbf{Ph.D., Multimedia University, Malaysia} in Natural Language Processing.
	\par Thesis title: \emph{Low-Cost Multilingual Lexicon Construction for Under-Resourced Languages.} More details at \url{http://liantze.penguinattack.org/phd.html}
%
\entry*[2003 -- 2006]%
	\textbf{M.Sc.~Computer Science, Universiti Sains Malaysia, Malaysia} in Automated Translation.\par
	Thesis title: \emph{Using Conceptual Vectors to Improve Translation Selection}.
%
\entry*[1998 -- 2001]%
	\textbf{B.Sc.~(Hons) Computer Science, University of Warwick, United Kingdom.}\par
	\emph{First Class Honours.}  Department Prize for Outstanding Performance.
%
\end{rubric}
%Project Experience
\sectionTitle{\textsc{Project Experience}}{\faCode}
\begin{experiences}
	\experience
		{Sept. 2017}	{Master Thesis Intern}{imec}{Eindhoven, the Netherlands}
		{Aug. 2016}	{
						\begin{itemize}
							\item Title: a digital-intensive wakeup timer based on an RC frequency-locked loop for Internet of Things (IoT) applications
							\item A novel ultra-low-power oscillator was modeled with MATLAB and designed in Cadence to fully exploit the small area and low supply voltage advantages in advanced CMOS processes
							\item An automated measurement script was developed with Python and the results show the oscillator has the lowest supply voltage, the highest energy efficiency, and comparable stabilities w.r.t. the state-of-the-art
						\end{itemize}
					}
								{Integrated Circuit design, Verilog-HDL, MATLAB, Python, Git, \LaTeX}
		 {height = 1.25cm}		{figs/imec}
	\emptySeparator

	\experience
	{Aug. 2015}	{Intern}{Changzhou Research Institute of Zhejiang University}{Jiangsu, China}
	{June 2015}{
					\begin{itemize}
						\item Explored control algorithms in smart car systems
						\item Adapted the Kalman Filter into the navigation system of the car using data from multiple sensors, e.g., gyroscope and accelerometer
						\item Collaborated with other colleagues in the PCB design of the hardware
					\end{itemize}
				}
							{Control System, Embedded System, MATLAB, PCB, Team Collaboration}
	 {width = 1.5cm}		{figs/zju}
\end{experiences}

\begin{projects}
	\project %add name of the algorithm used!!!!!!!!!!!!
		{June 2014}   {Bachelor Thesis: Design of An Object Tracking Software}
		{Dec. 2013} {
											\begin{itemize}
												\item Conducted research on various object tracking algorithms
												\item Designed a Windows software that can track multiple objects in a video file with OpenCV libraries using Visio Studio
												\item Optimized the software to handle real-time camera input for demonstration purposes
											\end{itemize}
										}
										{C++, Object Tracking, Windows, OpenCV, Visio Studio}
\end{projects}

% % A extension of Awesome Source CV LaTeX Template
%
%
% This template is originally maintained by Christophe ROGER from:
% https://github.com/darwiin/awesome-neue-latex-cv
%
% This modification has been download from:
% https://github.com/innerTide/awesome-neue-latex-cv-extended
% Author:
% Yuefeng Wu
%
%
% Render an experience in the experiences environment
% Usage:
% \experience
%  {<End date>}      {<Title>}{<Enterprise>}{<Location>}
%  {<Start date}     {
%                      <Experience description (Could be a list)>
%                    }
%                    {<Technology list>}
%  {<Logo parameter>}	{<Logo File Name>}
\sectionTitle{\textsc{Internship}}{\faSuitcase}

\begin{experiences}
	\experience
    {Sept. 2017}	{Master Thesis Intern}{imec}{Eindhoven, the Netherlands}
    {Aug. 2016}	{
    				\begin{itemize}
              \item Title: a digital-intensive wakeup timer based on an RC frequency-locked loop for Internet of Things (IoT) applications
    					\item Proposed a novel ultra-low-power oscillator architecture to fully exploit the small area and low supply voltage advantages in advanced CMOS processes
              \item System-level modeling with MATLAB to estimate stabilities of the oscillator w.r.t. temperature, supply and noise
              \item Transistor-level implementation in Cadence and tape-out in TSMC 40-nm CMOS
              \item Measurements show the oscillator has the lowest supply voltage, the highest energy efficiency, and comparable stabilities w.r.t. the state-of-the-art
    				\end{itemize}
    			}
                {Analog/Mixed-Signal, Ultra-Low Power, Oscillator, MATLAB, Low Noise, Cadence, Layout, Measurement, Git}
     {height = 1.25cm}		{figs/imec}
	\emptySeparator

    \experience
    {Aug. 2015}	{Intern}{Changzhou Research Institute of Zhejiang University}{Jiangsu, China}
    {June 2015}{
    				\begin{itemize}
              \item Explored control algorithms in smart car systems
    					\item Adapted the Kalman Filter into the navigation system of the car using data from multiple sensors, e.g., gyroscope and accelerometer
              \item Collaborated with other colleagues in the PCB design of the hardware
    				\end{itemize}
    			}
                {Control System, Embedded System, MATLAB, PCB, Team Collaboration}
     {width = 1.5cm}		{figs/zju}

\end{experiences}

% %%
%% Author: Ali Asghar Momeni Vesalian (momeni.vesalian@gmail.com)
%% 12/30/2017 AD
%%

\begin{jrsection}{Projects}
    \jproject{
        \jrsubcaption{Bitex Exchange}
    }{
        \jrsubsubcaption{Apr 2018 - Present}
    }{
        \begin{jrdescription}
            Bitex Exchange project ordered by a Polish-Iranian company to implement a crypto currency exchange website’s functional and non-functional requirements such as
            balance management, matching engine, order services, and etc.\ Its serious implementation started around May 2018 and I have been contributing to the project
            since June 2018 as the chief architect.\ In fact, I had a crucial role in designing architecture model and implementing several infrastructural APIs.\ I also
            design and implement some important functional requirements such as matching engine.
        \end{jrdescription}
    }
    \jproject{
        \jrsubcaption{Navaco CORE Banking}
    }{
        \jrsubsubcaption{Sep 2017 - Present}
    }{
        \begin{jrdescription}
            Navaco CORE Banking project ordered by Maskan Bank to implement banks’ functional and non-functional requirements such as accounting,
            deposit, lending, card and etc.\ Its serious implementation started around 2017 and I contributed to the project from March 2017 to June 2018
            as a key member of architecture team.\ The team had a crucial role in the project and had some important tasks such as designing architecture model and
            implementing a framework that is used by other developers to reduce implementation time.\ In fact, I had a crucial role in designing architecture model and
            implementing several APIs for batch processing.\ I also designed and implemented some important banks’ functional requirements such as lending overnight calculation.
        \end{jrdescription}
    }
    \jproject{
        \jrsubcaption{Lotus CORE Banking}
    }{
        \jrsubsubcaption{Feb 2008 - Present}
    }{
        \begin{jrdescription}
            Lotus project ordered by Parsian Bank to implement banks’ functional and non-functional requirements such as accounting,
            deposit, lending, card and etc.\ Its serious implementation started around 2008.\ I contributed to the project from September 2009 to September 2017
            as a key member of Lending team which deals with loans, bank guarantees and collateral.\ In fact, I had a crucial role in designing and implementing
            several functionalities such as installment table calculation, product management, repayment, settlement and modern banking services.\ I, besides,
            cooperated in infrastructural components related to lending system.\ It is valuable to mention that Lotus project has been operational in
            all branches of Parsian Bank across the country since June 2015.
        \end{jrdescription}
    }
    \jproject{
        \jrsubcaption{Lotus Lending Conversion}
    }{
        \jrsubsubcaption{Mar 2011 - Dec 2015}
    }{
        \begin{jrdescription}
            Lending conversion project, which aimed to migrate all information from Negin CORE Banking Solution into Lotus CORE Banking Solution,
            started actively in March 2011.\ I was designated as the superior and architect in order to design and implement the project.\ Inherently
            critical constraints such as gradual data migration in several phases and live migration without disruption had to be done accurately, which
            made it difficult to implement the project.\ The conversion project composed of multi-threaded Spring Batch and PL-SQl routines.\ The project
            was utilized in three operational phases successfully without any faults.
        \end{jrdescription}
    }
\end{jrsection}

% \sectionTitle{\textsc{Patent}}{\faNewspaperO}
  \begin{enumerate}
    \item Jie Chen, \textsc{Zhihao Zhou}, Minghao Xue; \textit{A MEMS magmetic filed sensor with a comb tooth structure}; CN Patent 201310456380; filed September 30, 2013, and issued September 9, 2015.
    \item Jie Chen, Qiushi Liang, \textsc{Zhihao Zhou}; \textit{A dual-torsion-pendulum type MEMS magnetic filed sensor}; CN Patent 201310455985; filed September 30, 2013, and issued September 23, 2015.
  \end{enumerate}

%\sectionTitle{\textsc{Extracurricular Activity}}{\faSoccerBallO}

\begin{experiences}
	\experience
	{Aug. 2013}	{Trainee}{Xilinx Summer School}{Nanjing, China}
	{July 2013}	{
		\begin{itemize}
			\item Innovation training program organized by Xilinx in Southeast University
			\item Learned industrial project development and management skills
			\item Conducted a smart car project using a FPGA/ARM platform in collaboration with other trainees
		\end{itemize}
	}
	{Project Development, Project Management, Team Building, Verilog-HDL, C, FPGA, ARM}
	{height = 0.4cm}		{figs/xilinx}
	\emptySeparator

	\experience
	{June 2012}	{Member}{Red Cross Society of Southeast University}{Nanjing, China}
	{June 2011}	{
		\begin{itemize}
			\item Responsible for advertising organization events by making posters, flyers, etc.
			\item Planned and organized volunteer activities, and received a considerable amount of positive feedback
		\end{itemize}
	}
	{Volunteer Work, Time Management, Communication Skills, Coordination Capabilities}
	{width = 1.5cm}		{figs/redcross}
\end{experiences}

%% Awesome Source CV LaTeX Template
%
% This template has been downloaded from:
% https://github.com/darwiin/awesome-neue-latex-cv
%
% Author:
% Christophe Roger
%
% Template license:
% CC BY-SA 4.0 (https://creativecommons.org/licenses/by-sa/4.0/)

%Section Professional Skill
\sectionTitle{\textsc{Professional Skills}}{\faTasks}
\renewcommand{\arraystretch}{1.1}

	\begin{tabular}{>{}r>{}p{13cm}}
		\textsc{Development Tools:}	  			&   Cadence, LTspice, ADS, Vivado, Quartus \Rmnum{2}, Altium Designer, ANSYS, Git\\
		\textsc{Programming Languages:}  		&   Verilog-HDL, VHDL, C/C++, MATLAB, Python\\
		\textsc{Operating Systems:}	        &   Windows, Linux, macOS\\
    \textsc{Miscellaneous:}							&   Team Collaboration, \LaTeX, Microsoft Office, Computer Maintenance\\
	\end{tabular}

\sectionTitle{\textsc{Professional Skills}}{\faTasks}
\renewcommand{\arraystretch}{1.1}

	\begin{tabular}{>{}r>{}p{13cm}}
		%\textsc{Development Tools:}	  			&   Cadence, LTspice, ADS, Vivado, Quartus \Rmnum{2}, Altium Designer, ANSYS, Git\\
		\textsc{Programming Languages:}  		&   C/C++, MATLAB, Java, Python, Verilog-HDL, VHDL\\
		\textsc{Operating Systems:}	        &   Windows, Linux, macOS\\
    \textsc{Miscellaneous:}							&   Team Collaboration, \LaTeX, Microsoft Office, Computer Maintenance\\
	\end{tabular}

%% Awesome Source CV LaTeX Template
%
% This template has been downloaded from:
% https://github.com/darwiin/awesome-neue-latex-cv
%
% Author:
% Christophe Roger
%
% Template license:
% CC BY-SA 4.0 (https://creativecommons.org/licenses/by-sa/4.0/)

%Section: Languages
\sectionTitle{\textsc{Language}}{\faLanguage}
\begin{skills}
	\skill{Mandarin}{5}&\skill{English}{4} \\
    \skill{Dutch}{1}& \\

\end{skills}

\resheading{获奖情况}
  \begin{itemize}[leftmargin=*]
    \item \ressubsingleline{西南交通大学综合奖学金}{二等奖三次,三等奖两次}{2009.09 -- 2013.06}
    \item \ressubsingleline{高教社杯全国大学生数学建模竞赛}{四川赛区$\,$一等奖}{2012.09}
    \item \ressubsingleline{西南交通大学第六期大学生科研训练计划项目}{校级优秀项目}{2012.05}
    \item \ressubsingleline{高教社杯全国大学生数学建模竞赛}{全国$\,$二等奖}{2011.09}
    \item \ressubsingleline{“中国电机工程学会杯” 全国大学生电工数学建模竞赛}{全国$\,$三等奖}{2011.12}
    \item \ressubsingleline{第八届苏北数学建模联赛}{全国$\,$二等奖}{2011.05}
    \item \ressubsingleline{西南交通大学数学建模竞赛}{二等奖}{2011.05}
    % \item \ressubsingleline{数学学院优秀学生干部}{}{2011.11}
    \item \ressubsingleline{精神文明建设积极分子}{}{2010.11}
    % \item \ressubheading{高教社杯全国大学生数学建模竞赛$\,$(本科组)}{四川赛区$\,$一等奖}{《葡萄酒的评价》}{2012.09}
    % \item \ressubheading{西南交通大学第六期大学生科研训练计划项目}{校级优秀项目}{《公共养老保险风险建模与应用分析》}{2012.05}
    % \item \ressubheading{高教社杯全国大学生数学建模竞赛$\,$(本科组)}{全国$\,$二等奖}{《交巡警服务平台的设置与调度》}{2011.09}
    % \item \ressubheading{“中国电机工程学会杯” 全国大学生电工数学建模竞赛}{全国$\,$三等奖}{《风功率预测问题》}{2011.12}
    % \item \ressubheading{第八届苏北数学建模联赛}{全国$\,$二等奖}{《旅游线路的优化设计》}{2011.05}
    % \item \ressubheading{西南交通大学数学建模竞赛}{二等奖}{《成都市三环线——绕城高速西北区域公交路线的设计》}{2011.05}
    % \item \ressubsingleline{数学学院优秀学生干部}{}{2011.11}
    % \item \ressubsingleline{精神文明建设积极分子}{}{2010.11}
    % \item \ressubsingleline{西南交通大学综合奖学金}{三次二等,两次三等}{2009.09 -- 2013.06}
  \end{itemize}

\end{document}
