% !TEX TS-program = luatex
% Awesome Source CV LaTeX Template
%
% This template is originally maintained by Christophe ROGER from:
% https://github.com/darwiin/awesome-neue-latex-cv
%
% This version is based on the modification from:
% https://github.com/innerTide/awesome-neue-latex-cv-extended
% Author:
% Zhihao Zhou
%
%

%roman numeral
\makeatletter
\newcommand{\rmnum}[1]{\romannumeral #1}
\newcommand{\Rmnum}[1]{\expandafter\@slowromancap\romannumeral #1@}
\makeatother

\documentclass[localFont, alternative]{awesome-source-cv}

% Name should be {<Given name>} {<Family name>}.
% The display order is automatically reversed with Chinese support is enabled.

\name{Zhihao}{Zhou}
\photo{2.6cm}{figs/photo}
\tagline{M.Sc. of Electrical Engineering}
\socialinfo{
	\smartphone{+31(0)651655432}
	\email{zzh2616@hotmail.com}\\
	\linkedin{zhihao-zhou-cn1992}\\
  %   \linkedin{yuefeng-wu-cn1116}
	% \github{innerTide}\\
	\address{Burgwal 21, 2611 GE, Delft, the Netherlands}\\
	\infos{Born on \textbf{Mar. 10, 1992} in \textbf{Shandong, China}}\\
}
%------------------------------------------
\begin{document}

% \makecvheader
\makecvheaderwithoutphoto
%--------------------SECTIONS-----------------------------------
\cvsection{Education}
\begin{cventries}
  \cventryEdu
    {B.S. Computer Science, 3.78 GPA}
    {Pennsylvania State University}
    {University Park, PA}
    {Aug 2015 - May 2019}
\end{cventries}

%Project Experience
\sectionTitle{\textsc{Project Experience}}{\faCode}
\begin{experiences}
	\experience
		{Sept. 2017}	{Master Thesis Intern}{imec}{Eindhoven, the Netherlands}
		{Aug. 2016}	{
						\begin{itemize}
							\item Title: a digital-intensive wakeup timer based on an RC frequency-locked loop for Internet of Things (IoT) applications
							\item A novel ultra-low-power oscillator was modeled with MATLAB and designed in Cadence to fully exploit the small area and low supply voltage advantages in advanced CMOS processes
							\item An automated measurement script was developed with Python and the results show the oscillator has the lowest supply voltage, the highest energy efficiency, and comparable stabilities w.r.t. the state-of-the-art
						\end{itemize}
					}
								{Integrated Circuit design, Verilog-HDL, MATLAB, Python, Git, \LaTeX}
		 {height = 1.25cm}		{figs/imec}
	\emptySeparator

	\experience
	{Aug. 2015}	{Intern}{Changzhou Research Institute of Zhejiang University}{Jiangsu, China}
	{June 2015}{
					\begin{itemize}
						\item Explored control algorithms in smart car systems
						\item Adapted the Kalman Filter into the navigation system of the car using data from multiple sensors, e.g., gyroscope and accelerometer
						\item Collaborated with other colleagues in the PCB design of the hardware
					\end{itemize}
				}
							{Control System, Embedded System, MATLAB, PCB, Team Collaboration}
	 {width = 1.5cm}		{figs/zju}
\end{experiences}

\begin{projects}
	\project %add name of the algorithm used!!!!!!!!!!!!
		{June 2014}   {Bachelor Thesis: Design of An Object Tracking Software}
		{Dec. 2013} {
											\begin{itemize}
												\item Conducted research on various object tracking algorithms
												\item Designed a Windows software that can track multiple objects in a video file with OpenCV libraries using Visio Studio
												\item Optimized the software to handle real-time camera input for demonstration purposes
											\end{itemize}
										}
										{C++, Object Tracking, Windows, OpenCV, Visio Studio}
\end{projects}

% % A extension of Awesome Source CV LaTeX Template
%
%
% This template is originally maintained by Christophe ROGER from:
% https://github.com/darwiin/awesome-neue-latex-cv
%
% This modification has been download from:
% https://github.com/innerTide/awesome-neue-latex-cv-extended
% Author:
% Yuefeng Wu
%
%
% Render an experience in the experiences environment
% Usage:
% \experience
%  {<End date>}      {<Title>}{<Enterprise>}{<Location>}
%  {<Start date}     {
%                      <Experience description (Could be a list)>
%                    }
%                    {<Technology list>}
%  {<Logo parameter>}	{<Logo File Name>}
\sectionTitle{\textsc{Internship}}{\faSuitcase}

\begin{experiences}
	\experience
    {Sept. 2017}	{Master Thesis Intern}{imec}{Eindhoven, the Netherlands}
    {Aug. 2016}	{
    				\begin{itemize}
              \item Title: a digital-intensive wakeup timer based on an RC frequency-locked loop for Internet of Things (IoT) applications
    					\item Proposed a novel ultra-low-power oscillator architecture to fully exploit the small area and low supply voltage advantages in advanced CMOS processes
              \item System-level modeling with MATLAB to estimate stabilities of the oscillator w.r.t. temperature, supply and noise
              \item Transistor-level implementation in Cadence and tape-out in TSMC 40-nm CMOS
              \item Measurements show the oscillator has the lowest supply voltage, the highest energy efficiency, and comparable stabilities w.r.t. the state-of-the-art
    				\end{itemize}
    			}
                {Analog/Mixed-Signal, Ultra-Low Power, Oscillator, MATLAB, Low Noise, Cadence, Layout, Measurement, Git}
     {height = 1.25cm}		{figs/imec}
	\emptySeparator

    \experience
    {Aug. 2015}	{Intern}{Changzhou Research Institute of Zhejiang University}{Jiangsu, China}
    {June 2015}{
    				\begin{itemize}
              \item Explored control algorithms in smart car systems
    					\item Adapted the Kalman Filter into the navigation system of the car using data from multiple sensors, e.g., gyroscope and accelerometer
              \item Collaborated with other colleagues in the PCB design of the hardware
    				\end{itemize}
    			}
                {Control System, Embedded System, MATLAB, PCB, Team Collaboration}
     {width = 1.5cm}		{figs/zju}

\end{experiences}

% % Awesome Source CV LaTeX Template
%
% This template is originally maintained by Christophe ROGER from:
% https://github.com/darwiin/awesome-neue-latex-cv
%
% This modification has been download from:
% https://github.com/innerTide/awesome-neue-latex-cv-extended
% Author:
% Yuefeng Wu
%
% Template license:
% CC BY-SA 4.0 (https://creativecommons.org/licenses/by-sa/4.0/)
% Section: Projects
% Usage:
% \project
%  {<End date>}      {<Project Title>}
%  {<Start date}     {
%                      <Contribution to the project>
%                    }
%                    {<Technology list>}
%
%


\sectionTitle{\textsc{Project Experience}}{\faCode}
\begin{projects}

  \project
    {June 2016} {All Digital Phase-Locked Loop (ADPLL)}
    {May. 2016} {
                      \begin{itemize}
                        \item Course project of Digital RF
                        \item Learned knowledge about digital RF and frequency synthesis
                        \item Built a time-domain model of the ADPLL based on its phase operation
                        \item System-level design with the model for behavior-level noise simulation in using MATLAB
                      \end{itemize}
                    }
                    {Digital RF, ADPLL, Frequency Synthesis, Modeling, MATLAB}
  \emptySeparator


  \project
    {May. 2016}   {Transistor Fabrication}
    {Apr. 2016} {
                      \begin{itemize}
                        \item Course project of IC-technology lab.
                        \item Learned basic CMOS fabrication steps and their physical mechanisms with a 1-μm Bi-CMOS process in Else Kooi Lab at Delft University of Technology
                        \item Simulation and hands-on operation of the fabrication of MOS transistors in a clean room
                        \item Lab measurement of the fabricated transistors using a microscope and a probe station
                      \end{itemize}
                    }
                    {CMOS process, Simulation, Fabrication, Clean Room, Measurement}
  \emptySeparator

  \project
    {Mar. 2016}   {Audio Amplifier Design}
    {Feb. 2016} {
                      \begin{itemize}
                        \item Course project of Analog CMOS Design
                        \item Designed a class-AB amplifier architecture to handle a low-ohmic load with a rail-to-rail swing
                        \item Implemented the amplifier using LTspice in 0.18-μm CMOS, and it achieved a high SNR, a high SFDR, and a low IM3
                        \item Optimized the gain and phase margins of the amplifier to achieve a stable operation within the given bandwidth
                      \end{itemize}
                    }
                    {Analog Design, Amplifier Design, Low Noise, Stability Margins, LTspice}
  \emptySeparator

  \project
    {Jan. 2016}   {Time-to-Digital Converter (TDC)}
    {Dec. 2015} {
                      \begin{itemize}
                        \item Course Project of Digital IC Design
                        \item Designed a 10-bit TDC with a 4-bit delay line and a counter to save area and power
                        \item Implemented the TDC using Cadence in UMC 90-nm CMOS, and it achieved a worst-case 27-ps resolution with both DNL and INL smaller than 1 at every process corner
                        \item Optimized the area of the TDC in the layout
                      \end{itemize}
                    }
                    {Mixed-Signal Design, TDC, Corner Simulation, Layout, Cadence}
  \emptySeparator

  \project
    {Dec. 2015}   {Low-Noise Amplifier (LNA)}
    {Nov. 2015} {
                      \begin{itemize}
                        \item Course project of Microwave Circuit Design
                        \item Learned impedance matching, stability, and noise figure in microwave/RF designs
                        \item Designed a CMOS LNA with an inductive degeneration for simultaneous input noise and impedance matching using ADS
                        \item Designed the corresponding output matching network of the LNA to achieve stability
                      \end{itemize}
                    }
                    {Microwave/RF Design, LNA, Matching Network, Stability, ADS}
    \emptySeparator

    % \project %add name of the algorithm used!!!!!!!!!!!!
    %   {June 2014}   {Bachelor Thesis: Design of An Object Tracking Software}
    %   {Dec. 2013} {
    %                     \begin{itemize}
    %                       \item Conducted research on various object tracking algorithms
    %                       \item Designed a Windows software that can track multiple objects in a video file with OpenCV libraries using Visio Studio
    %                       \item Optimized the software to handle real-time camera input for demonstration purposes
    %                     \end{itemize}
    %                   }
    %                   {C++, Object Tracking, Windows, OpenCV, Visio Studio}
    %   \emptySeparator

  \project
    {May. 2014}   {Micro-Electro-Mechanical System (MEMS) Magnetic Sensor}
    {Mar. 2013} {
                      \begin{itemize}
                        \item Student research project at Ministry of Education Key Lab. of MEMS, funded by National Science Foundation of China (No. 61201032)
                        \item Designed a two-dimensional MEMS magnetic sensor in collaboration with other students using ANSYS
                        \item Simulation and optimization of the sensor under various electrical and magnetic (EM) conditions and at process corners
                        \item Resulted in two patent applications
                      \end{itemize}
                    }
                    {MEMS, Sensor, Team Collaboration, Modeling, Simulation, ANSYS}
  %   \emptySeparator
  %
  % \project
  %   {Apr. 2013}   {Smart Car based on Camera Input}
  %   {Mar. 2013} {
  %                     \begin{itemize}
  %                       \item A prototype for the 7\textsuperscript{th} Freescale Smart Car competition at Southeast University
  %                       \item Built a smart car model with two PWM motors for driving and steering, a greyscale camera for capturing road information, and a microcontroller for overall control
  %                       \item Designed the PCB for the microcontroller in collaboration with my teammates
  %                       \item Implemented the PID motor control part of the car control software in μC/OS-\Rmnum{2}
  %                     \end{itemize}
  %                   }
  %                   {C, μC/OS-\Rmnum{2}, PCB Design, Altium Designer, Team Collaboration}
  %   \emptySeparator
  %
  %   \project
  %     {Dec. 2012}   {Digital Altimeter}
  %     {June 2012} {
  %                       \begin{itemize}
  %                         \item Funded by Student Research Training Program of Southeast University
  %                         \item Designed the altimeter hardware with a microcontroller PCB, a barometric sensor, a temperature sensor, and a LCD display
  %                         \item Implemented the altimeter software using barometric information with temperature error correction in μC/OS-\Rmnum{2}
  %                       \end{itemize}
  %                     }
  %                     {C, μC/OS-\Rmnum{2}, Sensor, PCB Design, Altium Designer}

\end{projects}

% \sectionTitle{\textsc{Patent}}{\faNewspaperO}
  \begin{enumerate}
    \item Jie Chen, \textsc{Zhihao Zhou}, Minghao Xue; \textit{A MEMS magmetic filed sensor with a comb tooth structure}; CN Patent 201310456380; filed September 30, 2013, and issued September 9, 2015.
    \item Jie Chen, Qiushi Liang, \textsc{Zhihao Zhou}; \textit{A dual-torsion-pendulum type MEMS magnetic filed sensor}; CN Patent 201310455985; filed September 30, 2013, and issued September 23, 2015.
  \end{enumerate}

%\sectionTitle{\textsc{Extracurricular Activity}}{\faSoccerBallO}

\begin{experiences}
	\experience
	{Aug. 2013}	{Trainee}{Xilinx Summer School}{Nanjing, China}
	{July 2013}	{
		\begin{itemize}
			\item Innovation training program organized by Xilinx in Southeast University
			\item Learned industrial project development and management skills
			\item Conducted a smart car project using a FPGA/ARM platform in collaboration with other trainees
		\end{itemize}
	}
	{Project Development, Project Management, Team Building, Verilog-HDL, C, FPGA, ARM}
	{height = 0.4cm}		{figs/xilinx}
	\emptySeparator

	\experience
	{June 2012}	{Member}{Red Cross Society of Southeast University}{Nanjing, China}
	{June 2011}	{
		\begin{itemize}
			\item Responsible for advertising organization events by making posters, flyers, etc.
			\item Planned and organized volunteer activities, and received a considerable amount of positive feedback
		\end{itemize}
	}
	{Volunteer Work, Time Management, Communication Skills, Coordination Capabilities}
	{width = 1.5cm}		{figs/redcross}
\end{experiences}

%\resheading{技能专长}
  \begin{itemize}[leftmargin=*]
    \item \textbf{语言}: C/C++,C\#,HTML,CSS,JavaScript,\href{http://cn.mathworks.com/products/matlab/}{MATLAB},\href{http://www.latex-project.org/}{\LaTeX}
    \item \textbf{系统}: Windows,Linux
    \item \textbf{软件/工具}: MS Visual Studio,eclipse,Chrome Developer Tools,Photoshop,Sublime Text,Atom,Git,SVN,Node.js,SAS,MATLAB
    \item \textbf{项目/框架}: \href{http://www.bootcss.com/}{Bootstrap},\href{https://jquery.com/}{jQuery},\href{http://jeasyui.com/}{EasyUI},less,sass,\href{http://gulpjs.com/}{gulp},ASP.NET
    % ,\href{https://angularjs.org/}{AngularJS},\href{http://gulpjs.com/}{gulp},sass/compass
    \item \textbf{数学}:数据分析、数值计算、数值优化(积分方法和智能算法)
    \item \textbf{英语}: 能够流畅的阅读科技博文,并熟练的用英文检索开发中遇到的问题
  \end{itemize}
\sectionTitle{\textsc{Professional Skills}}{\faTasks}
\renewcommand{\arraystretch}{1.1}

	\begin{tabular}{>{}r>{}p{13cm}}
		%\textsc{Development Tools:}	  			&   Cadence, LTspice, ADS, Vivado, Quartus \Rmnum{2}, Altium Designer, ANSYS, Git\\
		\textsc{Programming Languages:}  		&   C/C++, MATLAB, Java, Python, Verilog-HDL, VHDL\\
		\textsc{Operating Systems:}	        &   Windows, Linux, macOS\\
    \textsc{Miscellaneous:}							&   Team Collaboration, \LaTeX, Microsoft Office, Computer Maintenance\\
	\end{tabular}

%% Awesome Source CV LaTeX Template
%
% This template has been downloaded from:
% https://github.com/darwiin/awesome-neue-latex-cv
%
% Author:
% Christophe Roger
%
% Template license:
% CC BY-SA 4.0 (https://creativecommons.org/licenses/by-sa/4.0/)

%Section: Languages
\sectionTitle{\textsc{Language}}{\faLanguage}
\begin{skills}
	\skill{Mandarin}{5}&\skill{English}{4} \\
    \skill{Dutch}{1}& \\

\end{skills}

%!TEX root = ../resume.tex

% Awards

\subsection*{Honors and Awards}

Frank H.\ T.\ Rhodes Award (2014) $\bullet$ Michael W. Mitchell Prize (2013) $\bullet$ Dean's List (all semesters) $\bullet$ Gertrude Spencer Prize Honorable Mention (2011) $\bullet$ Lockheed Martin Foundation Scholarship (2010) $\bullet$ Eagle Scout with 3 palms


\end{document}
