\chapter{Conclusion}

Starting from the cleft sentences, we have explored the relationship between focus effect and pronoun resolution from a number of perspectives, including syntax, pragmatics and quantitative research data collected by others and our own experiment. We have seen that, focus as an special information structure is constantly interacting with other elements of discourses although it is usually assumed to have prominent cognitive status.

The results of our experiment agree with previous reseaches in some ways, such as the non-facilitative effect of focus in intra-sentential conditions, and the different preferences for antecedent in French and English pronoun resolution. However, some questions remain unsolved in our studies, including the precise mechanism behind object preference in English, the contradictory results about the congruence between focus and anaphora, and after all the influence of the specific reading strategies taken by the participants. The influence of L2 is explicitly reflected in the slower response time for nearly all test units in French, but whether L1 (Chinese) has an impact in our research still needs further verification. 

Focus marking and pronoun resolution are linguistic phenomena that involve complicated cognitive processes. In this regard, our experiment procedure is limited in many aspects. First of all is the scale. The number of test sentences as well as participants could be expanded. The second problem is the too simplistic experiment procedure and analytical methodology. We believe that the our research could be greatly improved by the inclusion of larger corpus materials and more mature statistical tools. Nevertheless, this thesis offers a glimpse into the studies of information structure and pronoun resolution. Hopefully, insights from our study could help further explorations in this field.



