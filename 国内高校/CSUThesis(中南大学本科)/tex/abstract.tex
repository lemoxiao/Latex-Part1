%# -*- coding: utf-8-unix -*-
%%==================================================

\begin{abstract}
本项目为年产50万吨MTO工厂的初步设计。通过分析当前国内外MTO生产和研究现状,对生产工艺进行了选择论证。然后运用Aspen软件模拟初步的工艺流程,并通过对一系列工艺参数,如精馏塔的塔板数—产品纯度、进料塔板数—产品纯度、产品纯度—回流比、再沸器负荷—回流比等进行灵敏度分析,优化设备操作条件,提高工艺的合理性和经济性。本设计还针对工艺流程进行换热网络设计和对全局换热网络进行了优化和评估,通过内部流股之间相互换热以减少公用工程的消耗,最终优化后节约$79.4\%$的热公用工程资源和$73.7\%$的冷公用工程资源。本设计还运用水夹点技术优化了用水网络,根据水硬度分类处理水操作单元,并合理再生利用,使得本项目新鲜水用量和废水排放量达到最小,优化后的用水网络节约用水$53.59\%$。本设计对于MTO工厂的生产和设计建造具有一定的现实指导意义。\\

\keywords{\zihao{-4} 工厂\quad 设计\quad MTO \quad 工艺 \quad 水夹点  \quad 网络 \quad 控制}
\end{abstract}

\begin{englishabstract}

This project is the preliminary design of a MTO plant with an annual output of 500,000 tons of light olefins. Based on the current production and research situation all through the world, the production method was selected and demonstrated. Aspen software was used to simulate the preliminary process. Heat integration method was applied to optimize the heat exchange network. Rational heat exchange between process streams were suggested which resulted in the decreasing of utilities consumption and exchanger number. The heat integration leaded to energy saving of $79.4\%$ of heat utilities and $73.7\%$ of the cold utilities. In addition, the water pinch technology was also implemented to optimize the water network. The water operating unit was classified according to water hardness, with a reasonable recycling. The amount of fresh water consumption and wastewater emission was minimized. The optimized water network achieved $53.59\%$ water saving. Finally, a preliminary economic analysis to the entire project was estimated in order to get the project construction cost and profitability. In summary, this design is of some practical significance for the production and design of the MTO industry.

\englishkeywords{\zihao{-4} Plant design\;Sensitivity analysis  \; Energy balance\; calculation \; Water pinch  Dynamic control}
\end{englishabstract}

