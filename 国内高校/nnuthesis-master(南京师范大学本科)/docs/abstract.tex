% Abstract
\clearpage
\thispagestyle{plain}
\phantomsection
\addcontentsline{toc}{chapter}{Abstract}

\centerline{\zihao{3}\bfseries Abstract}

\linespread{1.4}\zihao{-4}
\bigskip

This thesis explores the relationship between focus structure and pronoun resolution among non-native speakers of English and French. Firstly we reviewed the existing literature on the mechanism of focus effect and pronoun resolution. Then through a self-paced reading test, we find that focus, in the form of cleft structure does not necessarily increase the salience of a informational unit, thus may not in some cases make it a preferred antecedent for pronoun resolution. This result is line with previous researches on this topic. In our experiment, We also find that focused subject in French and focused object in English are processed faster, but focused subjects in both languages leads to longer response time of anaphora. Furthermore, our research also shows that the congruence between anaphora and focus does not make the latter more accessible. In this regard, we argue that the problem of whether there is subject or object preference in English and French is more complicated than the results of current studies.

\bigskip
\noindent\textbf{\zihao{4} Keywords:} 
focus effect, pronoun resolution, self-paced reading, English, French

