\chapter{The Current Study}

\section{Experiment Design}
In this study, we conducted a self-paced reading test in English and French respectively, aiming to explore how cleft structure affects pronoun resolution of non-native speakers of English and French. Both the test results of the selected sentences and the test procedure will be analyzed.

\subsection{Test Materials}
Twenty four groups of  non-ambiguous test sentences fall into six types (A--F) differed by their positions of focus and directions of anaphora, as are illustrated in \autoref{tb:corpus}. The selection of these test sentences adopts the method of \emph{Latin square}, thus each of the in total six lists contains twenty four sentences selected from each group, and each type of sentences occurs four times within each list. The benefit of this arrangement is that sentences in every list cover all six types, but do not come from the same group, so that subjects of the experiment picking one random list do not come cross sentences of the same semantic context. Because there are six lists, each list is tested three times.

The types of the sentences in every group are arranged as follows: \begin{enumerate}
    \item the pattern of anaphora is consistant, pointing to the subject and object of these sentences alternatively;
    \item type A and B are subject-clefted sentences; type C and D are object-clefted sentences;
    \item type E and F are sentences without focus structure.
\end{enumerate}   
Aside from twenty four test sentences selected from twenty four groups, each list also contains six \emph{filler items} (type G), which are sentences contains neither pronoun resolution nor cleft structure. All test sentences and filler items are of word length from seven to twelve, which is also a factor we take into consideration in our analysis. 

\subsection{Participants}
The participants are thirty-six final-year English and French major students from Nanjing Normal University. They all have beening learning English for over ten years, and are TEM-8 certificate holders. The French speaking participants have been learning the language for over three years, and they have TFS-4 certificates.

\subsection{Procedure}
The participants took full control of the pace of their reading by pressing the \textsf{SPACE} bottom to spontaneously display and hide consecutive words one by one of the test sentences. Response time for each word was recorded in miliseconds. We pay special attention to the response time of focus and anaphora positions of test sentences. Focus is the first word after \emph{It is} in English and \emph{C'est} in French. Anaphora takes the form of pronouns (he or she) in our tests.

In addition, half of the test sentences are followed by test questions, i.\,e.\,, either true or false statements relating to the previous sentences. Test questions, when displayed, participants press \textsf{A} on the keyboard if the statement is true based on the context and \textsf{L} if they do not match. For test sentences without following questions, participants simply press \textsf{SPACE} again to continue. Response time for test questions were also recorded.

\begin{landscape}\thispagestyle{empty}\centering
\begin{table}[htb]\caption{Test Materials}\label{tb:corpus}
\begin{tabular}{cp{.5\linewidth}p{.31\linewidth}c}
    \toprule
    \multicolumn{1}{l}{\textbf{Type}} & \multicolumn{1}{l}{\textbf{Test Sentence}} & \multicolumn{1}{l}{\textbf{Question}} & \multicolumn{1}{l}{\textbf{Answer}}\\
    \midrule
    A & It is Mary who hit Jason when she was in Paris. & \multirow{6}*{Mary was in New York.} & \multirow{6}*{\textsf{L}}\\
    B & It is Mary who hit Jason when he was in Paris. \\
    C & It is Jason whom Mary hit when she was in Paris. \\
    D & It is Jason whom Mary hit when he was in Paris. \\
    E & Mary hit Jason when she was in Paris. \\
    F & Mary hit Jason when he was in Paris. \\
    G & The bride invited the nun to her wedding & The nun was invited to the wedding. & \textsf{A} \\
    \midrule
    A & C'est Mary qui a frappée Jason quand elle était à Paris. & \multirow{6}*{Mary était à New York.} & \multirow{6}*{\textsf{L}} \\
    B & C'est Mary qui a frappée Jason quand il était à Paris.\\
    C & C'est Jason que Mary a frappé quand elle était à Paris.\\
    D & C'est Jason que Mary a frappe quand il était à Paris.\\
    E & Mary a frappée Jason quand elle était à Paris.\\
    F & Mary a frappée Jason quand il était à Paris.\\
    G & La Mariée a invitée la religieuse à son mariage. & La religieuse a été invitée à la mariage. & \textsf{A} \\
    \bottomrule
\end{tabular}
\end{table}
\end{landscape}


\section{Results}

We mesured the average response time for focus and anaphora positions as well as the questions, with the longest and the shortest response time filtered. In our experiment, participants take shorter time to process focus and longer time to process anaphora than words before and after the two positions in both languages. Detailed results are shown in \autoref{tb:rt_en} and \autoref{tb:rt_fr}. 

\begin{table}[htb]\caption{Average Response Time of English Test Materials}\label{tb:rt_en}
    \centering
    \begin{tabular}{lllllll}
        \toprule
        & A & B & C & D & E & F \\
        \midrule
        Focus & 283 & 253 & 223 & 260 \\
        Anaphora &  321 & 290 & 314 & 284 & 294 & 269 \\
        Question & 1734 & 1547 & 1143 & 1755 & 688 & 582 \\
        \bottomrule
    \end{tabular}
\end{table}

\begin{table}[htb]\caption{Average Response Time of French Test Materials}\label{tb:rt_fr}
    \centering
    \begin{tabular}{lllllll}
        \toprule
        & A & B & C & D & E & F \\
        \midrule
        Focus & 355 & 393 & 465 & 481 \\
        Anaphora & 446 & 351 & 333 & 539 & 349 & 447 \\
        Question & 3193 & 3349 & 2260 & 1910 & 656 & 549 \\
        \bottomrule
    \end{tabular}
\end{table}

Apart from focus and anaphora response time, we are also seeing that French verbs are processed longer than the English counterparts. Furthermore, we detected irregularly longer response time of some words sporadically.  For example, one participant spent \SI{1866}{ms} on the word \emph{whom} in the sentence \emph{It is Catherine whom Matthew arrested when she was living in Africa} (Type D, List A), while on focus and anaphora, the participant spent \SI{760}{ms} and \SI{447}{ms}, and around \SI{450}{ms} on other words. Another noticeable irregularity occurs in the sentence \emph{It is Michelle whom Roland mistreated when he was an adolescent}. One participant stayed for \SI{1183}{ms} on the word \emph{Roland}, and \SI{687}{ms} on \emph{adolescent}, but on other words, no more than \SI{346}{ms} was spent. 

\section{Discussion}
\subsection{General Analysis of English and French Data}
Generally, participants taking French tests tend to take longer time to read almost all testing units than participants taking English tests. The reason behind this result is two-fold: firstly, the participants have been learning English for over ten years, but they did not start taking French courses until entering college. Therefore, the profiency level of the L2 is a major cause of the difference between the response time of French and English; secondly, as we will be discussing later, French words are more complex morphologically, due to the conjugaison system. When encountering French verbs, participants also need to pay attention to the suffix \emph{-e}, if exists, which carries opposite meanings in subject and object-clefted sentences. For example, in 
\begin{align}
    \mbox{\emph{C'est Bella qui a hébergée Danis quand elle était célibataire.}}\\
    \label{bella}  
    \mbox{(\emph{It is Bella who housed Danis when she was single.})} \notag
\end{align}
the feminine form \emph{hébergée} signals that the agent, \emph{Bella}, is female, but in
\begin{align}
    \mbox{\emph{C'est Jane que Paul a aidée quand il vivait à Lyon.}}\\
    \mbox{(\emph{It is Jane whom Paul helped when he was in Lyon})} \notag
\end{align}
the word \emph{aidée} is suffixed by \emph{-e} only because the agent Jane is female.


However, there are some similar patterns in the two languages. In every list of French and English, focus processing appears to be more time-saving while anaphora resolution more time-consuming when compared with other testing units. We attribute this phenomenon to the memory burdens participants took during the tests. As the focused element usually occur at the beginning of a sentence (W3 in English and W2 in French), it becomes more prominent naturally and easier to be understood than anaphora which is usually located at the last four or five word. As other informational units drag memory burdens to the end of the sentence, it takes longer time for participants to recall the previous information, which is the same reason why response time for questions is even much longer. 

\subsection{Comparison with Previous Studies}
In our experiment, the result that focus does not necessarily render the antecedent more accesible in the same sentence is in line with \citet{colonna2015}. Data shows that non-focused sentence type E and F make the question--answer procedure significantly faster than that of sentence types with focus. Since focus and anaphora happen in the same test sentence, our results can be one proof of \citet{grosz1995}'s argument that focus is less prominent than other information structure within a single processing unit. From the point of view of syntactic complexity, we may also argue that the shorter sentence length of non-focused structure plays a bigger role here due to weaker challenges on short-time memory.

Subject preference in French and object preference in English, as proposed by \citet{reichle2014}, is verified in our tests, as indicated by the shorter processing time of focus of sentence type C and D in English and type A and B in French. Apart from the cognitive and grammatical evidence which suggests that the French \emph{c'est} structure enjoys special status \citep{lambrecht2001}, we may propose that the mandatory concatenation of \emph{cest} and \emph{est} in French, rather than optional form \emph{it's} in English helps the French focus to be put at a more forward and prominent position. However, in both languages, focused subject brought about longer response time of anaphora. The exact reason is is still a myth to us, and we either do not know why English objects are preferred, particularly by non-native speakers. 

Lastly, our experiment also finds that the congruence of focus and anaphora does not make the response of question easier. That is, the fact that anaphora points to focus does not improve the efficiency of pronoun resolution. In English materials, response time of anaphora in sentence type A and type C, where focus and anaphora are consistantly subject or object, is longer than that of sentence type B and type B where focus and anaphora do not match. In French materials, anaphora processing was sped up by object congruences, but was slowed by subject congruences. This discovery partly contradicts with the research of \citet{patterson2017}, according to whom, sentence type A and C should take less response time, which is however not the case in our study.

\subsection{Limitations}
One major factor of influence we have difficulty in assigning to the current logic is the exact cognitive approaches taken by the participants. We are not clear whether 
\begin{enumerate}
    \item certain words, or even names in our materials such as ``Santa Claus'' or ``Mary'' bear certain connotations or special memories for the participants so that they linger on this or even the next test sentence;
    \item they take distinct methodologies when offered the test materials of different languages;
    \item and their specific strategies have decisie influence on pronoun resolution.
\end{enumerate}
 Additionally, it needs to be clarified that to what extent does the response to the previous test sentences/questions affect the response time as well as the test-taking strategies of the following ones.

The different processing speeds among our participants can be both subtle and significant in the data collected. Indeed, some people are faster readers than others \citep{aravind2017}. However, a thorough explanation for the nuances in focus and anaphora processing requires the larger scale and more sophisticated corpus mining. Thus we cannot guarantee that our results will hold in other similar experiments. This problem of replicability \citep{brandt2014} also elicits us to doubt the effectiveness of the self-paced reading procedure, which is apparently not a perfect imitation of the normal reading process in the real world.  