%
% 说明:使用 xelatex 编译,在 texlive 2017 中编译通过
%
\documentclass[UTF8]{ctexart}
\usepackage[a4paper,top=2.0cm,bottom=2.5cm,left=2.8cm,right=2.8cm,%
            includehead,includefoot]{geometry}
\usepackage{amsmath,amssymb,amsfonts,bm}
\usepackage{graphicx,xcolor}
\usepackage[xetex,colorlinks,linkcolor=blue]{hyperref}
\usepackage{makecell,interfaces-makecell}
\usepackage{calc}
\renewcommand{\baselinestretch}{1.3}

%%%%% ===== 自定义命令 ===========================================================
\newcommand{\ul}[1]{\underline{\makebox[#1]{}}}
\newcommand{\ull}[2]{\underline{\makebox[#1]{\kaishu #2}}}
\newcommand{\ulaa}[1]{\underline{\makebox[3em]{\Large\textcircled{\normalsize #1}}}}
\newcommand{\dis}{\displaystyle}

\pagestyle{plain}

\begin{document}
\zihao{-4} % 小四号字体

%%%%% ===== 试卷头 ===============================================================
\begin{center}
{\heiti\LARGE 华东师范大学期末试卷 (A)} \bigskip

$20xx-20xx$ 学年第 X 学期

\bigskip\bigskip

%%%%% ===== 课程信息,包括课程名,课程性质等
\setlength{\tabcolsep}{1mm}
\renewcommand{\arraystretch}{1.4}
\begin{tabular}{p{7.5cm}p{6.5cm}}
  课程名称:\ull{4.0cm}{\zihao{4} 数~值~分~析} & \\
  学生姓名:\ul{4.0cm}          &
  学\hspace{10.5mm}号:\ul{4.0cm} \\
  专\hspace{2em}业:\ull{4.0cm}{数学与应用数学}  &
  年级/班级:\ull{4.0cm}{20xx级} \\
  课程性质:\ {\kaishu 专业X修} \\
\end{tabular}\smallskip

%%%%% ===== 得分表,其中 numexer 表示题号个数,这里设为 8,可根据实际需要修改,其他可以不用修改。
\newcounter{numexer}\setcounter{numexer}{8} % 题号个数,这里为 8
\newcounter{numcol}\setcounter{numcol}{\value{numexer}+2}
\newlength{\cellwidth}\setlength{\cellwidth}{\textwidth*\ratio{0.6pt}{\value{numexer} pt}}
\begin{tabular}{|*{\thenumcol}{c|}} \hline
  \repeatcell{\thenumexer}{rows=1,text=\makebox[\cellwidth]{\zhnumber{\column}}}
   & \makebox[0.12\textwidth]{总分} &\makebox[0.16\textwidth]{阅卷人签名} \\ \hline
  \repeatcell{\thenumcol}{rows=1,end=\\ \hline} \\ \hline
\end{tabular}
\end{center}

\medskip
\noindent\dotfill
\medskip

\linespread{1.5}\selectfont
\noindent{\heiti 一、填充题} (每空 2 分,共 24 分)

\newcounter{forlist}
\begin{list}{\arabic{forlist}.}
            {\setlength{\topsep}{2mm}
             \setlength{\listparindent}{0pt}
             \setlength{\labelsep}{5pt}
             \setlength{\itemsep}{0.8em}
             \setlength{\parsep}{5pt}
             \usecounter{forlist}}

\item
十进制数与二进制数的转换: $(123)_{10}= (\ulaa{1})_2$,
$(110)_2 = (\ulaa{2})_{10}$。

\item
十进制数与二进制数的转换: $(123)_{10}= (\ulaa{3})_2$,
$(110)_2 = (\ulaa{4})_{10}$。

\item
十进制数与二进制数的转换: $(123)_{10}= (\ulaa{5})_2$,
$(110)_2 = (\ulaa{6})_{10}$。


\end{list}


\newpage
\linespread{1.3}\selectfont

\begin{list}{\heiti\Chinese{forlist}、}
            {\setlength{\topsep}{1mm}
             \setlength{\listparindent}{0pt}
             \setlength{\itemsep}{2em}
             \setlength{\labelsep}{0pt}
             \usecounter{forlist}
             \setcounter{forlist}{1}}

\item (15 分) % ========================================================
设 $x_0,x_1,\ldots,x_n$ 为互异节点,求证: \\[1ex]
(1)$\dis\sum\limits_{j=0}^n x_j^k l_j(x) \equiv x^k $\quad ($k=0,1,\ldots,n$); \\
(2)$\dis\sum\limits_{j=0}^n (x_j-x)^k l_j(x) \equiv 0$\quad ($k=0,1,\ldots,n$)。

\item (15 分) % ========================================================
设 $x_0,x_1,\ldots,x_n$ 为互异节点,求证: \\[1ex]
(1)$\dis\sum\limits_{j=0}^n x_j^k l_j(x) \equiv x^k $\quad ($k=0,1,\ldots,n$); \\
(2)$\dis\sum\limits_{j=0}^n (x_j-x)^k l_j(x) \equiv 0$\quad ($k=0,1,\ldots,n$)。

\item (15 分) % ========================================================
设 $x_0,x_1,\ldots,x_n$ 为互异节点,求证: \\[1ex]
(1)$\dis\sum\limits_{j=0}^n x_j^k l_j(x) \equiv x^k $\quad ($k=0,1,\ldots,n$); \\
(2)$\dis\sum\limits_{j=0}^n (x_j-x)^k l_j(x) \equiv 0$\quad ($k=0,1,\ldots,n$)。


\end{list}

\end{document}