%
% 使用 xelatex 编译
%
\documentclass[10pt,compress,t]{ctexbeamer}
\usetheme{Warsaw}
\usefonttheme[onlymath]{serif}

%%%%% ===== 常用宏包 =======================================================
\usepackage{amsmath,amssymb,amsfonts,bm}
\usepackage{graphicx}
\usepackage{booktabs}


\begin{document}

\title[短标题]{学术报告标题\\ 长标题可以强制换行}
% \subtitle{也可以有个副标题}

\author[报告人姓名]{报告人姓名}

\institute[Math.ECNU]{\zihao{5} 华东师范大学~数学系}

\date[2017.10]{2017年10月}

\begin{frame}[plain]
  \titlepage
\end{frame}

\begin{frame}{内容提要}
  \tableofcontents[hideallsubsections]
\end{frame}

\section{背景介绍}

% 在每节前插入目录
\AtBeginSection[]{\frame{\tableofcontents[currentsection,hideallsubsections]}}

\begin{frame}
  \frametitle{背景介绍}
\begin{itemize}
\item 考虑问题
    $$ a^2+b^2=c^2.$$

\bigskip
\item
\uncover<2>{
 问题应用背景
    \begin{itemize}
        \item  xxxxx
        \item  xxxx
        \item  xxxxx
        \item $\cdots\ \cdots$
    \end{itemize}
}

\uncover<3->{
 问题应用背景2
}

\end{itemize}

\end{frame}

\section{定义与定理}
\begin{frame}{定义与定理}

  \begin{definition}
    这是连续的定义, 这是连续的定义, 这是连续的定义,
    这是连续的定义, 这是连续的定义, 这是连续的定义,
    这是连续的定义, 这是连续的定义, 这是连续的定义.
  \end{definition}
\end{frame}

\begin{frame}{定义与定理 (续)}
  \begin{theorem}[中值定理]
    这是中值定理, 这是中值定理, 这是中值定理, 这是中值定理,
    这是中值定理, 这是中值定理, 这是中值定理, 这是中值定理,
    这是中值定理, 这是中值定理, 这是中值定理, 这是中值定理,
    这是中值定理, 这是中值定理, 这是中值定理, 这是中值定理.
  \end{theorem}
\end{frame}

\section{算法描述}
\begin{frame}{算法描述}
\begin{itemize}
  \item 基本思想
  \begin{itemize}
    \item xxxxxx
    \item xxxxxx
  \end{itemize}
  \bigskip

  \item 主要优点
  \begin{itemize}
    \item  xxxx
    \item  xxxxx
  \end{itemize}
\end{itemize}
\end{frame}


\begin{frame}{算法 1}

  算法 1

\end{frame}


\section{数值实验}

\begin{frame}{数值算例}

\begin{center}
{Numerical results for Example 1}\smallskip

\begin{tabular}{ccccccccccc} \toprule
    &&&\multicolumn{2}{c}{GMRES(C)}
    &&\multicolumn{2}{c}{GMRES(L)}
    &&\multicolumn{2}{c}{GMRES(P)} \\ \cmidrule{4-5}\cmidrule{7-8}\cmidrule{10-11}
 $\theta$ & $N$ && Iter & CPU && Iter & CPU && Iter & CPU \\\hline
  0.5
 & $2^{11}$ &&  33&    0.04 &&  13&    0.02 &&  12&   0.01 \\
 & $2^{12}$ &&  33&    0.12 &&  14&    0.04 &&  12&   0.03 \\
 & $2^{13}$ &&  33&    0.26 &&  14&    0.09 &&  12&   0.08 \\
 & $2^{14}$ &&  33&    0.53 &&  15&    0.19 &&  12&   0.15 \\ \midrule
  0.8
 & $2^{11}$ &&  33&    0.04 &&  13&    0.02 &&  12&   0.01 \\
 & $2^{12}$ &&  33&    0.11 &&  14&    0.04 &&  12&   0.03 \\
 & $2^{13}$ &&  33&    0.25 &&  15&    0.10 &&  12&   0.08 \\
 & $2^{14}$ &&  33&    0.53 &&  16&    0.21 &&  12&   0.15 \\ \bottomrule
\end{tabular}
\end{center}

\end{frame}

\section{结论与展望}
\begin{frame}{结论与展望}

   这里是结论与展望 conclusion 和 remarks

\end{frame}


\begin{frame}[c,plain]
\begin{center}
\Huge\color{red}\heiti\bfseries 谢\quad 谢!

  Thank you!
\end{center}
\end{frame}

\end{document} 