% 用 xelatex 编译

\documentclass{ecnumaster}

%%%%% ===== 自定义命令
\renewcommand{\C}{\mathbb{C}}
\newcommand{\R}{\mathbb{R}}

\begin{document}

%%%%% ===== 中文封面信息
\graduateyear{20xx} % 毕业年份
\class{O241.6} % 分类号(数值线性代数是 O241.6)
\ctitle{\uline{论文标题论文标题标题\linebreak 如果一行放不下就放两行}}
\def\cctitle{论文标题} % 在原创性声明中使用, 不能出现手工换行
\caffil{数学系}
\cmajor{XXXX} % 计算数学
\cdirection{XXXX} % 数值代数
\csupervisor{某某某\, 教授}
\cauthor{XXXX}
\studentid{51088888888}
\cdate{20xx 年 xx 月}

%%%%% ===== 英文封面信息
\etitle{\uline{Title of Thesis Title of Thesis\linebreak
       Title Title Title of Thesis Title of\linebreak
       Thesis Title Title Title Title Title}}
\eaffil{Mathematics}
\emajor{xxxx xxxx} % Computational Mathematics
\edirection{xxxx xxxx} % Numerical Algebra
\esupervisor{XXX Xxxxxxx (Professor)}
\eauthor{ZHANG San}
\edate{xxx, 20xx}

%%%%% ===== 生成封面 =====
\newgeometry{top=2.0cm,bottom=2.0cm,left=2.5cm,right=2.5cm}
{ \renewcommand{\baselinestretch}{1.6} \makecover }
\clearpage{\pagestyle{empty}\cleardoublepage}

%%%%% ===== 原创性声明与著作权使用声明 =====
%%%%% ===== 原创性声明与著作权使用声明
\newpage
\thispagestyle{empty}
\pdfbookmark[0]{原创性声明}{Declaration}

\vspace*{1em}

{
\linespread{1.4}\zihao{-4}
\centerline{\zihao{3}\STSong 华东师范大学学位论文原创性声明}

\bigskip

郑重声明:本人呈交的学位论文《\cctitle》,
是在华东师范大学攻读硕士/博士(请勾选)
学位期间,在导师的指导下进行的研究工作及取得的研究成果。
除文中已经注明引用的内容外,本论文不包含其他个人已经发表或撰写过的研究成果。
对本文的研究做出重要贡献的个人和集体,均已在文中作了明确说明并表示谢意。

\vspace{1em}


{\STSong 作者签名}:$\underline{\hspace{4cm}}$ \hfill
{\STSong 日\quad 期}: \qquad\ 年 \quad\ 月 \quad\ 日 \qquad\mbox{}

\vspace{4em}

\centerline{\zihao{3}\STSong 华东师范大学学位论文著作权使用声明}
\bigskip


《\cctitle》
系本人在华东师范大学攻读学位期间在导师指导下完成的硕士/博士(请勾选)学位论文,
本论文的著作权归本人所有。
本人同意华东师范大学根据相关规定保留和使用此学位论文,
并向主管部门和学校指定的相关机构送交学位论文的印刷版和电子版;
允许学位论文进入华东师范大学图书馆及数据库被查阅、借阅;
同意学校将学位论文加入全国博士、硕士学位论文共建单位数据库进行检索,
将学位论文的标题和摘要汇编出版,采用影印、缩印或者其它方式合理复制学位论文。

本学位论文属于(请勾选)
\newcounter{muni}
\begin{list}{{\hfill\upshape (\qquad)\ \arabic{muni}. }}{%
     \usecounter{muni}\leftmargin6.5em\labelwidth4.2em\labelsep0.2em
     \itemsep0.5em\itemindent0pt\parsep0pt\topsep0pt}

\item 经华东师范大学相关部门审查核定的“内部”或“涉密”学位论文*,
  于 \qquad 年 \quad 月 \quad  日解密, 解密后适用上述授权。

\item 不保密,适用上述授权。
\end{list}

\vskip0.8cm


{\STSong 导师签名}:$\underline{\hspace{4cm}}$ \hfill
{\STSong 本人签名}:$\underline{\hspace{4cm}}$

\bigskip

{\mbox{}\hfill 年\qquad 月\qquad  日 }

\vfill

\parbox[t]{0.946\textwidth}{\zihao{5}
*“涉密”学位论文应是已经华东师范大学学位评定委员会办公室或保密委员会
审定过的学位论文(需附获批的《华东师范大学研究生申请学位论文“涉密”审批表》
方为有效),未经上述部门审定的学位论文均为公开学位论文。
此声明栏不填写的,默认为公开学位论文,均适用上述授权)。\\
}}

\clearpage{\pagestyle{empty}\cleardoublepage}

%%%%% ===== 答辩委员会成员 =====
%%%% ===== 答辩委员会成员
\thispagestyle{empty}
\pdfbookmark[0]{答辩委员会}{Committee}
\vspace*{2em}

\makeatletter
\begin{center}\STSong\zihao{3}
 \underline{\ \@cauthor\ } 硕士学位论文答辩委员会成员名单
\end{center}
\makeatother

\begin{center}\zihao{4}
\renewcommand{\arraystretch}{1.4}
  \begin{tabular}{|c|c|c|c|} \hline
   ~~~~~姓~名~~~~~ & ~~~~~职~称~~~~~
   & \hspace{6em}单~位\hspace{6em} & ~~~备~注~~~\\\hline
         XXX    &   教授    &  XXXXX大学数学系  & 主席  \\ \hline
         XXX    &   教授    &  XXXXX大学数学系  &       \\ \hline
         XXX    &   教授    &  XXXXX大学数学系  &       \\ \hline
                &           &                   &       \\ \hline
                &           &                   &       \\ \hline
  \end{tabular}
\end{center}

\clearpage{\pagestyle{empty}\cleardoublepage}

\frontmatter
\restoregeometry
%%%%% ===== 中文摘要 =====
\include{Abstract_chs}
\clearpage{\pagestyle{empty}\cleardoublepage}

%%%%% ===== 英文摘要 =====
\include{Abstract_eng}
\clearpage{\pagestyle{empty}\cleardoublepage}

%%%%% ===== 生成目录
\setcounter{tocdepth}{1}
\phantomsection\pdfbookmark[0]{目录}{ccontents}
\tableofcontents
%\listoffigures % 插图目录
%\listoftables  % 表格目录
\clearpage{\pagestyle{empty}\cleardoublepage}

%%%%%% ===== 正文部分 ===== %%%%%
\mainmatter
\linespread{1.4}\selectfont
%\setlength{\baselineskip}{0.88175cm}

\chapter{引言}

引言部分, 介绍论文研究课题的应用背景或者问题来源,
一些基本概念, 现有成果等等.

\section{问题的提出}
问题的提出, 问题的提出, 问题的提出, 问题的提出, 问题的提出.
问题的提出, 问题的提出, 问题的提出, 问题的提出, 问题的提出.
问题的提出, 问题的提出, 问题的提出, 问题的提出, 问题的提出.

问题的提出, 问题的提出, 问题的提出, 问题的提出, 问题的提出.
问题的提出, 问题的提出, 问题的提出, 问题的提出, 问题的提出.
问题的提出, 问题的提出, 问题的提出, 问题的提出, 问题的提出.

\section{现有成果}
现有成果介绍, 现有成果介绍, 现有成果介绍, 现有成果介绍, 现有成果介绍.
现有成果介绍, 现有成果介绍, 现有成果介绍, 现有成果介绍, 现有成果介绍.
现有成果介绍, 现有成果介绍, 现有成果介绍, 现有成果介绍, 现有成果介绍.

\begin{theorem}
  这是定理, 这是定理, 这是定理, 这是定理, 这是定理, 这是定理.
  这是定理, 这是定理, 这是定理, 这是定理, 这是定理, 这是定理.
  这是定理, 这是定理, 这是定理, 这是定理, 这是定理, 这是定理.
  这是定理, 这是定理, 这是定理, 这是定理, 这是定理, 这是定理.
\end{theorem}
\begin{proof}
  这是证明, 这是证明, 这是证明, 这是证明, 这是证明, 这是证明.
  这是证明, 这是证明, 这是证明, 这是证明, 这是证明, 这是证明.
  这是证明, 这是证明, 这是证明, 这是证明, 这是证明, 这是证明.
  这是证明, 这是证明, 这是证明, 这是证明, 这是证明, 这是证明.

  这是证明, 这是证明, 这是证明, 这是证明, 这是证明, 这是证明.
  这是证明, 这是证明, 这是证明, 这是证明, 这是证明, 这是证明.
  这是证明, 这是证明, 这是证明, 这是证明, 这是证明, 这是证明.
  这是证明, 这是证明, 这是证明, 这是证明, 这是证明, 这是证明.
\end{proof}

现有成果介绍, 现有成果介绍, 现有成果介绍, 现有成果介绍, 现有成果介绍.
现有成果介绍, 现有成果介绍, 现有成果介绍, 现有成果介绍, 现有成果介绍.
现有成果介绍, 现有成果介绍, 现有成果介绍, 现有成果介绍, 现有成果介绍.
现有成果介绍, 现有成果介绍, 现有成果介绍, 现有成果介绍, 现有成果介绍.
现有成果介绍, 现有成果介绍, 现有成果介绍, 现有成果介绍, 现有成果介绍.
现有成果介绍, 现有成果介绍, 现有成果介绍, 现有成果介绍, 现有成果介绍.

\chapter{准备工作}

论文中需要用到的一些知识或工具等. 论文中需要用到的一些知识或工具等.
论文中需要用到的一些知识或工具等. 论文中需要用到的一些知识或工具等.
论文中需要用到的一些知识或工具等. 论文中需要用到的一些知识或工具等.
论文中需要用到的一些知识或工具等. 论文中需要用到的一些知识或工具等.
论文中需要用到的一些知识或工具等. 论文中需要用到的一些知识或工具等.

\section{一些定义}
这里给出一些定义, 这里给出一些定义, 这里给出一些定义,
这里给出一些定义, 这里给出一些定义, 这里给出一些定义,
这里给出一些定义, 这里给出一些定义, 这里给出一些定义,
这里给出一些定义, 这里给出一些定义, 这里给出一些定义,
这里给出一些定义, 这里给出一些定义, 这里给出一些定义,
这里给出一些定义, 这里给出一些定义, 这里给出一些定义,
这里给出一些定义, 这里给出一些定义, 这里给出一些定义,
这里给出一些定义, 这里给出一些定义, 这里给出一些定义,
这里给出一些定义, 这里给出一些定义, 这里给出一些定义.

\begin{definition}
  这是定义, 这是定义, 这是定义, 这是定义, 这是定义, 这是定义,
  这是定义, 这是定义, 这是定义, 这是定义, 这是定义, 这是定义,
  这是定义, 这是定义, 这是定义, 这是定义, 这是定义, 这是定义,
  这是定义, 这是定义, 这是定义, 这是定义, 这是定义, 这是定义.
\end{definition}

\section{一些引理和推论}

这里给出一些引理和推论, 这里给出一些引理和推论, 这里给出一些引理和推论,
这里给出一些引理和推论, 这里给出一些引理和推论, 这里给出一些引理和推论,
这里给出一些引理和推论, 这里给出一些引理和推论, 这里给出一些引理和推论,
这里给出一些引理和推论, 这里给出一些引理和推论, 这里给出一些引理和推论,
这里给出一些引理和推论, 这里给出一些引理和推论, 这里给出一些引理和推论,
这里给出一些引理和推论, 这里给出一些引理和推论, 这里给出一些引理和推论.

\begin{lemma}
  这是引理, 这是引理, 这是引理, 这是引理, 这是引理, 这是引理,
  这是引理, 这是引理, 这是引理, 这是引理, 这是引理, 这是引理,
  这是引理, 这是引理, 这是引理, 这是引理, 这是引理, 这是引理,
  这是引理, 这是引理, 这是引理, 这是引理, 这是引理, 这是引理.
\end{lemma}

\begin{lemma}
  这是引理, 这是引理, 这是引理, 这是引理, 这是引理, 这是引理,
  这是引理, 这是引理, 这是引理, 这是引理, 这是引理, 这是引理,
  这是引理, 这是引理, 这是引理, 这是引理, 这是引理, 这是引理,
  这是引理, 这是引理, 这是引理, 这是引理, 这是引理, 这是引理.
\end{lemma}

\subsection{推论}
这里给出一些推论, 这里给出一些推论, 这里给出一些推论,
这里给出一些推论, 这里给出一些推论, 这里给出一些推论,
这里给出一些推论, 这里给出一些推论, 这里给出一些推论,
这里给出一些推论, 这里给出一些推论, 这里给出一些推论,
这里给出一些推论, 这里给出一些推论, 这里给出一些推论,
这里给出一些推论, 这里给出一些推论, 这里给出一些推论,
这里给出一些推论, 这里给出一些推论, 这里给出一些推论,
这里给出一些推论, 这里给出一些推论, 这里给出一些推论。

\begin{corollary}
  这是推论, 这是推论, 这是推论, 这是推论, 这是推论, 这是推论,
  这是推论, 这是推论, 这是推论, 这是推论, 这是推论, 这是推论,
  这是推论, 这是推论, 这是推论, 这是推论, 这是推论, 这是推论,
  这是推论, 这是推论, 这是推论, 这是推论, 这是推论, 这是推论.
\end{corollary}

\section{一些命题和性质}
现有成果介绍, 现有成果介绍, 现有成果介绍, 现有成果介绍, 现有成果介绍.
现有成果介绍, 现有成果介绍, 现有成果介绍, 现有成果介绍, 现有成果介绍.
现有成果介绍, 现有成果介绍, 现有成果介绍, 现有成果介绍, 现有成果介绍.

\begin{proposition}
  这是命题, 这是命题, 这是命题, 这是命题, 这是命题, 这是命题,
  这是命题, 这是命题, 这是命题, 这是命题, 这是命题, 这是命题,
  这是命题, 这是命题, 这是命题, 这是命题, 这是命题, 这是命题.
\end{proposition}

\begin{property}
  这是性质, 这是性质, 这是性质, 这是性质, 这是性质, 这是性质,
  这是性质, 这是性质, 这是性质, 这是性质, 这是性质, 这是性质,
  这是性质, 这是性质, 这是性质, 这是性质, 这是性质, 这是性质.
\end{property}

现有成果介绍, 现有成果介绍, 现有成果介绍, 现有成果介绍, 现有成果介绍.
现有成果介绍, 现有成果介绍, 现有成果介绍, 现有成果介绍, 现有成果介绍.
现有成果介绍, 现有成果介绍, 现有成果介绍, 现有成果介绍, 现有成果介绍.


\clearpage{\pagestyle{empty}\cleardoublepage}
\chapter{其他章节}
其他章节, 其他章节, 其他章节, 其他章节, 其他章节, 其他章节,
其他章节, 其他章节, 其他章节, 其他章节, 其他章节, 其他章节,
其他章节, 其他章节, 其他章节, 其他章节, 其他章节, 其他章节.

其他章节, 其他章节, 其他章节, 其他章节, 其他章节, 其他章节,
其他章节, 其他章节, 其他章节, 其他章节, 其他章节, 其他章节,
其他章节, 其他章节, 其他章节, 其他章节, 其他章节, 其他章节.


\section{一些注记}
现有成果介绍, 现有成果介绍, 现有成果介绍, 现有成果介绍, 现有成果介绍.
现有成果介绍, 现有成果介绍, 现有成果介绍, 现有成果介绍, 现有成果介绍.
现有成果介绍, 现有成果介绍, 现有成果介绍, 现有成果介绍, 现有成果介绍.
现有成果介绍, 现有成果介绍, 现有成果介绍, 现有成果介绍, 现有成果介绍.
现有成果介绍, 现有成果介绍, 现有成果介绍, 现有成果介绍, 现有成果介绍.

\begin{remark}
  这是注记, 这是注记, 这是注记, 这是注记, 这是注记, 这是注记,
  这是注记, 这是注记, 这是注记, 这是注记, 这是注记, 这是注记,
  这是注记, 这是注记, 这是注记, 这是注记, 这是注记, 这是注记.
\end{remark}

\section{算法}
算法示例, 算法示例, 算法示例, 算法示例, 算法示例, 算法示例,
算法示例, 算法示例, 算法示例, 算法示例, 算法示例, 算法示例,
算法示例, 算法示例, 算法示例, 算法示例, 算法示例, 算法示例.

算法示例, 算法示例, 算法示例, 算法示例, 算法示例, 算法示例,
算法示例, 算法示例, 算法示例, 算法示例, 算法示例, 算法示例,
算法示例, 算法示例, 算法示例, 算法示例, 算法示例, 算法示例.

算法示例, 算法示例, 算法示例, 算法示例, 算法示例, 算法示例,
算法示例, 算法示例, 算法示例, 算法示例, 算法示例, 算法示例,
算法示例, 算法示例, 算法示例, 算法示例, 算法示例, 算法示例.

\begin{algorithm}[H]
\caption{算法示例算法示例\label{Alg:}}
  \begin{algorithmic}[1]
  \State $v_1 = r/\|r\|_2$
  \For{$j=1$ to $m$}
  \State $z = A v_j$
  \For{$i=1$ to $j$}
  \State $h_{i,j} = (v_i,z)$  \Comment{内积}
  \State $z = z - h_{i,j}v_i$
  \EndFor
  \State $h_{j+1,j}=\|z\|_2$
  \If{$h_{j+1,j}=0$}
  \State break
  \EndIf
  \State $v_{j+1}=z/h_{j+1,j}$
  \EndFor
  \end{algorithmic}
\end{algorithm}

\clearpage{\pagestyle{empty}\cleardoublepage}

\backmatter
\linespread{1.1}\selectfont

%%%% ===== 参考文献 =====
\begin{thebibliography}{99}
\addcontentsline{toc}{chapter}{参考文献}
\thispagestyle{plain}

\bibitem{CPZ08}
\newblock K. C. Chang, K. Pearson and T. Zhang,
\newblock Perron-Frobenius Theorem for nonnegative tensors,
\newblock \emph{Commun. Math. Sci.}, 6 (2008), 507--520.

\bibitem{NW99}
\newblock J. Nocedal and S. J. Wright,
\newblock \emph{Numerical Optimization},
\newblock Springer, New York, 1999.

\end{thebibliography}

\clearpage{\pagestyle{empty}\cleardoublepage}
\linespread{1.4}\selectfont
\chapter*{附录}
\addcontentsline{toc}{chapter}{附录}

附录部分, 附录部分, 附录部分, 附录部分, 附录部分,
附录部分, 附录部分, 附录部分, 附录部分, 附录部分,
附录部分, 附录部分, 附录部分, 附录部分, 附录部分,
附录部分, 附录部分, 附录部分, 附录部分, 附录部分.

附录部分, 附录部分, 附录部分, 附录部分, 附录部分,
附录部分, 附录部分, 附录部分, 附录部分, 附录部分,
附录部分, 附录部分, 附录部分, 附录部分, 附录部分,
附录部分, 附录部分, 附录部分, 附录部分, 附录部分.

附录部分, 附录部分, 附录部分, 附录部分, 附录部分,
附录部分, 附录部分, 附录部分, 附录部分, 附录部分,
附录部分, 附录部分, 附录部分, 附录部分, 附录部分,
附录部分, 附录部分, 附录部分, 附录部分, 附录部分.


\clearpage{\pagestyle{empty}\cleardoublepage}
\chapter*{致谢}
\addcontentsline{toc}{chapter}{致谢}

致谢部分, 致谢部分, 致谢部分, 致谢部分, 致谢部分,
致谢部分, 致谢部分, 致谢部分, 致谢部分, 致谢部分,
致谢部分, 致谢部分, 致谢部分, 致谢部分, 致谢部分,
致谢部分, 致谢部分, 致谢部分, 致谢部分, 致谢部分.

致谢部分, 致谢部分, 致谢部分, 致谢部分, 致谢部分,
致谢部分, 致谢部分, 致谢部分, 致谢部分, 致谢部分,
致谢部分, 致谢部分, 致谢部分, 致谢部分, 致谢部分,
致谢部分, 致谢部分, 致谢部分, 致谢部分, 致谢部分.

致谢部分, 致谢部分, 致谢部分, 致谢部分, 致谢部分,
致谢部分, 致谢部分, 致谢部分, 致谢部分, 致谢部分,
致谢部分, 致谢部分, 致谢部分, 致谢部分, 致谢部分,
致谢部分, 致谢部分, 致谢部分, 致谢部分, 致谢部分.

致谢部分, 致谢部分, 致谢部分, 致谢部分, 致谢部分,
致谢部分, 致谢部分, 致谢部分, 致谢部分, 致谢部分,
致谢部分, 致谢部分, 致谢部分, 致谢部分, 致谢部分,
致谢部分, 致谢部分, 致谢部分, 致谢部分, 致谢部分.

\clearpage{\pagestyle{empty}\cleardoublepage}
\chapter*{研究成果}
\addcontentsline{toc}{chapter}{研究成果}

研究生期间所取得的研究成果.

\end{document}
