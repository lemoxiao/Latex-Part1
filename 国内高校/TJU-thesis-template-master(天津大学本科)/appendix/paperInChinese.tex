% !Mode:: "TeX:UTF-8"

\titlecontents{chapter}[2em]{\vspace{.5\baselineskip}\xiaosan\song}
             {\prechaptername\CJKnumber{\thecontentslabel}\postchaptername\qquad}{}
             {}             % 设置该选项为空是为了不让目录中显示页码
\addcontentsline{toc}{chapter}{中文译文}
\setcounter{page}{1}            % 单独从 1 开始编页码
\markboth{中文译文}{中文译文}   % 用于将章节号添加到页眉中
\chapter*{中文译文}

\trtitle{葛底斯堡演说}

87年前,我们的先辈们在这个大陆上接生了一个新型的共和国,她受孕于自由的理念,并献身于一切人生来平等的理想。

现在我们进行了一场重大的内战,以考验这个共和国,或者任何一个受孕于自由和献身于上述理想的共和国是否能够长久生存下去。我们站在这场战争中的一个重要战场上聚集。烈士们为使这个共和国能够生存下去而献出了自己的生命,我们来到这里,是要把这个战场的一部分奉献给他们作为最后安息之所。我们这样做是完全应该而且是非常恰当的。但是,从更广泛的意义上来说,这块土地我们不能够奉献,不能够圣化,不能够神化。正是那些活着的或者已经死去的曾经在这里战斗过的英雄们才使得这块土地成为神圣之土。我们无力使之增减一分。我们在这里说什么,世人不会注意,也不会长期记住,但是英雄们的行为永远不会被人们遗忘。

这更要求我们这些活着的人去继续英雄们为之战斗,并使之前进的未尽事业。倒是我们应该在这里把自己奉献于仍然留在我们面前的伟大任务——我们要从这些光荣的死者身上汲取更多的献身精神,来完成他们已经完全彻底为之献身的事业;我们要在这里下定最大的决心,不让这些死者白白牺牲;我们要使共和国在上帝保佑下得到自由的新生,要使这个民有、民治、民享的政府永世长存。