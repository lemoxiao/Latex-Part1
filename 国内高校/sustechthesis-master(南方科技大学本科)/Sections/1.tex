% !Mode:: "TeX:UTF-8"
\section{数学符号测试}
\subsection{行间公式}
根据WinEdt的\TeX{} Symbols:$\sum_{i=1}^{n}
\prod_{j=1}^{m}\oint\sqrt[k]{\aleph^\hbar_\imath\times\frac{\nabla\jmath}{\ell}}$
\subsection{行内公式}
同理根据WinEdt的\TeX{} Symbols:\par
首先使用\verb|$$|:$$\sum_{i=1}^{n}
\prod_{j=1}^{m}\oint\sqrt[k]{\aleph^\hbar_\imath\times\frac{\nabla\jmath}{\ell}}$$

其次使用\verb|\[\]|:\[\sum_{i=1}^{n}
\prod_{j=1}^{m}\oint\sqrt[k]{\aleph^\hbar_\imath\times\frac{\nabla\jmath}{\ell}}\]
\subsubsection{EQUATION环境}
\begin{equation}\label{eq:test_input:1}
\sum_{i=1}^{n}
\prod_{j=1}^{m}\oint\sqrt[k]{\aleph^\hbar_\imath\times\frac{\nabla\jmath}{\ell}}
\end{equation}
根据式\eqref{eq:test_input:1}可以知道,……。
\subsubsection{ALIGN环境}
\begin{align}
\min_{\substack{1\leq i\leq n \\ 1\leq j\leq m}}\quad \Gamma\Delta\Theta\Lambda\Xi\Pi\Sigma\Upsilon\Phi\Psi\Omega \label{eq:test_input:2} \\
s.t.\begin{cases}
      1, & \mbox{if i==j,}\\
      0, & \mbox{otherwise}.
    \end{cases}\label{eq:test_input:3}
\end{align}
由式\eqref{eq:test_input:2}和式\eqref{eq:test_input:3}可以知道,……。