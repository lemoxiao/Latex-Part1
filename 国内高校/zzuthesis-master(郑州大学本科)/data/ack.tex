%%================================================
%% Filename: ack.tex
%% Encoding: UTF-8
%% Author: Yuan Xiaoshuai - yxshuai@gmail.com
%% Created: 2012-01-12 18:09
%% Last modified: 2016-08-28 21:05
%%================================================
\begin{ack}

韶华易逝,光阴荏苒,余自入渝倏忽而又三年矣。曩者,离鲁地,弃吴越奔巴蜀,恢恢乎
若漏网之鲫,惶惶乎似惊弓之鸟,几无容身之地。幸遇刘徐二相荐,得拜于恩师魏公门下
,而今已历六秋。先生治学严谨,博通中外,蔚为家,研习学术,废寝忘食,已臻忘我之
境界。犹忆当年,乍入门墙,耻无寸功先生不以余愚钝,委以重托,是以矢勤矢勇,思惟
速战,毋负所托。然欲速不心忧如焚。全赖先生数次点拨,每自躬亲,甚于子夜,共为进
退,方有所成。于缀词成文,苦心孤诣,字斟句酌,独具匠心,深为叹服,并于耳提面命
之际益良多。先生之德才,若高山仰止,亦非庸庸如吾辈者所能望其项背。每至周常言:
“修身者智之府也,爱施者仁之端也,取予者义之符也,耻辱者勇之决也立名者行之极也
。凡此五者,皆安身立命之本,不可偏废。”言之谆谆,听之诚若醍醐灌顶。退而思之,
果斯然也,大为拜服。兼之师母,待余恩厚,视若螟多得给养,感激涕零。余素朴陋,尚
无剖符丹书之功,反受此殊遇,非结草衔不可报也!今虽即辞师门,必常怀乌鸟之情,反
哺之心,诚不失其望也!

师门同侪,学强张骞,皆忠义之士,情同手足;乐至四国,许都耀琼,此良实,多蒙所助
;兰莉二姊,性行淑均,爱如姐弟。其他诸君,皆为志虑忠纯士,多有广益。区区不才,
有何德能,安得广助若此?感荷之心,亦自拳拳。实验中曾多次求教于杨(明莉)、杜(
军)二老师,受益匪浅;中心实验室(光辉)、鲜(晓红)二位老师及刘姊渝萍为实验工
作大开方便之门,谨表谢忱家中上至期颐之祖母,中至慈爱之父母、姊姊,下至始龀之甥
男,均不遗力以支持,融融亲情,曷其幸甚!谨撰此文,鱼传尺素,答报椿萱!

嗟乎!聚散无常,盛况难再;书不尽意,略陈固陋。临别赠言,幸承恩于公;登高作赋,
是所望于群贤。斗胆献拙,情之不已;一言均赋,四韵俱成。洒陵江,以酹逝去之岁月。

\begin{center}
西入渝州已六霜,貔貅帐里铸鱼肠。\\
书山文海研学术,翠袖红巾戏皮黄。\\
莫叹时舛惟虎踞,且图运转季鹰扬。\\
今朝叩别师尊去,一片丹心化碧江。
\footnote{重庆大学化工博士季孟波学位论文《抗溺水性气体多孔电极的研究》致谢词}
\end{center}

\end{ack}
