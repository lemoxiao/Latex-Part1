\section{}
\subsection{批量下载图片bash脚本}\label{download_shell}
\begin{lstlisting}[caption=批量下载图片bash脚本,language=bash]
#!/bin/bash
MAX=\$2;
for ((MAX;MAX>0; MAX--));
        do wget -vv  --no-check-certificate "$1" -O "$MAX";
done;
\end{lstlisting}

\subsection{简单平均法灰度化图片}\label{gray_pic}
\begin{lstlisting}[caption=简单平均法灰度化图片,language=matlab]
pic=imread('../abc/1');
[h,w,d]=size(pic);
for m=1:h
    for n=1:w
        sum=0;
        for v=pic(m,n,:)
            sum=sum+v/3;
        end
        pic2(m,n)=sum;
    end
end
figure,imshow(pic),figure,imshow(pic2);
\end{lstlisting}

\subsection{增强对比度}\label{contrast}
\begin{lstlisting}[caption=增强对比度,language=matlab]
pic=imread('../icbc/1');
pic=rgb2gray(pic);%先灰度化处理
pic2=imadjust(pic,[0.3 0.4],[0 1]);%增强对比度
imcontrast(gcf);%使用imcontrast工具箱进行动态的对比度调整
figure,imshow(pic2),figure,imshow(pic);
\end{lstlisting}


\subsection{采用均值滤波的积分投影算法}\label{means_filter}
\begin{lstlisting}[caption=采用均值滤波的积分投影算法,language=matlab]
pic=imread('../test.bmp');
words_count=5;%已知识别的字符个数为5
pic=rgb2gray(pic);
pic=im2bw(pic);
[h,w,d]=size(pic);
row=[];%图像在x轴上的积分投影

%计算积分投影
for x=1:w
    row(x)=sum(pic(:,x,1));
end

%为了演示方便,我们另外给出滤波后的积分投影
row1=row;
%对原来的积分投影进行滤波,注意滤波后的积分投影会比原来的少两个数据,分
%别位于首尾 
for x=2:w-1
    new=sum(row1(x-1:x+1))/3;
    row1(x)=new;
end

%使用滤波后的积分投影进行分析,从峰值到谷值扫描
for std=max(row):-1:min(row1)
    blocks=[];%保存每一个谷坡
    inblock=0;%是否处于谷坡
    for x=1:length(row1)
        %如果从谷顶刚进入谷坡,则新建一个谷坡
        if (row1(x)<std && inblock==0)
            blocks=[blocks;x 0];
            inblock=1;
        %如果从谷坡爬回谷顶,则标记刚才的新建谷坡    
        elseif(row1(x)>=std && inblock==1)
            blocks(end)=x;
            inblock=0;
        end
    end
    %如果字符贴边上,则手动结束
    if inblock==1
        blocks(end)=length(row1);
    end
    [block_count,n]=size(blocks);
    %当谷坡数量与要识别的字符个数相等就退出扫描
    if block_count==words_count
        break;
    end
end

%绘制积分投影滤波前后的对比图
figure,subplot(2,1,1),hold on;
for x=1:block_count
    rectangle('FaceColor','c','EdgeColor','none'...
    ,'Position',[blocks(x,1) min(row) blocks(x,2)-blocks(x,1)
max(row)-min(row)-1]);
end

plot([0,length(row1)],[std,std],'g--');
plot(row,'b'),plot(row1,'r');
legend('scanning line','before filter','after filter');
subplot(2,1,2),imshow(pic);
for x=1:block_count
    line([blocks(x,1),blocks(x,1)],[h,0]);
    line([blocks(x,2),blocks(x,2)],[h,0]);
end
hold off;
\end{lstlisting}


\subsection{带权重的连通区域聚类法}\label{segmentation_with_weight}
\begin{lstlisting}[caption=带权重的连通区域聚类法,language=matlab]
pic=imread('../test.bmp');
%已知识别的字符个数为5
k=5;

%灰度并二值化
pic=rgb2gray(pic);
pic=im2bw(pic);

figure,subplot(2,1,1);
imshow(pic);%绘制原图

[height,width]=size(pic);
%采用4连通寻找图像连通域
[L,num]=bwlabel(~pic,8);

%计算连通域的质心
r=regionprops(L);

centroids=[];%质心
for i=1:length(r)
    centroids(i,:)=r(i).Centroid;
end

%根据连通域面积,加大质心的个数,以增加权重
for i=1:length(r)
    for x=2:r(i).Area
        centroids=[centroids;r(i).Centroid];
    end
end

%使用K-means把质心聚成k类
km=kmeans(centroids,k,'distance','sqEuclidean','start','uniform','replicates',50);

%在图像上标记k类区域
for i=1:length(r)
    L(find(L==i))=km(i)+num;
end

%彩色绘制k类区域
RGB =label2rgb(L);
subplot(2,1,2),imshow(RGB),hold on;

%绘制各个区域的质点
plot(centroids(1:length(r),1),centroids(1:length(r),2),'b*');

%绘制聚类区域的质点
for i=1:k
    M=zeros(height,width);
    M(find(L==i+num))=1;
    s=regionprops(M);
    plot(s.Centroid(1),s.Centroid(2),'c*');
end
legend('centroid of each connected object','centroid of each cluster');
hold off;
\end{lstlisting}

\subsection{Jitendra采样法}\label{jitendra}
\begin{lstlisting}[caption=演示程序,language=matlab]
clc;
close;
close all;
img=imread('../tmp/1.bmp');
img=img(:,:,1);
img=im2double(img);

%把图片的周围空出一个像数宽,方便提取边缘
[h,w]=size(img);
img=[ones(h,1) img ones(h,1)];
img=[ones(1,w+2);img;ones(1,w+2)];

n=100;
figure(1);
hold on;
subplot(1,3,1);
imshow(img);
title('original image');
img=rot90(img');
%提取边缘
[x,y,t,c]=bdry_extract(img);
subplot(1,3,2);
plot(x,y,'b.');
title('edge image');
axis([0,140,0,140]);
axis square; %长宽比例为 1
%使用Jitendra采样法进行采样
[xi,yi,ti]=get_samples(x,y,t,n);
subplot(1,3,3);
plot(xi,yi,'b+');
title(sprintf('after sample(n=\%d)',n));
axis([0,140,0,140]);
axis square; %长宽比例为 1
grid on;
hold off;
\end{lstlisting}


\begin{lstlisting}[caption=Jitendra采样法,language=matlab]
function [xi,yi,ti]=get_samples(x,y,t,nsamp)
N=length(x);
k=3;
Nstart=min(k*nsamp,N);

ind0=randperm(N);%生成一个1到N的乱序列
ind0=ind0(1:Nstart);
%将x,y打乱,但对应点保持不变
xi=x(ind0);
yi=y(ind0);
ti=t(ind0);
xi=xi(:);
yi=yi(:);
ti=ti(:);
%计算xi yi各点的距离
d2=dist2([xi yi],[xi yi]);
d2=d2+diag(Inf*ones(Nstart,1));

s=1;
while s
    % find closest pair
    [a,b]=min(d2);%寻找每一列的最小值a,并把其所在行数保存在b,a为单行
    [c,d]=min(a);%寻找a中最小的值c,并把其所在列数保存在d,矩阵d2最小值就为d
    I=b(d);
    J=d;
    % remove one of the points删除d2中的最小点,并删除xi,yi,ti相应的点
    xi(J)=[];
    yi(J)=[];
    ti(J)=[];
    d2(:,J)=[];
    d2(J,:)=[];
    if size(d2,1)==nsamp
        s=0;
    end
end
      
\end{lstlisting}

\begin{lstlisting}[caption=利用梯度来提取边缘,language=matlab]
function [x,y,t,c]=bdry_extract(V);
Vg=V; % if no smoothing is needed
%画等高线,就是轮廓
c=contourc(Vg,[.5 .5]);
%计算数值梯度
[G1,G2]=gradient(Vg);

%把c(1,:)里面不是0.5标记为1,是的标记为0,储存到fz
fz=c(1,:)~=0.5;
%把c(1,:)里面为0.5的,将其上下对应的两个数设置为NaN
c(:,find(~fz))=NaN;
%B为c,去掉c(1,:)里面为0.5后
B=c(:,find(fz));

npts=size(B,2);
t=zeros(npts,1);
for n=1:npts
    x0=round(B(1,n));
    y0=round(B(2,n));
    t(n)=atan2(G2(y0,x0),G1(y0,x0))+pi/2;
end

x=B(1,:)';
y=B(2,:)';

\end{lstlisting}
