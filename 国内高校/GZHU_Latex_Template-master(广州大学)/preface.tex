%请在下面输入前言内容。
目前世界互联网发展速度块,中国也紧跟世界发展脚步,网络安全问题也随之成为需要重点研究的课题,信息安全问题也显得越来越重要。信息安全漏洞小至影响人们日常生活,大至影响企业生产与运营,乃至危害国家政府等重要领域。信息安全的领域包括协议安全、计算机操作系统安全、安全机制(数据加密、数字签名、信息认证等),直至安全系统,在任何一个因素出现缺口都可以威胁全局安全。

当今电子商务发展迅速,股票、网上购物、网上银行等都离不开数字签名,数字签名有能与所签文件绑定,签名者不可否认自己的签名,容易被验证,不易被伪造等特点,故数字签名在信息安全中扮演着非常重要的一个角色。自从1976年数字签名被Diffie和Hellman提出以来,数字签名技术引起了计算机网络界还有密码应用界的关注。

1985年,Elgamal第一次在有限域上基于离散对数问题设计了Elgamal数字签名方案,这是数字签名历史上一个重要的里程碑。为了减少计算量,提高安全性,使系统能够做到更多的适应于不同的网络应用环境中,在此之后密码学研究者们在Elgamal型签名方案的基础上相继提出了许多种有效的改进签名方案。从原始Elgamal签名方案、AWV方案、Harn方案、具有信息恢复功能的数字签名方案和基于复合问题的Elgamal型方案等出发,同时分析了它们各自的优缺点;1994年ham 对Elgamal签名方案及类似的签名方案进行了有效的分析总结,提出了广义Elgamal签名方案,又一次对密码学界产生了巨大影响。此后密码工作学着们又提出不少Elgamal签名方案,如美国国家标准与技术研究所的建议标准DDS。

本人试图实现Elgamal签名方案,并在此基础上实现基于Elgamal算法的代理签名方案,通过C语言编程,实现密钥对的产生、对消息值的签名与验证过程,并对它们进行安全分析。
