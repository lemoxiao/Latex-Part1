\section{引言}

\subsection{选题背景与意义}
\label{sec:background}
  安全传输层协议 (Transport Layer Security,下面简称 TLS ) 每天被全球数百万用户使用,作为互联网安全的核心构建块,不仅应用到浏览器的 HTTPS 协议中,还有其他应用层协议也使用了 TLS,比如 SSH、WSS 协议等。由于 TLS 1.2 及以下版本中的各种安全缺陷和设计缺陷\cite{7877537},\cite{8632026},会受到降级攻击、中间人攻击、BEAST 攻击、POODLE攻击等,最为严重的openssl心脏出血漏洞\cite{8001980},属于实现上的漏洞而不是协议上的漏洞,还有 TLS 1.2 的完整握手需要两轮往返时间,耗时长,即使使用会话恢复也需要一轮往返时间。
  
  无论是基于实现还是基于规范,都促使 TLS 工作组在起草下一版协议时采用“部署前分析”设计范例\cite{7546518}。TLS 协议主要目标是在两个通信应用程序之间提供数据保密性和数据完整性。IETF( 国际互联网工程任务组)用了超过四年时间,起草了28份草案后,于2018年8月发布 TLS 1.3\cite{RFC8446} 最终版本(RCF 8446)。相比较于 TLS 1.2,TLS 1.3 删除未使用或者不安全的功能,并且加密了更多的握手信息,减少了握手延迟。TLS协议的主要目标是在通信双方提供安全信道,包括对服务器和客户端的身份验证,信道上传输数据的机密性,保证通信双方传输内容的完整性。非对称加密算法中:在 TLS 1.3 中密钥交换默认使用椭圆曲线加密法,删除静态 RSA 密钥交换。对称加密算法上:删除 CBC 模式密码、RC4 流密码,保证信息机密性和完整性使用 AEAD 算法\cite{RFC5166},是 TLS 1.3 中唯一保留的对称加密方式,只有 AES-GCM 和 ChaCha20-Poly1305 两种,更安全和难以破解。删除 SHA-1 哈希函数,哈希算法使用具有更长密钥的 SHA-2 哈希函数,其中包括 SHA-256 和 SHA-384。
  
  此外,TLS 1.3 使用新的 PSK 密钥协商。TLS 1.3 不仅在安全性上有提升,速度上也提升不少。TLS 1.3 相比较于 TLS 1.2 的两次往返才能建立连接的完整握手,针对 TLS 1.2 中两次往返耗时长,速度较慢比较慢,为了有更快是网络访问速度,TlS 1.3 完整握手使用了也减少到一次往返和会话恢复的零轮往返时长协议握手,大大提升连接速度,比 TLS 1.2 减少了一个握手往返,仅在传输层上速度将提升一倍。
  
  虽然 TLS 1.3 和零轮往返时长协议不能减少传输的往返延迟,但可以减少建立 HTTPS 连接所需的往返次数,从而减少握手花费的时间。有更快的握手速度固然是好,但同时不能忽略一些安全问题。利用 WebSocket 编程简单实现 TLS 协议,分析和重现 TLS 1.3 协议中 0—RTT 的重放攻击,并在后台服务器中实现抵抗重放攻击方法,测试抵抗重放攻击的实用性。

\subsection{国内外研究现状和相关工作}
\label{sec:related_work}
  有关 TLS 1.3 的讨论,多数停留在非正式发布前\cite{7883842}\cite{ARTICLE_typical1}\cite{ARTICLE_typical}的草案,基于 DH 密钥交换的零轮往返时长协议,但在正式发布前已经删除,最终使用的是基于 PSK 的握手恢复。在国内,只能在期刊上阅读到少量关于 TLS 1.3 的文章,而且文章内容讲述过于简单,不够全面,关于 TLS 1.3 协议最重要的部分零轮往返时长协议没有更多讲解,国内知名通信软件微信,参考 TLS 1.3 实现的安全通信协议 MMTLS\cite{MMTLS},其中实现零轮往返时长协议,比较于 TLS 协议,MMTLS 协议由于微信客户端每个人可用,于是删除了客户端认证相关的功能,同时在微信客户端程序中内置服务器的签名公钥,握手中不在需要进行服务器认证,减少发送的流量,关于 MMTLS 抗重放攻击,根据微信特有的后台架构,提出了基于客户端和服务器端时间序列的防重放策略,保证超过时间的重放包能被服务器拒绝,通过由 Proxy 层和 Logic 框架协同控制。
  
  支持 TLS 1.3的代码库,最广泛的当然有Openssl,Google 的 boringssl、guntls 等。国外相关工作,谷歌在基于 UDP 协议上实现的 QUIC Crypto 协议\cite{8280429}中首次实现了零轮往返时长协议,由于 QUIC 更早的使用零轮往返时长协议,实现的标准也提供给 TLS 1.3作为参考,但到了 TLS 1.3 正式版本发布,QUIC 反而会基于 TLS 1.3,并在以后的 HTTP/3.0 中使用,促进网络协议的发展。Facebook 使用 C++14 标准实现强大,高性能的 TLS 库,代码库命名为 Fizz,在 QUIC 基础改做出改动,在手机 APP 上实现零轮往返,实现更快的连接速度,并且有效地处理安全性问题,实现部署零轮往返时长协议,发现建立连接所需的时间降低了 41\%。
  
  在密钥交换的同时减少延迟开销已成为学术界和工业界的密钥交换)(key exchange, KE)协议的主要设计目标。在这方面特别感兴趣的是零轮往返时长协议,其允许客户端在零往返时间中发送加密数据,从而最小化等待时间。比如 Google 的 QUIC 协议和 TLS 1.3。零轮往返密钥交换的主要挑战是为协议中发送的第一个加密数据,称为早期数据,实现前向保密和防止重放攻击的安全性。有不少说法称不可能为此消息实现前向保密,因为用于加密早期数据的密钥必须依赖于接收者的非短暂密钥。如果接收者的非密钥在以后泄露给攻击者,攻击者应该可以执行与实际会话中的接收者相同的计算来计算会话密钥。
  
  在相关研究中,表明不能为早期数据提供前向保密性是错误的想法。研究人员构建了第一个零轮往返时长协议,利用一种可穿透的密钥封装方案,它为所有传输的有效负载消息提供完全的前向保密,并自动适应重放攻击。
  由于在没有先前连接的信息的情况下,是不可能在 0-RTT 中进行身份验证和建立加密密钥,因此 0-RTT 密钥交换协议必须利用在某些先前通信中获得的密钥材料来建立0-RTT密钥。在早期版本的 QUIC 中使用的一种非常常见的方法是基于 DH 密钥交换,但在正式发布的 TLS 1.3 中采用的是从预共享对称密钥(PSK)中导出 0-RTT 密钥。

\subsection{研究内容及主要贡献}

分析 TLS 1.3 协议,利用 WebSocket 简单实现握手协议和记录层协议,重现 TLS 1.3 中 0-RTT 的重放攻击。实现对抗 0-RTT 重放的方法并测试效果。得到以下结果:

\begin{itemize}
  \item[-] TLS 1.3 比旧版本更安全可靠
  \item[-] 0-RTT 握手在连接时速度更快
  \item[-] 相关实现可以正确抵抗 0-RTT 重放攻击
\end{itemize}

\subsection{论文章节安排}
\label{sec:arrangement}

第一章引言

第二章实现简单 TLS 1.3 协议

第三章分析零轮往返时长协议中的重放攻击

第四章测试和实验结果

第五章总结与展望

\afterpage{\null\newpage}
\newpage