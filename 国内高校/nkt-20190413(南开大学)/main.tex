
% -*- coding: utf-8 -*-
%%
%%
%%
%%
%%
%%
%%  本模板可以使用以下两种方式编译:
%%
%%     1. PDFLaTeX
%%
%%     2. XeLaTeX [推荐]
%%
%%  注意:
%%    1. 在改变编译方式前应先删除 *.toc 和 *.aux 文件,
%%       因为不同编译方式产生的辅助文件格式可能并不相同。
%%
%%
\documentclass[12pt,openright]{book}

\usepackage{ifxetex}
\ifxetex
  \usepackage[bookmarksnumbered]{hyperref}
\else
  \usepackage[unicode,bookmarksnumbered]{hyperref}
\fi

\usepackage[emptydoublepage]{NKThesis}   % 中文
%\usepackage[emptydoublepage,English]{NKThesis} % 英文
\usepackage{amssymb}
%   根据需要选择 biblatex 宏包选项.
\usepackage[backend = bibtex8, defernumbers = true,  sorting=none,  style = nkthesis]{biblatex}
\hypersetup{colorlinks=true,
            pdfborder=0 0 1,
            citecolor=black,
            linkcolor=black}
%\usepackage{tikz}
\usepackage{amsmath}
\addbibresource{nkthesis.bib}
\DeclareBibliographyCategory{cited}
\AtEveryCitekey{\addtocategory{cited}{\thefield{entrykey}}}

\includeonly{
abstract,
manual,
acknowledgements,
references,
appendices,
resume
}
\newtheorem{Theorem}{\hskip 2em 定理}[chapter]
\newtheorem{Lemma}[Theorem]{\hskip 2em 引理}
\newtheorem{Corollary}[Theorem]{\hskip 2em 推论}
\newtheorem{Proposition}[Theorem]{\hskip 2em 命题}
\newtheorem{Definition}[Theorem]{\hskip 2em 定义}
\newtheorem{Example}[Theorem]{\hskip 2em 例}
\newcommand{\upcite}[1]{\textsuperscript{\textsuperscript{\cite{#1}}}}
\begin{document}

%  设置基本信息
%  注意:  逗号`,'是项目分隔符. 如果某一项的值出现逗号, 应放在花括号内, 如 {,}
%
\NKTsetup{%
  论文题目(中文) = 现代绿色化学中的物理有机问题,
  副标题         = ——酵母菌催化反应机理和咪唑类离子液体酸度的研究,
  论文题目(英文) =  Physical Organic Concerns in Green Chemistry: Mechanistic
                   Aspects of Baker's Yeast Mediated Reduction and
                   Measurements of Acidity of Imidazolium Ionic Liquids,
  论文作者       = 张十三,
  学号           = 1020161111,
  指导教师       = 张三\quad 教授,
  申请学位       = 理学硕士,
  培养单位       = 某某学院,
  学科专业       = 某某某某,
  研究方向       = 某某某某,
  中图分类号     = ,
  UDC            = ,
  学校代码       = 10055,
  密级           = 公开,
                   % 公开 | 限制 | 秘密 | 机密, 若为公开, 不填以下三项
  保密期限       = ,
  审批表编号     = ,
  批准日期       = ,
  论文完成时间   = 二〇一九年三月,
  答辩日期       = ,
  论文类别       = 学历硕士,
                   % 博士 | 学历硕士 | 硕士专业学位 | 高校教师 | 同等学力硕士
  学院(单位)       = 某某学院,
  答辩委员会主席       = 李四,
  评阅人       = 李四\quad 王五,      
  专业           = 某某某某,
  联系电话       = 1234567,
  Email          = 123@123.com,
  通讯地址(邮编) = 天津市南开区卫津路94号(300071),
  备注           = }


%# -*- coding: utf-8-unix -*-
%%==================================================

\begin{abstract}
本项目为年产50万吨MTO工厂的初步设计。通过分析当前国内外MTO生产和研究现状,对生产工艺进行了选择论证。然后运用Aspen软件模拟初步的工艺流程,并通过对一系列工艺参数,如精馏塔的塔板数—产品纯度、进料塔板数—产品纯度、产品纯度—回流比、再沸器负荷—回流比等进行灵敏度分析,优化设备操作条件,提高工艺的合理性和经济性。本设计还针对工艺流程进行换热网络设计和对全局换热网络进行了优化和评估,通过内部流股之间相互换热以减少公用工程的消耗,最终优化后节约$79.4\%$的热公用工程资源和$73.7\%$的冷公用工程资源。本设计还运用水夹点技术优化了用水网络,根据水硬度分类处理水操作单元,并合理再生利用,使得本项目新鲜水用量和废水排放量达到最小,优化后的用水网络节约用水$53.59\%$。本设计对于MTO工厂的生产和设计建造具有一定的现实指导意义。\\

\keywords{\zihao{-4} 工厂\quad 设计\quad MTO \quad 工艺 \quad 水夹点  \quad 网络 \quad 控制}
\end{abstract}

\begin{englishabstract}

This project is the preliminary design of a MTO plant with an annual output of 500,000 tons of light olefins. Based on the current production and research situation all through the world, the production method was selected and demonstrated. Aspen software was used to simulate the preliminary process. Heat integration method was applied to optimize the heat exchange network. Rational heat exchange between process streams were suggested which resulted in the decreasing of utilities consumption and exchanger number. The heat integration leaded to energy saving of $79.4\%$ of heat utilities and $73.7\%$ of the cold utilities. In addition, the water pinch technology was also implemented to optimize the water network. The water operating unit was classified according to water hardness, with a reasonable recycling. The amount of fresh water consumption and wastewater emission was minimized. The optimized water network achieved $53.59\%$ water saving. Finally, a preliminary economic analysis to the entire project was estimated in order to get the project construction cost and profitability. In summary, this design is of some practical significance for the production and design of the MTO industry.

\englishkeywords{\zihao{-4} Plant design\;Sensitivity analysis  \; Energy balance\; calculation \; Water pinch  Dynamic control}
\end{englishabstract}


\tableofcontents
% ----------------------------------------------------------------------------
% User Manual
% ----------------------------------------------------------------------------
\documentclass[thesis.tex]{subfiles}
\begin{document}

\chapter[\vspace{-2\baselineskip}]{QUThesis User Manual}
\begin{quote}[Lewis Carroll, Alice in Wonderland][flushright]
``Begin at the beginning and go on till you come to the end, then stop.''
\end{quote}

\lettrine{T}{he quthesis style package} is designed for Doctorate of Philosophy
dissertations at the Queensland University of Technology (QUT), but may also be
useful at many other institutions. The goals of the package are threefold:
(i)~to comply with QUT's thesis submission guidelines, (ii)~to provide
comprehensive base-level functionality, structure and version control so you
can focus on content, and (iii)~to produce beautiful thesis documents that are
a pleasure to read. To this end, quthesis provides a number of builtin commands
and options, which are described in this manual.
\subsection*{Package Options}
The quthesis package is designed to be used with the \code{book} documentclass.
The package is initialized in the typical manner,

\begin{lstlisting}[language=tex]
    \documentclass[11pt,a4paper]{book}
    \usepackage[options]{quthesis}
\end{lstlisting}

\noindent The package accepts the following options:

\noindent\begin{longtable}{l p{8.5cm}}
\code|print|    & Print binding offsets of 0.5in are added to the inner edge, and
                  hyperlink coloring is disabled \\
\code|onehalfspacing| & Increase line spacing in the content, while preserving
                  header spacing \\
\code|strict|   & Comply with the QUT style guidelines (default)\\
\code|relaxed|  & Remove material that isn't of interest to the reader;
                  breaks compliance with the official QUT guidelines \\
\code|dropcaps| & Initializes the lettrine package so that paragraph beginning
                  words can be formatted using dropped capitals (dropcaps), as
                  per the first character of this manual \\
\code|calendas| & Use the Calendas font for headings, or a similar fallback if
                  Calendas is not available \\
\code|nonumber| & Remove numbering from sections and subsections (more suitable
                  for historical/literature theses)
\end{longtable}


\subsection*{Features}
\begin{tightemize}
\item Automatically generated cover page, and templates for front matter
      (dedication, acknowledgements, abstract)
\item Margins and font size optimized for readability
\item Simple and elegant headers and footers
\item Explicit breaking of chapter titles without affecting headers
\item Customized \code{listings} environment to better match the prose
      formatting
\item Better default settings for \code{hyperref}, including link colors, PDF
      metadata and references
\item Adjusted footnote size and spacing from main prose
\end{tightemize}


\subsection*{Provided Environments}
\begin{itemize}
\item \code|\begin{quote}[author][alignment] ... \end{quote}| \\ Format a
quote, like the one at the beginning of the manual. The command has two
optional arguments. The \code{author} formats a right-justified attribution
after the quote. The \code{alignment} argument specifies the horizontal
alignment of the quote on the page. It may be one of
\code{flushleft|center|flushright}.
\item \code|\begin{verticenter} ... \end{verticenter}| \\ Vertically center a
block of content with optical adjustment (the content is actually positioned
slightly above the true center).
\item \code|\begin{tightemize} ... \end{tightemize}| \\ A renewed itemize
environment with tighter spacing between items.
\item \code|\begin{biography}[path/to/portrait.png] ... \end{biography}| \\ An
environment for providing a short author biography at the end of the thesis.
Text is wrapped around the optional portrait figure. The environment opens left
instead of the usual right so that it appears on the back page as per novels.
\item \code|\begin{verse} ... \end{verse}| \\ A renewed verse environment that
provides two extra features over the regular verse environment: (1) lines can
be input verbatim, with blank lines indicating stanzas, and (2) page breaks are
disallowed mid-stanza.

\end{itemize}


\subsection*{Provided Commands}
\begin{itemize}
\item \code|\chapter[Alpha Name]{Chapter Name}| \\
A custom chapter command that, along with behaving like a normal chapter, also
allows unnumbered chapters and chapters with alpha names. This is useful for
introductory chapters such as \\\code|\chapter[Introduction]{The State of the Union}|.
\item \code|\chap, \sec, \subsec, \fig, \eqn| \\
These commands wrap \code|\ref| to provide two utilities: (i) enforce
namespacing in the use of labels, and (ii) enforce consistency of referring to
labels. A figure with the label \code|\label{fig:results}| for example, can be
referred to by \code|\fig{results}| which will evaluate to \code{Figure 2}. In
the five variants, the namespace in the label is always the command name
followed by a colon.
\item \code|\code{inline code}| \\
An alias of the \code{lstinline} command.
\item \code|\ie, \eg, \cf, \etc, \wrt, \dof, \etal| \\
Italicized abbreviations which respect the spacing and deduplication (for
sentence ending abbreviations) of the final period: \ie, \eg, \cf, \etc, \wrt,
\dof, \etal.
\end{itemize}

\end{document}

%!TEX program = XeLaTeX
%!TeX root =main.tex

\bibliography{paper.bib}

%%% Local Variables: 
%%% mode: latex
%%% TeX-master: "main.tex"
%%% End: 

\documentclass[../thesis.tex]{subfiles} % so that this document can be compiled on its own

\begin{document}

\begin{danksagung*}
\todo{Ihr Text hier.}
\end{danksagung*}

\begin{acknowledgements*}
\todo{Enter your text here.}
\end{acknowledgements*}

\end{document}
%\newpage
\appendix

%%附录第一个章节
\section{第一附录}


%%变量列举

\begin{table}[H]
\caption{Symbol Table-Constants}
\centering
\begin{tabular}{lll}
\toprule
Symbol & Definition  & Units\\
\midrule[2pt]
\multicolumn{3}{c}{\textbf{Constants} }\\
$DL$&Expectancy of poisson-distribution &  unitless \\
$NCL$ &Never- Change-Lane& unitless\\
$CCL$&Cooperative-Change-Lane& unitless\\
$ACL$&Aggressive-Change-Lane& unitless\\
$FCL$&Friendly-Change-Lane& unitless\\
$SCC$&Self-driving-Cooperative-Car& unitless\\
$NSC$&None-Self-drive-Car& unitless\\
\bottomrule
\end{tabular}
\end{table}


\section{第二附录}
\textcolor[rgb]{0.98,0.00,0.00}{\textbf{Simulation Code}}
\begin{python}
import java.util.*;  
public class test {  
    public static void main (String[]args){   
        int day=0;  
        int month=0;  
        int year=0;  
        int sum=0;  
        int leap;     
        System.out.print("请输入年,月,日\n");     
        Scanner input = new Scanner(System.in);  
        year=input.nextInt();  
        month=input.nextInt();  
        day=input.nextInt();  
        switch(month) /*先计算某月以前月份的总天数*/    
        {     
        case 1:  
            sum=0;break;     
        case 2:  
            sum=31;break;     
        case 3:  
            sum=59;break;     
        case 4:  
            sum=90;break;     
        case 5:  
            sum=120;break;     
        case 6:  
            sum=151;break;     
        case 7:  
            sum=181;break;     
        case 8:  
            sum=212;break;     
        case 9:  
            sum=243;break;     
        case 10:  
            sum=273;break;     
        case 11:  
            sum=304;break;     
        case 12:  
            sum=334;break;     
        default:  
            System.out.println("data error");break;  
        }     
        sum=sum+day; /*再加上某天的天数*/    
        if(year%400==0||(year%4==0&&year%100!=0))/*判断是不是闰年*/    
            leap=1;     
        else    
            leap=0;     
        if(leap==1 && month>2)/*如果是闰年且月份大于2,总天数应该加一天*/    
            sum++;     
        System.out.println("It is the the day:"+sum);  
        }  
} 
\end{python}



% !Mode:: "TeX:UTF-8"
\begin{resume}
\vspace*{\baselineskip}
\noindent
\begin{minipage}[t]{4cm}
\vspace{-\baselineskip}
\includegraphics[width=3.3cm]{biophoto}
\end{minipage}%
\hfill%
\begin{minipage}[t]{10cm}
\vspace{-\baselineskip}
%===============================
姓\qquad 名:×××

性\qquad 别:×

民\qquad 族:×族

出生年月:×××× 年× 月

籍\qquad 贯:××××
%===============================
\end{minipage}
\vspace*{\baselineskip}
%===============================

\textbf{学习经历}   % 自大学起

×××× 年× 月考入××大学××院××专业,×××× 年× 月本科毕业并获得××学士学位。

×××× 年× 月-- ×××× 年× 月,在燕山大学××学院××学科学习。

\textbf{获奖情况}   % 自大学起

×××× -- ×××× 年,燕山大学校级三好学生

×××× -- ×××× 年,燕山大学校级一等奖学金

×××× -- ×××× 年,燕山大学××学院三好学生

×××× -- ×××× 年,燕山大学××学院优秀团干部

(不含科研学术获奖)。

\textbf{工作经历}   % 没有可不写(将本行删除即可)

\end{resume}

\end{document}
