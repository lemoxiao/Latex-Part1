
% -*- coding: utf-8 -*-
%%
%%
%%
%%
%%
%%
%%  本模板可以使用以下两种方式编译:
%%
%%     1. PDFLaTeX
%%
%%     2. XeLaTeX [推荐]
%%
%%  注意:
%%    1. 在改变编译方式前应先删除 *.toc 和 *.aux 文件,
%%       因为不同编译方式产生的辅助文件格式可能并不相同。
%%
%%
\documentclass[12pt,openright]{book}

\usepackage{ifxetex}
\ifxetex
  \usepackage[bookmarksnumbered]{hyperref}
\else
  \usepackage[unicode,bookmarksnumbered]{hyperref}
\fi

\usepackage[emptydoublepage]{NKThesis}   % 中文
%\usepackage[emptydoublepage,English]{NKThesis} % 英文
\usepackage{amssymb}
%   根据需要选择 biblatex 宏包选项.
\usepackage[backend = bibtex8, defernumbers = true,  sorting=none,  style = nkthesis]{biblatex}
\hypersetup{colorlinks=true,
            pdfborder=0 0 1,
            citecolor=black,
            linkcolor=black}
%\usepackage{tikz}
\usepackage{amsmath}
\addbibresource{nkthesis.bib}
\DeclareBibliographyCategory{cited}
\AtEveryCitekey{\addtocategory{cited}{\thefield{entrykey}}}

\includeonly{
abstract,
manual,
acknowledgements,
references,
appendices,
resume
}
\newtheorem{Theorem}{\hskip 2em 定理}[chapter]
\newtheorem{Lemma}[Theorem]{\hskip 2em 引理}
\newtheorem{Corollary}[Theorem]{\hskip 2em 推论}
\newtheorem{Proposition}[Theorem]{\hskip 2em 命题}
\newtheorem{Definition}[Theorem]{\hskip 2em 定义}
\newtheorem{Example}[Theorem]{\hskip 2em 例}
\newcommand{\upcite}[1]{\textsuperscript{\textsuperscript{\cite{#1}}}}
\begin{document}

%  设置基本信息
%  注意:  逗号`,'是项目分隔符. 如果某一项的值出现逗号, 应放在花括号内, 如 {,}
%
\NKTsetup{%
  论文题目(中文) = 现代绿色化学中的物理有机问题,
  副标题         = ——酵母菌催化反应机理和咪唑类离子液体酸度的研究,
  论文题目(英文) =  Physical Organic Concerns in Green Chemistry: Mechanistic
                   Aspects of Baker's Yeast Mediated Reduction and
                   Measurements of Acidity of Imidazolium Ionic Liquids,
  论文作者       = 张十三,
  学号           = 1020161111,
  指导教师       = 张三\quad 教授,
  申请学位       = 理学硕士,
  培养单位       = 某某学院,
  学科专业       = 某某某某,
  研究方向       = 某某某某,
  中图分类号     = ,
  UDC            = ,
  学校代码       = 10055,
  密级           = 公开,
                   % 公开 | 限制 | 秘密 | 机密, 若为公开, 不填以下三项
  保密期限       = ,
  审批表编号     = ,
  批准日期       = ,
  论文完成时间   = 二〇一九年三月,
  答辩日期       = ,
  论文类别       = 学历硕士,
                   % 博士 | 学历硕士 | 硕士专业学位 | 高校教师 | 同等学力硕士
  学院(单位)       = 某某学院,
  答辩委员会主席       = 李四,
  评阅人       = 李四\quad 王五,      
  专业           = 某某某某,
  联系电话       = 1234567,
  Email          = 123@123.com,
  通讯地址(邮编) = 天津市南开区卫津路94号(300071),
  备注           = }


% Abstract
\clearpage
\thispagestyle{plain}
\phantomsection
\addcontentsline{toc}{chapter}{Abstract}

\centerline{\zihao{3}\bfseries Abstract}

\linespread{1.4}\zihao{-4}
\bigskip

This thesis explores the relationship between focus structure and pronoun resolution among non-native speakers of English and French. Firstly we reviewed the existing literature on the mechanism of focus effect and pronoun resolution. Then through a self-paced reading test, we find that focus, in the form of cleft structure does not necessarily increase the salience of a informational unit, thus may not in some cases make it a preferred antecedent for pronoun resolution. This result is line with previous researches on this topic. In our experiment, We also find that focused subject in French and focused object in English are processed faster, but focused subjects in both languages leads to longer response time of anaphora. Furthermore, our research also shows that the congruence between anaphora and focus does not make the latter more accessible. In this regard, we argue that the problem of whether there is subject or object preference in English and French is more complicated than the results of current studies.

\bigskip
\noindent\textbf{\zihao{4} Keywords:} 
focus effect, pronoun resolution, self-paced reading, English, French


\tableofcontents
% ----------------------------------------------------------------------------
% User Manual
% ----------------------------------------------------------------------------
\documentclass[thesis.tex]{subfiles}
\begin{document}

\chapter[\vspace{-2\baselineskip}]{QUThesis User Manual}
\begin{quote}[Lewis Carroll, Alice in Wonderland][flushright]
``Begin at the beginning and go on till you come to the end, then stop.''
\end{quote}

\lettrine{T}{he quthesis style package} is designed for Doctorate of Philosophy
dissertations at the Queensland University of Technology (QUT), but may also be
useful at many other institutions. The goals of the package are threefold:
(i)~to comply with QUT's thesis submission guidelines, (ii)~to provide
comprehensive base-level functionality, structure and version control so you
can focus on content, and (iii)~to produce beautiful thesis documents that are
a pleasure to read. To this end, quthesis provides a number of builtin commands
and options, which are described in this manual.
\subsection*{Package Options}
The quthesis package is designed to be used with the \code{book} documentclass.
The package is initialized in the typical manner,

\begin{lstlisting}[language=tex]
    \documentclass[11pt,a4paper]{book}
    \usepackage[options]{quthesis}
\end{lstlisting}

\noindent The package accepts the following options:

\noindent\begin{longtable}{l p{8.5cm}}
\code|print|    & Print binding offsets of 0.5in are added to the inner edge, and
                  hyperlink coloring is disabled \\
\code|onehalfspacing| & Increase line spacing in the content, while preserving
                  header spacing \\
\code|strict|   & Comply with the QUT style guidelines (default)\\
\code|relaxed|  & Remove material that isn't of interest to the reader;
                  breaks compliance with the official QUT guidelines \\
\code|dropcaps| & Initializes the lettrine package so that paragraph beginning
                  words can be formatted using dropped capitals (dropcaps), as
                  per the first character of this manual \\
\code|calendas| & Use the Calendas font for headings, or a similar fallback if
                  Calendas is not available \\
\code|nonumber| & Remove numbering from sections and subsections (more suitable
                  for historical/literature theses)
\end{longtable}


\subsection*{Features}
\begin{tightemize}
\item Automatically generated cover page, and templates for front matter
      (dedication, acknowledgements, abstract)
\item Margins and font size optimized for readability
\item Simple and elegant headers and footers
\item Explicit breaking of chapter titles without affecting headers
\item Customized \code{listings} environment to better match the prose
      formatting
\item Better default settings for \code{hyperref}, including link colors, PDF
      metadata and references
\item Adjusted footnote size and spacing from main prose
\end{tightemize}


\subsection*{Provided Environments}
\begin{itemize}
\item \code|\begin{quote}[author][alignment] ... \end{quote}| \\ Format a
quote, like the one at the beginning of the manual. The command has two
optional arguments. The \code{author} formats a right-justified attribution
after the quote. The \code{alignment} argument specifies the horizontal
alignment of the quote on the page. It may be one of
\code{flushleft|center|flushright}.
\item \code|\begin{verticenter} ... \end{verticenter}| \\ Vertically center a
block of content with optical adjustment (the content is actually positioned
slightly above the true center).
\item \code|\begin{tightemize} ... \end{tightemize}| \\ A renewed itemize
environment with tighter spacing between items.
\item \code|\begin{biography}[path/to/portrait.png] ... \end{biography}| \\ An
environment for providing a short author biography at the end of the thesis.
Text is wrapped around the optional portrait figure. The environment opens left
instead of the usual right so that it appears on the back page as per novels.
\item \code|\begin{verse} ... \end{verse}| \\ A renewed verse environment that
provides two extra features over the regular verse environment: (1) lines can
be input verbatim, with blank lines indicating stanzas, and (2) page breaks are
disallowed mid-stanza.

\end{itemize}


\subsection*{Provided Commands}
\begin{itemize}
\item \code|\chapter[Alpha Name]{Chapter Name}| \\
A custom chapter command that, along with behaving like a normal chapter, also
allows unnumbered chapters and chapters with alpha names. This is useful for
introductory chapters such as \\\code|\chapter[Introduction]{The State of the Union}|.
\item \code|\chap, \sec, \subsec, \fig, \eqn| \\
These commands wrap \code|\ref| to provide two utilities: (i) enforce
namespacing in the use of labels, and (ii) enforce consistency of referring to
labels. A figure with the label \code|\label{fig:results}| for example, can be
referred to by \code|\fig{results}| which will evaluate to \code{Figure 2}. In
the five variants, the namespace in the label is always the command name
followed by a colon.
\item \code|\code{inline code}| \\
An alias of the \code{lstinline} command.
\item \code|\ie, \eg, \cf, \etc, \wrt, \dof, \etal| \\
Italicized abbreviations which respect the spacing and deduplication (for
sentence ending abbreviations) of the final period: \ie, \eg, \cf, \etc, \wrt,
\dof, \etal.
\end{itemize}

\end{document}

\subsection{Sous section présentant les Références}

\subsubsection{Références bibliographique}

Référence bibliographique simple~\cite{Amendola2017}.

Multiples références bibliographiques en une seule commande~\cite{Petkov2005,Raman2009,Colinet2004}.


\subsubsection{Références aux équations, figures, et tables}

Voir l'équation~\eqref{eq:equation}. Voir \autoref{fig:label} et la figure~\ref{fig:label}.

Voir le tableau~\ref{table:tableaucomplexe} page~\pageref{table:tableaucomplexe}.


\defaultfont

\BiAppendixChapter{��~~~~л}{Acknowledgement}

\ifxueweidoctor               %��ʿҳü
\fancyhead[CO]{\CJKfamily{song}\xiaowu\leftmark}
\else
\fancyhead[CO]{\CJKfamily{song}\xiaowu ��������ҵ��ѧ\cxueke \cxuewei ѧλ����}
\fi
\fancyhead[CE]{\CJKfamily{song}\xiaowu ��������ҵ��ѧ\cxueke \cxuewei ѧλ���� }%

������ģ����UFO@bbs.hit.edu.cn�ġ���������ҵ��ѧ��ѧ��ʿ��˶ʿ������ģ�塷�Ļ����ϣ�
���ںܶ��˵İ�������ɵģ��ڴ�һ�������DZ�ʾ��л��

�ر��лStanley����������ģ�忪Դ��ĿPluto�Լ���������ģ��Ĵ����޸ģ�
ʹ֮���ӷ��Ϲ�������ģ��Ҫ��

�ر��л�������϶���վ��~Tex~�İ���~Tex��nebula������cucme��
������ʼ���ն�ȫ��֧��ģ�����������Ϊ�����˴����Ĺ�����

��л���괺~(HIT bbs ID: dengnch)�������˴�����ʱ��������ģ���
һϵ�в�����ʹ�ø�~\LaTeX~ģ��Ͷ�Ӧ��~Word~ģ��ĸ�ʽ������ȫһ�¡�

��лˮľ�廪��~\TeX~��~\LaTeX~��ĸ�λ����Ϊ���ṩ�ĸ��ְ�����
�ر���~snoopyzhao~���ѣ���������ĵ�Ϊ��ģ�����������ѣ�ʹ��ģ�����������
˳�����С�

������ĸ�л�������϶���~bbs~վ~Tex~���������ѵĴ���֧�֣�



ֵ���������֮�ʣ������������˽��������ʦ��ͬѧ�����Գ�ֿ��л�⣡

���ȸ�л�ҵĵ�ʦ{\bf ijijij}���ڣ������ĵ��о�����������{\bf
ij}��ʦ����Ľ�����չ���ġ�
����ѧ���ϲ��Ͻ�ȡ������������ִ��׷��ľ�������ѧϰ�İ�����{\bf
ij}��ʦ��������̵���ʶ������dz���Ľ�������������ӡ��


��л{\bf ijijij}���ں�{\bf ijijij}���ڶ���ѧϰ�͹����İ�����
�����ڷܵĹ������硢��۵�����̬�ȶ�����ظ�Ⱦ���ҡ���л{\bf
ijijij}���ں�{\bf ijijij}���ڶ���ѧҵ�������ϵĹ��ġ�


��л��ʿ��{\bf ijijij}��{\bf ijijij}��{\bf ijijij}��{\bf ijijij}��
���ҵ���˽�����ͻ���֧�֡���лʵ�������е��ֵܽ����ǣ�
����Ҷȹ����ⳤ�õ�ѧϰ���о��׶Σ������ҽ�����⣬����˼�롣

����ر�Ҫ��л�ҵ������ǣ����Ƕ���Ҫ�����٣��������ҵĶ��ǹػ���֧�ֺ����⡣

%\newpage
\appendix

%%附录第一个章节
\section{第一附录}


%%变量列举

\begin{table}[H]
\caption{Symbol Table-Constants}
\centering
\begin{tabular}{lll}
\toprule
Symbol & Definition  & Units\\
\midrule[2pt]
\multicolumn{3}{c}{\textbf{Constants} }\\
$DL$&Expectancy of poisson-distribution &  unitless \\
$NCL$ &Never- Change-Lane& unitless\\
$CCL$&Cooperative-Change-Lane& unitless\\
$ACL$&Aggressive-Change-Lane& unitless\\
$FCL$&Friendly-Change-Lane& unitless\\
$SCC$&Self-driving-Cooperative-Car& unitless\\
$NSC$&None-Self-drive-Car& unitless\\
\bottomrule
\end{tabular}
\end{table}


\section{第二附录}
\textcolor[rgb]{0.98,0.00,0.00}{\textbf{Simulation Code}}
\begin{python}
import java.util.*;  
public class test {  
    public static void main (String[]args){   
        int day=0;  
        int month=0;  
        int year=0;  
        int sum=0;  
        int leap;     
        System.out.print("请输入年,月,日\n");     
        Scanner input = new Scanner(System.in);  
        year=input.nextInt();  
        month=input.nextInt();  
        day=input.nextInt();  
        switch(month) /*先计算某月以前月份的总天数*/    
        {     
        case 1:  
            sum=0;break;     
        case 2:  
            sum=31;break;     
        case 3:  
            sum=59;break;     
        case 4:  
            sum=90;break;     
        case 5:  
            sum=120;break;     
        case 6:  
            sum=151;break;     
        case 7:  
            sum=181;break;     
        case 8:  
            sum=212;break;     
        case 9:  
            sum=243;break;     
        case 10:  
            sum=273;break;     
        case 11:  
            sum=304;break;     
        case 12:  
            sum=334;break;     
        default:  
            System.out.println("data error");break;  
        }     
        sum=sum+day; /*再加上某天的天数*/    
        if(year%400==0||(year%4==0&&year%100!=0))/*判断是不是闰年*/    
            leap=1;     
        else    
            leap=0;     
        if(leap==1 && month>2)/*如果是闰年且月份大于2,总天数应该加一天*/    
            sum++;     
        System.out.println("It is the the day:"+sum);  
        }  
} 
\end{python}



\resumeitem{个人简历:}
\noindent xxxx 年 xx 月 xx 日出生于 xx 省 xx 县。\\
\noindent xxxx 年 9 月考入 xx 大学 xx 系 xx 专业,xxxx 年 7 月本科毕业并获得 xx 学士学位。\\
\noindent xxxx 年 9 月免试进入 xx 大学 xx 系攻读 xx 学位至今。

\resumeitem{发表论文:} % 发表的和录用的合在一起
\begin{enumerate}[{[}1{]}]
\item Yang Y, Ren T L, Zhang L T, et al. Miniature microphone with silicon-
  based ferroelectric thin films. Integrated Ferroelectrics, 2003,
  52:229-235. (SCI 收录, 检索号:758FZ.)
\item 杨轶, 张宁欣, 任天令, 等. 硅基铁电微声学器件中薄膜残余应力的研究. 中国机
  械工程, 2005, 16(14):1289-1291. (EI 收录, 检索号:0534931 2907.)
\item 杨轶, 张宁欣, 任天令, 等. 集成铁电器件中的关键工艺研究. 仪器仪表学报,
  2003, 24(S4):192-193. (EI 源刊.)
\item Yang Y, Ren T L, Zhu Y P, et al. PMUTs for handwriting recognition. In
  press. (已被 Integrated Ferroelectrics 录用. SCI 源刊.)
\item Wu X M, Yang Y, Cai J, et al. Measurements of ferroelectric MEMS
  microphones. Integrated Ferroelectrics, 2005, 69:417-429. (SCI 收录, 检索号
  :896KM.)
\item 贾泽, 杨轶, 陈兢, 等. 用于压电和电容微麦克风的体硅腐蚀相关研究. 压电与声
  光, 2006, 28(1):117-119. (EI 收录, 检索号:06129773469.)
\item 伍晓明, 杨轶, 张宁欣, 等. 基于MEMS技术的集成铁电硅微麦克风. 中国集成电路, 
  2003, 53:59-61.
\end{enumerate}

\resumeitem{研究成果:} % 有就写,没有就删除
\begin{enumerate}[{[}1{]}]
\item 任天令, 杨轶, 朱一平, 等. 硅基铁电微声学传感器畴极化区域控制和电极连接的
  方法: 中国, CN1602118A. (中国专利公开号.)
\item Ren T L, Yang Y, Zhu Y P, et al. Piezoelectric micro acoustic sensor
  based on ferroelectric materials: USA, No.11/215, 102. (美国发明专利申请号.)
\end{enumerate}

\end{document}
