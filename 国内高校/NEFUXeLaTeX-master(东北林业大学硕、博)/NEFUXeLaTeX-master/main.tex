%!Tex Program = xelatex
% -*-coding: utf-8 -*-
%  \author{初砚硕}
\def\usewhat{xelatex} % 定义编译方式 pdflatex, dvipdfmx, or xelatex____ps:只有xelatex 能使用Arial字体

\def\version{1.1} % 模板版本号

\def\xueke{Engineering} % 定义学科 Engineering, Science, Management, or Arts

\documentclass[12pt,openany,twoside]{book}

\usepackage{amsmath}                                                                       % 排版数学公式
\usepackage{latexsym}
\usepackage{amsfonts}                                                                      %数学符号字体库宏包套件,它包含有:amsfonts、amssymb、eufrak 和 eucal 四个宏包。
\usepackage{amssymb}                                                                       % 定义AMS的数学符号命令
\usepackage{mathrsfs}                                                                      % 数学RSFS书写字体
\usepackage{bm}                                                                            % 数学黑体
\usepackage{graphicx}                                                                      % 支持插图,图形宏包graphics的扩展宏包
\usepackage{color,xcolor}                                                                  % 支持彩色
\usepackage[paperwidth=21cm,paperheight=29.7cm,
            top=2.54cm,bottom=2.54cm,left=2.54cm,right=2.54cm,
            headheight=1.5cm,headsep=0.07cm,footskip=1.75cm]{geometry}                     % 页面设置上下2.54cm,左右2.54cm,页眉1.5cm,页脚1.75cm
\usepackage{amscd}
\usepackage[linesnumbered,ruled,vlined]{algorithm2e}
\usepackage{diagbox}
\usepackage{minted}
\usepackage{listings}                                                                      %引入代码环境
\usepackage{titlesec}                                                                      %设置章节格式
\usepackage{enumerate}                                 							%更改enumerate环境格式
\usepackage{hyperref}
%url % 引用的宏包

\graphicspath{{figures/}} %定义所有的eps文件在 figures 子目录下


\begin{document}

%������ hyperref �� arydshln �����ݴ�����Ŀ¼����ʧЧ�����⡣
\def\temp{\relax}
\let\temp\addcontentsline
\gdef\addcontentsline{\phantomsection\temp}
\newcommand*{\subfigencaptionlist}{} % ��ͼ�μ���Ŀ¼ʱ��

\makeatletter
\gdef\hitfor{\@for}
\gdef\hitempty{}
\gdef\hittwo{\tw@}
\makeatother

\newcommand{\mr}[1]{\mathrm{#1}} %�����������\mr������\mathrm
\def \ReferenceEName {References} %%����ο����׵ı���
\def \ReferenceCName {�����}

%����ͼ���½�˫��������
\newcommand{\figenname}{Fig}
\newcommand{\listfigenname}{List of Figures}
\newfloatlist[chapter]{figen}{fen}{\listfigenname}{\figenname}
\newfixedcaption{\figencaption}{figen}
\renewcommand{\thefigen}{\thechapter-\arabic{figure}}
\makeatletter
\renewcommand{\@cftmakefentitle}{\chapter*{\listfigenname\@mkboth{\bfseries\listfigenname}{\bfseries\listfigenname}}}
\makeatother

\newcommand{\FigureBiCaption}[2]
{\renewcommand{\figurename}{ͼ}
\caption{\protect\setlength{\baselineskip}{1.5em}#1} %\protect\setlength{\baselinestretch}{1.3}\selectfont
\vspace{-0.5ex}
\figencaption{\protect\setlength{\baselineskip}{1.5em}#2}%
%%
%%����ͼ�μ���Ŀ¼
\makeatletter
\def\hittemp{}
 \hitfor \hittemp:=\subfigencaptionlist \do {%
        \ifx \hitempty\hittemp\relax \else
          \addcontentsline{fen}{subfigen}{\protect\numberline\hittemp}
        \fi}
 \gdef\subfigencaptionlist{}
\makeatother
}


\setcounter{fendepth}{2} %Ӣ��ͼ��Ŀ¼����� 1(ֻ��һ��Ŀ¼) 2(������Ŀ¼)
\setcounter{lofdepth}{2} %����ͼ��Ŀ¼����� 1(ֻ��һ��Ŀ¼) 2(������Ŀ¼)
\makeatletter
\renewcommand*{\l@subfigure}{\@dottedxxxline{\ext@subfigure}{2}{3.8em}{1.5em}} %����ͼ��Ŀ¼ subfigure
\gdef\l@subfigen{\@dottedtocline{0}{3.8em}{1.5em}}%Ӣ��ͼ��Ŀ¼ latex
\newif\ifsubfigtoc
\ifnum \tw@ > \@nameuse{c@fendepth} \subfigtocfalse \else \subfigtoctrue \fi
\makeatother
\newbox\tempbox
\newcommand{\SubfigEnCaption}[1]
{\makeatletter
 \ifsubfigtoc
    %����Ŀ¼���������һ��Ҫ�� ��ͼ ֮��,�������ݴ��� subfigencaptionlist
    \xdef\subfigencaptionlist{\subfigencaptionlist,%
        {{\thesubfigure}\protect\ignorespaces{#1}}}
\else
    \relax
\fi
\makeatother
%����caption
\vspace{-1.7ex}
\sbox{\tempbox}{\thesubfigure\hskip\subfiglabelskip #1}%
\ifthenelse{\lengthtest{\wd\tempbox > \linewidth}}%
{\\\parbox[t]{\linewidth}{\flushleft\noindent\thesubfigure\hskip\subfiglabelskip #1\hangafter=1\hangindent=15pt}}%
{\\[2ex]\centerline{\thesubfigure\hskip\subfiglabelskip #1}}
}

%\newcommand{\SubfigureCaption}[2]  % Two Parameters, the first one is the width of the subfigure,
%{
%\addtocounter{subfigure}{-1}       % the second one is the caption of the subfigure
%\vspace{-2ex}
%\subfigure[#2]{\rule{#1}{0pt}}
%}

\newcommand{\tblenname}{Table} %define tbl instead of table
\newcommand{\listtblenname}{List of Tables}
\newfloatlist[chapter]{tblen}{ten}{\listtblenname}{\tblenname}
\newfixedcaption{\tblencaption}{tblen}
\renewcommand{\thetblen}{\thechapter-\arabic{table}}% ��tblen����table����Ϊtable��tablen���һ�£���tablen��\longbitoccaption��������Ч��
\makeatletter
\renewcommand{\@cftmaketentitle}{\chapter*{\listtblenname\@mkboth{\bfseries\listtblenname}{\bfseries\listtblenname}}}
\makeatother

\newcommand{\TableBiCaption}[2]
{
\renewcommand{\tablename}{��}
\caption{\protect\setlength{\baselineskip}{1.5em}#1}
\vspace{-2ex}
\tblencaption{\protect\setlength{\baselineskip}{1.5em}#2}
\vspace{1ex}
}

%%%% �������caption����Ӣ�ı���Ŀ¼��������ʾ
\makeatletter
\def\@cont@LT@LTBiToeCaption#1[#2]#3{%
  \LT@makecaption#1\fnum@table{#3}%
  \def\@tempa{#2}%
  \ifx\@tempa\@empty\else
    {\let\\\space
      %\phantomsection
      \addcontentsline{ten}{tblen}{\protect\numberline{\thetable}{#2}}}%%\addcontentsline{lot}{table}{\protect\numberline{}{#2}}}%
  \fi}
\def\LT@c@ption#1[#2]#3{%
  \LT@makecaption#1\fnum@table{#3}%
  \def\@tempa{#2}%
  \ifx\@tempa\@empty\else
     {\let\\\space
     %\phantomsection
     \addcontentsline{lot}{table}{\protect\numberline{\thetable}{#2}}}%
  \fi}
\let\@cont@oldLT@c@ption\LT@c@ption
\newcommand*{\LTBiTocCaption}[5]{
  \@if@contemptyarg{#1}{\caption{#2}}{\caption[#1]{#2}}%
  \global\let\@cont@oldtablename\tablename
  \gdef\tablename{Table} %#3
  \global\let\LT@c@ption\@cont@LT@LTBiToeCaption
  \\
  \@if@contemptyarg{#4}{\caption{#5}}{\caption[#4]{#5}}%
  \global\let\tablename\@cont@oldtablename
  \global\let\LT@c@ption\@cont@oldLT@c@ption}
\makeatother

\renewcommand{\cfttblendotsep}{1} %�Զ���ͼ��Ŀ¼�еĵ����С
\renewcommand{\cftfigendotsep}{1}

%\renewcommand{\tablename}{��}  %jdg�ṩ��һ�ַ�����Ӣ�ij��������ӵ�����Ŀ¼��ȥ�����涨��
%\newcommand{\LTBiCaption}[2]   %\bicaptiontwotoc ��Ҫ���ǽ��������⡣
%{%
%\caption{#1} \gdef\tablename{Table}
%\\ %[-3.5ex]
%\caption{#2}
%\gdef\tablename{��}\\ %[-1.5ex]
%}

%%%---��ʽ�з�������----start----
%\begin{formulasymb}{ʽ��}{-3pt}%-3pt,-20pt�����Ϸ��ļ�ࡣ
%  \fdesfirst{��һ��ǩ}{���ƿ��ƿ��ƿ��ƿ���}
%  \fdes{������ǩ}{���ƿ��ƿ��ƿ��ƿ���}
%\end{formulasymb}
\newenvironment{formulades}[1]%
{\noindent\begin{list}{}{%
\setlength\topsep{0pt}
\settowidth{\labelwidth}{#1}
\setlength{\labelsep}{1mm}
\setlength{\leftmargin}{\labelwidth+\labelsep}
}}{\end{list}}
\newenvironment{formulasymb}[2]%-\!-\!-\!-
{\vspace*{#2}\newcommand{\fdesfirst}[2]%
{\begin{formulades}{#1\hspace*{26pt}##1~\cdash}\item[#1\hspace*{26pt}##1~\cdash]{##2}\end{formulades}\vspace*{-21pt}}%�Լ�����
\newcommand{\fdes}[2]{\begin{formulades}{#1\hspace{26pt}##1~\cdash}\item[##1~\cdash]{##2}\end{formulades}\vspace*{-21pt}}}%�Լ�����
{\vspace{21pt}\relax}%21pt����
%%----��ʽ�з�������----end-----

\makeatletter%���¶���BiChapter�����ʵ�ֱ����ֶ����У�����Ӱ��Ŀ¼
\def\BiChapter{\relax\@ifnextchar [{\@BiChapter}{\@@BiChapter}}
\def\@BiChapter[#1]#2#3{\chapter[#1]{#2}
    \addcontentsline{toe}{chapter}{\bfseries Chapter \thechapter\hspace{0.5em} #3}}
\def\@@BiChapter#1#2{\chapter{#1}
    \addcontentsline{toe}{chapter}{\bfseries Chapter \thechapter\hspace{0.5em} #2}}
\makeatother
%\newcommand{\BiChapter}[2]
%{
%    \chapter{#1}
%    \addcontentsline{toe}{chapter}{\bfseries Chapter \thechapter\hspace{0.5em} #2}
%}

\newcommand{\BiSection}[2]
{   \section{#1}
    \addcontentsline{toe}{section}{\protect\numberline{\csname thesection\endcsname}#2}
}

\newcommand{\BiSubsection}[2]
{    \subsection{#1}
    \addcontentsline{toe}{subsection}{\protect\numberline{\csname thesubsection\endcsname}#2}
}

\newcommand{\BiSubsubsection}[2]
{    \subsubsection{#1}
    \addcontentsline{toe}{subsubsection}{\protect\numberline{\csname thesubsubsection\endcsname}#2}
}

\newcommand{\BiAppendixChapter}[2] % �ø�¼���������ڷ������£�������
{\phantomsection
\markboth{#1}{#1}%\markboth{\MakeUppercase{#1}}{\MakeUppercase{#1}}
\addcontentsline{toc}{chapter}{\hei #1}
\addcontentsline{toe}{chapter}{\bfseries #2}  \chapter*{#1}
}

\newcommand{\BiAppChapter}[2]    % �ø�¼�������������½ڵ�������¼
{\phantomsection
\markboth{#1}{#1}  \chapter{#1}   %\markboth{\MakeUppercase{#1}}{\MakeUppercase{#1}}
%\addcontentsline{toc}{chapter}{\hei #1}
\addcontentsline{toe}{chapter}{\bfseries Appendix A~~#2}
}

\renewcommand{\thefigure}{\arabic{chapter}-\arabic{figure}}%ʹͼ���Ϊ 7-1 �ĸ�ʽ %\protect{~}
%\makeatletter
%\renewcommand\fnum@figure{\figurename\nobreakspace\thefigure\protect{~~~~~~~~~}} %
%\makeatother

\renewcommand{\thesubfigure}{\alph{subfigure})}%ʹ��ͼ���Ϊ a)�ĸ�ʽ
\makeatletter
\renewcommand{\p@subfigure}{\thefigure(} %%ʹ��ͼ����Ϊ 7-1(a) �ĸ�ʽ
\makeatother
%\renewcommand{\thesubfigure}{\alph{subfigure}}
%\makeatletter
%\renewcommand{\p@subfigure}{\thefigure} %%ʹ��ͼ����Ϊ 7-1a �ĸ�ʽ
%\renewcommand{\@thesubfigure}{\thesubfigure)\hskip\subfiglabelskip}%ʹ��ͼ���Ϊ a)�ĸ�ʽ
%\makeatother

\renewcommand{\thetable}{\arabic{chapter}-\arabic{table}}%%ʹ�����Ϊ 7-1 �ĸ�ʽ
\renewcommand{\theequation}{\arabic{chapter}-\arabic{equation}}%%ʹ��ʽ���Ϊ 7-1 �ĸ�ʽ

\setlength\jot{2.5ex}%���ù�ʽ֮��Ĵ�ֱ���� Ĭ��value = 3pt

%���� ѧ�� ѧλ
\def \xuekeEngineering {Engineering}
\def \xuekeScience {Science}
\def \xuekeManagement {Management}
\def \xuekeArts {Arts}

\ifx \xueke \xuekeEngineering
\newcommand{\cxueke}{��ѧ}
\newcommand{\exueke}{Engineering}
\fi

\ifx \xueke \xuekeScience
\newcommand{\cxueke}{��ѧ}
\newcommand{\exueke}{Science}
\fi

\ifx \xueke \xuekeManagement
\newcommand{\cxueke}{����ѧ}
\newcommand{\exueke}{Management}
\fi

\ifx \xueke \xuekeArts
\newcommand{\cxueke}{��ѧ}
\newcommand{\exueke}{Arts}
\fi

\newcommand{\cdash}{\mbox{��\!\!\!\!��\!\!\!\!��}}%�����������ۺŵ�����
\newcommand{\dif}{\mathrm{d}}%����ѧģʽ������΢��dx
 % 文本格式定义
\pagestyle{plain}                                                                          %plain格式页眉页脚
\titleformat{\section}{\centering\heiti\zihao{-2}}{实验\,\thesection: }{1em}{}              %小二号黑体居中
\titleformat{\subsection}{\heiti\zihao{-3}}{\thesubsection}{1em}{}                         %小三号黑体
\titleformat{\subsubsection}{\heiti\zihao{-4}}{\thesubsubsection}{1em}{}                   %小四号黑体
\lstset{
    basicstyle=\tt,                                                                        %行号
    numbers=left,
    rulesepcolor=\color{red!20!green!20!blue!20},
    escapeinside=``,
    xleftmargin=2em,xrightmargin=2em, aboveskip=1em,                                       %背景框
    framexleftmargin=1.5mm,
    frame=shadowbox,                                                                       %背景色
    backgroundcolor=\color[RGB]{245,245,244},                                              %样式
    keywordstyle=\color{blue}\bfseries,
    identifierstyle=\bf,
    numberstyle=\color[RGB]{0,192,192},
    commentstyle=\it\color[RGB]{96,96,96},
    stringstyle=\rmfamily\slshape\color[RGB]{128,0,0},                                     %显示空格
    showstringspaces=false
}                                                                                          %代码显示格式设置 感谢@信的灵感

\hypersetup{hidelinks}
\fancyhead[CO]{\song \xiaowu \leftmark}
\fancyhead[CE]{\song \xiaowu \leftmark}
\frontmatter
\thispagestyle{empty}
\begin{center}
    \includegraphics[scale=0.7]{figures/cdut.png}
\end{center}
\vskip1.5cm
\begin{center}
    \makebox[109mm][s]{\heiti\zihao{-0}\bf 本科生实验报告}
\end{center}
\vskip2cm
\begin{center}
    \makebox[20mm][s]{\heiti\zihao{4} 实验课程}\underline{\makebox[130mm][c]{\heiti\zihao{3}\LaTeX 书写行为规范}}\\
    \vskip1cm
    \makebox[20mm][s]{\heiti\zihao{4} 实验名称}\underline{\makebox[130mm][c]{\heiti \zihao{3} 利用\LaTeX 书写成都理工大学实验报告}}\\
    \vskip1cm
    \makebox[20mm][s]{\heiti\zihao{4} 专业名称}\underline{\makebox[130mm][c]{\heiti\zihao{3} 专业全称(有专业方向的用小括号标明)}}\\
    \vskip1cm
    \makebox[20mm][s]{\heiti\zihao{4} 学生姓名}\underline{\makebox[130mm][c]{\heiti\zihao{3} 您的姓名}}\\
    \vskip1cm
    \makebox[20mm][s]{\heiti\zihao{4} 学生学号}\underline{\makebox[130mm][c]{\heiti\zihao{3} 您的学号}}\\
    \vskip1cm
    \makebox[20mm][s]{\heiti\zihao{4} 指导教师}\underline{\makebox[130mm][c]{\heiti\zihao{3} 您的授课或者指导老师}}\\
    \vskip1cm
    \makebox[20mm][s]{\heiti\zihao{4} 实验地点}\underline{\makebox[130mm][c]{\heiti\zihao{3} 授课地点(如:6C403)}}
    \vskip1cm
    \makebox[20mm][s]{\heiti\zihao{4} 实验成绩}\underline{\makebox[130mm][c]{\heiti\zihao{3} 由指导老师书写}}\\
\end{center}
\vskip1.85cm
\vfill\begin{center}
    {\songti\zihao{3}二〇一八年三月二十一日}
\end{center}
\newpage
\thispagestyle{empty}
\tableofcontents
\newpage
\setcounter{page}{1} % 封面

%% 目录

\defaultfont
\clearpage{\pagestyle{empty}\cleardoublepage}
\pdfbookmark[0]{目~~~~录}{mulu}
\tableofcontents    % 中文目录
\clearpage{\pagestyle{empty}\cleardoublepage}

%\addtocontents{toc}{\protect\vskip1\baselineskip} % 中文目录增加空行

\clearpage{\pagestyle{empty}\cleardoublepage}     % 清除目录后面空页的页眉和页脚

\mainmatter\defaultfont\sloppy\raggedbottom
\fancyhead[CO]{\song \xiaowu 东北林业大学硕士学位论文}
\fancyhead[CE]{\song \xiaowu \leftmark}%
\titleformat{\chapter}{\center\xiaoer\bf\hei\arial}{\chaptername}{0.5em}{}% 实现英文摘要字体和正文章标题字体不一样
\titlespacing{\chapter}{0pt}{-6.35mm}{15pt}


%!Tex Program = xelatex
% -*-coding: utf-8 -*-
        \BiChapter{模版简介}{Review}

        \BiSection{模版来源}{} 

本模版是从哈工大\LaTeX 模版2.0改版而来。

\BiSection{模版使用方法}{} 
本模版的插图、公式、图表命令与哈工大\LaTeX 模版2.0一致。
其具体方式可查看哈工大\LaTeX 模版2.0的模版说明。
不同之处有如下几点:
\begin{nefulist}
  \item 列表命令
  
  \begin{verbatim}
    \begin{nefulist}
      \item *
      \item *
      \item *      
    \end{nefulist}
  \end{verbatim}
  \item 公式引用命令
  
  \begin{verbatim}
    \nefuEqRef{label}
  \end{verbatim}
\end{nefulist}
 \BiSection{模版更改之处}{} 
\begin{nefulist}
  \item 页面大小更改
  \item 页眉页脚  
  
  页眉双杠线的粗细与word一致,
  奇数偶数页眉交替。
  
  \item 字体设置
  \item 封皮设置  
  \item  参考文献应用
  
  在文后的参考文献列表使用与哈工大一致,使用05版的国标(林大太落后了该用新标准了)。
  但在正文中引用三篇连续的参考文献时候,林大规定用波浪线“$\sim$”。
\end{nefulist}



\chapter{Conclusion and Future Scope}


\section{Future Scope}



\section{Conclusion}

   % 结论
%参考文献

\fancyhead[CO]{\song \xiaowu \leftmark}
\fancyhead[CE]{\song \xiaowu \leftmark}%
\defaultfont

\bibliographystyle{chinesebst2005_UTF8}%
\addcontentsline{toc}{chapter}{参考文献}      % 参考文献加入到中文目录
\addcontentsline{toe}{chapter}{\bfseries  References} % 参考文献加入到英文目录
\addtolength{\bibsep}{-0.8em}
%\nocite{*}
\renewcommand\bibname{参考文献}%book 类
%如果文档类是article之类的, 用\renewcommand\refname{参考文献}
%如果文档类是book之类的, 用\renewcommand\bibname{参考文献}

\bibliography{reference/reference}

\newpage
\appendix

\section{Appendix A: Simulated Skew-T}
\lipsum

\section{Appendix B: Simulated Hodographs}
\lipsum
% 致谢
%----------------------------------------------------------------------------------------
%	Publications
%----------------------------------------------------------------------------------------
\lettersection{\faEditS \hspace{0.1cm} Publications}
%----------------------------------------------------------------------------------------
%	BGP Origin Validation (RPKI)
%----------------------------------------------------------------------------------------
\begin{tabularx}{1\linewidth}{>{\raggedleft\scshape}p{2.5cm}X}
\gray SURFnet & \textbf{\href{http://rp.delaat.net/2012-2013/p59/report.pdf}{BGP Origin Validation (RPKI)}} \hfill June 2013\\
\end{tabularx}

\vspace{2pt}
The goal is to increase the adoption rate of RPKI. Hence we've build SURFnet's RPKI Dashboard, which provides network operators with a number of RPKI statistics and assists them in cleaning up possible invalid prefixes.\\ {\faAlternateExternalLink}  \url{https://labs.ripe.net/Members/javy_de_koning/rpki-dashboard} {\faAlternateExternalLink} \url{http://goo.gl/bwOxjD}
\vspace{12pt}

%----------------------------------------------------------------------------------------
%	Defending against DNS reflection amplification attacks
%----------------------------------------------------------------------------------------
\begin{tabularx}{1\linewidth}{>{\raggedleft\scshape}p{2.5cm}X}
\gray NLnet Labs & \textbf{\href{http://www.nlnetlabs.nl/downloads/publications/report-rrl-dekoning-rozekrans.pdf}{Defending against DNS reflection amplification attacks}} \hfill February 2013\\
\end{tabularx}

\vspace{2pt}
The research goal was to find out if the proposed mechanisms to defend against a DNS amplification attacks are effective. The main conclusion is that response rate limiting is a proper defense mechanism against current amplification attacks, but it is not effective against more sophisticated attacks.{\faAlternateExternalLink} \url{http://goo.gl/yWn9vS} 
\vspace{12pt}

%----------------------------------------------------------------------------------------
%	ABN HACK
%----------------------------------------------------------------------------------------
\begin{tabularx}{1\linewidth}{>{\raggedleft\scshape}p{2.5cm}X}
\gray UvA & \textbf{\href{http://staff.science.uva.nl/~delaat/news/2013-02-12/security_in_mobile_banking_ssn.pdf}{Security in mobile banking}} \hfill December 2012\\
\end{tabularx}

\vspace{2pt}
The goal of the research was to find out how the security used in Android based mobile banking applications is implemented. During this project we discovered a serious weakness in the ABN AMRO mobile banking Android app which was exposing PIN and transaction information. {\faAlternateExternalLink} \url{http://goo.gl/P66Xvn} {\faAlternateExternalLink} \url{https://www.security.nl/posting/40061}
\vspace{12pt}    % 所发文章
\documentclass[../thesis.tex]{subfiles} % so that this document can be compiled on its own

\begin{document}

\begin{danksagung*}
\todo{Ihr Text hier.}
\end{danksagung*}

\begin{acknowledgements*}
\todo{Enter your text here.}
\end{acknowledgements*}

\end{document}% 致谢
%!Tex Program = xelatex
% -*-coding: utf-8 -*-
\pagestyle{empty}
\newcommand{\subchapterstyle}%
  {\song\rmfamily\bfseries\xiaoer}  
\addcontentsline{toc}{chapter}{独创性声明}
\begin{center}{\subchapterstyle 独创性声明}\end{center}

 本人声明所呈交的学位论文是本人在导师指导下进行的研究工作及取得的研
 究成果。据我所知,除了文中特别加以标注和致谢的地方外,论文中不包含其他
 人已经发表或撰写过的研究成果,也不包含为获得\underline{\textbf{~东北林业大学~}}或其他教育
 机构的学位或证书而使用过的材料。与我一同工作的同志对本研究所做的任何贡献均己在论文中作了明确的说明并表示谢意

\vspace{\baselineskip}
\hspace{4em}学位论文作者签名:\hfill 签字日期:\hspace{2.5em}年\hspace{1.5em}月\hspace{1.5em}日

\vspace{2\baselineskip}
\addcontentsline{toc}{chapter}{学位论文版权使用授权书}
\begin{center}{\subchapterstyle 学位论文版权使用授权书}\end{center}

本学位论文作者完全了解\underline{\textbf{~东北林业大学~}}有关保留、使用学位论文的规
定,有权保留并向国家有关部门或机构送交论文的复印件和磁盘,允许论文被查
阅和借阅。本人授权\underline{\textbf{~东北林业大学~}}可以将学位论文的全部或部分内容编入有
关数据库进行检索,可以采用影印、缩印或扫描等复制手段保存、汇编学位论
文。
(保密的学位论文在解密后适用本授权书)

%\vspace{\baselineskip}
%本学位论文属于(请在以下相应方框内打“$\surd$”):
%
%保密\xiaoer$\square$\xiaosi ,在~~~~~~~~~~~~~年解密后适用本授权书
%
%不保密\xiaoer$\square$\xiaosi

\vspace{\baselineskip}
\hspace{4em}学位论文作者签名:\hfill 签字日期:\hspace{2.5em}年\hspace{1.5em}月\hspace{1.5em}日

\vspace{\baselineskip}
\hspace{4em}导师签名:\hfill 签字日期:\hspace{2.5em}年\hspace{1.5em}月\hspace{1.5em}日

\vspace{2\baselineskip}
学位论文作者毕业后去向:\\

\vspace{\baselineskip}
\hspace{4em}工作单位:  \hfill 电话:

\vspace{\baselineskip}
\hspace{4em}通讯地址:  \hfill  邮编:

   % 承诺

\clearpage


\end{document}
