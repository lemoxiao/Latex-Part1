%!Tex Program = xelatex
% -*-coding: utf-8 -*-


\newcommand{\xiaochu}{\fontsize{36pt}{36pt}\selectfont}       % 一号, 1.倍行距
\newcommand{\yihao}{\fontsize{26pt}{26pt}\selectfont}       % 一号, 1.倍行距
\newcommand{\xiaoyi}{\fontsize{24pt}{24pt}\selectfont}      % 小一, 1.倍行距
\newcommand{\erhao}{\fontsize{22pt}{22pt}\selectfont}       % 二号, 1.倍行距
\newcommand{\xiaoer}{\fontsize{18pt}{18pt}\selectfont}      % 小二, 单倍行距
\newcommand{\sanhao}{\fontsize{16pt}{16pt}\selectfont}      % 三号, 1.倍行距
\newcommand{\xiaosan}{\fontsize{15pt}{15pt}\selectfont}     % 小三, 1.倍行距
\newcommand{\sihao}{\fontsize{14pt}{14pt}\selectfont}       % 四号, 1.0倍行距
\newcommand{\xiaosi}{\fontsize{12pt}{12pt}\selectfont}      % 小四, 1.倍行距
\newcommand{\wuhao}{\fontsize{10.5pt}{10.5pt}\selectfont}   % 五号, 单倍行距
\newcommand{\xiaowu}{\fontsize{9pt}{9pt}\selectfont}        % 小五, 单倍行距



%避免宏包 hyperref 和 arydshln 不兼容带来的目录链接失效的问题。
\def\temp{\relax}
\let\temp\addcontentsline
\gdef\addcontentsline{\phantomsection\temp}

\newcommand{\nefuEqRef}[1]{(\ref{#1})}

\makeatletter

%\gdef\hitempty{}%全局定义空,貌似没用

%\renewcommand\labelenumi{(\theenumi )} %调整列表环境为(1)

\newcommand{\mr}[1]{\mathrm{#1}} %定义新命令,用\mr来代替\mathrm

%重新定义BiChapter命令,可实现标题手动换行,但不影响目录
\def\BiChapter{\relax\@ifnextchar [{\@BiChapter}{\@@BiChapter}}

\def\@BiChapter[#1]#2#3{\chapter[#1]{#2}
    \addcontentsline{toe}{chapter}{\bfseries \xiaosi Chapter \thechapter\hspace{0.5em} #3}}
\def\@@BiChapter#1#2{\chapter{#1}
    \addcontentsline{toe}{chapter}{\bfseries \xiaosi Chapter \thechapter\hspace{0.5em}{\boldmath #2}}}

\newcommand{\BiSection}[2]
{   \section{#1}
    \addcontentsline{toe}{section}{\protect\numberline{\csname thesection\endcsname}#2}
}

\newcommand{\BiSubsection}[2]
{    \subsection{#1}
    \addcontentsline{toe}{subsection}{\protect\numberline{\csname thesubsection\endcsname}#2}
}

\newcommand{\BiSubsubsection}[2]
{    \subsubsection{#1}
    \addcontentsline{toe}{subsubsection}{\protect\numberline{\csname thesubsubsection\endcsname}#2}
}

\newcommand{\BiAppendixChapter}[2] % 该附录命令适用于发表文章,简历等
{\phantomsection
\markboth{#1}{#1}
\addcontentsline{toc}{chapter}{\xiaosi #1}
\addcontentsline{toe}{chapter}{\bfseries \xiaosi #2}  \chapter*{#1}
}

\newcommand{\BiAppChapter}[2]    % 该附录命令适用于有章节的完整附录
{\phantomsection  \chapter{#1}   %\markboth{\MakeUppercase{#1}}{\MakeUppercase{#1}} %为了winedt中project tree 中toc正确显示,不要挪到下一行;
\addcontentsline{toe}{chapter}{\bfseries \xiaosi Appendix \thechapter~~#2}
}

\renewcommand{\thefigure}{\arabic{chapter}-\arabic{figure}}%使图编号为 7-1 的格式 %\protect{~}
\renewcommand{\thesubfigure}{\alph{subfigure})}%使子图编号为 a)的格式
\renewcommand{\p@subfigure}{\thefigure~} %%使子图引用为 7-1 a) 的格式,母图编号和子图编号之间用~加一个空格
\renewcommand{\thetable}{\arabic{chapter}-\arabic{table}}%%使表编号为 7-1 的格式
\renewcommand{\theequation}{\arabic{chapter}-\arabic{equation}}%% 使公式编号为 7-1 的格式

\makeatother

%定义 学科 学位
\def \xuekeEngineering {Engineering}
\def \xuekeScience {Science}
\def \xuekeManagement {Management}
\def \xuekeArts {Arts}

\ifx \xueke \xuekeEngineering
\newcommand{\cxueke}{工学}
\newcommand{\exueke}{Engineering}
\fi

\ifx \xueke \xuekeScience
\newcommand{\cxueke}{理学}
\newcommand{\exueke}{Science}
\fi

\ifx \xueke \xuekeManagement
\newcommand{\cxueke}{管理学}
\newcommand{\exueke}{Management}
\fi

\ifx \xueke \xuekeArts
\newcommand{\cxueke}{文学}
\newcommand{\exueke}{Arts}
\fi
