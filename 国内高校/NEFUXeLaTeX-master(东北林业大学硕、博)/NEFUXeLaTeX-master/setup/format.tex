%!Tex Program = xelatex
% -*-coding: utf-8 -*-
\allowdisplaybreaks[4]

%\CJKcaption{gb_452_UTF8} %_UTF8 xelate不用

\setlength{\parindent}{2em}

\renewcommand\contentsname{目录}%目~~~~录

\renewcommand\chaptername{~\thechapter~}

\setcounter{secnumdepth}{4} \setcounter{tocdepth}{2}


\titleformat{\chapter}{\center\xiaoer\bf\hei}{\chaptername}{0.5em}{}
\titlespacing{\chapter}{0pt}{-6.35mm}{15pt}
%\titleformat{\chapter*}{\center\xiaoer\bf\hei\arial}{\chaptername}{0.5em}{}
%\titlespacing{\chapter*}{0pt}{-6.35mm}{15pt}

\titleformat{\section}{\xiaosan\bf\hei\arial}{\thesection}{0.5em}{}
\titlespacing{\section}{0pt}{6pt}{6pt}
\titleformat{\subsection}{\sihao\bf\hei\arial}{\thesubsection}{0.5em}{}
\titlespacing{\subsection}{0pt}{6pt}{0mm}
\titleformat{\subsubsection}[runin]{\song\arial}{\thesubsubsection}{0.5em}{}[\quad\quad ]
\titlespacing{\subsubsection}{0pt}{0pt}{0pt}

\titlecontents{chapter}[0em]{\bf\hei}{\arial\thecontentslabel~}{}{\bf\titlerule*[4pt]{.}\contentspage}
\titlecontents{section}[0em]{}{\thecontentslabel~}{}{\titlerule*[4pt]{.}\contentspage}
\titlecontents{subsection}[0em]{}{\thecontentslabel~}{}{\titlerule*[4pt]{.}\contentspage}

%\dottedcontents{section}[40pt]{}{22pt}{0.3pc}
%\dottedcontents{subsection}[62pt]{}{32pt}{0.3pc}



% 缩小目录中各级标题之间的缩进,使它们相隔一个字符距离,也就是12pt
\makeatletter
\renewcommand*\l@chapter{\@dottedtocline{0}{0em}{4em}}% 控制英文目录: 细点\@dottedtocline  粗点\@dottedtoclinebold
\renewcommand*\l@section{\@dottedtocline{1}{1.5em}{1.8em}}
\renewcommand*\l@subsection{\@dottedtocline{2}{2.5em}{2.5em}}


% 定义页眉和页脚 使用fancyhdr 宏包
%\newcommand{\makeheadrule}{
%    \rule[0pt]{\textwidth}{0.75pt}\\[-16.5pt]
%\rule{\textwidth}{0.8mm}
%}
%\renewcommand{\headrule}{
%    {\if@fancyplain\let\headrulewidth\plainheadrulewidth\fi
%     \makeheadrule}
%}
%\renewcommand{\headrule}{
%    \hrule height0.4true mm width\headwidth \vskip0.2true mm
%    \hrule height0.2true mm width\headwidth \vskip-0.8true mm
%}

\renewcommand{\headrule}{
    \vskip0.5true mm
    \hrule height0.5true mm width\textwidth \vskip0.3true mm
    \hrule height0.3true mm width\textwidth \vskip-1.6true mm
}

\pagestyle{fancyplain}

%去掉章节标题中的数字
%%不要注销这一行,否则页眉会变成:“第1章1  绪论”样式
\renewcommand{\chaptermark}[1]{\markboth{\chaptertitlename~\ #1}{}}
\fancyhf{}

%在book文件类别下,\leftmark自动存录各章之章名,\rightmark记录节标题
%% 页眉字号 要求 小五
%根据单双面打印设置不同的页眉;
%\fancyhead[CO]{\song \xiaowu 东北林业大学硕士学位论文}
%\fancyhead[CE]{\song \xiaowu \leftmark}%
\fancyfoot[C,C]{\xiaowu -~\thepage~-}


\renewcommand\frontmatter{\cleardoublepage
  \@mainmatterfalse
  \pagenumbering{Roman}}

% 设置行距和段落间垂直距离
\renewcommand{\CJKglue}{\hskip 0.56pt plus 0.08\baselineskip} % 加大字间距,使每行34个字

% 调整列表环境的垂直间距
\setitemize{itemindent=3em,leftmargin=0pt,itemsep=0ex,listparindent=2em,partopsep=0pt,parsep=0ex,topsep=0ex}
\setenumerate{itemindent=3em,leftmargin=0pt,itemsep=0ex,listparindent=2em,partopsep=0pt,parsep=0ex,topsep=0ex}
\setdescription{itemindent=3em,leftmargin=0pt,itemsep=0ex,listparindent=2em,partopsep=0pt,parsep=0ex,topsep=0ex}


\renewcommand\@biblabel[1]{[#1]\hspace{0.5em}} % 不去除参考文献里标号两边的括号
\newcommand{\ucite}[1]{$^{\mbox{\scriptsize \cite{#1}}}$} % 增加 \ucite 命令使显示的引用为上标形式
\newcommand{\citeup}[1]{$^{\mbox{\scriptsize \cite{#1}}}$} % for WinEdt users

% 定制浮动图形和表格标题样式
\renewcommand{\figurename}{图}
\renewcommand{\tablename}{表}
\captionnamefont{\wuhao}
\captiontitlefont{\wuhao}
\captiondelim{~~}
\renewcommand{\subcapsize}{\wuhao}
\setlength{\abovecaptionskip}{0pt}
\setlength{\belowcaptionskip}{0pt}
%将章标题中的中文数字(一、二、三)变为阿拉伯数字(1,2,3)

%%\renewcommand\CJKthechapter{{\@arabic\c@chapter}}

% 自定义项目列表标签及格式 \begin{nefulist} 列表项 \end{nefulist}
\newcounter{nefuctr} %自定义新计数器
\newenvironment{nefulist}{%%%%%定义新环境
\begin{list}{{(\arabic{nefuctr})}} %%标签格式
    {
     \usecounter{nefuctr}
     \setlength{\leftmargin}{0cm}     %左边界
     \setlength{\parsep}{0ex}         %段落间距
     \setlength{\topsep}{0pt}         %列表到上下文的垂直距离
     \setlength{\itemsep}{0ex}        %标签间距
     \setlength{\labelwidth}{0.5em}     %标号和列表项之间的距离,默认0.5em
     \setlength{\labelsep}{0.5em}     %标号和列表项之间的距离,默认0.5em
     \setlength{\itemindent}{3em}    % 标签缩进量
     \setlength{\listparindent}{2em} % 段落缩进量
    }}
{\end{list}}%%%%%

% 自定义项目列表标签及格式 \begin{publist} 列表项 \end{publist}
\newcounter{pubctr} %自定义新计数器
\newenvironment{publist}{%%%%%定义新环境
\begin{list}{\arabic{pubctr}} %%标签格式
    {
     \usecounter{pubctr}
     \setlength{\leftmargin}{1.5em}     % 左边界 \leftmargin =\itemindent + \labelwidth + \labelsep
     \setlength{\itemindent}{0em}     % 标号缩进量
     \setlength{\labelsep}{1em}       % 标号和列表项之间的距离,默认0.5em
     \setlength{\rightmargin}{0em}    % 右边界
     \setlength{\topsep}{0ex}         % 列表到上下文的垂直距离
     \setlength{\parsep}{0ex}         % 段落间距
     \setlength{\itemsep}{0ex}        % 标签间距
     \setlength{\listparindent}{0pt} % 段落缩进量
    }}
{\end{list}}%%%%%

% 默认字体
\renewcommand\normalsize{
  \@setfontsize\normalsize{12pt}{12pt}
  \setlength\abovedisplayskip{4pt}
  \setlength\abovedisplayshortskip{4pt}
  \setlength\belowdisplayskip{\abovedisplayskip}
  \setlength\belowdisplayshortskip{\abovedisplayshortskip}
  \let\@listi\@listI}
\def\defaultfont{\renewcommand{\baselinestretch}{1.57}\normalsize\selectfont}
\predisplaypenalty=0  %公式之前可以换页,公式出现在页面顶部

% 封面、摘要、版权、致谢格式定义
\def\ctitle#1{\def\@ctitle{#1}}\def\@ctitle{}
\def\cdegree#1{\def\@cdegree{#1}}\def\@cdegree{}
\def\caffil#1{\def\@caffil{#1}}\def\@caffil{}
\def\csubject#1{\def\@csubject{#1}}\def\@csubject{}
\def\cauthor#1{\def\@cauthor{#1}}\def\@cauthor{}
\def\csupervisor#1{\def\@csupervisor{#1}}\def\@csupervisor{}
\def\cchief#1{\def\@cchief{#1}}\def\@cchief{}
\def\creview#1{\def\@creview{#1}}\def\@creview{}


\def\cassosupervisor#1{\def\@cassosupervisor{~ & {\hei 副 \hfill 导 \hfill 师:} & #1\\}}\def\@cassosupervisor{}
\def\ccosupervisor#1{\def\@ccosupervisor{~ & {\hei 联 \hfill 合\hfill 导 \hfill 师:} & #1\\}}\def\@ccosupervisor{}
\def\cdate#1{\def\@cdate{#1}}\def\@cdate{}% 论文提交日期
\def\cdatequestion#1{\def\@cdatequestion{#1}}\def\@cdatequestion{}% 答辩日期
\def\cdatedegree#1{\def\@cdatedegree{#1}}\def\@cdatedegree{}% 授予学位日期

\long\def\cabstract#1{\long\def\@cabstract{#1}}\long\def\@cabstract{}
\def\ckeywords#1{\def\@ckeywords{#1}}\def\@ckeywords{}

\def\etitle#1{\def\@etitle{#1}}\def\@etitle{}
\def\edegree#1{\def\@edegree{#1}}\def\@edegree{}
\def\eaffil#1{\def\@eaffil{#1}}\def\@eaffil{}
\def\esubject#1{\def\@esubject{#1}}\def\@esubject{}
\def\eauthor#1{\def\@eauthor{#1}}\def\@eauthor{}
\def\esupervisor#1{\def\@esupervisor{#1}}\def\@esupervisor{}
%\def\eassosupervisor#1{\def\@eassosupervisor{#1}}\def\@eassosupervisor{}
\def\eassosupervisor#1{\def\@eassosupervisor{~ & \textbf{Associate Supervisor:} & #1\\}}\def\@eassosupervisor{}
%\def\ecosupervisor#1{\def\@ecosupervisor{#1}}\def\@ecosupervisor{}
\def\ecosupervisor#1{\def\@ecosupervisor{~ & \textbf{Co Supervisor:} & #1\\}}\def\@ecosupervisor{}
\def\edate#1{\def\@edate{#1}}\def\@edate{}
\def\edatequestion#1{\def\@edatequestion{#1}}\def\@edatequestion{}

\long\def\eabstract#1{\long\def\@eabstract{#1}}\long\def\@eabstract{}
\long\def\NotationList#1{\long\def\@NotationList{#1}}\long\def\@NotationList{}
\def\ekeywords#1{\def\@ekeywords{#1}}\def\@ekeywords{}
\def\natclassifiedindex#1{\def\@natclassifiedindex{#1}}\def\@natclassifiedindex{}
\def\internatclassifiedindex#1{\def\@internatclassifiedindex{#1}}\def\@internatclassifiedindex{}
\def\statesecrets#1{\def\@statesecrets{#1}}\def\@statesecrets{}

% 定义封面
\def\makecover{
      \begin{titlepage}
            % 封面一
            %\vspace*{0.3cm}
%            \begin{center}{
%                  \xiaoyi
%                  \begin{center}{
%                        \song\bfseries 硕士学位论文
%                  }\end{center}
%            }
%
%            \vspace{0.6cm}
%
%            \parbox[t][2.9cm][t]{\textwidth}{
%                  \erhao
%                  \begin{center}{
%                        \hei\@ctitle
%                  }\end{center}
%            }
%
%            \parbox[t][5.1cm][t]{\textwidth}{
%                  \erhao %英文标题太长时可以采用\xiaoer
%                  \begin{center}{
%                        \bfseries\@etitle
%                  }\end{center}
%            }
%
%            \parbox[t][8.2cm][t]{\textwidth}{
%                  \xiaoer
%                  \begin{center}{
%                        \song\bfseries  \@cauthor
%                  }\end{center}
%            }
%
%            \parbox[t][1cm][t]{\textwidth}{
%                  \begin{center}{
%                        \song \xiaoer\bfseries \@cdate
%                  }\end{center}
%            }\end{center}
%%%%%%%%%%%%%%%%%%%%%%%%%%%%%%%%%%%%%%%%%%%%%%%%%%%%%%%%%%%%%%%%%%%%%%%%%%%%%%%%%%%%%%%%%%%%%%%%%%%%%%%%%%%%%%%%%%%%%%%%%
            %内封
            \newpage
            \thispagestyle{empty}
                        \newlength{\widthtemp}
                        \settowidth{\widthtemp}{\song \sihao 学校代码:}
                        {\setlength{\parindent}{2.2em}
                        \parbox[t][17mm][t]{\textwidth}
                        {\begin{spacing}{1.25}
                              \hspace*{10.34cm}{\song \sihao 学校代码:10213}\song \xiaoer\\
                              \indent\hspace*{10.34cm}\makebox[\widthtemp][s]{\song \sihao 学\hfill 号:}\song \sihao  05001\song \xiaoer\\
                        \end{spacing}}}
\begin{center}
            \vspace{3.25mm}
            \parbox[t][3.3cm][t]{\textwidth}{\begin{center}{\xiaochu\xw 学\ \ 位\ \ 论\ \ 文}\end{center}}
            \parbox[t][3.4cm][t]{\textwidth}{\begin{center}\erhao\hei\@ctitle\end{center}}
            \parbox[t][3.75cm][t]{\textwidth}{\begin{center}\xiaoyi\kai\@cauthor\end{center}}
            \newlength{\lsspace}
            \setlength{\lsspace}{0.78cm}
      \begin{minipage}[t]{\textwidth}
            {\sihao%\hspace{0.56cm}
                              \song %\renewcommand{\arraystretch}{0.6}
                              \settowidth{\widthtemp}{\hei \sihao 授予学位日期}
                              \vspace{0.6em}\hspace{1ex}
                              \renewcommand{\arraystretch}{0}
\begin{tabular}{r@{:}lr@{:}l}
{\hei \rule{0pt}{\lsspace}指导教师姓名}&\@csupervisor&\multicolumn{2}{l}{东北林业大学}\\
{\hei \rule{0pt}{\lsspace}申请学位级别}&\@cdegree &\makebox[\widthtemp][s]{\hei 学\hfill 科\hfill 专\hfill 业}&\@csubject\\
{\hei \rule{0pt}{\lsspace}论文提交日期}&\@cdate  &{\hei 论文答辩日期}&\@cdatequestion  \\% \@cdegree\\
{\hei \rule{0pt}{\lsspace}授予学位单位}&东北林业大学&{\hei 授予学位日期}&\@cdatedegree \\
\multicolumn{4}{l}{\rule{0pt}{9mm}}
\end{tabular}}
      \end{minipage}
      {
      \settowidth{\widthtemp}{\hei\sihao 答辩委员会主席}
                                  \sihao\song
                                  \hspace*{5em}\begin{tabular}{c@{:}l}
                                    {\hei\sihao 答辩委员会主席} & {\@cchief}\\
                                    {\rule{0pt}{0.8\lsspace}\makebox[\widthtemp][s]{\hei\sihao 论\hfill\ 文\hfill\ 评\hfill\ 阅\hfill\ 人}} & {\@creview}\\
                                  \end{tabular}
      }
      \vspace{14mm}
                         \begin{center}
                       \includegraphics[height=1.53cm,width=7.00cm,bb=0 0 887 191]{nefupic/nefu_pic.png}
                       \end{center}
\end{center}
%\@cauthor
%%%%%%%%%%%%%%%%%%%%%%%%%%%%%%%%%%%%%%%%%%%%%%%%%%%%%%%%%%%%%%%%%%%%%%%%%%%%%%%%%%%%%%%%%%%%%%%%%%%%%%%%%%%%%%%%%%%%%%%%%
            % 英文封面
            \newpage
            \thispagestyle{empty}
            \vspace*{1.5mm}
            {\setlength{\parindent}{23.62em}
                        \parbox[t][29mm][t]{\textwidth}
                        {\begin{spacing}{1.3}
                              {\xiaosi University Code:10213}\song \xiaosan\\
                              {\indent\xiaosi Register Code :05001}
                        \end{spacing}}}
            \begin{center}
                  \parbox[t][41mm][t]{\textwidth}{\xiaoer
                        \begin{center} { Dissertation for the Degree of Master}\end{center}
                  } %与中文保持一致,删除in {\exueke}
                  \parbox[t][68mm][t]{\textwidth}{\erhao
                        \begin{center} {
                              \@etitle
                        }\end{center}
                  }
                  {
                        {
                              \sihao\noindent\hspace*{6.6mm}
                              \renewcommand{\arraystretch}{1.5}
                                    \begin{tabularx}{\textwidth}{@{}l@{~}X@{}}
                                          \textbf{Candidate:}                     &  \@eauthor\\
                                          \textbf{Supervisor:}                    &  \@esupervisor\\
                                          \textbf{Associate Supervisor:}          &  \@eassosupervisor\\
                                          \textbf{Academic Degree Applied for:}   &  \@edegree\\
                                          \textbf{Specialty:}                     &  \@esubject\\
                                          \textbf{Date of Oral Examination:}      &  \@edatequestion\\
                                          \textbf{University:} &  Northeast Forestry University
                                    \end{tabularx}
                              \renewcommand{\arraystretch}{1}

                        }
                  }
            \end{center}
      \end{titlepage}

      %%%%%%%%%%%%%%%%%%%   Abstract and keywords  %%%%%%%%%%%%%%%%%%%%%%%
      \clearpage
      \chapter*{摘要}
      \phantomsection
      \markboth{摘要}{摘要}
      %\vspace{1em}
      \addcontentsline{toc}{chapter}{\xiaosi 摘要}
      \addcontentsline{toe}{chapter}{\bfseries \xiaosi Abstract (In Chinese)}
      %\BiAppendixChapter{摘要}{}
      \setcounter{page}{1}
      \song\defaultfont
      \@cabstract
      \vspace{\baselineskip}

      \hangafter=1\hangindent=52.3pt\noindent
      {\hei 关键词:} \@ckeywords

      %%%%%%%%%%%%%%%%%%%   English Abstract  %%%%%%%%%%%%%%%%%%%%%%%%%%%%%%
      \clearpage
      \BiAppendixChapter{\textbf\timesnewroman{Abstract}}{Abstract (In English)} % 不要挪到下一行,生成正确的摘要toc
      \@eabstract
      \vspace{\baselineskip}

      \hangafter=1\hangindent=60pt\noindent
      {\bfseries{Keywords:}}  \@ekeywords

}

%%%%%%%%%%%%%%%%%%%%%%%%%%%%%%%%%%%%%%%%%%%%%%%%%%%%%%%%%%%%%%%
% 英文目录格式
\def\@dotsep{0.75}           % 定义英文目录的点间距
\setlength\leftmargini {0pt}
\setlength\leftmarginii {0pt}
\setlength\leftmarginiii {0pt}
\setlength\leftmarginiv {0pt}
\setlength\leftmarginv {0pt}
\setlength\leftmarginvi {0pt}

\def\engcontentsname{\bfseries Contents}
\newcommand\tableofengcontents{
   \pdfbookmark[0]{Contents}{econtent}
     \@restonecolfalse
   \chapter*{\engcontentsname  %chapter*上移一行,避免在toc中出现。
       \@mkboth{%
          \engcontentsname}{\engcontentsname}}
   \@starttoc{toe}%
   \if@restonecol\twocolumn\fi
}

\urlstyle{same}  %论文中引用的网址的字体默认与正文中字体不一致,这里修正为一致的。

\renewcommand\endtable{\vspace{-4mm}\end@float}

\makeatother
