\chapter{使い方}

\section{図と表}

普通に使えるよね、例えば\autoref{fig:logo}と\autoref{tab:city}。

\begin{figure}[H]
  \includegraphics[width=3cm]{nuaa-logo.pdf}
  \caption[バッジ]{学校のバッジ\label{fig:logo}}
\end{figure}

\begin{table}[htb]
  \caption[人数の表]{人数の表 (source: Wikipedia)\label{tab:city}}
  \begin{tabular}{lr}
    \toprule
    都市 & 人数 \\
    \midrule
    メキシコ & 20,116,842\\
    上海 & 19,210,000\\
    北京 & 15,796,450\\
    イスタンブール & 14,160,467\\
    \bottomrule
  \end{tabular}
\end{table}

\section{なぜ \texttt{\textbackslash zhcn} が必要ですか}

\zhcn{因为同一个汉字在两种语言中的写法不一样,如}\autoref{fig:compare}。

\begin{figure}[H]
  \subfloat[日語字]{\fontsize{60bp}{60bp}\selectfont 字}\hfil
  \subfloat[\zhcn{中国字}]{\fontsize{60bp}{60bp}\selectfont\zhcn{字}}\quad
  \caption{字$\times 2$\label{fig:compare}}
\end{figure}

\zhcn{并且,如果使用 IPA 之类字符数量不多的字体写中文的话,有可能会遇到中文独有的汉字,
导致最后的 pdf 里出现无法显示汉字的框。
所以一定要根据的语言,使用对应的字体。}

\section{参考文献について}

\zhcn{目前还无法在 biber 的参考文献中指定语言/字体…在这里偷个懒,使用英文示例中的例子:}

Cite one paper\cite{r1}, or multiple\cite{r2,r3,r4}.

Here is inline cited paper\inlinecite{r6}, and another paper\inlinecite{r7,r8,r9}.

\zhcn{理论上基本能处理日文的参考文献数据库的,但笔者实在有点偷懒,}えへへ~
