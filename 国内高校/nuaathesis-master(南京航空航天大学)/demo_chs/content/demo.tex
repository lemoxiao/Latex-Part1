\chapter{使用示例}

本章介绍一些常用的宏包的常用方法,希望能为读者写作时提供参考。

\section{插图}

首先讨论一下插图的格式,在 \LaTeX{} 环境下,
\begin{enumerate}
\item 推荐使用宏包来绘制插图,如 \pkg{tikz},它兼容所有 \LaTeX{} 环境,
字体能与全文统一,质量最佳,但是需要的学习成本较大。
请务必先阅读 \pkg{tikz} 文档的第1章教程,
然后可以去 texample\footnote{\url{http://texample.net/tikz}} 等网站上找类似的例子,
也可以使用 GeoGebra\footnote{\url{https://www.geogebra.org}} 之类的工具来生成\TeX 代码,
效果可以参见\autoref{fig:tikzrot};
\item 其次推荐使用其他绘图工具生成的 \verb|PDF|、 \verb|EPS| 格式的矢量图,
\verb|svg| 格式可以通过 inkscape 软件转换成带 \TeX{}文本代码的 \verb|PDF|。效果可以参见\autoref{fig:logo};
\item 当然,\verb|PNG|、 \verb|jpeg| 之类的位图格式也能做插图;
\item 最后,不要忘记论文是\textbf{单色印刷}的,请确保插图在黑白打印的情况下的清晰度。
\end{enumerate}

\begin{figure}[htb]
  \newcounter{density}
  \setcounter{density}{20}
  \begin{tikzpicture}
  \newcounter{density}
  \setcounter{density}{20}
  \def\couleur{Dandelion}
  \path[coordinate] (0,0) coordinate(A)
              ++( 90:4cm) coordinate(B)
              ++(0:4cm) coordinate(C)
              ++(-90:4cm) coordinate(D);
  \draw (A) node[left] {A}
    (B) node[left] {B}
    (C) node[right] {C}
    (D) node[right] {D};
  \draw[fill=\couleur!\thedensity] (A)--(B)--(C)--(D)--cycle;
  \foreach \x in {1,...,40}{%
      \pgfmathsetcounter{density}{\thedensity+25}
      \setcounter{density}{\thedensity}
      \path[coordinate] coordinate(X) at (A){};
      \path[coordinate] (A) -- (B) coordinate[pos=.1](A)
                          -- (C) coordinate[pos=.1](B)
                          -- (D) coordinate[pos=.1](C)
                          -- (X) coordinate[pos=.1](D);
      \draw[fill=\couleur!\thedensity] (A)--(B)--(C)--(D)--cycle;
  }
\end{tikzpicture}

  \caption{tikz例子}
  \label{fig:tikzrot}
\end{figure}

\begin{figure}[htb]
  \includegraphics[width=4cm]{nuaa-logo.pdf}
  \caption{一个校徽}
  \label{fig:logo}
\end{figure}

如果需要多个插图共用一个题注的话,需要加载额外的宏包,
一般选用 \pkg{subcaption} 或 \pkg{subfig},这两个宏包是互斥的。
需要注意的是 \pkg{subcaption} 貌似与 \pkg{geometry} 有些冲突,
会导致多行的图表的最后一行无法居中,而 \pkg{geometry} 是设置页边距的必用宏包。
所以个人推荐使用  \pkg{subfig},效果可以参考\autoref{fig:sub2}。

\begin{figure}[htb]
  \subfloat[左边的大校徽\label{fig:sub1}]{\includegraphics[width=4cm]{nuaa-logo.pdf}}\quad
  \subfloat[短标题:小校徽][小校徽,题注很长,不过请各位放心,它会自动换行\label{fig:sub2}]
  {\includegraphics[width=3cm]{nuaa-logo.pdf}}
  \caption{包含两张图片的插图}
  \label{fig:subfigs}
\end{figure}

如果需要插入图表的话,可以考虑使用 \pkg{pgfplots} 宏包,效果参见\autoref{fig:plots};
也可以用 Matplotlib、MatLab、Mathematica 之类的工具导出成兼容格式的图片。

\begin{figure}[htb]
  \subfloat[二维图像\label{fig:func}]{%\documentclass{ctexart}
%\usepackage{pgfplots}
%\pgfplotsset{compat=1.16}
%\begin{document}
\begin{tikzpicture}
  \pgfplotstableread[
    % col sep=comma
  ]{data/plot_2d.csv}{\Data}
  \begin{axis}[
    width=.45\textwidth,
    xmin=0, xmax=16, xtick distance=4,
    xlabel={序列},
    ymin=0, ymax=1,
    ylabel={正确率},
    grid=both,
    legend pos=south east,
  ]
    \addplot+ table[x=idx, y=parray] {\Data};
    \addlegendentry{环境1};
    \addplot+[mark=o] table[x=idx, y=pround] {\Data};
    \addlegendentry{环境2};
  \end{axis}
\end{tikzpicture}
%\end{document}
} \quad
  \subfloat[三维图像\label{fig:sum}]{%\documentclass{minimal}
%\usepackage{pgfplots}
%\pgfplotsset{compat=1.16}
%\begin{document}
\begin{tikzpicture}
  \pgfplotstableread[
    % col sep=comma
  ]{data/plot_3d.csv}{\Data}
  \begin{axis}[
    width=.45\textwidth,
    view={-30}{30},
    xmin=0, xmax=16, xtick distance=4,
    xlabel={Num},
    ymin=1, ymax=20, ytick distance=5,
    ylabel={Round},
    zmin=0, zmax=1, ztick distance=.2,
    zlabel={PDF},
    z tick label style={
      /pgf/number format/.cd,
        fixed,
        fixed zerofill,
        precision=1,
      /tikz/.cd
      },
    grid=major,
  ]
    \addplot3[
      surf,
      mesh/rows=17,
      patch type=rectangle,
      opacity=1,fill opacity=0.1,
      colormap/cool
    ] table[x=num, y=round, z=p] {\Data};
  \end{axis}
\end{tikzpicture}
%\end{document}
}
  \caption{拙作中利用 \pkg{pgfplot} 绘制的图表}
  \label{fig:plots}
\end{figure}

如果真的需要让十几张图片共用一个题注的话,
需要手工拆分成多个 \env{float} 并用 \cs{ContinuedFloat} 来拼接,
不过直接多次使用 \cs{caption} 会在图表清单里产生多个重复条目,需要一点点小技巧
(设置图表目录标题为空)。
建议将浮动位置指定为 \verb|t|,以确保分散至多页的图能占用整个页面,手工分页才能靠谱。
效果可以参见\autoref{fig:subfigss} 的\autoref{fig:logo6}。

\begin{figure}[t]
  \subfloat[校徽$\times 1$]{\includegraphics[width=4cm]{nuaa-logo.pdf}}\quad
  \subfloat[校徽$\times 2$]{\includegraphics[width=.4\textwidth]{nuaa-logo.pdf}}\\
  \subfloat[校徽$\times 3$]{\includegraphics[width=.4\textwidth]{nuaa-logo.pdf}}\quad
  \subfloat[校徽$\times 4$]{\includegraphics[width=4cm]{nuaa-logo.pdf}}
  \caption{包含多张图片的插图}
  \label{fig:subfigss}
\end{figure}
\begin{figure}[t]
  \ContinuedFloat
  \subfloat[校徽$\times 5$]{\includegraphics[width=4cm]{nuaa-logo.pdf}}\quad
  \subfloat[校徽$\times 6$ \label{fig:logo6}]{\includegraphics[width=4cm]{nuaa-logo.pdf}}\\
  \subfloat[校徽$\times 7$]{\includegraphics[width=4cm]{nuaa-logo.pdf}}\quad
  \subfloat[校徽$\times 8$]{\includegraphics[width=4cm]{nuaa-logo.pdf}}
  % 指定图表清单中的标题为[],即可将其消除,避免目录中出现重复条目
  \caption[]{包含多张图片的插图(续)}
\end{figure}

如果需要插入一张很大的图片的话,可以使用 \pkg{rotating} 提供的 \env{sidewaysfigure},
它能将插图放置在单独的页面上,如果文档使用 \verb|twoside| 选项的话,它会根据页面方向,
设置 \ang{90} 或 \ang{270} 旋转,可能需要编译两遍才能设置正确的旋转方向。
不过可能有一个问题,\env{sidewaysfigure} 中使用 \cs{subfloat} 可能无法准确标号,
需要手工重置 \texttt{subfigure} 计数器。
效果参见\autoref{fig:fullpage1} 和\autoref{fig:fullpage2}。

\setcounter{subfigure}{0}
\begin{sidewaysfigure}
  \subfloat[First caption\label{fig:fp1}]{\includegraphics[width=.8\textheight]{nuaa-jianqi.pdf}} \\
  \subfloat[Second caption]{\includegraphics[height=2cm]{nuaa-jianqi.pdf}}
  \caption{一幅占用完整页面的图片}
  \label{fig:fullpage1}
\end{sidewaysfigure}

\setcounter{subfigure}{0}
\begin{sidewaysfigure}
  \subfloat[First caption]{\includegraphics[height=2cm]{nuaa-jianqi.pdf}} \\
  \subfloat[Second caption]{\includegraphics[width=.8\textheight]{nuaa-jianqi.pdf}}
  \caption{又一幅占用完整页面的图片}
  \label{fig:fullpage2}
\end{sidewaysfigure}

\section{表格}

由于封面页,本模板预先加载了 \pkg{array} 和 \pkg{tabu},如果需要其他表格的宏包,
请自行加载。

如果需要插入一个简易的表格,可以只使用 \env{tabular} 环境,如\autoref{tab:city}。
\begin{table}[htb]
  \caption[城市人口]{城市人口数量排名 (source: Wikipedia)\label{tab:city}}
  \begin{tabular}{lr}
    \toprule
    城市 & 人口 \\
    \midrule
    Mexico City & 20,116,842\\
    Shanghai & 19,210,000\\
    Peking & 15,796,450\\
    Istanbul & 14,160,467\\
    \bottomrule
  \end{tabular}
\end{table}

也可以使用 \env{tabu} 环境,它可以更灵活地设置列宽,但它有一些 bug,如\autoref{tab:tabu}。
\begin{table}[htb]
  \caption{\env{tabu} 注意事项 \label{tab:tabu}}
  \begin{tabu} to .9\textwidth {XX[2]<{\strut}} \toprule
    默认列 & 有修正的列 \\ \midrule
    \env{tabu} 的 bug? \par This line is BAD & 注意左侧最后一行后的垂直空格 \\ \midrule
    注意对比最后一行 &
      bug 会影响多行的 \env{tabu} 表格 \par
      bug 的修正方法是在段落后面加 \cs{strut} \par
      This line is Good \\ \midrule
    垂直居中没效果 & 改用 \env{tabular} \\ \midrule
    与新版 \pkg{array} 不兼容 & 谨慎使用,切勿用 \texttt{tabu spread} \\ \bottomrule
  \end{tabu}
\end{table}

如果需要对某一列的小数点对齐,或者带有单位,或者需要做四舍五入的处理,可以尝试配合 \pkg{siunitx} 一起使用。
非常推荐看一下 \pkg{siunitx} 文档的,至少看一下“Hints for using siunitx”一节的输出结果,
\autoref{tab:xmpl:mixed} 来自于该文档的 7.14 节。

\begin{table}[htb]
  \caption{Tables where numbers have different units}
  \label{tab:xmpl:mixed}
  \begin{tabular}
    {
      >{$}l<{$}
      S[table-format = 2.3(1)]
      S[table-format = 3.3(1)]
    }
    \toprule
      & {One} & {Two} \\
    \midrule
    a / \si{\angstrom}   &  1.234(2) &   5.678(4) \\
    \beta / \si{\degree} & 90.34(4)  & 104.45(5)  \\
    \mu / \si{\per\mm}   &  0.532    &   0.894    \\
    \bottomrule
  \end{tabular}
  \hfil
  \begin{tabular}
    {S[table-format=1.3]@{\,}s[table-unit-alignment = left]}
    \toprule
    \multicolumn{2}{c}{Heading} \\
    \midrule
    1.234 & \metre   \\
    0.835 & \candela \\
    4.23  & \joule\per\mole \\
    \bottomrule
  \end{tabular}
\end{table}

如果表格内容很多,导致无法放在一页内的话,需要用 \env{longtable} 或 \env{longtabu} 进行分页。
\autoref{tab:performance} 是来自 \cquthesis{} 的一个长表格的例子。

\begin{longtable}[c]{c*{6}{r}}
	\caption[实验数据]{实验数据,这个题注十分的长,注意这在索引中的处理方式,还有 \cs{caption} 后面的双反斜杠}\label{tab:performance}\\
	\toprule
	\multirow{2}{*}{测试程序} & \multicolumn{1}{c}{正常运行} & \multicolumn{1}{c}{同步} & \multicolumn{1}{c}{检查点} & \multicolumn{1}{c}{卷回恢复}
	& \multicolumn{1}{c}{进程迁移} & \multicolumn{1}{c}{检查点} \\
	& \multicolumn{1}{c}{时间 (s)}& \multicolumn{1}{c}{时间 (s)}&
	\multicolumn{1}{c}{时间 (s)}& \multicolumn{1}{c}{时间 (s)}& \multicolumn{1}{c}{时间 (s)}& \multicolumn{1}{c}{文件 (KB)} \\ \midrule
	\endfirsthead
	\multicolumn{7}{c}{\nuaafontcaption 续表~\thetable\hskip1em 实验数据}\\
	\toprule
	\multirow{2}{*}{测试程序} & \multicolumn{1}{c}{正常运行} & \multicolumn{1}{c}{同步} & \multicolumn{1}{c}{检查点} & \multicolumn{1}{c}{卷回恢复}
	& \multicolumn{1}{c}{进程迁移} & \multicolumn{1}{c}{检查点} \\
	& \multicolumn{1}{c}{时间 (s)}& \multicolumn{1}{c}{时间 (s)}&
	\multicolumn{1}{c}{时间 (s)}& \multicolumn{1}{c}{时间 (s)}& \multicolumn{1}{c}{时间 (s)}& \multicolumn{1}{c}{文件(KB)} \\ \midrule
	\endhead
	\hline
	\multicolumn{7}{r}{续下页}
	\endfoot
	\endlastfoot
	CG.A.2 & 23.05 & 0.002 & 0.116 & 0.035 & 0.589 & 32491 \\
	CG.A.4 & 15.06 & 0.003 & 0.067 & 0.021 & 0.351 & 18211 \\
	CG.A.8 & 13.38 & 0.004 & 0.072 & 0.023 & 0.210 & 9890 \\
	CG.B.2 & 867.45 & 0.002 & 0.864 & 0.232 & 3.256 & 228562 \\
	CG.B.4 & 501.61 & 0.003 & 0.438 & 0.136 & 2.075 & 123862 \\
	CG.B.8 & 384.65 & 0.004 & 0.457 & 0.108 & 1.235 & 63777 \\
	MG.A.2 & 112.27 & 0.002 & 0.846 & 0.237 & 3.930 & 236473 \\
	MG.A.4 & 59.84 & 0.003 & 0.442 & 0.128 & 2.070 & 123875 \\
	MG.A.8 & 31.38 & 0.003 & 0.476 & 0.114 & 1.041 & 60627 \\
	MG.B.2 & 526.28 & 0.002 & 0.821 & 0.238 & 4.176 & 236635 \\
	MG.B.4 & 280.11 & 0.003 & 0.432 & 0.130 & 1.706 & 123793 \\
	MG.B.8 & 148.29 & 0.003 & 0.442 & 0.116 & 0.893 & 60600 \\
	LU.A.2 & 2116.54 & 0.002 & 0.110 & 0.030 & 0.532 & 28754 \\
	LU.A.4 & 1102.50 & 0.002 & 0.069 & 0.017 & 0.255 & 14915 \\
	LU.A.8 & 574.47 & 0.003 & 0.067 & 0.016 & 0.192 & 8655 \\
	LU.B.2 & 9712.87 & 0.002 & 0.357 & 0.104 & 1.734 & 101975 \\
	LU.B.4 & 4757.80 & 0.003 & 0.190 & 0.056 & 0.808 & 53522 \\
	LU.B.8 & 2444.05 & 0.004 & 0.222 & 0.057 & 0.548 & 30134 \\
	CG.B.2 & 867.45 & 0.002 & 0.864 & 0.232 & 3.256 & 228562 \\
	CG.B.4 & 501.61 & 0.003 & 0.438 & 0.136 & 2.075 & 123862 \\
	CG.B.8 & 384.65 & 0.004 & 0.457 & 0.108 & 1.235 & 63777 \\
	MG.A.2 & 112.27 & 0.002 & 0.846 & 0.237 & 3.930 & 236473 \\
	MG.A.4 & 59.84 & 0.003 & 0.442 & 0.128 & 2.070 & 123875 \\
	MG.A.8 & 31.38 & 0.003 & 0.476 & 0.114 & 1.041 & 60627 \\
	MG.B.2 & 526.28 & 0.002 & 0.821 & 0.238 & 4.176 & 236635 \\
	MG.B.4 & 280.11 & 0.003 & 0.432 & 0.130 & 1.706 & 123793 \\
	MG.B.8 & 148.29 & 0.003 & 0.442 & 0.116 & 0.893 & 60600 \\
	LU.A.2 & 2116.54 & 0.002 & 0.110 & 0.030 & 0.532 & 28754 \\
	LU.A.4 & 1102.50 & 0.002 & 0.069 & 0.017 & 0.255 & 14915 \\
	LU.A.8 & 574.47 & 0.003 & 0.067 & 0.016 & 0.192 & 8655 \\
	LU.B.2 & 9712.87 & 0.002 & 0.357 & 0.104 & 1.734 & 101975 \\
	LU.B.4 & 4757.80 & 0.003 & 0.190 & 0.056 & 0.808 & 53522 \\
	LU.B.8 & 2444.05 & 0.004 & 0.222 & 0.057 & 0.548 & 30134 \\
	EP.A.2 & 123.81 & 0.002 & 0.010 & 0.003 & 0.074 & 1834 \\
	EP.A.4 & 61.92 & 0.003 & 0.011 & 0.004 & 0.073 & 1743 \\
	EP.A.8 & 31.06 & 0.004 & 0.017 & 0.005 & 0.073 & 1661 \\
	EP.B.2 & 495.49 & 0.001 & 0.009 & 0.003 & 0.196 & 2011 \\
	EP.B.4 & 247.69 & 0.002 & 0.012 & 0.004 & 0.122 & 1663 \\
	EP.B.8 & 126.74 & 0.003 & 0.017 & 0.005 & 0.083 & 1656 \\
	\bottomrule
\end{longtable}

\section{数字与国际单位}

本模板预加载 \pkg{siunitx} 来格式化文中的内联数字,该宏包有大量可定制的参数,
请务必阅读其文档,并在文档导言部分设置格式。

\begin{itemize}
  \item 旋转角度为 \ang{90}、\ang{270}
  \item 分辨率 \num{1920x1080} 的像素数量约为 \num{2.07e6}
  \item 电脑显示器的像素间距为 \SI{1.8}{\nm}、\SI{180}{\um} 还是 \SI{18}{\mm}?
  \item 重力加速度 $g=\SI{9.8}{\kg\per\square\second}$、
  $g=\SI[inter-unit-product=\ensuremath{{}\cdot{}}]{9.8}{\kg\per\square\second}$,
  亦或 $g=\SI[per-mode=symbol]{9.8}{\kg\per\square\second}$
\end{itemize}

\section{中英文之间空格}

很遗憾,目前 \LaTeX{} 和 \CTeX{} 虽然能处理普通汉字与英文之间的间隔,
但是汉字与宏之间的空格仍然需要手工调整,请务必按以下的规则撰写原稿:
\begin{itemize}
  \item[\ding{51}] 如\autoref{fig:sub2} 所示:\verb|如\autoref{fig:sub2} 所示|,这个宏返回的是“图 x.xx”,
  所以前面两个汉字之间不能加空格,后面数字与汉字之间必须加空格;
  \item[\ding{51}] 距离为 1.7~个天文单位:\verb|距离为 1.7~个天文单位|,前面可以不加空格(\CTeX 会修正),
  后面必须加 \verb|~| 以防止在 “1.7”与“个”之间换行。此时更推荐写成 \SI{1.7}{au}:\verb|\SI{1.7}{au}|。
\end{itemize}
