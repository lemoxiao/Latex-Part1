% !Mode:: "TeX:UTF-8"
\begin{abstract}
摘要是论文内容的高度概括,应具有独立性和自含性,即不阅读论文的全文,就能通过摘要了解整个论文的必要信息。摘要应包括本论文研究的目的、理论与实际意义、主要研究内容、研究方法等,重点突出研究成果和结论。

摘要的内容要完整、客观、准确,应做到不遗漏、不拔高、不添加。摘要应按层次逐段简要写出,避免将摘要写成目录式的内容介绍。摘要在叙述研究内容、研究方法和主要结论时,除作者的价值和经验判断可以使用第一人称外,一般使用第三人称,采用"分析了……原因"、"认为……"、"对……进行了探讨"等记述方法进行描述。避免主观性的评价意见,避免对背景、目的、意义、概念和一般性(常识性)理论叙述过多。

摘要需采用规范的名词术语(包括地名、机构名和人名)。对个别新术语或无中文译文的术语,可用外文或在中文译文后加括号注明外文。摘要中不宜使用公式、化学结构式、图表、非常用的缩写词和非公知公用的符号与术语,不标注引用文献编号。

摘要的字数(以汉字计),硕士学位论文一般为500-650字,博士学位论文为900-1200字,均以能将规定内容阐述清楚为原则,文字要精练,段落衔接要流畅。摘要页不需写出论文题目。

英文摘要与中文摘要的内容应完全一致,在语法、用词上应准确无误,语言简练通顺。中文摘要在前,英文摘要在后。

\end{abstract}

\begin{keywords}
光子晶体光纤;色散;双折射;遗传算法;有限元法;太赫兹
\end{keywords}

\cleardoublepage

\begin{englishabstract}
As a novel kind of optical waveguide, Photonic Crystal Fiber (PCF) has significant advantages of orders of magnitudes than original optical fibers. For instance flexible dispersion, extreme large or small mode area and high birefringence. Therefore PCF
can replace the traditional optical fibers with great advantages of  performances. The design and fabrication also become essential in modern optics.

Dispersion plays an crucial role in optical communication network and nonlinear optics. ...
\end{englishabstract}

\begin{englishkeywords}
Photonic crystal fiber; Dispersion; Birefringence; Genetic Algorithm; Finite Element Method; Terahertz
\end{englishkeywords} 

\cleardoublepage