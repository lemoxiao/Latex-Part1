% !Mode:: "TeX:UTF-8"
\chapter{公式}
\label{chap:equ}
本章介绍基本公式的输入方法;矩阵和向量的输入;方程组的输入;多行公式的换行与对齐。
\LaTeX 中数学公式的输入依赖于数学环境。

在正文中用到的简短公式,可以直接使用两个美元符号“\$”括起来,如:
\begin{verbatim}
直角三角形三边长度满足关系式$a^2+b^2=c^2$
\end{verbatim}
得到的结果是:\\
直角三角形三边长度满足关系式$a^2+b^2=c^2$\\

而对于一些较为重要或者较复杂、需要编号的公式,则需要使用各种数学环境,例如使用equation环境:
\begin{verbatim}
\begin{equation}\label{chp-mode}
\mathbfit{E}=\mathrm{Re}(\mathbfit{E}(\mathbfit{r}))e^{j\omega t}
\end{equation}
\end{verbatim}
得到的结果是:
\begin{equation}\label{chp-mode}
\mathbfit{E}=\mathrm{Re}(\mathbfit{E}(\mathbfit{r}))e^{j\omega t}
\end{equation}
对它的引用方式为:\verb|公式\eqref{chp-mode}|\\
得到的结果为:公式\eqref{chp-mode}

如果不想对公式进行编号,则可以使用equation*环境:
\begin{verbatim}
\begin{equation*}\label{chp-m2}
\mathbfit{E}=\mathrm{Re}(\mathbfit{E}(\mathbfit{r}))e^{j\omega t}
\end{equation*}
\end{verbatim}
得到的结果是:
\begin{equation*}\label{chp-m2}
\mathbfit{E}=\mathrm{Re}(\mathbfit{E}(\mathbfit{r}))e^{j\omega t}
\end{equation*}

\section{上下标}
\verb|a_1+b^2\times c_1^2=0|输出结果为:$a_1+b^2\times c_1^2=0$

\section{分式}
命令\verb|\frac, \dfrac\ tfrac|可以用来输出分式:
\begin{verbatim}
\begin{equation}\label{fr}
  \sin\dfrac{\cos\dfrac{a}{b}}{c}=
  \sin\frac{\cos\frac{a}{b}}{c}=
  \sin\tfrac{\cos\tfrac{a}{b}}{c}
\end{equation}
\end{verbatim}
输出的结果是:
\begin{equation}\label{fr}
\sin\dfrac{\cos\dfrac{a}{b}}{c}=
\sin\frac{\cos\frac{a}{b}}{c}=
\sin\tfrac{\cos\tfrac{a}{b}}{c}
\end{equation}

当使用括号来括起纵向尺寸较大的对象例如分式时,要使用\verb|\left| 和
\verb|\right| 命令使括号在纵向上伸长。例如:
\begin{verbatim}
\begin{equation}\label{frr}
  \left(\frac{a}{b}\right)=(\frac{a}{b})
\end{equation}
\end{verbatim}
的输出结果是:
\begin{equation}\label{frr}
\left(\frac{a}{b}\right)=(\frac{a}{b})
\end{equation}

\section{矢量点乘与叉乘}
矢量点乘:\verb|$\mathbfit{A}\cdot\mathbfit{B}$|输出:$\mathbfit{A}\cdot\mathbfit{B}$

矢量叉乘:\verb|$\mathbfit{C}\times\mathbfit{D}$|输出:$\mathbfit{C}\times\mathbfit{D}$

\section{求和与积分}
命令\verb|\sum|和命令\verb|\int|负责输出求和与积分号。例如:
\begin{verbatim}
\begin{equation}\label{equ-sum}
  \sum_{i=1}^n\sin\beta_i^2=0
\end{equation}
\end{verbatim}
输出结果为:
\begin{equation}\label{equ-sum}
\sum_{i=1}^n\sin\beta_i^2=0
\end{equation}
\begin{verbatim}
\begin{equation}\label{equ-int}
  \int_a^b\frac{c}{d}\,\mathrm{d}x=0
\end{equation}
\end{verbatim}
输出结果为:
\begin{equation}\label{equ-int}
  \int_a^b\frac{c}{d}\,\mathrm{d}x=0
\end{equation}


\section{矩阵与数组}
矩阵与数组使用array环境:
\begin{verbatim}
\begin{equation}\label{equ-array}
  \left(
    \begin{array}{c} a \\ c \end{array}
  \right)=
  \left(
    \begin{array}{cc} a & b \\ c & d \end{array}
  \right)
\end{equation}
\end{verbatim}
输出结果是:
\begin{equation}\label{equ-array}
  \left(
  \begin{array}{c} a \\ c \end{array}
  \right)=
  \left(
  \begin{array}{cc} a & b \\ c & d \end{array}
  \right)
\end{equation}
也可以使用matrix环境:
\begin{verbatim}
\begin{equation}\label{equ-matrix}
  \begin{matrix} 0 & 1 \\ 1 & 0\end{matrix}=
  \begin{pmatrix}0 &-i \\ i & 0\end{pmatrix}=
  \begin{bmatrix}1 & 0 \\ 0 &-1\end{bmatrix}=
  \begin{vmatrix}a & b \\ c & d\end{vmatrix}
\end{equation}
\end{verbatim}
输出结果是:
\begin{equation}\label{equ-matrix}
  \begin{matrix} 0 & 1 \\ 1 & 0\end{matrix}=
  \begin{pmatrix}0 &-i \\ i & 0\end{pmatrix}=
  \begin{bmatrix}1 & 0 \\ 0 &-1\end{bmatrix}=
  \begin{vmatrix}a & b \\ c & d\end{vmatrix}
\end{equation}

\section{多行公式与对齐方法}
多行公式排列,每个公式都有自己的编号通常使用align环境。例如:
\begin{verbatim}
\begin{align}
  a_1+a_2+a_3 &=0 \label{equ-s1}\\
  b_1+b_2+b_3+b_4 &=0 \label{equ-s2}\\
  c_1+c_2 &=0 \label{equ-v1}
\end{align}
\end{verbatim}
输出结果为:
\begin{align}
  a_1+a_2+a_3 &=0 \label{equ-s1}\\
  b_1+b_2+b_3+b_4 &=0 \label{equ-s2}\\
  c_1+c_2 &=0 \label{equ-v1}
\end{align}
其中符号“\&”为对齐符号。这里实现了等号对齐。

\section{带有大括号的方程组}
与多行公式不同,方程组左侧使用“\verb|\left{|”加了一个大括号,另外只有一个公式编号,因此采用equation和aligned结合的方式,例如:
\begin{verbatim}
\begin{equation}\label{equ-fml}
  \left\{
  \begin{aligned}
    x^2+y^2 &=0\\
    x+y+z^2 &=0\\
    x^2+y+z &=0
  \end{aligned}
  \right.
\end{equation}
\end{verbatim}
输出结果为:
\begin{equation}\label{equ-fml}
  \left\{
  \begin{aligned}
    x^2+y^2 &=0\\
    x+y+z^2 &=0\\
    x^2+y+z &=0
  \end{aligned}
  \right.
\end{equation}

