\chapter{绪论}
本章介绍了无线传感网的结构及其应用领域。分析规模化无线传感网的特点,对规模化传感网数据认证的需求和面临的挑战进行了分析。概述了本文的研究内容,并对文章的组织结构予以说明。
\section{本文研究背景和意义}
无线传感器网络(Wireless Sensor Networks,无线传感网)是一种特殊组织结构的移动自组网\upcite{c:sensor},在环境监测、工业控制、资源监控、智能家居、医疗保健和军事等各种领域都有广泛应用,有非常重要的地位和作用。
随着无线传感网技术的不断发展,很多应用进入了日常生活中,物联网技术也将成为未来发展的重要方向。
无线传感网的各种技术发展紧跟具体的应用需求,随着各种应用场景需要的安全性越来越高,安全问题也成为了阻碍无线传感网大规模发展的一个制约。

大范围监测在环境监测和军事侦察等诸多关系国家社会重大安全的领域都具有重要的地位和作用。在环境监测领域,往往面临范围野外受限甚至恶劣条件,在海洋等资源监测领域,水声通信等基础技术还不是很完善, 在军事侦察对抗领域更是要应对破坏攻击情况,传统的大范围实时监测机制和系统都难以得到有效部署,使用无线传感技术成为了最好的解决方案。规模化无线传感网因此应运而生,而且为满足大范围监测的需要,无线传感网的规模越来越大。

规模化无线传感网面临的安全威胁更多,攻击的影响更大,而且由于传感器节点的特点,传统的安全机制和协议无法直接适用于无线传感网,使得安全问题更加凸显。因此针对规模化无线传感网安全机制的研究成为了热门研究方向。

\subsection{无线传感网概述}
\subsubsection{无线传感网结构}
无线传感器节点被部署在目标监测区域,大量的传感器节点通过无线广播的方式,以一定的算法自组织成为一个多跳的无线网络。如图~\ref{fig:cluster}所示,是一个典型无线传感网的结构\upcite{c:cluster},由三部分组成:监测区域的传感器节点、与外部网络连接的网关或基站、远程数据中心。在监测区域的传感器节点一般通过算法组成若干的簇,每个簇通过簇头节点与其他簇或者基站通信,这样的方案节约了节点的能量。簇内节点收集到监测数据以后通过簇头节点的整合,形成报文通过一定的路径发送给基站,基站进一步通过外部网络设备,如互联网、卫星等将监测数据传输到远程数据中心。

\begin{figure}[htbp]
  \centering
  \includegraphics[width=5in]{cluster}
  \caption{无线传感网系统结构}
  \label{fig:cluster}
\end{figure}


无线传感网中,基站的计算和存储能力都比较强,
基站的功能可以是一个数据处理中心,向网络广播控制信息,从监测区域获取数据。
也可以是一个网络网关,负责数据向远程数据中心的传输。

\subsubsection{无线传感器节点结构}
传感器节点是无线传感网的基本组成单元,负责数据采集、发送等基本功能。
无线传感器节点一般仅具有很小的存储空间,较弱的计算能力,因此单个节点无法完成复杂的感知任务,需要大量的节点协同工作。

随着电子技术的发展,无线传感器节点的性能也有了很大的提升,如Crossbow公司研发的TelosB,CPU频率为8MHz,有10KB的RAM,使用2.4GHz无线电,能达到250Kbps数据传输,使用两节AAA电池(5号电池)供电。国产传感器节点典型的有美新的MEMSIC无线模块,工作频率可选433 MHz、868-915MHz或2.4GHz,拥有5年电池寿命,支持10-100米的发射范围,拥有19.2kbps-240kbps的数据传输速率。

\begin{figure}[htbp]
  \centering
  \includegraphics[width=5in]{node}
  \caption{无线传感网节点结构}
  \label{fig:node}
\end{figure}

这些传感器节点的设计原理基本相同,主要包括4个模块:传感模块、数据处理模块、无线通信模块和能量供应模块。
如图~\ref{fig:node}所示,是一个典型的无线传感器节点的结构图。传感模块主要负责从感知区域通过传感器获取数据,并将数据转化为适合进行网络传输的数字信号;数据处理模块主要包括处理和存储功能,负责控制传感器节点的运行,对传感模块获取的数据进行处理和存储,数据报文的整合与认证都是由数据处理模块完成,一般该模块需要嵌入式系统的支持,如UC Berkeley的开源嵌入式系统TinyOS\upcite{c:tinyos}等;无线通信模块负责与其他传感器节点或基站之间的通信,传感器节点一般使用内置天线进行数据收发;能量供应模块负责给其他模块供应能量,大部分传感器节点使用微型电池作为电源,因此能量非常有限。传感器节点中还包括一些负责定位、同步等功能的部件。

\subsubsection{无线传感网协议结构}

\begin{figure}[htbp]
  \centering
  \includegraphics[width=5in]{construction}
  \caption{无线传感网协议结构}
  \label{fig:construction}
\end{figure}
无线传感网的通信协议栈和相关网络管理技术是当前的主要研究内容,协议结构如图~\ref{fig:construction}所示。
因为无线传感网是面向特定需求的网络,因此针对不同的部署环境,不同的网络部署结构,要对通信协议栈进行优化,使能量消耗、抗节点损耗、抗攻击能力等适应传感网的应用需求。

类似于OSI网络模型,无线传感网的通信协议栈由物理层、数据链路层、网络层、传输层、应用层组成:

物理层:物理层是通信协议栈的最底层,主要功能是将数据调制成适合传输的数字信号,通过无线电、红外灯无线介质完成传感器节点的数据收发。

数据链路层:数据链路层负责装配数据帧,对数据帧进行MAC校验,进行差错控制,向网络层提供透明可靠的数据传输服务。

网络层:主要负责无线传感网中的路由功能,将数据通过有效路径传送到目标节点,向传输层提供端对端的数据传输服务。

传输层:传输层负责数据报文的传送和控制,为应用层提供可靠的传输服务,对网络进行流量控制,进行服务质量控制(QOS)。

应用层:直接为应用提供服务,提供相应的应用协议和服务接口。

传感网管理协议提供了拓扑管理、QOS管理、安全管理、能量管理和网络管理等功能,实现对无线传感网以及各个节点的监控和管理。
\subsubsection{无线传感网的应用前景}
分布式传感网在军事中的应用是无线传感网的雏形,随着电子技术的不断发展,传感器节点的性能不断提升,无线传感网各种协议的完善和发展,使无线传感网在环境监测、军事侦察、智能家居、智能公路等各个领域得到了大量的应用,其应用前景十分广泛。

\begin{compactitem}
  \item 环境监测:无线传感网能完成大范围监测的任务,在自然数据采集中发挥重要作用,尤其是海洋监测传感网和内陆水文传感网等应用领域。如Li 等人将无线传感网部署在水产养殖水域,对水环境数据进行检测\upcite{c:water}。
  \item 军事侦察:由于无线传感网具有自组网、部署简单、容许节点失效等特点,适合部署在危险的敌对区域,完成军事侦察、战场环境监测等任务,因而在军事领域有很大应用前景,是现代化电子战的重要战略武器。如美国海军将开发的自主分布式DADS(Deployable Autonomous Distributed System)用于沿海广大海域的警戒、反潜和反水雷\upcite{c:DADS}。
  \item 智能家居:智能家居是通过无线传感器将房间中的各种家电等设备连接起来,实现家居环境的监测以及远程控制,构建出智能的居住环境\upcite{c:homes}。
  \item 智能公路:通过部署在公路上的无线传感器节点以及车载传感器节点,共同组成智能公路传感网络,对交通状况实现自动监测,引导车流等,实现自动化的公路交通管理。
\end{compactitem}



\subsection{规模化无线传感网数据认证}
\subsubsection{规模化无线传感网的特点}
规模化无线传感网是为满足大范围监测的需要而产生的,如国内著名的绿野千传项目,在浙江省天目山建立的大规林业监测传感网,部署的自组织传感网节点超过2000个,网络中传输路径跳数超过 20 跳
\upcite{c:lvye}。
规模化无线传感网具有如下的特点:
\begin{enumerate}\setlength{\itemsep}{-\itemsep}
  \item 节点数量大,覆盖面积广,节点失效较为频繁,网络拓扑结构相对不稳定。
  \item 一般部署于恶劣区域,甚至是敌对攻击区域,恶意攻击的频度增加。
  \item 节点的计算和存储能力更为受限,网络的能量较为敏感,对机制的轻量化要求更突出。
\end{enumerate}

\subsubsection{规模化无线传感网的数据认证需求}
无线传感网中的认证包括身份认证和数据认证。身份认证是对网络中节点的合法身份的一种判定机制,是数据认证的基础。无线传感网数据认证主要包括两个方面:
\begin{compactitem}
  \item 数据来源合法性,主要以身份认证为基础,通过数据报文中的认证机制判定数据报文的来源的合法性。
  \item 数据完整性,通过数据认证的机制,确保节点收到的数据报文没有被非法进行篡改。
\end{compactitem}

在环境监测等领域,规模化传感网每天都会产生海量的感知数据。在军事侦察领域,随着侦察区域的扩大,侦察精度的提高,传感网感知的数据量飞速增长。尤其在实时监测场景,数据量大、传输实时性要求高,无线传感器节点的性能限制使得规模化传感网实现可靠传输具有非常的难度,合适的数据认证机制可以为其提供有力支持。在无线传感网中数据泄露、错误数据甚至虚假数据会对网络的安全造成重大影响。尤其在重要战略场景或军事场景,还要考虑破坏攻击的可能,因此数据认证更为安全攸关。
\subsubsection{规模化无线传感网数据认证面临的挑战}

复杂环境下数据高安全性要求对数据认证提出的挑战。实时监测传感网通常部署环境恶劣,而且缺乏基础设施的建设,由于自然环境和主动攻击等对节点的破坏,使节点的失效率很高,网络拓扑结构动态变化,数据传输质量不够稳定,而且存在突发大故障潜因,需要在容灾抗毁前提下进行数据认证,确保传输的安全性。

端对端传输为数据认证提出的挑战。完全依靠广播等数据传输机制,在规模化无线传感网中,传输效率过低,消耗的节点能量和通信资源过大,而且容易受到泛洪攻击的影响。有效利用规模化传感网中端对端数据传输,能够有效的保证传输效率。在端对端传输中,由于多跳传输的原因,当路径中出现妥协节点时,整条路径容易被攻破,从而造成数据传输被攻击,因而在多路径端对端的数据传输中,有效利用数据认证机制加强路径上的安全保障是安全传输的关键。

轻量级认证机制及其实现技术为数据认证提出的挑战。规模化无线传感网传输的数据量大,要求处理快捷。在节点资源能力受限,通信能耗受限的前提下,需要计算、存储、通信都轻量级的水平,保障网络安全、传输可靠性、高效性和数据可信,具有很大难度。传统的的认证机制使用的密码算法复杂度未达到轻量级,不适合规模化传感网网络资源受限的特点,我们需要设计适合实时性较高的规模无线传感网达到轻量级算法。

攻击对抗对数据认证提出的挑战。无线传感网一般部署在恶劣环境中,而且具有自组网络的多跳性、无中心性和自组织性等特征,致使其通信协议栈的各个层级都容易遭受到各种形式的攻击,我们需要设计能够适应有限节点能量,有限计算能力的数据认证算法,对抗各种攻击,保证无线传感网传输数据的来源合法性和完整性。

\section{本文研究内容}
本文根据规模化无线传感网的安全需求以及其特点,针对其数据认证关键技术展开研究,使用多节点联合的技术思路研究数据认证模型和机制,并设计实现了关键算法。
主要工作如下:

\begin{enumerate}\setlength{\itemsep}{-\itemsep}
  \item 提出了多跳长路径上多节点联合数据认证的模型,设计了多跳长路径上多节点联合数据认证协议,并设计了路径上节点关系的维护算法,对协议的安全性能进行了分析评价。
  \item 针对多跳长路径上多写点联合数据认证协议的不足,对算法进行了优化,提出了多路径抗节点失效机制和动态步长多节点联合数据认证机制,并对优化方案的安全性能进行了分析评价。
  \item 围绕多跳长路径多节点联合数据认证机制的需求,对密钥分配方案进行了深入研究,提出了基于单向hash链的密钥分配方案,并对认证中的MAC进行了研究,提出了适应数据认证机制需求的MAC码。
\end{enumerate}


\section{本文组织结构}
本文一共分为七章。

第一章\quad 绪论,介绍了课题的选题背景,描述了无线传感网的特点,介绍了无线传感网的相关安全技术,列出了本文的主要研究内容和本文组织结构。

第二章\quad 相关研究概述,本章首先对无线传感网的安全技术进行了概述,然后重点对数据认证和密钥分配两种安全技术进行了论述。

第三章\quad 多跳长路径上多节点联合数据认证,本章提出了无线传感网中多跳长路径多节点联合的数据认证模型,及其设计目标。
重点介绍了关键算法与协议的设计实现,对多节点联合数据认证机制的安全性能进行了分析评价。

第四章\quad 数据认证方案优化,本章针对多跳长路径上多节点联合数据认证进行了优化,提出了多路径抗节点失效和动态步长多节点联合数据认证两个优化方案,并对它们的安全性能进行了分析评价。

第五章\quad 密钥分配与MAC设计,本章对多节点联合数据认证中的密钥分配方案以及使用的MAC的设计进行了介绍,提出了基于单向hash链的密钥分配方案,以及适应多节点联合数据认证的MAC码。

第六章\quad 仿真实验与结果分析,本章在仿真平台上对多跳长路径多节点联合数据认证机制,以及其优化方案进行了仿真实验,对它们的安全性能结果进行了评价。

第七章\quad 总结与展望,本章对全文的工作做了总结,指出了数据认证机制现阶段的不足以及未来研究中需要研究及完善的地方。

