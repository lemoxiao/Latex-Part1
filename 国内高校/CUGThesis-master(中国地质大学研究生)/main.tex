%!TEX program = xelatex
\documentclass[master,oldfontcfg,euler,twoside,openany]{cugthesis}
% 默认twoside 双面打印
% 将master修改为bachelor, doctor or master (暂时只支持硕士毕业论文模版)
% 要使用adobe字体,添加adobefonts选项
% 使用euler数学字体,如不愿使用,去掉euler
% 使用外文写作,请添加notchinese
% oldfontcfg,使用老板的字体设置,建议初学者开启

% 设置图形文件的搜索路径
\graphicspath{{figures/}}
\newcommand{\tabincell}[2]{\begin{tabular}{@{}#1@{}}#2\end{tabular}}
%仅用于本示例文档中显示特殊字符串
\usepackage{xltxtra}
%仅用于子图
%\usepackage{graphicx}
%\usepackage{subfigure}

%%%%%%%%%%%%%%%%%%%%%%%%%%%%%%
%% 封面部分
%%%%%%%%%%%%%%%%%%%%%%%%%%%%%%

  % 中文封面内容
  \title{MapReduce中间结果重用优化及分析}%一般情况下扉页和封皮、书脊共用一个标题文本,可以不用定义\spinetitle(仅硕博有用), \covertitle(本硕博均有用)和\encovertitle(仅本科有用)。特殊情况见下。
  \spinetitle{MapReduce中间结果重用优化及分析}
  %特殊情况1:本例中\title命令里含有换行控制字符,这会导致制作书脊的时候出现错误,例如如果你注释掉\spinetitle{...}这一行就会报错。这时需要定义一个不含换行等命令的\spinetitle,这并不表示\spinetitle里不能有任何命令——只能使用有限的命令。
  %特殊情况2:本例中标题过长,所以需要缩小书脊标题的字号。
  %特殊情况3:本例中中英文混排,由于tex竖排的原理限制,中英文基线不重合,所以需要人工调整英文的基线。具体调整量根据不同字体有所不同。
  \covertitle{MapReduce中间结果重用优化及分析}
  %\covertitle{中文题目第一行\\中文题目第二行}
  %不要在此调整封皮字体大小! Do not set Cover Page font size here!
  %特殊情况4:本例中\title中含有多个换行,导致标题超过了两行。根据制本厂规定,封皮标题不能超过两行。因此需要定义封皮使用的标题\covertitle. 如果你注释掉这一行,就会发现封皮不符合规定。
  \encovertitle{MapReduce Performance Acceleration and Analytics with Intermediate Results Reusing}
  %\encovertitle{English Title Line 1\\English Title Line 2\\English Title Line 3}
  %不要在此调整封皮字体大小! Do not set Cover Page font size here!
  %特殊情况5:仅本科生有用。本科封皮中有英文标题,不超过三行。与上类似。

  \author{徐\ 锦\ 来}
  \depart{信息工程学院}%系别,硕博请用系代号,本科请用全称如
  %\depart{数理化和信息工程系}
  \major{软件工程专业}%专业,硕博请用全称,本科不需要
  \advisor{罗忠文\ 教授}
  %\coadvisor{姚宏\ 教授}%第二导师,没有请注释掉
  \studentid{120121352}%
  \submitdate{二〇一五年五月}
  %\numsubmitdate{2015.04}
  % 英文封面内容
  \entitle{MapReduce Performance Acceleration and Analytics \\with Intermediate Results Reusing}
  \enauthor{Jinlai Xu}
  \enmajor{Software Engineering}
  \enadvisor{Prof. Zhongwen Luo}
  %\encoadvisor{Prof. Hong Yao}%另外一个导师
  \ensubmitdate{2015.05}
  
%%%%%%%%%%%%%%%%%%%%%%%%%%%%%%%%%%%%%%%%%%%%%%%%%%%%%%%%%%%%%%%%%%%%%
% If you use another language instead of chinese and english, then you
% should define some strings and provide information in your language.
%%%%%%%%%%%%%%%%%%%%%%%%%%%%%%%%%%%%%%%%%%%%%%%%%%%%%%%%%%%%%%%%%%%%%
%  \otherustcstr{zhong guo ke xue ji shu da xue}%A translation of `University of Science and Technology of China' in your language
%  \otherthesisstr{shuo shi xue wei lun wen}%A translation of `A dissertation for doctor(master/bachelor)'s degree' in your language
%  \otherauthorstr{xing ming}%A translation of `Author' in your language
%  \otherdepartmentstr{yuan xi}%A translation of `Department' in your language
%  \otherstudentidstr{xue hao}%A translation of `Student ID' in your language
%  \othersupervisorstr{dao shi}%A translation of `Supervisor' in your language
%  \otherfinishedtimestr{ri qi}%A translation of `Finished Time' in your language
%  \otherspecialitystr{zhuan ye}%A translation of `Speciality' in your language
%  \othertitle{zhong guo ke xue ji shu da xue tong yong xue wen lun wen shi li wen dang}
%  \otherauthor{zhao qian sun}
%  \otheradvisor{zhou wu zheng}
%  \othercoadvisor{feng chen zhu}
%  \othersubmitdate{hou nian ma yue}
%  \othermajor{mou zhuan ye}
%  \otherdepart{mou xi}

\begin{document}

  % 封面
  \maketitle

%特别注意,以下述顺序为准,在对应部分添加文档部件,切勿颠倒顺序:
%本科论文的文档部件顺序是:
%    frontmatter:致谢、目录、中文摘要、英文摘要、
%    mainmatter: 正文章节
%    backmatter: 参考文献或资料注释、附录
%硕博论文的文档部件顺序是:
%    frontmatter:中文摘要、英文摘要、目录、符号说明
%    mainmatter: 正文章节
%    backmatter: 参考文献、附录、致谢、发表论文
%%%%%%%%%%%%%%%%%%%%%%%%%%%%%%
%% 前言部分
%%%%%%%%%%%%%%%%%%%%%%%%%%%%%%
\frontmatter
%\pagenumbering{}
\makeatletter
\ifustc@bachelor
	%%%%%%%%%%%%%%%%%
	%本科论文修改这里
	%%%%%%%%%%%%%%%%%
	% 致谢
	% !TEX root = ../main.tex
\begin{thanks}

感谢原中科大本硕博论文模版的制作人员

包括但不限于ywg,XPS,Liuqs,Guolicai,刘青松等原模版的制作者和维护者

我在这里只是按照中国地质大学的work版本的硕士论文模版进行了一定的修改

以满足使用\LaTeX 撰写中国地质大学硕士论文的目的

感谢大家对本模板更新工作的支持!

本模板以及本示例文档还存在许多不足之处,欢迎大家测试并及时提供反馈。

\begin{flushright}
seeksky@CUG

欢迎访问我的博客 http://jinlaixu.net
\end{flushright}


在中国地质大学完成本科和硕士连读学业的七年里,我所从事的学习和研究工作,都是在导师以及系里其他老师和同学的指导和帮助下进行的。在完成论文之际,请容许我对他们表达诚挚的谢意。

首先感谢导师XXX教授和XXX副教授多年的指导和教诲,是他们把我带到了计算机视觉的研究领域。X老师严谨的研究态度及忘我的工作精神,X老师认真细致的治学态度及宽广的胸怀,都将使我受益终身。

感谢班主任XXX老师和XX老师多年的关怀。感谢XXX、XX、XX等老师,他们本科及研究生阶段的指导给我研究生阶段的研究工作打下了基础。

感谢XX、XXX、XXX、XX、XXX、XXX、XXX、XX等师兄师姐们的指点和照顾;感谢XXX、XX、XXX等几位同班同学,与你们的讨论使我受益良多;感谢XXX、XX、XXX、XX、XXX等师弟师妹,我们在XXX实验室共同学习共同生活,一起走过了这段愉快而难忘的岁月。

感谢科大,感谢一路走过来的兄弟姐妹们,在最宝贵年华里,是你们伴随着我的成长。

最后,感谢我家人一贯的鼓励和支持,你们是我追求学业的坚强后盾。

\vskip 18pt

\begin{flushright}

~~~~赵钱孙~~~~

\today

\end{flushright}

\end{thanks}

	
	%目录部分
	%目录
	\tableofcontents
	%默认表格、插图、算法索引名称分别为“表格索引”、“插图索引”和“算法索引”
	%如果需要自行修改lot,lof,loa的名称,请定义
	%\ustclotname{...}
	%\ustclofname{...}
	%\ustcloaname{...}

	% 表格索引
	\ustclot
	% 插图索引
	\ustclof
	%算法索引 
	%如果需要使用算法环境并列出算法索引,请加入补充宏包。
	\ustcloa
	
	% 摘要
	%# -*- coding: utf-8-unix -*-
%%==================================================

\begin{abstract}
本项目为年产50万吨MTO工厂的初步设计。通过分析当前国内外MTO生产和研究现状,对生产工艺进行了选择论证。然后运用Aspen软件模拟初步的工艺流程,并通过对一系列工艺参数,如精馏塔的塔板数—产品纯度、进料塔板数—产品纯度、产品纯度—回流比、再沸器负荷—回流比等进行灵敏度分析,优化设备操作条件,提高工艺的合理性和经济性。本设计还针对工艺流程进行换热网络设计和对全局换热网络进行了优化和评估,通过内部流股之间相互换热以减少公用工程的消耗,最终优化后节约$79.4\%$的热公用工程资源和$73.7\%$的冷公用工程资源。本设计还运用水夹点技术优化了用水网络,根据水硬度分类处理水操作单元,并合理再生利用,使得本项目新鲜水用量和废水排放量达到最小,优化后的用水网络节约用水$53.59\%$。本设计对于MTO工厂的生产和设计建造具有一定的现实指导意义。\\

\keywords{\zihao{-4} 工厂\quad 设计\quad MTO \quad 工艺 \quad 水夹点  \quad 网络 \quad 控制}
\end{abstract}

\begin{englishabstract}

This project is the preliminary design of a MTO plant with an annual output of 500,000 tons of light olefins. Based on the current production and research situation all through the world, the production method was selected and demonstrated. Aspen software was used to simulate the preliminary process. Heat integration method was applied to optimize the heat exchange network. Rational heat exchange between process streams were suggested which resulted in the decreasing of utilities consumption and exchanger number. The heat integration leaded to energy saving of $79.4\%$ of heat utilities and $73.7\%$ of the cold utilities. In addition, the water pinch technology was also implemented to optimize the water network. The water operating unit was classified according to water hardness, with a reasonable recycling. The amount of fresh water consumption and wastewater emission was minimized. The optimized water network achieved $53.59\%$ water saving. Finally, a preliminary economic analysis to the entire project was estimated in order to get the project construction cost and profitability. In summary, this design is of some practical significance for the production and design of the MTO industry.

\englishkeywords{\zihao{-4} Plant design\;Sensitivity analysis  \; Energy balance\; calculation \; Water pinch  Dynamic control}
\end{englishabstract}

%此文件中含有中英文摘要
\else
	%%%%%%%%%%%%%%%%%
	%硕博论文修改这里
	%%%%%%%%%%%%%%%%%
  % 个人简介
  % !Mode:: "TeX:UTF-8"
\begin{resume}
\vspace*{\baselineskip}
\noindent
\begin{minipage}[t]{4cm}
\vspace{-\baselineskip}
\includegraphics[width=3.3cm]{biophoto}
\end{minipage}%
\hfill%
\begin{minipage}[t]{10cm}
\vspace{-\baselineskip}
%===============================
姓\qquad 名:×××

性\qquad 别:×

民\qquad 族:×族

出生年月:×××× 年× 月

籍\qquad 贯:××××
%===============================
\end{minipage}
\vspace*{\baselineskip}
%===============================

\textbf{学习经历}   % 自大学起

×××× 年× 月考入××大学××院××专业,×××× 年× 月本科毕业并获得××学士学位。

×××× 年× 月-- ×××× 年× 月,在燕山大学××学院××学科学习。

\textbf{获奖情况}   % 自大学起

×××× -- ×××× 年,燕山大学校级三好学生

×××× -- ×××× 年,燕山大学校级一等奖学金

×××× -- ×××× 年,燕山大学××学院三好学生

×××× -- ×××× 年,燕山大学××学院优秀团干部

(不含科研学术获奖)。

\textbf{工作经历}   % 没有可不写(将本行删除即可)

\end{resume}

  % 摘要
  %# -*- coding: utf-8-unix -*-
%%==================================================

\begin{abstract}
本项目为年产50万吨MTO工厂的初步设计。通过分析当前国内外MTO生产和研究现状,对生产工艺进行了选择论证。然后运用Aspen软件模拟初步的工艺流程,并通过对一系列工艺参数,如精馏塔的塔板数—产品纯度、进料塔板数—产品纯度、产品纯度—回流比、再沸器负荷—回流比等进行灵敏度分析,优化设备操作条件,提高工艺的合理性和经济性。本设计还针对工艺流程进行换热网络设计和对全局换热网络进行了优化和评估,通过内部流股之间相互换热以减少公用工程的消耗,最终优化后节约$79.4\%$的热公用工程资源和$73.7\%$的冷公用工程资源。本设计还运用水夹点技术优化了用水网络,根据水硬度分类处理水操作单元,并合理再生利用,使得本项目新鲜水用量和废水排放量达到最小,优化后的用水网络节约用水$53.59\%$。本设计对于MTO工厂的生产和设计建造具有一定的现实指导意义。\\

\keywords{\zihao{-4} 工厂\quad 设计\quad MTO \quad 工艺 \quad 水夹点  \quad 网络 \quad 控制}
\end{abstract}

\begin{englishabstract}

This project is the preliminary design of a MTO plant with an annual output of 500,000 tons of light olefins. Based on the current production and research situation all through the world, the production method was selected and demonstrated. Aspen software was used to simulate the preliminary process. Heat integration method was applied to optimize the heat exchange network. Rational heat exchange between process streams were suggested which resulted in the decreasing of utilities consumption and exchanger number. The heat integration leaded to energy saving of $79.4\%$ of heat utilities and $73.7\%$ of the cold utilities. In addition, the water pinch technology was also implemented to optimize the water network. The water operating unit was classified according to water hardness, with a reasonable recycling. The amount of fresh water consumption and wastewater emission was minimized. The optimized water network achieved $53.59\%$ water saving. Finally, a preliminary economic analysis to the entire project was estimated in order to get the project construction cost and profitability. In summary, this design is of some practical significance for the production and design of the MTO industry.

\englishkeywords{\zihao{-4} Plant design\;Sensitivity analysis  \; Energy balance\; calculation \; Water pinch  Dynamic control}
\end{englishabstract}

%此文件中含有中英文摘要
	% 目录
	\tableofcontents
	%默认表格、插图、算法索引名称分别为“表格索引”、“插图索引”和“算法索引”
	%如果需要自行修改lot,lof,loa的名称,请定义
	%\ustclotname{...}
	%\ustclofname{...}
	%\ustcloaname{...}

	% 表格索引
	\ustclot
	% 插图索引
	\ustclof
	%算法索引 
	%如果需要使用算法环境并列出算法索引,请加入补充宏包。
	%\ustcloa
	
	%符号说明,需要加入补充包
	\include{chapter/denotation}%不是必需的,如果不想列出请注释掉
\fi
\makeatother

%%%%%%%%%%%%%%%%%%%%%%%%%%%%%%
%% 正文部分
%%%%%%%%%%%%%%%%%%%%%%%%%%%%%%
\mainmatter

  \chapter{Introduction}
\label{chap:intro}

\section{What is Lorem Ipsum?}
\label{sec:apadia}

Lorem Ipsum is simply dummy text of the printing and typesetting industry. Lorem Ipsum has been the industry's standard dummy text ever since the 1500s, when an unknown printer took a galley of type and scrambled it to make a type specimen book \cite{banerjee:pedersen:2003}. It has survived not only five centuries, but also the leap into electronic typesetting, remaining essentially unchanged. It was popularised in the 1960s with the release of Letraset sheets containing Lorem Ipsum passages, and more recently with desktop publishing software like Aldus PageMaker including versions of Lorem Ipsum \cite{berment:phd:2004}.



\section{Where Does It Come From?}
\label{sec:where}

Contrary to popular belief, Lorem Ipsum is not simply random text. It has roots in a piece of classical Latin literature from 45 BC, making it over 2000 years old. Richard McClintock, a Latin professor at Hampden-Sydney College in Virginia, looked up one of the more obscure Latin words, consectetur, from a Lorem Ipsum passage, and going through the cites of the word in classical literature, discovered the undoubtable source \cite{azarova:etal:2002,budanitsky:hirst:2006}. Lorem Ipsum comes from sections 1.10.32 and 1.10.33 of ``de Finibus Bonorum et Malorum'' (The Extremes of Good and Evil) by Cicero, written in 45 BC. This book is a treatise on the theory of ethics, very popular during the Renaissance. The first line of Lorem Ipsum, ``''Lorem ipsum dolor sit amet\ldots'', comes from a line in section 1.10.32.

The standard chunk of Lorem Ipsum used since the 1500s is reproduced below for those interested. Sections 1.10.32 and 1.10.33 from ``de Finibus Bonorum et Malorum'' by Cicero are also reproduced in their exact original form, accompanied by English versions from the 1914 translation by H. Rackham. 

\begin{equation}
-\frac{(x_0 - \mu)^2}{2 \sigma^2} = -\ln 2
\end{equation}


\section{Examples}
\label{sec:examples}

The first few paragraphs of Lorem Ipsum are given below.

\subsection{First Paragraph}

Lorem ipsum dolor sit amet, consectetur adipiscing elit. Donec posuere, neque quis feugiat egestas, quam sapien dictum justo, eu vulputate nunc metus sed dui. Integer molestie leo quis libero facilisis, dictum pretium quam ornare. Vestibulum ante ipsum primis in faucibus orci luctus et ultrices posuere cubilia Curae; Vivamus luctus rutrum magna non convallis. Praesent vestibulum consequat eros, et fringilla nisi suscipit id. Nam vulputate justo dui, eu rutrum est accumsan ut. Sed molestie erat vitae mi blandit, in volutpat urna lobortis. Vestibulum mollis rutrum gravida. Fusce dolor nulla, condimentum vel pretium ut, venenatis eget leo. Ut semper placerat mauris, ut tempus est tempor vel. Interdum et malesuada fames ac ante ipsum primis in faucibus. In vitae feugiat diam. Pellentesque accumsan consequat turpis aliquam elementum.


\subsection{Next Two Paragraphs}

Vivamus dignissim arcu nunc, non aliquam sem porta vitae. Sed sodales accumsan dui sit amet egestas. Maecenas rhoncus a erat eget accumsan. 

\begin{table}[hbt!]\centering
\caption{Number of Jewels}

\begin{tabular}{l c}
\hline
Type & Quantity \\\hline
Sapphire & 6\\
Diamond & 23\\
Gold & 56\\
Silver & 235\\
Bronze & 324\\\hline
\end{tabular}
\end{table}

\begin{itemize}
\item Etiam vitae pulvinar metus, sed fringilla orci. 
\item Duis dapibus dolor risus, non ultrices enim porta sit amet. 
\item Ut eu libero augue. 
\end{itemize}

Nulla ipsum augue, feugiat ac laoreet quis, pretium ut magna. Class aptent taciti sociosqu ad litora torquent per conubia nostra, per inceptos himenaeos. Integer blandit placerat dictum.

\begin{figure}[hbt!]\centering
\includegraphics[width=.5\textwidth]{green}
\caption{Example figure}
\end{figure}

Sed dolor justo, scelerisque sed rutrum quis, porttitor a mauris. Cras non auctor felis, rutrum fringilla risus. Integer at convallis erat, sit amet luctus turpis. Duis sed rutrum eros, quis tempus risus. Etiam pellentesque nisi odio, eget dignissim eros ultrices et. Aliquam leo massa, fermentum vel odio sed, ullamcorper molestie lorem. Integer lorem felis, adipiscing sit amet interdum eget, auctor at lorem. Aliquam ultricies tortor eu nibh facilisis tincidunt.


\subsubsection{Some Notes}

Duis sed rutrum eros, quis tempus risus. Etiam pellentesque nisi odio, eget dignissim eros ultrices et. Aliquam leo massa, fermentum vel odio sed, ullamcorper molestie lorem.

\subsubsection{And Further}
Duis sed rutrum eros, quis tempus risus. Etiam pellentesque nisi odio, eget dignissim eros ultrices et. Aliquam leo massa, fermentum vel odio sed, ullamcorper molestie lorem.


\subsection{Least-Squares with Forgetting Factor AdaptiveLaw}

\section{Mechatronic Suspension System; History and a Brief Background}
Nulla ipsum augue, feugiat ac laoreet quis, pretium ut magna. Class aptent taciti sociosqu ad litora torquent per conubia nostra, per inceptos himenaeos. Integer blandit placerat dictum.



\section{Summary}
Nulla ipsum augue, feugiat ac laoreet quis, pretium ut magna. Class aptent taciti sociosqu ad litora torquent per conubia nostra, per inceptos himenaeos. Integer blandit placerat dictum.

Sed dolor justo, scelerisque sed rutrum quis, porttitor a mauris. Cras non auctor felis, rutrum fringilla risus. Integer at convallis erat, sit amet luctus turpis. Duis sed rutrum eros, quis tempus risus. Etiam pellentesque nisi odio, eget dignissim eros ultrices et. Aliquam leo massa, fermentum vel odio sed, ullamcorper molestie lorem. Integer lorem felis, adipiscing sit amet interdum eget, auctor at lorem. Aliquam ultricies tortor eu nibh facilisis tincidunt.
  \include{chapter/chap-analytics}
  \include{chapter/chap-IntermediateAnalysis}
  \include{chapter/chap-memomr}
  \include{chapter/chap-experiment}
  %\include{chapter/chap-application}
  \include{chapter/chap-summary}
  %自行添加
  %\include{chapter/...}

%%%%%%%%%%%%%%%%%%%%%%%%%%%%%%
%% 附件部分
%%%%%%%%%%%%%%%%%%%%%%%%%%%%%%
\backmatter
  \makeatletter
  \ifustc@bachelor\relax\else
    % 致谢
    % !TEX root = ../main.tex
\begin{thanks}

感谢原中科大本硕博论文模版的制作人员

包括但不限于ywg,XPS,Liuqs,Guolicai,刘青松等原模版的制作者和维护者

我在这里只是按照中国地质大学的work版本的硕士论文模版进行了一定的修改

以满足使用\LaTeX 撰写中国地质大学硕士论文的目的

感谢大家对本模板更新工作的支持!

本模板以及本示例文档还存在许多不足之处,欢迎大家测试并及时提供反馈。

\begin{flushright}
seeksky@CUG

欢迎访问我的博客 http://jinlaixu.net
\end{flushright}


在中国地质大学完成本科和硕士连读学业的七年里,我所从事的学习和研究工作,都是在导师以及系里其他老师和同学的指导和帮助下进行的。在完成论文之际,请容许我对他们表达诚挚的谢意。

首先感谢导师XXX教授和XXX副教授多年的指导和教诲,是他们把我带到了计算机视觉的研究领域。X老师严谨的研究态度及忘我的工作精神,X老师认真细致的治学态度及宽广的胸怀,都将使我受益终身。

感谢班主任XXX老师和XX老师多年的关怀。感谢XXX、XX、XX等老师,他们本科及研究生阶段的指导给我研究生阶段的研究工作打下了基础。

感谢XX、XXX、XXX、XX、XXX、XXX、XXX、XX等师兄师姐们的指点和照顾;感谢XXX、XX、XXX等几位同班同学,与你们的讨论使我受益良多;感谢XXX、XX、XXX、XX、XXX等师弟师妹,我们在XXX实验室共同学习共同生活,一起走过了这段愉快而难忘的岁月。

感谢科大,感谢一路走过来的兄弟姐妹们,在最宝贵年华里,是你们伴随着我的成长。

最后,感谢我家人一贯的鼓励和支持,你们是我追求学业的坚强后盾。

\vskip 18pt

\begin{flushright}

~~~~赵钱孙~~~~

\today

\end{flushright}

\end{thanks}
%硕博致谢部分
    % 发表文章目录
    %\include{chapter/pub}
  \fi
  \makeatother
  % 参考文献
  % 使用 BibTeX
  % 选择参考文献的排版格式。注意ustcbib这个格式不保证完全符合要求,请自行决定是否使用
  \bibliographystyle{cugbib}%{GBT7714-2005NLang-UTF8}
  %\bibliographystyle{gbt7714-2005}%{GBT7714-2005NLang-UTF8}
  \bibliography{bib/ref}
  \nocite{*} % for every item
  % 不使用 BibTeX
  % \begin{thebibliography}{99}
\addcontentsline{toc}{chapter}{References}

\bibitem[Aravind, Hackl \& Wexler(2017)]{aravind2017}
Aravind, A.\,, Hackl, M.\, \& Wexler, K.\,, 2017, Syntactic and Pragmatic Factors in Children's Comprehension of Cleft Constructions [J], \emph{Langauge Acquisition}, 1: 1-31.
\bibitem[Beaver \& Clark(2008)]{beaver2008}
	Beaver, D.\,I.\,\& Clark, B.\,Z.\,, 2008, \emph{Sense and Sensitivity: How Focus Determines Meaning} [M], Malden, MA: Blackwell Publishing.
\bibitem[Brandt et al(2014)]{brandt2014}
	Brandt, M.\,et al, 2014, The Replication Recipe: What makes for a convincing replication? [J], \emph{Journal of Experimental Social Psychology}, 50: 217--224.
\bibitem[Colonna, Schimke \& Hemforth(2012)]{colonna2012}
	Colonna, S.\,, Schimke, S.\,, \& Hemforth, B.\,, 2012, Information structure effects on anaphora resolution in German and French: A crosslinguistic study of pronoun resolution [J], \emph{Linguistics}, 55(5): 991--1013.
\bibitem[Colonna, Schimke \& Hemforth(2015)]{colonna2015}
	Colonna, S.\,, Schimke, S.\,, \& Hemforth, B.\,, 2015, Deifferent effects of focus in intra- and inter-sentential pronoun resolution in German [J], \emph{Language, Cognition and Neuroscience}, 30(10): 1306--1325.
\bibitem[Cowles, Walenski \& Kluender(2007)]{cowles2007}
	Cowles, H.\,H.\,, Walenski, M.\,, \& Kluender, R.\,, 2007, Linguistic and cognitive prominence in anaphora resolution: topic, contrastive focus and pronouns [J], \emph{Topoi}, 26(1): 3--18.
\bibitem[Grosz, Weinstein \& Joshi(1995)]{grosz1995}
	Grosz, B.\,J.\,, Weinstein, S.\,, \& Joshi, A.\,K.\,, 1995, Centering: A framework for modeling the local coherence of discourse [J], \emph{Computational Linguistics}, 21: 203--205.
\bibitem[Engelkamp \& Zimmer(1982)]{engelkamp1982}
	Engelkamp, J.\, \& Zimmer, H.\,D.\,, 1982, The interaction of subjectivization and concept placement in the processing of cleft sentences [J], \emph{Quarterly Journal of Experimental Psychology: Human Experimental Psychology}, 34(A): 463--478.
\bibitem[Hakes, Evans \& Brannon(1976)]{hakes1976}
	Hakes, D.\,T.\,, Evans, J.\,S.\, \& Brannon, L.\,L.\,, 1976, Understanding sentences with relative clauses [J], \emph{Memory and Cognition}, 4(3): 283--290.
\bibitem[Halliday(1967)]{halliday1967}
	Halliday, M.\,, 1967, Notes on Transitivity and Theme in English (Part 2) [J], \emph{Journal of Lingyuistics}, 3: 206.
\bibitem[Kaiser(2011)]{kaiser2011}
	Kaiser, E.\,, 2011, Focusing on pronouns: Consequences of subjecthooh, pronominalisation and contrastive focus [J], \emph{Language and cognitive Processes}, 26(10), 1625--1666.
\bibitem[Kornai(2008)]{kornai2008}
	Kornai, A.\,, 2008, \emph{Mathematical Linguistics} [M], London: Springer-Verlag. 
\bibitem[Lambrecht(2001)]{lambrecht2001}
	Lambrecht, K.\,, 2001, A framework for the analysis of cleft constructions [J], \emph{Linguistics}, 39(3): 463--516.
\bibitem[Liu, Xu \& Liang(2017)]{liu2017}
	Liu, H.\,, Xu, C.\, \& Liang, J.\,, 2017, Dependency distance: A new perspective on syntactic patterns in natural langauges [J], \emph{Physics of Life Reviews}, 21: 171--193.
\bibitem[Manning \& Sch\"{u}tze(1999)]{manning1999}
	Manning, C.\,D.\, \& Sch\"{u}tze, H.\,, 1999, \emph{Foundations of Statistical Natural Language Processing} [M], London: The MIT Press. 
\bibitem[Mitkov(1999)]{mitkov1999}
	Mitkov, R.\,, 1999, \emph{Anaphora Resolution: The State of the Art} [R], Paper based on the COLING'98/ACL'98 tutorial on anaphora resolution.
\bibitem[O'Grady et al(2010)]{ogrady2010}
	O'Grady, W.\, et al, 2010, \emph{Contemporary Linguistics: An Introduction} [M], Boston: Bedford \& St. Martin's.
\bibitem[Patterson, Esaulova \& Felser(2017)]{patterson2017}
	Patterson, C.\,, Esaulova, Y.\, \& Felser, C.\,, 2017, The impact of focus on pronoun resolution in native and non-native sentence comprehension [J], \emph{Second Langauge Research}, 33(4): 403--429.
\bibitem[Pinker(2014)]{pinker2014}
	Pinker, S.\,, 2014, \emph{The Sense of Style: The Thinking Person's Guide to Writing in the 21$^{st}$ Century} [M], New York: Penguin Books.
\bibitem[Reichle(2014)]{reichle2014}
	Reichle, R.\,, 2014, Cleft type and focus structure processing in French, \emph{Language, Cognition and Neuroscience}, 29(1): 107--124.
\bibitem[Sayed(2003)]{sayed2003}
	Sayed, I.\,Q.\,, 2003, \emph{Issues in Anaphora Resolution} [OL], retrieved from: \url{https://nlp.stanford.edu/courses/cs224n/2003/fp/iqsayed/project_report.pdf}.
\bibitem[Sgall, Hajicová \& Panevová(1986)]{sgall1986}
	Sgall, P.\,, Hajicová, E.\, \& Panevová, J.\,, 1986, \emph{The Meaning of the Sentence in its Semantic and Pragmatic Aspects} [M], Praha: Academia.
\bibitem[徐晓东, 陈庆荣(2014)]{徐晓东2014}
	徐晓东, 陈庆荣, 2014, 汉语焦点信息影响代词回指的电生理机制 [J], 《心理科学进展》, 22(6): 902--910.
\bibitem[徐晓东,倪传斌,陈丽娟(2013)]{徐晓东2013}
	徐晓东,倪传斌,陈丽娟, 2013, 话题结构和动词语义对代词回指的影响:一项基于语言产生和语言理解任务的实证研究 [J], 《现代外语(季刊)》, 36(4): 331--339. 
\end{thebibliography}


  % 附录,没有请注释掉
  \begin{appendix}
    \include{chapter/chap-req}
  \end{appendix}

  

  

\end{document}
