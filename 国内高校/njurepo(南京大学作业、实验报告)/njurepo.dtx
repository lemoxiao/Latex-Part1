% \iffalse meta-comment
%
% Copyright (C) 2019 by Zangwei Zheng <zhengzangw@gmail.com>
%
% This file may be distributed and/or modified under the conditions of
% the LaTeX Project Public License, either version 1.3c of this license
% or (at your option) any later version. The latest version of this 
% license is in:
%
%   http://www.latex-project.org/lppl.txt
%
% and version 1.3c or later is part of all distributions of LaTeX
% version 2005/12/01 or later.
%
% \fi
%
% \iffalse
%<*driver>
\ProvidesFile{njurepo.dtx}[2019/01/29 1.0.1 Nanjing University Report Template]
\documentclass{ltxdoc}
\usepackage{dtx-style}
    \EnableCrossrefs
    \CodelineIndex
    \RecordChanges
\begin{document}
    \DocInput{njurepo.dtx}
    \PrintChanges
    \PrintIndex
\end{document}
%</driver>
% \fi
%
% \CheckSum{0}
%
% \CharacterTable
%  {Upper-case    \A\B\C\D\E\F\G\H\I\J\K\L\M\N\O\P\Q\R\S\T\U\V\W\X\Y\Z
%   Lower-case    \a\b\c\d\e\f\g\h\i\j\k\l\m\n\o\p\q\r\s\t\u\v\w\x\y\z
%   Digits        \0\1\2\3\4\5\6\7\8\9
%   Exclamation   \!     Double quote  \"     Hash (number) \#
%   Dollar        \$     Percent       \%     Ampersand     \&
%   Acute accent  \'     Left paren    \(     Right paren   \)
%   Asterisk      \*     Plus          \+     Comma         \,
%   Minus         \-     Point         \.     Solidus       \/
%   Colon         \:     Semicolon     \;     Less than     \<
%   Equals        \=     Greater than  \>     Question mark \?
%   Commercial at \@     Left bracket  \[     Backslash     \\
%   Right bracket \]     Circumflex    \^     Underscore    \_
%   Grave accent  \`     Left brace    \{     Vertical bar  \|
%   Right brace   \}     Tilde         \~}
%
% \DoNotIndex{\newenvironment,\@bsphack,\@empty,\@esphack,\sfcode}
% \DoNotIndex{\addtocounter,\label,\let,\linewidth,\newcounter}
% \DoNotIndex{\noindent,\normalfont,\par,\parskip,\phantomsection}
% \DoNotIndex{\providecommand,\ProvidesPackage,\refstepcounter}
% \DoNotIndex{\RequirePackage,\setcounter,\setlength,\string,\strut}
% \DoNotIndex{\textbackslash,\texttt,\ttfamily,\usepackage}
% \DoNotIndex{\begin,\end,\begingroup,\endgroup,\par,\\}
% \DoNotIndex{\if,\ifx,\ifdim,\ifnum,\ifcase,\else,\or,\fi}
% \DoNotIndex{\let,\def,\xdef,\edef,\newcommand,\renewcommand}
% \DoNotIndex{\expandafter,\csname,\endcsname,\relax,\protect}
% \DoNotIndex{\Huge,\huge,\LARGE,\Large,\large,\normalsize}
% \DoNotIndex{\small,\footnotesize,\scriptsize,\tiny}
% \DoNotIndex{\normalfont,\bfseries,\slshape,\sffamily,\interlinepenalty}
% \DoNotIndex{\textbf,\textit,\textsf,\textsc}
% \DoNotIndex{\hfil,\par,\hskip,\vskip,\vspace,\quad}
% \DoNotIndex{\centering,\raggedright,\ref}
% \DoNotIndex{\c@secnumdepth,\@startsection,\@setfontsize}
% \DoNotIndex{\ ,\@plus,\@minus,\p@,\z@,\@m,\@M,\@ne,\m@ne}
% \DoNotIndex{\@@par,\DeclareOperation,\RequirePackage,\LoadClass}
% \DoNotIndex{\AtBeginDocument,\AtEndDocument}
%
% \changes{v1.0.0}{2019/01/22}{Initial version}
% \changes{v1.0.1}{2019/01/29}{Add more ability}
% \changes{v1.1.0}{2019/01/29}{Stable version}
%
% \GetFileInfo{\jobname.dtx}
%
% \def\indexname{索引}
% \def\glossaryname{修改记录}
% \IndexPrologue{\section{\indexname}}
% \GlossaryPrologue{\section{\glossaryname}}

% \title{\bfseries\color{violet}\njurepo: 南京大学本科生范用报告}
% \author{郑奘巍 \\[5pt]\texttt{zhengzangw@gmail.com}}
% \date{\fileversion\ (\filedate)}
% \maketitle\thispagestyle{empty}
%
% \begin{abstract}\noindent
% 此宏包旨在建立一个免于配置的、指令相对简单的南京大学作业、实验报告通用模板。
% \end{abstract}
%
%
% \vskip2cm
% \def\abstractname{免责声明}
% \begin{abstract}
% \noindent
% \begin{enumerate}
% \item 本模板的发布遵守 \LaTeX\ Project Public License,使用前请认真阅读协议内
%   容。
% \item \textbf{本模板为作者自己通常使用的报告模板,与南京大学官方没有任何关系}。任何使用该宏包进行实验报告制作时,请\textbf{务必根据课程要求进行写作}。由于使用本模板而引起的作业验收问题,均与本模板作者无关。
% \item 本模板借鉴\thuthesis{}宏包的大量内容,需要稳定模板的同学也可以选择使用清华大学的\thuthesis{}宏包并自己进行配置。
% \item 任何个人或组织以本模板为基础进行修改、扩展而生成的新的专用模板,请严格遵
%   守 \LaTeX\ Project Public License 协议。由于违犯协议而引起的任何纠纷争端均与
%   本模板作者无关。
% \end{enumerate}
% \end{abstract}
%
% \clearpage
% \pagestyle{fancy}
% \begin{multicols}{2}[
%   \setlength{\columnseprule}{.4pt}
%   \setlength{\columnsep}{18pt}]
%   \tableofcontents
% \end{multicols}
% \clearpage
%
% \section{模板介绍}
% \njurepo\ (\textbf{N}an\textbf{jing} \textbf{U}niversity \LaTeX\ Versatile \textbf{Repo}rt Template)是根据作者用\LaTeX{}制作南京大学课程实验报告的模板文件,可帮助本科生快速的制作实验报告和作业。
% 本文档将尽量完整的介绍模板的使用方法,如有不清楚之处可以参考示例文档或者根据第 3.1 节说明提问,有兴趣者都可以参与完善此手册,也非常欢迎对代码的贡献。
% 
% \section{安装}
% \label{sec:installation}
% \njurepo 开发版需要自行前往github主页:\\
% https://github.com/zhengzangw/njurepo下载。
%
% \subsection{字体安装}
% 字体存放在font文件夹中,使用模板前先自行安装。
%
% \subsection{模板的组成}
% 下表列出了\njurepo 的主要文件及其功能介绍:
%
% \begin{longtable}{l|p{8cm}}
% \toprule
% {\heiti 文件(夹)} & {\heiti 功能描述}\\\midrule
% \endfirsthead
% \midrule
% {\heiti 文件(夹)} & {\heiti 功能描述}\\\midrule
% \endhead
% \endfoot
% \endlastfoot
% njurepo.ins & \textsc{DocStrip} 驱动文件(开发用) \\
% njurepo.dtx & \textsc{DocStrip} 源文件(开发用)\\\midrule
% example.pdf & 实例文档\\
% main.tex & 主文件\\
% figs/ & 图片路径\\
% figs/logos/ & 示例文档图片路径\\
% fonts/ & 字体\\
% parts/ & 具体内容\\
% parts/examples & 示例文档具体内容\\
% ref/ & 参考文献和参考文献样式文件\\
% njurepo.cls & 模板类文件\\
% \textbf{njurepo.pdf} & 用户手册(本文档)\\ \bottomrule
% \end{longtable}
%
% \subsection{生成模板}
% 使用Makefile或\XeLaTeX 生成模板文件
% \begin{shell}
% make cls
% xelatex njurepo.dtx # 两句选一句即可
% \end{shell}
% \subsection{生成论文}
% \subsubsection{latexmk}
% latexmk 命令支持全自动生成\LaTeX{}编写的文档,并且支持使用不同的工具链来进行生成,它会自动运行多次工具直到交叉引用都被解决。下面给出了一个用 latexmk 调用 xelatex 生成最终文档的示例:
% \begin{shell}
% latexmk -xelatex main
% \end{shell}
% \subsubsection{make}
% \njurepo{}提供了一个Makefile:
% \begin{shell}
% make clean
% make cls # 生成 njurepo.cls
% make doc # 生成说明文档 njurepo.pdf
% make main # 生成示例文档main.pdf
% \end{shell}
% \subsection{升级}
% 在github上下载最新版,运行:
% \begin{shell}
% make cls
% \end{shell}
% 生成新的类文件和配置文件即可。也可以直接拷贝 njurepo.cls,免去上面命令的执行。
% 
%
% \section{使用说明}
% \subsection{示例文件}
% 推荐从模板自带的示例文档入手,其中包括了论文写作用到的所有命令及其使用方法,只需要用自己的内容进行相应替换就可以。对于不清楚的命令可以查阅本手册。下面的例子描述了模板中章节的组织形式,来自于示例文档,具体内容可以参考模板附带的 main.tex 和 parts/examples/。
% \begin{latex}
% \documentclass[language=english,open=any]{njurepo}
% \begin{document}
% \frontmatter
% \thispagestyle{empty}
\begin{center}
    \includegraphics[scale=0.7]{figures/cdut.png}
\end{center}
\vskip1.5cm
\begin{center}
    \makebox[109mm][s]{\heiti\zihao{-0}\bf 本科生实验报告}
\end{center}
\vskip2cm
\begin{center}
    \makebox[20mm][s]{\heiti\zihao{4} 实验课程}\underline{\makebox[130mm][c]{\heiti\zihao{3}\LaTeX 书写行为规范}}\\
    \vskip1cm
    \makebox[20mm][s]{\heiti\zihao{4} 实验名称}\underline{\makebox[130mm][c]{\heiti \zihao{3} 利用\LaTeX 书写成都理工大学实验报告}}\\
    \vskip1cm
    \makebox[20mm][s]{\heiti\zihao{4} 专业名称}\underline{\makebox[130mm][c]{\heiti\zihao{3} 专业全称(有专业方向的用小括号标明)}}\\
    \vskip1cm
    \makebox[20mm][s]{\heiti\zihao{4} 学生姓名}\underline{\makebox[130mm][c]{\heiti\zihao{3} 您的姓名}}\\
    \vskip1cm
    \makebox[20mm][s]{\heiti\zihao{4} 学生学号}\underline{\makebox[130mm][c]{\heiti\zihao{3} 您的学号}}\\
    \vskip1cm
    \makebox[20mm][s]{\heiti\zihao{4} 指导教师}\underline{\makebox[130mm][c]{\heiti\zihao{3} 您的授课或者指导老师}}\\
    \vskip1cm
    \makebox[20mm][s]{\heiti\zihao{4} 实验地点}\underline{\makebox[130mm][c]{\heiti\zihao{3} 授课地点(如:6C403)}}
    \vskip1cm
    \makebox[20mm][s]{\heiti\zihao{4} 实验成绩}\underline{\makebox[130mm][c]{\heiti\zihao{3} 由指导老师书写}}\\
\end{center}
\vskip1.85cm
\vfill\begin{center}
    {\songti\zihao{3}二〇一八年三月二十一日}
\end{center}
\newpage
\thispagestyle{empty}
\tableofcontents
\newpage
\setcounter{page}{1}
% %# -*- coding: utf-8-unix -*-
%%==================================================

\begin{abstract}
本项目为年产50万吨MTO工厂的初步设计。通过分析当前国内外MTO生产和研究现状,对生产工艺进行了选择论证。然后运用Aspen软件模拟初步的工艺流程,并通过对一系列工艺参数,如精馏塔的塔板数—产品纯度、进料塔板数—产品纯度、产品纯度—回流比、再沸器负荷—回流比等进行灵敏度分析,优化设备操作条件,提高工艺的合理性和经济性。本设计还针对工艺流程进行换热网络设计和对全局换热网络进行了优化和评估,通过内部流股之间相互换热以减少公用工程的消耗,最终优化后节约$79.4\%$的热公用工程资源和$73.7\%$的冷公用工程资源。本设计还运用水夹点技术优化了用水网络,根据水硬度分类处理水操作单元,并合理再生利用,使得本项目新鲜水用量和废水排放量达到最小,优化后的用水网络节约用水$53.59\%$。本设计对于MTO工厂的生产和设计建造具有一定的现实指导意义。\\

\keywords{\zihao{-4} 工厂\quad 设计\quad MTO \quad 工艺 \quad 水夹点  \quad 网络 \quad 控制}
\end{abstract}

\begin{englishabstract}

This project is the preliminary design of a MTO plant with an annual output of 500,000 tons of light olefins. Based on the current production and research situation all through the world, the production method was selected and demonstrated. Aspen software was used to simulate the preliminary process. Heat integration method was applied to optimize the heat exchange network. Rational heat exchange between process streams were suggested which resulted in the decreasing of utilities consumption and exchanger number. The heat integration leaded to energy saving of $79.4\%$ of heat utilities and $73.7\%$ of the cold utilities. In addition, the water pinch technology was also implemented to optimize the water network. The water operating unit was classified according to water hardness, with a reasonable recycling. The amount of fresh water consumption and wastewater emission was minimized. The optimized water network achieved $53.59\%$ water saving. Finally, a preliminary economic analysis to the entire project was estimated in order to get the project construction cost and profitability. In summary, this design is of some practical significance for the production and design of the MTO industry.

\englishkeywords{\zihao{-4} Plant design\;Sensitivity analysis  \; Energy balance\; calculation \; Water pinch  Dynamic control}
\end{englishabstract}


% \maketitlepage
% \makecover
% \makeabstract
% \tableofcontents
% \input{parts/examples/denotation}
% \mainmatter
% \maketitle
% \input{parts/examples/problemsolving}
% \input{parts/examples/mathpro}
% \chapter{绪论}
本章介绍了无线传感网的结构及其应用领域。分析规模化无线传感网的特点,对规模化传感网数据认证的需求和面临的挑战进行了分析。概述了本文的研究内容,并对文章的组织结构予以说明。
\section{本文研究背景和意义}
无线传感器网络(Wireless Sensor Networks,无线传感网)是一种特殊组织结构的移动自组网\upcite{c:sensor},在环境监测、工业控制、资源监控、智能家居、医疗保健和军事等各种领域都有广泛应用,有非常重要的地位和作用。
随着无线传感网技术的不断发展,很多应用进入了日常生活中,物联网技术也将成为未来发展的重要方向。
无线传感网的各种技术发展紧跟具体的应用需求,随着各种应用场景需要的安全性越来越高,安全问题也成为了阻碍无线传感网大规模发展的一个制约。

大范围监测在环境监测和军事侦察等诸多关系国家社会重大安全的领域都具有重要的地位和作用。在环境监测领域,往往面临范围野外受限甚至恶劣条件,在海洋等资源监测领域,水声通信等基础技术还不是很完善, 在军事侦察对抗领域更是要应对破坏攻击情况,传统的大范围实时监测机制和系统都难以得到有效部署,使用无线传感技术成为了最好的解决方案。规模化无线传感网因此应运而生,而且为满足大范围监测的需要,无线传感网的规模越来越大。

规模化无线传感网面临的安全威胁更多,攻击的影响更大,而且由于传感器节点的特点,传统的安全机制和协议无法直接适用于无线传感网,使得安全问题更加凸显。因此针对规模化无线传感网安全机制的研究成为了热门研究方向。

\subsection{无线传感网概述}
\subsubsection{无线传感网结构}
无线传感器节点被部署在目标监测区域,大量的传感器节点通过无线广播的方式,以一定的算法自组织成为一个多跳的无线网络。如图~\ref{fig:cluster}所示,是一个典型无线传感网的结构\upcite{c:cluster},由三部分组成:监测区域的传感器节点、与外部网络连接的网关或基站、远程数据中心。在监测区域的传感器节点一般通过算法组成若干的簇,每个簇通过簇头节点与其他簇或者基站通信,这样的方案节约了节点的能量。簇内节点收集到监测数据以后通过簇头节点的整合,形成报文通过一定的路径发送给基站,基站进一步通过外部网络设备,如互联网、卫星等将监测数据传输到远程数据中心。

\begin{figure}[htbp]
  \centering
  \includegraphics[width=5in]{cluster}
  \caption{无线传感网系统结构}
  \label{fig:cluster}
\end{figure}


无线传感网中,基站的计算和存储能力都比较强,
基站的功能可以是一个数据处理中心,向网络广播控制信息,从监测区域获取数据。
也可以是一个网络网关,负责数据向远程数据中心的传输。

\subsubsection{无线传感器节点结构}
传感器节点是无线传感网的基本组成单元,负责数据采集、发送等基本功能。
无线传感器节点一般仅具有很小的存储空间,较弱的计算能力,因此单个节点无法完成复杂的感知任务,需要大量的节点协同工作。

随着电子技术的发展,无线传感器节点的性能也有了很大的提升,如Crossbow公司研发的TelosB,CPU频率为8MHz,有10KB的RAM,使用2.4GHz无线电,能达到250Kbps数据传输,使用两节AAA电池(5号电池)供电。国产传感器节点典型的有美新的MEMSIC无线模块,工作频率可选433 MHz、868-915MHz或2.4GHz,拥有5年电池寿命,支持10-100米的发射范围,拥有19.2kbps-240kbps的数据传输速率。

\begin{figure}[htbp]
  \centering
  \includegraphics[width=5in]{node}
  \caption{无线传感网节点结构}
  \label{fig:node}
\end{figure}

这些传感器节点的设计原理基本相同,主要包括4个模块:传感模块、数据处理模块、无线通信模块和能量供应模块。
如图~\ref{fig:node}所示,是一个典型的无线传感器节点的结构图。传感模块主要负责从感知区域通过传感器获取数据,并将数据转化为适合进行网络传输的数字信号;数据处理模块主要包括处理和存储功能,负责控制传感器节点的运行,对传感模块获取的数据进行处理和存储,数据报文的整合与认证都是由数据处理模块完成,一般该模块需要嵌入式系统的支持,如UC Berkeley的开源嵌入式系统TinyOS\upcite{c:tinyos}等;无线通信模块负责与其他传感器节点或基站之间的通信,传感器节点一般使用内置天线进行数据收发;能量供应模块负责给其他模块供应能量,大部分传感器节点使用微型电池作为电源,因此能量非常有限。传感器节点中还包括一些负责定位、同步等功能的部件。

\subsubsection{无线传感网协议结构}

\begin{figure}[htbp]
  \centering
  \includegraphics[width=5in]{construction}
  \caption{无线传感网协议结构}
  \label{fig:construction}
\end{figure}
无线传感网的通信协议栈和相关网络管理技术是当前的主要研究内容,协议结构如图~\ref{fig:construction}所示。
因为无线传感网是面向特定需求的网络,因此针对不同的部署环境,不同的网络部署结构,要对通信协议栈进行优化,使能量消耗、抗节点损耗、抗攻击能力等适应传感网的应用需求。

类似于OSI网络模型,无线传感网的通信协议栈由物理层、数据链路层、网络层、传输层、应用层组成:

物理层:物理层是通信协议栈的最底层,主要功能是将数据调制成适合传输的数字信号,通过无线电、红外灯无线介质完成传感器节点的数据收发。

数据链路层:数据链路层负责装配数据帧,对数据帧进行MAC校验,进行差错控制,向网络层提供透明可靠的数据传输服务。

网络层:主要负责无线传感网中的路由功能,将数据通过有效路径传送到目标节点,向传输层提供端对端的数据传输服务。

传输层:传输层负责数据报文的传送和控制,为应用层提供可靠的传输服务,对网络进行流量控制,进行服务质量控制(QOS)。

应用层:直接为应用提供服务,提供相应的应用协议和服务接口。

传感网管理协议提供了拓扑管理、QOS管理、安全管理、能量管理和网络管理等功能,实现对无线传感网以及各个节点的监控和管理。
\subsubsection{无线传感网的应用前景}
分布式传感网在军事中的应用是无线传感网的雏形,随着电子技术的不断发展,传感器节点的性能不断提升,无线传感网各种协议的完善和发展,使无线传感网在环境监测、军事侦察、智能家居、智能公路等各个领域得到了大量的应用,其应用前景十分广泛。

\begin{compactitem}
  \item 环境监测:无线传感网能完成大范围监测的任务,在自然数据采集中发挥重要作用,尤其是海洋监测传感网和内陆水文传感网等应用领域。如Li 等人将无线传感网部署在水产养殖水域,对水环境数据进行检测\upcite{c:water}。
  \item 军事侦察:由于无线传感网具有自组网、部署简单、容许节点失效等特点,适合部署在危险的敌对区域,完成军事侦察、战场环境监测等任务,因而在军事领域有很大应用前景,是现代化电子战的重要战略武器。如美国海军将开发的自主分布式DADS(Deployable Autonomous Distributed System)用于沿海广大海域的警戒、反潜和反水雷\upcite{c:DADS}。
  \item 智能家居:智能家居是通过无线传感器将房间中的各种家电等设备连接起来,实现家居环境的监测以及远程控制,构建出智能的居住环境\upcite{c:homes}。
  \item 智能公路:通过部署在公路上的无线传感器节点以及车载传感器节点,共同组成智能公路传感网络,对交通状况实现自动监测,引导车流等,实现自动化的公路交通管理。
\end{compactitem}



\subsection{规模化无线传感网数据认证}
\subsubsection{规模化无线传感网的特点}
规模化无线传感网是为满足大范围监测的需要而产生的,如国内著名的绿野千传项目,在浙江省天目山建立的大规林业监测传感网,部署的自组织传感网节点超过2000个,网络中传输路径跳数超过 20 跳
\upcite{c:lvye}。
规模化无线传感网具有如下的特点:
\begin{enumerate}\setlength{\itemsep}{-\itemsep}
  \item 节点数量大,覆盖面积广,节点失效较为频繁,网络拓扑结构相对不稳定。
  \item 一般部署于恶劣区域,甚至是敌对攻击区域,恶意攻击的频度增加。
  \item 节点的计算和存储能力更为受限,网络的能量较为敏感,对机制的轻量化要求更突出。
\end{enumerate}

\subsubsection{规模化无线传感网的数据认证需求}
无线传感网中的认证包括身份认证和数据认证。身份认证是对网络中节点的合法身份的一种判定机制,是数据认证的基础。无线传感网数据认证主要包括两个方面:
\begin{compactitem}
  \item 数据来源合法性,主要以身份认证为基础,通过数据报文中的认证机制判定数据报文的来源的合法性。
  \item 数据完整性,通过数据认证的机制,确保节点收到的数据报文没有被非法进行篡改。
\end{compactitem}

在环境监测等领域,规模化传感网每天都会产生海量的感知数据。在军事侦察领域,随着侦察区域的扩大,侦察精度的提高,传感网感知的数据量飞速增长。尤其在实时监测场景,数据量大、传输实时性要求高,无线传感器节点的性能限制使得规模化传感网实现可靠传输具有非常的难度,合适的数据认证机制可以为其提供有力支持。在无线传感网中数据泄露、错误数据甚至虚假数据会对网络的安全造成重大影响。尤其在重要战略场景或军事场景,还要考虑破坏攻击的可能,因此数据认证更为安全攸关。
\subsubsection{规模化无线传感网数据认证面临的挑战}

复杂环境下数据高安全性要求对数据认证提出的挑战。实时监测传感网通常部署环境恶劣,而且缺乏基础设施的建设,由于自然环境和主动攻击等对节点的破坏,使节点的失效率很高,网络拓扑结构动态变化,数据传输质量不够稳定,而且存在突发大故障潜因,需要在容灾抗毁前提下进行数据认证,确保传输的安全性。

端对端传输为数据认证提出的挑战。完全依靠广播等数据传输机制,在规模化无线传感网中,传输效率过低,消耗的节点能量和通信资源过大,而且容易受到泛洪攻击的影响。有效利用规模化传感网中端对端数据传输,能够有效的保证传输效率。在端对端传输中,由于多跳传输的原因,当路径中出现妥协节点时,整条路径容易被攻破,从而造成数据传输被攻击,因而在多路径端对端的数据传输中,有效利用数据认证机制加强路径上的安全保障是安全传输的关键。

轻量级认证机制及其实现技术为数据认证提出的挑战。规模化无线传感网传输的数据量大,要求处理快捷。在节点资源能力受限,通信能耗受限的前提下,需要计算、存储、通信都轻量级的水平,保障网络安全、传输可靠性、高效性和数据可信,具有很大难度。传统的的认证机制使用的密码算法复杂度未达到轻量级,不适合规模化传感网网络资源受限的特点,我们需要设计适合实时性较高的规模无线传感网达到轻量级算法。

攻击对抗对数据认证提出的挑战。无线传感网一般部署在恶劣环境中,而且具有自组网络的多跳性、无中心性和自组织性等特征,致使其通信协议栈的各个层级都容易遭受到各种形式的攻击,我们需要设计能够适应有限节点能量,有限计算能力的数据认证算法,对抗各种攻击,保证无线传感网传输数据的来源合法性和完整性。

\section{本文研究内容}
本文根据规模化无线传感网的安全需求以及其特点,针对其数据认证关键技术展开研究,使用多节点联合的技术思路研究数据认证模型和机制,并设计实现了关键算法。
主要工作如下:

\begin{enumerate}\setlength{\itemsep}{-\itemsep}
  \item 提出了多跳长路径上多节点联合数据认证的模型,设计了多跳长路径上多节点联合数据认证协议,并设计了路径上节点关系的维护算法,对协议的安全性能进行了分析评价。
  \item 针对多跳长路径上多写点联合数据认证协议的不足,对算法进行了优化,提出了多路径抗节点失效机制和动态步长多节点联合数据认证机制,并对优化方案的安全性能进行了分析评价。
  \item 围绕多跳长路径多节点联合数据认证机制的需求,对密钥分配方案进行了深入研究,提出了基于单向hash链的密钥分配方案,并对认证中的MAC进行了研究,提出了适应数据认证机制需求的MAC码。
\end{enumerate}


\section{本文组织结构}
本文一共分为七章。

第一章\quad 绪论,介绍了课题的选题背景,描述了无线传感网的特点,介绍了无线传感网的相关安全技术,列出了本文的主要研究内容和本文组织结构。

第二章\quad 相关研究概述,本章首先对无线传感网的安全技术进行了概述,然后重点对数据认证和密钥分配两种安全技术进行了论述。

第三章\quad 多跳长路径上多节点联合数据认证,本章提出了无线传感网中多跳长路径多节点联合的数据认证模型,及其设计目标。
重点介绍了关键算法与协议的设计实现,对多节点联合数据认证机制的安全性能进行了分析评价。

第四章\quad 数据认证方案优化,本章针对多跳长路径上多节点联合数据认证进行了优化,提出了多路径抗节点失效和动态步长多节点联合数据认证两个优化方案,并对它们的安全性能进行了分析评价。

第五章\quad 密钥分配与MAC设计,本章对多节点联合数据认证中的密钥分配方案以及使用的MAC的设计进行了介绍,提出了基于单向hash链的密钥分配方案,以及适应多节点联合数据认证的MAC码。

第六章\quad 仿真实验与结果分析,本章在仿真平台上对多跳长路径多节点联合数据认证机制,以及其优化方案进行了仿真实验,对它们的安全性能结果进行了评价。

第七章\quad 总结与展望,本章对全文的工作做了总结,指出了数据认证机制现阶段的不足以及未来研究中需要研究及完善的地方。


% \include{parts/examples/chap02}
% \include{parts/examples/digitalexp}
% \include{parts/examples/code}
% \backmatter
% \listoffigures
% \listoftables
% \listofequations
% \bibliographystyle{ref/numeric} % ref/numeric,ref/author-year,plainnat,IEEEtran
% \bibliography{ref/refs}
% \begin{acknowledgements}
Thanks to all who have helped.
\end{acknowledgements}
% \begin{appendix}
%   \section{}
\subsection{批量下载图片bash脚本}\label{download_shell}
\begin{lstlisting}[caption=批量下载图片bash脚本,language=bash]
#!/bin/bash
MAX=\$2;
for ((MAX;MAX>0; MAX--));
        do wget -vv  --no-check-certificate "$1" -O "$MAX";
done;
\end{lstlisting}

\subsection{简单平均法灰度化图片}\label{gray_pic}
\begin{lstlisting}[caption=简单平均法灰度化图片,language=matlab]
pic=imread('../abc/1');
[h,w,d]=size(pic);
for m=1:h
    for n=1:w
        sum=0;
        for v=pic(m,n,:)
            sum=sum+v/3;
        end
        pic2(m,n)=sum;
    end
end
figure,imshow(pic),figure,imshow(pic2);
\end{lstlisting}

\subsection{增强对比度}\label{contrast}
\begin{lstlisting}[caption=增强对比度,language=matlab]
pic=imread('../icbc/1');
pic=rgb2gray(pic);%先灰度化处理
pic2=imadjust(pic,[0.3 0.4],[0 1]);%增强对比度
imcontrast(gcf);%使用imcontrast工具箱进行动态的对比度调整
figure,imshow(pic2),figure,imshow(pic);
\end{lstlisting}


\subsection{采用均值滤波的积分投影算法}\label{means_filter}
\begin{lstlisting}[caption=采用均值滤波的积分投影算法,language=matlab]
pic=imread('../test.bmp');
words_count=5;%已知识别的字符个数为5
pic=rgb2gray(pic);
pic=im2bw(pic);
[h,w,d]=size(pic);
row=[];%图像在x轴上的积分投影

%计算积分投影
for x=1:w
    row(x)=sum(pic(:,x,1));
end

%为了演示方便,我们另外给出滤波后的积分投影
row1=row;
%对原来的积分投影进行滤波,注意滤波后的积分投影会比原来的少两个数据,分
%别位于首尾 
for x=2:w-1
    new=sum(row1(x-1:x+1))/3;
    row1(x)=new;
end

%使用滤波后的积分投影进行分析,从峰值到谷值扫描
for std=max(row):-1:min(row1)
    blocks=[];%保存每一个谷坡
    inblock=0;%是否处于谷坡
    for x=1:length(row1)
        %如果从谷顶刚进入谷坡,则新建一个谷坡
        if (row1(x)<std && inblock==0)
            blocks=[blocks;x 0];
            inblock=1;
        %如果从谷坡爬回谷顶,则标记刚才的新建谷坡    
        elseif(row1(x)>=std && inblock==1)
            blocks(end)=x;
            inblock=0;
        end
    end
    %如果字符贴边上,则手动结束
    if inblock==1
        blocks(end)=length(row1);
    end
    [block_count,n]=size(blocks);
    %当谷坡数量与要识别的字符个数相等就退出扫描
    if block_count==words_count
        break;
    end
end

%绘制积分投影滤波前后的对比图
figure,subplot(2,1,1),hold on;
for x=1:block_count
    rectangle('FaceColor','c','EdgeColor','none'...
    ,'Position',[blocks(x,1) min(row) blocks(x,2)-blocks(x,1)
max(row)-min(row)-1]);
end

plot([0,length(row1)],[std,std],'g--');
plot(row,'b'),plot(row1,'r');
legend('scanning line','before filter','after filter');
subplot(2,1,2),imshow(pic);
for x=1:block_count
    line([blocks(x,1),blocks(x,1)],[h,0]);
    line([blocks(x,2),blocks(x,2)],[h,0]);
end
hold off;
\end{lstlisting}


\subsection{带权重的连通区域聚类法}\label{segmentation_with_weight}
\begin{lstlisting}[caption=带权重的连通区域聚类法,language=matlab]
pic=imread('../test.bmp');
%已知识别的字符个数为5
k=5;

%灰度并二值化
pic=rgb2gray(pic);
pic=im2bw(pic);

figure,subplot(2,1,1);
imshow(pic);%绘制原图

[height,width]=size(pic);
%采用4连通寻找图像连通域
[L,num]=bwlabel(~pic,8);

%计算连通域的质心
r=regionprops(L);

centroids=[];%质心
for i=1:length(r)
    centroids(i,:)=r(i).Centroid;
end

%根据连通域面积,加大质心的个数,以增加权重
for i=1:length(r)
    for x=2:r(i).Area
        centroids=[centroids;r(i).Centroid];
    end
end

%使用K-means把质心聚成k类
km=kmeans(centroids,k,'distance','sqEuclidean','start','uniform','replicates',50);

%在图像上标记k类区域
for i=1:length(r)
    L(find(L==i))=km(i)+num;
end

%彩色绘制k类区域
RGB =label2rgb(L);
subplot(2,1,2),imshow(RGB),hold on;

%绘制各个区域的质点
plot(centroids(1:length(r),1),centroids(1:length(r),2),'b*');

%绘制聚类区域的质点
for i=1:k
    M=zeros(height,width);
    M(find(L==i+num))=1;
    s=regionprops(M);
    plot(s.Centroid(1),s.Centroid(2),'c*');
end
legend('centroid of each connected object','centroid of each cluster');
hold off;
\end{lstlisting}

\subsection{Jitendra采样法}\label{jitendra}
\begin{lstlisting}[caption=演示程序,language=matlab]
clc;
close;
close all;
img=imread('../tmp/1.bmp');
img=img(:,:,1);
img=im2double(img);

%把图片的周围空出一个像数宽,方便提取边缘
[h,w]=size(img);
img=[ones(h,1) img ones(h,1)];
img=[ones(1,w+2);img;ones(1,w+2)];

n=100;
figure(1);
hold on;
subplot(1,3,1);
imshow(img);
title('original image');
img=rot90(img');
%提取边缘
[x,y,t,c]=bdry_extract(img);
subplot(1,3,2);
plot(x,y,'b.');
title('edge image');
axis([0,140,0,140]);
axis square; %长宽比例为 1
%使用Jitendra采样法进行采样
[xi,yi,ti]=get_samples(x,y,t,n);
subplot(1,3,3);
plot(xi,yi,'b+');
title(sprintf('after sample(n=\%d)',n));
axis([0,140,0,140]);
axis square; %长宽比例为 1
grid on;
hold off;
\end{lstlisting}


\begin{lstlisting}[caption=Jitendra采样法,language=matlab]
function [xi,yi,ti]=get_samples(x,y,t,nsamp)
N=length(x);
k=3;
Nstart=min(k*nsamp,N);

ind0=randperm(N);%生成一个1到N的乱序列
ind0=ind0(1:Nstart);
%将x,y打乱,但对应点保持不变
xi=x(ind0);
yi=y(ind0);
ti=t(ind0);
xi=xi(:);
yi=yi(:);
ti=ti(:);
%计算xi yi各点的距离
d2=dist2([xi yi],[xi yi]);
d2=d2+diag(Inf*ones(Nstart,1));

s=1;
while s
    % find closest pair
    [a,b]=min(d2);%寻找每一列的最小值a,并把其所在行数保存在b,a为单行
    [c,d]=min(a);%寻找a中最小的值c,并把其所在列数保存在d,矩阵d2最小值就为d
    I=b(d);
    J=d;
    % remove one of the points删除d2中的最小点,并删除xi,yi,ti相应的点
    xi(J)=[];
    yi(J)=[];
    ti(J)=[];
    d2(:,J)=[];
    d2(J,:)=[];
    if size(d2,1)==nsamp
        s=0;
    end
end
      
\end{lstlisting}

\begin{lstlisting}[caption=利用梯度来提取边缘,language=matlab]
function [x,y,t,c]=bdry_extract(V);
Vg=V; % if no smoothing is needed
%画等高线,就是轮廓
c=contourc(Vg,[.5 .5]);
%计算数值梯度
[G1,G2]=gradient(Vg);

%把c(1,:)里面不是0.5标记为1,是的标记为0,储存到fz
fz=c(1,:)~=0.5;
%把c(1,:)里面为0.5的,将其上下对应的两个数设置为NaN
c(:,find(~fz))=NaN;
%B为c,去掉c(1,:)里面为0.5后
B=c(:,find(fz));

npts=size(B,2);
t=zeros(npts,1);
for n=1:npts
    x0=round(B(1,n));
    y0=round(B(2,n));
    t(n)=atan2(G2(y0,x0),G1(y0,x0))+pi/2;
end

x=B(1,:)';
y=B(2,:)';

\end{lstlisting}

% \end{appendix}
% \end{document}
% \end{latex}
%
% \subsection{选项}
% \label{sec:option}
% \DescribeOption{language}
% 论文的主要语言(默认:中文)。可选:\option{chinese},\option{english}。决定了封面、标题、定理的语言。
% \DescribeOption{open}
% 正规出版物的章节出现在奇数页,也就是右手边的页面,这就是 \option{right},。在这种情况下,如果前一章的最后一页也是奇数,那么模板会自动生成一个纯粹的空白页。
% 提交的作业如果是电子稿的话,可以使用连续页,即使用\option{any}
% \DescribeOption{wide}
% 是否使用宽页面。如果生成作业的话,宽页面或许好看。
% \DescribeOption{awesomefont}
% 是否使用awesomefont图标。
%
% \subsection{字体配置}
% \label{sec:font-config}
% 使用\CTeX\ 默认字体配置
% \subsubsection{字体命令}
% \label{sec:fontcmds}
% \myentry{字体}
% \DescribeMacro{\songti}
% \DescribeMacro{\fangsong}
% \DescribeMacro{\heiti}
% \DescribeMacro{\kaishu}
% 用来切换宋体、仿宋、黑体、楷体四种基本字体。
% \myentry{字号}
% \DescribeMacro{\chuhao}
% \DescribeMacro{\xiaochu}
% \DescribeMacro{\yihao}
% \DescribeMacro{\xiaoyi}
% \DescribeMacro{\bahao}
% 定义字体大小,分别为
% \begin{center}
% \begin{tabular}{llllll}
% \toprule
% \cs{chuhao} & \cs{xiaochu} & \cs{yihao}  & \cs{xiaoyi}    & \cs{erhao}  & \cs{xiaoer}\\
% \cs{sanhao} & \cs{xiaosan} & \cs{sihao}  & \cs{banxiaosi} & \cs{xiaosi} & \cs{dawu}\\
% \cs{wuhao}  & \cs{xiaowu}  & \cs{liuhao} & \cs{xiaoliu}   & \cs{qihao}  & \cs{bahao}\\\bottomrule
% \end{tabular}
% \end{center}
% 使用方法为:\cs{command}\oarg{num},其中 command 为字号命令,num 为行距。比
% 如 \cs{xiaosi}[1.5] 表示选择小四字体,行距 1.5 倍。写作指南要求表格中的字体
% 是 \cs{dawu},模板已经设置好了。
% 对于英文,开发版中smallcaps默认使用了spinweradC字体。可以使用\cs{setmainfont}进行重新定义。
%
% \subsection{封面信息}
% 仿照parts/examples/cover.tex 进行设置
% \subsection{问求}
% 为问求特制了一些宏,具体可见parts/examples/problemsolving.tex 
% \subsection{表格}
% \begin{latex}
%  \figpf{parameter}{filename}
%  \figpfc{parameter}{filename}{caption}
% \end{latex}
% \subsection{图片}
% \begin{latex}
%  \tabncc{number per row}{content}{caption}
%  \tabnc{number per row}{content}
% \end{latex}
% \subsection{代码}
%预设了如下的lstlisting环境
% \begin{longtable}{ccccc}
% \toprule
% code & codedisplay & cplus & shell & commandshell \\
% verilog & python & & &\\  
% \bottomrule
% \end{longtable}
% \subsection{文字}
% \begin{latex}
% \href{link}{words} # 插入链接
% \magenta{品红色字}
% \CJKunderline{下划线字}
% \end{latex}
% 更多的预置宏包,可见\ref{sec:loadpkg}
%
%
% \section{致谢}
% 感谢以下宏包的作者为本宏包提供了借鉴:
% \begin{itemize}
%  \item 清华大学\thuthesis https://github.com/xueruini/thuthesis
%  \item 南京大学 NJUBachelor https://github.com/ZLCao/NJUBachelor
% \end{itemize}
% 
% \StopEventually{}
%
% \section{实现细节}
% \subsection{基本信息}
%    \begin{macrocode}
%<*cls>
\NeedsTeXFormat{LaTeX2e}
\ProvidesClass{njurepo}[2019/01/25 1.0.0 Nanjing University Report Template]
%    \end{macrocode}
%
% \subsection{定义选项}
% \label{sec:defoption}
% 使用kvoptions宏包进行选项设置
%    \begin{macrocode}
\hyphenation{NJU-repo}
\def\njurepo{\textsc{NJU}\-\textsc{repo}}
\def\thuthesis{\textsc{Thu}\-\textsc{Thesis}}
\def\version{1.0.1}
\RequirePackage{kvoptions}
\SetupKeyvalOptions{
    family=nju,
    prefix=nju@,
    setkeys=\kvsetkeys
}
\DeclareStringOption[chinese]{language}[chinese]
\DeclareStringOption[any]{open}[any]
\DeclareBoolOption{wide}
\DeclareBoolOption{color}
\DeclareBoolOption{draft}
\DeclareBoolOption{awesomefont}
\DeclareDefaultOption{\PassOptionsToClass{\CurrentOption}{ctexbook}}

\ProcessKeyvalOptions*
%    \end{macrocode}
%
% 检测选项是否合法
%    \begin{macrocode}
\newcommand\nju@validate@key[1]{%
  \@ifundefined{nju@\csname nju@#1\endcsname true}{%
    \ClassError{njurepo}{Invalid value '\csname nju#1\endcsname'}{}
    }{%
      \csname nju@\csname nju@#1\endcsname true\endcsname
    }
}
\newif\ifnju@chinese
\newif\ifnju@english
\nju@validate@key{language}
\newif\ifnju@any
\newif\ifnju@right
\nju@validate@key{open}
%    \end{macrocode}
% 
% 使用ctexbook宏包
%    \begin{macrocode}
\LoadClass[a4paper,openany,UTF8,zihao=-4,scheme=plain]{ctexbook}
%    \end{macrocode}
%
% \subsection{加载宏包}
% \label{sec:loadpkg}
% 用于开发的宏包
%    \begin{macrocode}
\RequirePackage{etoolbox}
\RequirePackage{ifxetex}
\RequirePackage{xparse}
%    \end{macrocode}
% 用于图片的宏包
%    \begin{macrocode}
\RequirePackage{graphicx}
\graphicspath{{figs/}}
\graphicspath{{figs/logo/}}
\RequirePackage[labelformat=simple]{subcaption}
\RequirePackage{pdfpages}
\includepdfset{fitpaper=true}
\RequirePackage{tikz,tikzducks}
\usetikzlibrary{decorations.pathmorphing,graphs,calc}
\RequirePackage{dirtree}
%    \end{macrocode}
% 用于表格的宏包
%    \begin{macrocode}
\RequirePackage{array}
\RequirePackage{longtable}
\RequirePackage{booktabs}
\RequirePackage{multirow}
\RequirePackage{bbding,stmaryrd}
\RequirePackage{tabularx}
\RequirePackage{diagbox}
\RequirePackage{makecell}
\RequirePackage{float}
%    \end{macrocode}
% 用于数学的宏包
%    \begin{macrocode}
\RequirePackage{CJKfntef}
\RequirePackage{amsmath}
\RequirePackage[amsmath, thmmarks, hyperref]{ntheorem}
\RequirePackage{physics}
%    \end{macrocode}
% 其它宏包
%    \begin{macrocode}
\RequirePackage[sort&compress]{natbib}
%    \end{macrocode}
%
% 超链接
%    \begin{macrocode}
\RequirePackage{hyperref}
\ifxetex
  \hypersetup{%
    CJKbookmarks=true}
\else
  \hypersetup{%
    unicode=true,
    CJKbookmarks=false}
\fi
\hypersetup{%
  linktoc=all,
  bookmarksnumbered=true,
  bookmarksopen=true,
  bookmarksopenlevel=1,
  breaklinks=true,
  colorlinks=false,
  plainpages=false,
  pdfborder=0 0 0}	
\urlstyle{same}
\def\UrlBreaks{%
  \do\/%
  \do\a\do\b\do\c\do\d\do\e\do\f\do\g\do\h\do\i\do\j\do\k\do\l%
     \do\m\do\n\do\o\do\p\do\q\do\r\do\s\do\t\do\u\do\v\do\w\do\x\do\y\do\z%
  \do\A\do\B\do\C\do\D\do\E\do\F\do\G\do\H\do\I\do\J\do\K\do\L%
     \do\M\do\N\do\O\do\P\do\Q\do\R\do\S\do\T\do\U\do\V\do\W\do\X\do\Y\do\Z%
  \do0\do1\do2\do3\do4\do5\do6\do7\do8\do9\do=\do/\do.\do:%
  \do\*\do\-\do\~\do\'\do\"\do\-}
\Urlmuskip=0mu plus 0.1mu
%    \end{macrocode}
%
% 页眉页脚设置
%    \begin{macrocode}
\RequirePackage{fancyhdr}
\RequirePackage{notoccite}	
%    \end{macrocode}
%
% \subsection{页面设置}
% 使用了thuthesis的非本科生默认配置。
%    \begin{macrocode}
\RequirePackage{geometry}
\ifnju@wide 
\geometry{
    a4paper, %210*297mm
    hcentering,
    ignoreall,
    nomarginpar,
    left=10mm,
    headheight=5mm,
    headsep=5mm,
    textheight=237mm,
    bottom=29mm,
    footskip=6mm
}\else
\geometry{
    a4paper, %210*297mm
    hcentering,
    ignoreall,
    nomarginpar,
    left=30mm,
    headheight=5mm,
    headsep=5mm,
    textheight=237mm,
    bottom=29mm,
    footskip=6mm
}
\fi
%    \end{macrocode}
%
% \subsection{主文档格式}
% \label{sec:mainbody}
%
% \begin{macro}{\cleardoublepage}
%    \begin{macrocode}
\let\nju@cleardoublepage\cleardoublepage
\newcommand{\nju@clearemptydoublepage}{%
  \clearpage{\pagestyle{nju@empty}\nju@cleardoublepage}}
\let\cleardoublepage\nju@clearemptydoublepage
%    \end{macrocode}
% \end{macro}
%
% \begin{macro}{\frontmatter}
% \begin{macro}{\mainmatter}
% \begin{macro}{\backmatter}
%    \begin{macrocode}
\renewcommand\frontmatter{%
    \ifnju@right\cleardoublepage\else\clearpage\fi
    \@mainmatterfalse
    \pagenumbering{Roman}
    \pagestyle{nju@empty}}
\renewcommand\mainmatter{%
    \ifnju@right\cleardoublepage\else\clearpage\fi
    \@mainmattertrue
    \pagenumbering{arabic}
    \pagestyle{nju@headings}}
\renewcommand\backmatter{%
    \ifnju@right\cleardoublepage\else\clearpage\fi
    \@mainmattertrue}
%    \end{macrocode}
% \end{macro}
% \end{macro}
% \end{macro}
%
% \subsection{字体与字号}
% \label{sec:font}
% \subsubsection{英文字体}
% 配置英文字体。
%    \begin{macrocode}
\newcommand\nju@fontset{\csname g__ctex_fontset_tl\endcsname}
\ifthenelse{\equal{\nju@fontset}{fandol}}{
  \setmainfont[
    Extension      = .otf,
    UprightFont    = *-regular,
    BoldFont       = *-bold,
    ItalicFont     = *-italic,
    BoldItalicFont = *-bolditalic,
  ]{texgyretermes}
  \setsansfont[
    Extension      = .otf,
    UprightFont    = *-regular,
    BoldFont       = *-bold,
    ItalicFont     = *-italic,
    BoldItalicFont = *-bolditalic,
  ]{texgyreheros}
  \setmonofont[
    Extension      = .otf,
    UprightFont    = *-regular,
    BoldFont       = *-bold,
    ItalicFont     = *-italic,
    BoldItalicFont = *-bolditalic,
    Scale          = MatchLowercase,
  ]{texgyrecursor}
}{
  \setmainfont{Times New Roman}
  \setsansfont{Arial}
  \ifthenelse{\equal{\nju@fontset}{mac}}{
    \setmonofont[Scale=MatchLowercase]{Menlo}
  }{
    \setmonofont[Scale=MatchLowercase]{Courier New}
  }
}
%    \end{macrocode}
%
% \subsubsection{数学环境字体}
% 配置数学字体(使用unicode-math)
%    \begin{macrocode}
\RequirePackage{unicode-math}
\unimathsetup{
  math-style = ISO,
  bold-style = ISO,
  nabla      = upright,
  partial    = upright,
}
\IfFontExistsTF{STIX2Math.otf}{
  \setmathfont[StylisticSet=8]{STIX2Math.otf}
  \setmathfont[range={scr,bfscr},StylisticSet=1]{STIX2Math.otf}
  \IfFontExistsTF{XITSMath-Regular.otf}{
    \setmathfont[range={\partial,\lbrace,\rbrace}]{XITSMath-Regular.otf}
  }{
    \setmathfont[range={\partial,\lbrace,\rbrace}]{xits-math.otf}
  }
}{
  \setmathfont[
    Extension    = .otf,
    BoldFont     = *bold,
    StylisticSet = 8,
  ]{xits-math}
  \setmathfont[range={cal,bfcal},StylisticSet=1]{xits-math.otf}
}
%    \end{macrocode}
%
% \subsubsection{数学环境符号}
% \begin{macro}{\ldots}
% 省略号一律居中,所以 \cs{ldots} 不再居于底部。
%    \begin{macrocode}
\ifnju@chinese
  \def\mathellipsis{\cdots}
\fi
%    \end{macrocode}
% \end{macro}
%
% \begin{macro}{\le}
% \begin{macro}{\ge}
% \begin{macro}{\leq}
% \begin{macro}{\geq}
% 小于等于号要使用倾斜的形式。
%    \begin{macrocode}
\protected\def\le{\leqslant}
\protected\def\ge{\geqslant}
\AtBeginDocument{%
  \renewcommand\leq{\leqslant}%
  \renewcommand\geq{\geqslant}%
}
%    \end{macrocode}
% \end{macro}
% \end{macro}
% \end{macro}
% \end{macro}
%
% \begin{macro}{\int}
% 积分号 \cs{int} 使用正体,并且上下限默认置于积分号上下两侧。
%    \begin{macrocode}
\removenolimits{%
  \int\iint\iiint\iiiint\oint\oiint\oiiint
  \intclockwise\varointclockwise\ointctrclockwise\sumint
  \intbar\intBar\fint\cirfnint\awint\rppolint
  \scpolint\npolint\pointint\sqint\intlarhk\intx
  \intcap\intcup\upint\lowint
}
%    \end{macrocode}
% \end{macro}
%
% \begin{macro}{\Re}
% \begin{macro}{\Im}
% \begin{macro}{\nabla}
% 实部、虚部操作符使用罗马体 $\mathrm{Re}$、$\mathrm{Im}$ 而不是 fraktur 体
% $\Re$、$\Im$。\cs{nabla} 使用粗正体。
%    \begin{macrocode}
\AtBeginDocument{%
  \renewcommand{\Re}{\operatorname{Re}}%
  \renewcommand{\Im}{\operatorname{Im}}%
  \renewcommand\nabla{\mbfnabla}%
}
%    \end{macrocode}
% \end{macro}
% \end{macro}
% \end{macro}
%
% \begin{macro}{\bm}
% \begin{macro}{\boldsymbol}
% 兼容旧的粗体命令:\pkg{bm} 的 \cs{bm} 和 \pkg{amsmath} 的 \cs{boldsymbol}。
%    \begin{macrocode}
\newcommand\bm{\symbf}
\renewcommand\boldsymbol{\symbf}
%    \end{macrocode}
% \end{macro}
% \end{macro}
%
% \begin{macro}{\square}
% 兼容 \pkg{amssymb} 中的命令。
%    \begin{macrocode}
\newcommand\square{\mdlgwhtsquare}
%    \end{macrocode}
% \end{macro}
%
% 允许太长的公式断行、分页等。
%    \begin{macrocode}
\allowdisplaybreaks[4]
\renewcommand\theequation{\ifnum \c@chapter>\z@ \thechapter-\fi\@arabic\c@equation}
%    \end{macrocode}
%
% 公式距前后文的距离由 4 个参数控制,参见 \cs{normalsize} 的定义。
%    \begin{macrocode}
\def\make@df@tag{\@ifstar\nju@make@df@tag@@\make@df@tag@@@}
\def\nju@make@df@tag@@#1{\gdef\df@tag{\nju@maketag{#1}\def\@currentlabel{#1}}}
\def\nju@maketag#1{\maketag@@@{(\ignorespaces #1\unskip\@@italiccorr)}}
\def\tagform@#1{\maketag@@@{(\ignorespaces #1\unskip\@@italiccorr)\equcaption{#1}}}
%    \end{macrocode}
% 修改 \cs{tagform} 会影响 \cs{eqref}。
%    \begin{macrocode}
\renewcommand{\eqref}[1]{\textup{(\ref{#1})}}
%    \end{macrocode}
%
% \subsubsection{中文字体}
% \pkg{ctex}在微软下使用雅黑字体,在macOS下使用苹方字体。这里不做更改。
%
% \subsubsection{字号}
% WORD 中的字号对应该关系如下(1bp = 72.27/72 pt):
% \begin{center}
% \begin{tabular}{llll}
% \toprule
% 初号 & 42bp & 14.82mm & 42.1575pt \\
% 小初 & 36bp & 12.70mm & 36.135 pt \\
% 一号 & 26bp & 9.17mm & 26.0975pt \\
% 小一 & 24bp & 8.47mm & 24.09pt \\
% 二号 & 22bp & 7.76mm & 22.0825pt \\
% 小二 & 18bp & 6.35mm & 18.0675pt \\
% 三号 & 16bp & 5.64mm & 16.06pt \\
% 小三 & 15bp & 5.29mm & 15.05625pt \\
% 四号 & 14bp & 4.94mm & 14.0525pt \\
% 小四 & 12bp & 4.23mm & 12.045pt \\
% 五号 & 10.5bp & 3.70mm & 10.59375pt \\
% 小五 & 9bp & 3.18mm & 9.03375pt \\
% 六号 & 7.5bp & 2.56mm & \\
% 小六 & 6.5bp & 2.29mm & \\
% 七号 & 5.5bp & 1.94mm & \\
% 八号 & 5bp & 1.76mm & \\\bottomrule
% \end{tabular}
% \end{center}
%
% \begin{macro}{\normalsize}
% 默认正文小四号 (12bp) 字,行距为固定值 20 bp。
%    \begin{macrocode}
\renewcommand\normalsize{%
  \@setfontsize\normalsize{12bp}{20bp}%
  \abovedisplayskip=12bp \@plus 2bp \@minus 2bp
  \abovedisplayshortskip=12bp \@plus 2bp \@minus 2bp
  \belowdisplayskip=\abovedisplayskip
  \belowdisplayshortskip=\abovedisplayshortskip}
%    \end{macrocode}
% \end{macro}
%
% \begin{macro}{\nju@def@fontsize}
% 根据习惯定义字号。用法:
% \cs{nju@def@fontsize}\marg{字号名称}\marg{磅数}
%
% 避免了字号选择和行距的紧耦合。所有字号定义时为单倍行距,并提供选项指定行距倍数。
%    \begin{macrocode}
\def\nju@def@fontsize#1#2{%
  \expandafter\newcommand\csname #1\endcsname[1][1.3]{%
    \fontsize{#2}{##1\dimexpr #2}\selectfont}}
%    \end{macrocode}
% \end{macro}
%
% \begin{macro}{\chuhao}
% \begin{macro}{\xiaochu}
% \begin{macro}{\yihao}
% \begin{macro}{\xiaoyi}
% \begin{macro}{\erhao}
% \begin{macro}{\xiaoer}
% \begin{macro}{\sanhao}
% \begin{macro}{\xiaosan}
% \begin{macro}{\sihao}
% \begin{macro}{\banxiaosi}
% \begin{macro}{\xiaosi}
% \begin{macro}{\dawu}
% \begin{macro}{\wuhao}
% \begin{macro}{\xiaowu}
% \begin{macro}{\liuhao}
% \begin{macro}{\xiaoliu}
% \begin{macro}{\qihao}
% \begin{macro}{\bahao}
% 一组字号定义。
%    \begin{macrocode}
\nju@def@fontsize{chuhao}{42bp}
\nju@def@fontsize{xiaochu}{36bp}
\nju@def@fontsize{yihao}{26bp}
\nju@def@fontsize{xiaoyi}{24bp}
\nju@def@fontsize{erhao}{22bp}
\nju@def@fontsize{xiaoer}{18bp}
\nju@def@fontsize{sanhao}{16bp}
\nju@def@fontsize{xiaosan}{15bp}
\nju@def@fontsize{sihao}{14bp}
\nju@def@fontsize{banxiaosi}{13bp}
\nju@def@fontsize{xiaosi}{12bp}
\nju@def@fontsize{dawu}{11bp}
\nju@def@fontsize{wuhao}{10.5bp}
\nju@def@fontsize{xiaowu}{9bp}
\nju@def@fontsize{liuhao}{7.5bp}
\nju@def@fontsize{xiaoliu}{6.5bp}
\nju@def@fontsize{qihao}{5.5bp}
\nju@def@fontsize{bahao}{5bp}
%    \end{macrocode}
% \end{macro}
% \end{macro}
% \end{macro}
% \end{macro}
% \end{macro}
% \end{macro}
% \end{macro}
% \end{macro}
% \end{macro}
% \end{macro}
% \end{macro}
% \end{macro}
% \end{macro}
% \end{macro}
% \end{macro}
% \end{macro}
% \end{macro}
% \end{macro}
%
%
% \subsubsection{中文标点}
%
% \newcommand\unicodechar[1]{U+#1(\symbol{"#1})}
% 由于 Unicode 的一些标点符号是中西文混用的:
% \unicodechar{00B7}、
% \unicodechar{2013}、
% \unicodechar{2014}、
% \unicodechar{2018}、
% \unicodechar{2019}、
% \unicodechar{201C}、
% \unicodechar{201D}、
% \unicodechar{2025}、
% \unicodechar{2026}、
% \unicodechar{2E3A},
% 所以要根据语言设置正确的字体。
% \footnote{\url{https://github.com/CTeX-org/ctex-kit/issues/389}}
% 所以要根据语言设置正确的字体。
%    \begin{macrocode}
\newcommand\nju@setchinese{%
  \xeCJKResetPunctClass
}
\newcommand\nju@setenglish{%
  \xeCJKDeclareCharClass{HalfLeft}{"2018, "201C}%
  \xeCJKDeclareCharClass{HalfRight}{
    "00B7, "2019, "201D, "2013, "2014, "2025, "2026, "2E3A,
  }%
}
\newcommand\nju@setdefaultlanguage{%
  \ifnju@chinese
    \nju@setchinese
  \else
    \nju@setenglish
  \fi
}
%    \end{macrocode}
%
% \subsection{局部设置}
% \subsubsection{页眉页脚}
% \label{sec:headerfooter}
%
% 定义页眉和页脚样式。
% \begin{macro}{\ps@nju@empty}
% \begin{macro}{\ps@nju@plain}
% \begin{macro}{\ps@nju@headings}
% \begin{itemize}
% \item \texttt{nju@empty}:页眉页脚都没有
% \item \texttt{nju@plain}:只显示页脚的页码。\cs{chapter} 自动调用
% \cs{thispagestyle\{nju@plain\}}。
% \item \texttt{nju@headings}:页眉页脚同时显示
% \end{itemize}
%    \begin{macrocode}
\fancypagestyle{nju@empty}{%
  \fancyhf{}
  \renewcommand{\headrulewidth}{0pt}
  \renewcommand{\footrulewidth}{0pt}}
\fancypagestyle{nju@plain}{%
  \fancyhead{}
  \fancyfoot[C]{\xiaowu\thepage}
  \renewcommand{\headrulewidth}{0pt}
  \renewcommand{\footrulewidth}{0pt}}
\fancypagestyle{nju@headings}{%
  \fancyhead{}
  \fancyhead[C]{\wuhao\normalfont\leftmark}
  \fancyfoot{}
  \fancyfoot[C]{\wuhao\thepage}
  \renewcommand{\headrulewidth}{0.4pt}
  \renewcommand{\footrulewidth}{0pt}}
%    \end{macrocode}
% \end{macro}
% \end{macro}
% \end{macro}
%
% \subsubsection{段落}
% \label{sec:paragraph}
%
% 全文首行缩进 2 字符,标点符号用全角
%    \begin{macrocode}
\ctexset{%
  punct=quanjiao,
  space=auto,
  autoindent=true}
%    \end{macrocode}
%
% \subsubsection{列表}
% 利用 \pkg{enumitem} 命令调整默认列表环境间的距离,以符合中文习惯。
%    \begin{macrocode}
\RequirePackage[shortlabels]{enumitem}
\RequirePackage{environ}
\setlist{nosep}
%    \end{macrocode}
%
%
% \subsubsection{脚注}
% \label{sec:footnote}
% 脚注符合中文习惯,数字带圈。
%    \begin{macrocode}
\ifthenelse{\equal{\nju@fontset}{mac}}{
  \newfontfamily\nju@circlefont{Songti SC Light}
}{
  \ifthenelse{\equal{\nju@fontset}{windows}}{
    \newfontfamily\nju@circlefont{SimSun}
  }{
    \IfFontExistsTF{XITS-Regular.otf}{
      \newfontfamily\nju@circlefont{XITS-Regular.otf}
    }{
      \newfontfamily\nju@circlefont{xits-regular.otf}
    }
  }
}
\def\nju@textcircled#1{%
  \ifnum\value{#1} >9%
    \ClassError{njurepo}%
      {Too many footnotes in this page.}{Keep footnote less than 10.}%
  \fi
  {\nju@circlefont\symbol{\numexpr\value{#1}+"245F\relax}}%
}
\renewcommand{\thefootnote}{\nju@textcircled{footnote}}
\renewcommand{\thempfootnote}{\nju@textcircled{mpfootnote}}
%    \end{macrocode}
%
% 定义脚注分割线,字号(宋体小五),以及悬挂缩进(1.5字符)。
%    \begin{macrocode}
\def\footnoterule{\vskip-3\p@\hrule\@width0.3\textwidth\@height0.4\p@\vskip2.6\p@}
\let\nju@footnotesize\footnotesize
\renewcommand\footnotesize{\nju@footnotesize\xiaowu[1.5]}
%\footnotemargin1.5em\relax
%    \end{macrocode}
%
% \cs{@makefnmark} 默认是上标样式,而在脚注部分要求为正文大小。利用\cs{patchcmd} 动态调整 \cs{@makefnmark} 的定义。
%    \begin{macrocode}
\let\nju@makefnmark\@makefnmark
\def\nju@@makefnmark{\hbox{{\normalfont\@thefnmark}}}
\pretocmd{\@makefntext}{\let\@makefnmark\nju@@makefnmark}{}{}
\apptocmd{\@makefntext}{\let\@makefnmark\nju@makefnmark}{}{}
%    \end{macrocode}
%
%
% \subsubsection{定理环境}
% \label{sec:equation}
%
% 定理标题使用黑体,正文使用宋体,冒号隔开。
%    \begin{macrocode}
\theorembodyfont{\normalfont}
\theoremheaderfont{\normalfont\heiti}
\theoremsymbol{\ensuremath{\square}}
\newtheorem*{proof}{证明}
\theoremstyle{plain}
\theoremsymbol{}
\theoremseparator{:}
\ifnju@chinese
  \newcommand\nju@assumption@name{假设}
  \newcommand\nju@definition@name{定义}
  \newcommand\nju@proposition@name{命题}
  \newcommand\nju@lemma@name{引理}
  \newcommand\nju@theorem@name{定理}
  \newcommand\nju@axiom@name{公理}
  \newcommand\nju@corollary@name{推论}
  \newcommand\nju@exercise@name{练习}
  \newcommand\nju@example@name{例}
  \newcommand\nju@remark@name{注释}
  \newcommand\nju@problem@name{问题}
  \newcommand\nju@conjecture@name{猜想}
  \newcommand\nju@solution@name{解}
\else
  \newcommand\nju@assumption@name{Assumption}
  \newcommand\nju@definition@name{Definition}
  \newcommand\nju@proposition@name{Proposition}
  \newcommand\nju@lemma@name{Lemma}
  \newcommand\nju@theorem@name{Theorem}
  \newcommand\nju@axiom@name{Axiom}
  \newcommand\nju@corollary@name{Corollary}
  \newcommand\nju@exercise@name{Exercise}
  \newcommand\nju@example@name{Example}
  \newcommand\nju@remark@name{Remark}
  \newcommand\nju@problem@name{Problem}
  \newcommand\nju@conjecture@name{Conjecture}
  \newcommand\nju@solution@name{Solution}
\fi
\theoremheaderfont{\bfseries}
\newtheorem{assumption}{\nju@assumption@name}[chapter]
\newtheorem{definition}{\nju@definition@name}[chapter]
\newtheorem{proposition}{\nju@proposition@name}[chapter]
\newtheorem{lemma}{\nju@lemma@name}[chapter]
\newtheorem{theorem}{\nju@theorem@name}[chapter]
\newtheorem{axiom}{\nju@axiom@name}[chapter]
\newtheorem{corollary}{\nju@corollary@name}[chapter]
\newtheorem{exercise}{\nju@exercise@name}[chapter]
\newtheorem{example}{\nju@example@name}[chapter]
\newtheorem{remark}{\nju@remark@name}[chapter]
\newtheorem{problem}{\nju@problem@name}[chapter]
\newtheorem{conjecture}{\nju@conjecture@name}[chapter]
\newtheorem{solution}{\nju@solution@name}[chapter]

%\RequirePackage{microtype}
\ifnju@chinese
\newcommand{\promisewords}{请独立完成作业,不得抄袭。\\若参考了其它资料,请给出引用。\\鼓励讨论,但需独立书写解题过程。}
\else
\newcommand{\promisewords}{I promise this work is done on my own with no plagiarism.}
\fi
\newcommand{\pshw}{\section*{\scshape Part I\ \ \ Homework}}
\newcommand{\pscr}{\section*{\scshape Part II\ \ \ Correction}}
\newcommand{\psfb}{\section*{\scshape Part III\ \ \ Feedback}}
\newcommand{\Hrule}{\noindent\rule{\linewidth}{0.5mm}}

\ifnju@awesomefont
\RequirePackage{awesomefont}
\fi

\theorempostwork{\vspace{-0.5cm}\Hrule}
\newtheorem*{pssolution}{\ifnju@awesomefont\faPencilSquareO\ \fi\nju@solution@name}
\RequirePackage[listings]{tcolorbox}
\newtcolorbox{ps@problem}[1]{fonttitle=\bfseries,title=#1,before skip=0.5cm, after skip=-0.5cm}
\newenvironment{psproblem}[1][]{
    \begin{ps@problem}{\ifnju@awesomefont\faQuestionCircle\ \fi\nju@problem@name\ #1}
}{
    \end{ps@problem}
}
%
% \subsubsection{浮动对象}
% \label{sec:float}
% 设置浮动对象和文字之间的距离
%    \begin{macrocode}
\setlength{\floatsep}{20bp \@plus4pt \@minus1pt}
\setlength{\intextsep}{20bp \@plus4pt \@minus2pt}
\setlength{\textfloatsep}{20bp \@plus4pt \@minus2pt}
\setlength{\@fptop}{0bp \@plus1.0fil}
\setlength{\@fpsep}{12bp \@plus2.0fil}
\setlength{\@fpbot}{0bp \@plus1.0fil}
%    \end{macrocode}
%
% 下面这组命令使浮动对象的缺省值稍微宽松一点,从而防止幅度对象占据过多的文本页面,
% 也可以防止在很大空白的浮动页上放置很小的图形。
%    \begin{macrocode}
\renewcommand{\textfraction}{0.15}
\renewcommand{\topfraction}{0.85}
\renewcommand{\bottomfraction}{0.65}
\renewcommand{\floatpagefraction}{0.60}
%    \end{macrocode}
%
% 定制浮动图形和表格标题样式
% \begin{itemize}
%   \item 图表标题字体为 11pt, 这里写作大五号
%   \item 去掉图表号后面的冒号。图序与图名文字之间空一个汉字符宽度。
%   \item 图:caption 在下,段前空 6 磅,段后空 12 磅
%   \item 表:caption 在上,段前空 12 磅,段后空 6 磅
% \end{itemize}
%
%    \begin{macrocode}
\let\old@tabular\@tabular
\def\nju@tabular{\dawu[1.5]\old@tabular}
\DeclareCaptionLabelFormat{nju}{{\dawu[1.5]\normalfont #1~#2}}
\DeclareCaptionLabelSeparator{nju}{\hspace{1em}}
\DeclareCaptionFont{nju}{\dawu[1.5]}
\captionsetup{labelformat=nju,labelsep=nju,font=nju,skip=6bp}
\captionsetup[table]{position=top}
\captionsetup[figure]{position=bottom}
\captionsetup[sub]{font=nju}
\renewcommand{\thesubfigure}{(\alph{subfigure})}
\renewcommand{\thesubtable}{(\alph{subtable})}
% \renewcommand{\p@subfigure}{:}
%    \end{macrocode}
% 我们采用 \pkg{longtable} 来处理跨页的表格。同样我们需要设置其默认字体为五号。
%    \begin{macrocode}
\let\nju@LT@array\LT@array
\def\LT@array{\dawu[1.5]\nju@LT@array} % set default font size
%    \end{macrocode}
%
% \begin{macro}{\hlinewd}
% 简单的表格使用三线表推荐用 \cs{hlinewd}。如果表格比较复杂还是用 \pkg{booktabs} 的命令好一些。
%    \begin{macrocode}
\def\hlinewd#1{%
  \noalign{\ifnum0=`}\fi\hrule \@height #1 \futurelet
    \reserved@a\@xhline}
%    \end{macrocode}
% \end{macro}
%
%
% \subsubsection{章节标题}
% \label{sec:theor}
%    \begin{macrocode}
\ifnju@chinese
  \ctexset{%
    chapter/name={第,章},
    appendixname=附录,
    contentsname={目\hspace{\ccwd}录},
    listfigurename=插图索引,
    listtablename=表格索引,
    figurename=图,
    tablename=表,
    bibname=参考文献,
    indexname=索引,
  }
  \newcommand\listequationname{公式索引}
  \newcommand\equationname{公式}
\else
  \newcommand\listequationname{List of Equations}
  \newcommand\equationname{Equation}
\fi
\newcommand{\cabstractname}{摘\hspace{\ccwd}要}
\newcommand{\eabstractname}{Abstract}
\let\CJK@todaysave=\today
\def\CJK@todaysmall@short{\the\year 年 \the\month 月}
\def\CJK@todaysmall{\the\year 年 \the\month 月 \the\day 日}
\def\CJK@todaybig@short{\zhdigits{\the\year}年\zhnumber{\the\month}月}
\def\CJK@todaybig{\zhdigits{\the\year}年\zhnumber{\the\month}月\zhnumber{\the\day}日}
\def\CJK@today{\CJK@todaysmall}
\renewcommand\today{\CJK@today}
\newcommand\CJKtoday[1][1]{%
  \ifcase#1\def\CJK@today{\CJK@todaysave}
    \or\def\CJK@today{\CJK@todaysmall}
    \or\def\CJK@today{\CJK@todaybig}
  \fi}
%    \end{macrocode}
%
% \pkg{fancyhdr} 定义页眉页脚很方便,但是有一个非常隐蔽的坑。通过 \pkg{fancyhdr}
% 定义的样式在第一次被调用时会修改 \cs{chaptermark},这会导致页眉信息错误(多余
% 章号并且英文大写)。这是因为在原始的 \file{book.cls} 中定义如下(大意):
% \begin{latex}
% \newcommand\chaptername{Chapter}
% \newcommand\@chapapp{\chaptername}
% \def\chaptermark#1{
%   \markboth{\MakeUppercase{\@chapapp\ \thechapter}}{}}
% \end{latex}
% 很显然这个 \cs{\@chapapp} 不适合中文,因此我们使用\cs{CTEXthechapter}(
% 如,“第 x 章”),同时会将 \cs{MakeUppercase} 去掉。也就是说我们会做如下动作:
% \begin{latex}
% \renewcommand{\chaptermark}[1]{\@mkboth{\CTEXthechapter\hskip\ccwd#1}{}}
% \end{latex}
% 但,\pkg{fancyhdr} 不知何故在 \cs{ps@fancy} 中对 \cs{chaptermark} 进行重定义
% (其实一模一样),而这个 \cs{ps@fancy} 会在 \cs{fancypagestyle} 中使用,如下:
% \begin{latex}
% \newcommand{\fancypagestyle}[2]{%
%   \@namedef{ps@#1}{\let\fancy@gbl\relax#2\relax\ps@fancy}}
% \end{latex}
% 这样的话,\cs{ps@fancy} 会在 \pkg{fancyhdr} 定义的任何样式首次样被激活时调用,从
% 而覆盖我们的 \cs{chaptermark} 定义(后续样式再激活不会重复覆盖)。所以我们采用如下
% 方法解决:
%    \begin{macrocode}
\AtBeginDocument{%
  \pagestyle{nju@empty}
  \renewcommand{\chaptermark}[1]{\@mkboth{\CTEXthechapter\hskip\ccwd#1}{}}}
%    \end{macrocode}
%
% 各级标题格式设置。
% \begin{description}
% \item[chapter] 章序号与章名之间空一个汉字符 黑体三号字,居中书写,单倍行距,段
%   前空 24 磅,段后空 18 磅。本科要求:段前段后间距 30/20 pt,行距 20pt。但正文
%   章节 30pt 的话和样例效果不一致。
% \item[section] 一级节标题,例如:\fbox{2.1 实验装置与实验方法}。节标题序号与标
%   题名之间空一个汉字符(下同)。采用黑体四号(14pt)字居左书写,行距为固定
%   值 20 磅,段前空 24 磅,段后空 6 磅。本科:25/12 pt,行距 18pt。
% \item[subsection] 二级节标题,例如:\fbox{2.1.1 实验装置}。采用黑体 13pt 字居左
%   书写,行距为固定值 20 磅,段前空 12 磅,段后空 6 磅。本科:中文黑体 12pt 字,
%   英文 13pt 字,段间距 12/6 pt,行距 15pt。
% \item[subsubsection] 三级节标题,例如:\fbox{2.1.2.1 归纳法}。采用黑体小四号
%   (12pt)字居左书写,行距为固定值 20 磅,段前空 12 磅,段后空 6 磅。
%
% \end{description}
%    \begin{macrocode}
\newcommand\nju@chapter@titleformat[1]{%
    \ifthenelse%
      {\equal{#1}{\eabstractname}}%
      {\bfseries #1}%
      {#1}%
  }
\ctexset{%
  chapter={
    afterindent=true,
    pagestyle={nju@headings},
    beforeskip={9bp},
    aftername=\hskip\ccwd,
    afterskip={24bp},
    format={\centering\sffamily\sanhao[1]},
    nameformat=\relax,
    numberformat=\relax,
    titleformat=\nju@chapter@titleformat,
    lofskip=0pt,
    lotskip=0pt,
  },
  section={
    afterindent=true,
    beforeskip={24bp\@plus 1ex \@minus .2ex},
    afterskip={6bp\@plus .2ex},
    format={\sffamily\sihao[1.429]},
  },
  subsection={
    afterindent=true,
    beforeskip={16bp\@plus 1ex \@minus .2ex},
    afterskip={6bp \@plus .2ex},
    format={\sffamily\banxiaosi[1.538]},
    numberformat={\sffamily\banxiaosi[1.538]},
  },
  subsubsection={
    afterindent=true,
    beforeskip={16bp\@plus 1ex \@minus .2ex},
    afterskip={6bp \@plus .2ex},
    format={\sffamily\xiaosi[1.667]},
  },
  paragraph/afterindent=true,
  subparagraph/afterindent=true}
%    \end{macrocode}
%
% \begin{macro}{\nju@chapter*}
% 默认的 \cs{chapter*} 很难同时满足研究生院和本科生的论文要求。本科论文要求所有的
% 章都出现在目录里,比如摘要、Abstract、主要符号表等,所以可以简单的扩展默
% 认\cs{chapter*} 实现这个目的。但是研究生又不要这些出现在目录中,而且致谢和声明
% 部分的章名、页眉和目录都不同,所以定义一个灵活的 \cs{nju@chapter*} 专门处理这些
% 要求。
%
% \cs{nju@chapter*}\oarg{tocline}\marg{title}\oarg{header}: tocline 是出现在目录
% 中的条目,如果为空则此 chapter 不出现在目录中,如果省略表示目录出现 title;
% title 是章标题;header 是页眉出现的标题,如果忽略则取 title。通过这个宏我才真
% 正体会到 \TeX\ macro 的力量!
%    \begin{macrocode}
\newcounter{nju@bookmark}
\NewDocumentCommand\nju@chapter{s o m o}{
  \IfBooleanF{#1}{%
    \ClassError{njurepo}{You have to use the star form: \string\nju@chapter*}{}
  }%
  \ifnju@right\cleardoublepage\else\clearpage\fi\phantomsection%
  \IfValueTF{#2}{%
    \ifthenelse{\equal{#2}{}}{%
      \addtocounter{nju@bookmark}\@ne
      \pdfbookmark[0]{#3}{njuchapter.\thenju@bookmark}
    }{%
      \addcontentsline{toc}{chapter}{#3}
    }
  }{%
    \addcontentsline{toc}{chapter}{#3}
  }%
  \ctexset{chapter/beforeskip=25bp}
  \chapter*{#3}%
  \ctexset{chapter/beforeskip=15bp}
  \IfValueTF{#4}{%
    \ifthenelse{\equal{#4}{}}
    {\@mkboth{}{}}
    {\@mkboth{#4}{#4}}
  }{%
    \@mkboth{#3}{#3}
  }
}
%    \end{macrocode}
% \end{macro}
%
%
% \subsubsection{目录}
% \label{sec:toc}
% 最多 4 层,即: x.x.x.x,对应的命令和层序号分别是:
% \cs{chapter}(0), \cs{section}(1), \cs{subsection}(2), \cs{subsubsection}(3)。
%    \begin{macrocode}
\setcounter{secnumdepth}{3}
\setcounter{tocdepth}{2}
%    \end{macrocode}
%
% 每章标题行前空 6 磅,后空 0 磅。章节名中英文用 Arial 字体,页码仍用 Times。
% \begin{macro}{\tableofcontents}
%    \begin{macrocode}
\renewcommand\tableofcontents{%
  \nju@chapter*[]{\contentsname}
  \xiaosi[1.65]\@starttoc{toc}\normalsize}
%    \end{macrocode}
% 调整目录样式
%    \begin{macrocode}
\def\@pnumwidth{2em}
\def\@tocrmarg{\@pnumwidth}
\def\@dotsep{1}
\renewcommand*\l@chapter[2]{%
  \ifnum \c@tocdepth >\m@ne
    \addpenalty{-\@highpenalty}%
    \vskip 4bp \@plus\p@
    \setlength\@tempdima{4em}%
    \begingroup
      \parindent \z@ \rightskip \@pnumwidth
      \parfillskip -\@pnumwidth
      \leavevmode
      \advance\leftskip\@tempdima
      \hskip -\leftskip
      {#1}%
      \leaders\hbox{$\m@th\mkern \@dotsep mu\hbox{.}\mkern \@dotsep mu$}\hfill%
      \nobreak{#2}\par
      \penalty\@highpenalty
    \endgroup
  \fi}

\patchcmd{\@dottedtocline}{\hb@xt@\@pnumwidth}{\hbox}{}{}
\renewcommand*\l@section{%
  \@dottedtocline{1}{\ccwd}{2.1em}}
\renewcommand*\l@subsection{%
  \@dottedtocline{2}{2\ccwd}{3em}}
\renewcommand*\l@subsubsection{%
  \@dottedtocline{3}{3.5em}{3.8em}}
%    \end{macrocode}
% \end{macro}
%
% \subsection{附加页面}
% \label{sec:etc}
%
% \subsubsection{封面}
% \label{sec:cover}
% 定义封面参数。
%    \begin{macrocode}
\def\nju@def@term#1{%
  \define@key{nju}{#1}{\csname #1\endcsname{##1}}
  \expandafter\gdef\csname #1\endcsname##1{%
    \expandafter\gdef\csname nju@#1\endcsname{##1}}
  \csname #1\endcsname{}}
\nju@def@term{ctitle}
\nju@def@term{csubtitle}
\nju@def@term{csubsubtitle}
\nju@def@term{etitle}
\nju@def@term{esubtitle}
\nju@def@term{esubsubtitle}
\nju@def@term{cauthor}
\nju@def@term{csupervisor}
\nju@def@term{cassosupervisor}
\nju@def@term{ccosupervisor}
\nju@def@term{eauthor}
\nju@def@term{esupervisor}
\nju@def@term{eassosupervisor}
\nju@def@term{ecosupervisor}
\nju@def@term{cdegree}
\nju@def@term{edegree}
\nju@def@term{cdepartment}
\nju@def@term{edepartment}
\nju@def@term{cmajor}
\nju@def@term{emajor}
\nju@def@term{cdate}
\nju@def@term{edate}
\nju@def@term{stdid}
\nju@def@term{mail}
\cdate{\CJK@todaybig@short}
\edate{\ifcase \month \or January\or February\or March\or April\or May%
       \or June\or July \or August\or September\or October\or November
       \or December\fi\unskip,\ \ \the\year}
%    \end{macrocode}
%
% \begin{environment}{cabstract}
% \begin{environment}{eabstract}
% 摘要最好以环境的形式出现(否则命令的形式会导致开始结束的括号距离太远,我不喜
% 欢),这就必须让环境能够自己保存内容留待以后使用。使用 \pkg{environ} 的
% \cs{Collect@Body} 来实现。
%    \begin{macrocode}
\newcommand{\nju@@cabstract}[1]{\long\gdef\nju@cabstract{#1}}
\newenvironment{cabstract}{\Collect@Body\nju@@cabstract}{}
\newcommand{\nju@@eabstract}[1]{\long\gdef\nju@eabstract{#1}}
\newenvironment{eabstract}{\Collect@Body\nju@@eabstract}{}
%    \end{macrocode}
% \end{environment}
% \end{environment}
%
% \begin{macro}{\nju@parse@keywords}
%   不同论文格式关键词之间的分割不太相同,我们用 \cs{ckeywords} 和
%    \cs{ekeywords} 来收集关键词列表,然后用本命令来生成符合要求的格式。
%    \begin{macrocode}
\def\nju@parse@keywords#1{
  \define@key{nju}{#1}{\csname #1\endcsname{##1}}
  \expandafter\gdef\csname nju@#1\endcsname{}
  \expandafter\gdef\csname #1\endcsname##1{
    \@for\reserved@a:=##1\do{
      \expandafter\ifx\csname nju@#1\endcsname\@empty\else
        \expandafter\g@addto@macro\csname nju@#1\endcsname{%
          \ignorespaces\csname nju@#1@separator\endcsname}
      \fi
      \expandafter\expandafter\expandafter\g@addto@macro%
        \expandafter\csname nju@#1\expandafter\endcsname\expandafter{\reserved@a}}}}
%    \end{macrocode}
% \end{macro}
% \begin{macro}{\ckeywords}
% \begin{macro}{\ekeywords}
% 利用 \cs{nju@parse@keywords} 来定义,内部通过 \cs{nju@ckeywords} 和
% \cs{nju@ekeywords} 来引用。
%    \begin{macrocode}
\nju@parse@keywords{ckeywords}
\nju@parse@keywords{ekeywords}
%    \end{macrocode}
% \end{macro}
% \end{macro}
%
% \begin{macro}{\njusetup}
% 由上可见,封面和封底有一大堆信息需要设置,为了简化操作界面,提供一
% 个 \cs{njusetup} 命令支持 key/value 的方式来设置。key 就是前面各个设置项的
% 名字。\note[说明:]{只能设置普通项,不支持环境项,
% 如 \texttt{cabstract} 和 \texttt{eabstract}。} 由于这些设置项被 \cs{makecover}
% 调用,所以此命令需要在 \cs{makecover} 之前被调用。
%    \begin{macrocode}
\def\njusetup{\kvsetkeys{nju}}
%    \end{macrocode}
% \end{macro}
%
% 定义封面用到的各种文字。
%    \begin{macrocode}
\def\nju@ckeywords@separator{;}
\def\nju@ekeywords@separator{;}
\def\nju@catalog@number@title{分类号}
\def\nju@id@title{编号}
\def\nju@title@sep{:}
\def\nju@schoolname{南京大学}
\def\nju@author@title{姓名}
\def\nju@department@title{系别}
\def\nju@major@title{专业}
\def\nju@supervisor@title{指导教师}
\def\nju@assosuper@title{辅导教师}
\def\nju@studentid@title{学号}
\def\nju@date@title{日期}
\def\nju@mail@title{邮箱}
\newcommand{\nju@ckeywords@title}{关键词:}
\def\nju@title@pre{}

\def\nju@eng@title@sep{:}
\def\nju@eng@author@title{Name}
\def\nju@eng@studentid@title{StdID}
\def\nju@eng@date@title{Date}
\def\nju@eng@mail@title{E-mail}
%    \end{macrocode}
%
% 中文小型标题
%    \begin{macrocode}
\renewcommand{\maketitle}{
  \nju@setup@pdfinfo
  \begin{center} {\LARGE \ifnju@chinese\nju@ctitle\else\nju@etitle\fi}
  \end{center}
  \hspace*{\fill}
  \ifnju@chinese
    \nju@author@title\nju@title@sep\CJKunderline{\nju@cauthor}
  \else
    \nju@eng@author@title\nju@eng@title@sep\underline{\nju@eauthor}
  \fi
  \hspace*{\fill}
  \ifx\nju@stdid\@empty\relax
  \else
    \ifnju@chinese
      \nju@studentid@title\nju@title@sep\CJKunderline{\nju@stdid}
    \else
      \nju@eng@studentid@title\nju@eng@title@sep\underline{\nju@stdid}
    \fi
  \fi
  \hspace*{\fill}
  \ifnju@chinese
    \nju@date@title\nju@title@sep\CJKunderline{\today}
  \else
    \nju@eng@date@title\nju@eng@title@sep\CJKunderline{\nju@edate}
  \fi
  \hspace*{\fill}\\
}
%    \end{macrocode}
%
% 别样封面
%    \begin{macrocode}
\newcommand{\maketitlepage}{
  \nju@setup@pdfinfo
  \begin{titlepage}
    \begin{center}
    \ifx\nju@esubsubtitle\@empty\relax  {\LARGE\sffamily\scshape\ifnju@chinese\nju@csubsubtitle\else\nju@esubsubtitle\fi\ }\\[1.5cm]
    \else
    {\LARGE\sffamily\scshape \ifnju@chinese\nju@csubsubtitle\else\nju@esubsubtitle\fi}\\[1.5cm]
    \fi
		{\Large\sffamily\scshape \ifnju@chinese\nju@csubtitle\else\nju@esubtitle\fi}\\
    \rule{\linewidth}{0.5mm} \\[0.4cm]
		{\huge\sffamily\bfseries \ifnju@chinese\nju@ctitle\else\nju@etitle\fi}\\
		\rule{\linewidth}{0.5mm} \\[1.5cm]
			
		\begin{center}
			\begin{tabular}{@{\hspace{0.5cm}}l@{\hspace{0.5cm}}l}
				\nju@eauthor & \nju@stdid\\
			\end{tabular}
		\end{center}
		\vfill
		{\large \nju@edate}
    \end{center}
    \ifnju@right\cleardoublepage\else\clearpage\fi
  \end{titlepage}
}
%    \end{macrocode}
%
% \myentry{封面第一页}
% \begin{macro}{\nju@first@titlepage}
% 题名使用一号黑体字,一行写不下时可分两行写,并采用 1.25 倍行距。
% 申请学位的学科门类: 小二号宋体字。
% 中文封面页边距:
%  上- 6.0 厘米,下- 5.5 厘米,左- 4.0 厘米,右- 4.0 厘米,装订线 0 厘米;
%
%    \begin{macrocode}
\newcommand\nju@underline[2][6em]{\hskip1pt\underline{\hb@xt@ #1{\hss#2\hss}}\hskip3pt}
\newlength{\nju@title@width}
\ifxetex % todo: ugly codes
  \newcommand{\nju@put@title}[2][\nju@title@width]{%
  \begin{CJKfilltwosides}[b]{#1}#2\end{CJKfilltwosides}}
\else
  \newcommand{\nju@put@title}[2][\nju@title@width]{%
  \begin{CJKfilltwosides}{#1}#2\end{CJKfilltwosides}}
\fi
\newcommand{\nju@first@titlepage}{
  \begin{center}
    \vspace*{-1.6cm}
    \parbox[b][2.4cm][t]{\textwidth}{%
      \rule{1cm}{0cm}}
      \vskip0.65cm
      \par\vskip2cm
      {\xiaochu\heiti\ziju{0.5}\textbf\nju@csubtitle}
      \vskip2.2cm\hskip0.8cm
      \noindent\heiti\xiaoer\nju@title@pre
      \parbox[t]{12cm}{%
      \ignorespaces\yihao[1.51]%
      \renewcommand{\CJKunderlinebasesep}{0.25cm}%
      \renewcommand{\ULthickness}{1.3pt}%
      \ifxetex
        \xeCJKsetup{underline/format=\color{black}}%
      \else
        \def\CJKunderlinecolor{\color{black}}%
      \fi
      \centering\CJKunderline*{\nju@ctitle}
      
    }%
      \vskip1.3cm
%    \end{macrocode}
%
% 作者及导师信息部分使用三号仿宋字
%    \begin{macrocode}
      \vskip0.75cm
      \ifx\nju@cassosupervisor\@empty%
        \def\nju@tempa{7.15cm}
      \else%
        \def\nju@tempa{8.15cm}
      \fi%
      \parbox[t][\nju@tempa][t]{\textwidth}{%
        {\fangsong\sanhao[1.95]%
         \hspace*{1.9cm}
         \setlength{\nju@title@width}{4em}
         \setlength{\extrarowheight}{6pt}
         \ifxetex % todo: ugly codes
           \begin{tabular}{p{\nju@title@width}@{}l@{\extracolsep{8pt}}l}
         \else
           \begin{tabular}{p{\nju@title@width}l@{}l}
         \fi
             \nju@put@title{\nju@department@title} & \nju@title@sep
               & \nju@cdepartment\\
             \nju@put@title{\nju@major@title}      & \nju@title@sep
               & \nju@cmajor\\
             \nju@put@title{\nju@author@title}     & \nju@title@sep
               & \nju@cauthor \\
             \nju@put@title{\nju@supervisor@title} & \nju@title@sep
               & \nju@csupervisor\\
             \ifx\nju@cassosupervisor\@empty\else%
               \nju@put@title{\nju@assosuper@title} & \nju@title@sep
               & \nju@cassosupervisor\\
             \fi
           \end{tabular}
        }}
%    \end{macrocode}
%
% 论文成文打印的日期,用三号宋体汉字,不用阿拉伯数字
% 本科:论文成文打印的日期用阿拉伯数字,采用小四号宋体
%    \begin{macrocode}
     \begin{center}
       {\vskip-1.0cm\xiaosi
         \songti\nju@cdate}
     \end{center}
    \end{center}} % end of titlepage
%    \end{macrocode}
% \end{macro}
%
% \myentry{英文封面}
% \begin{macro}{\nju@engcover}
%    \begin{macrocode}
\newcommand{\nju@engcover}{%
  \begin{center}
    \vspace*{-5pt}
    \parbox[t][5.2cm][t]{\paperwidth-7.2cm}{
      \renewcommand{\baselinestretch}{1.5}
      \begin{center}
        \erhao[1.1]\bfseries\sffamily\nju@etitle%
      \end{center}}
    \parbox[t][][b]{\paperwidth-7.2cm}{
      \renewcommand{\baselinestretch}{1.3}
      \begin{center}
        \sanhao\sffamily by\\[3bp]
        \bfseries\nju@eauthor%
        \ifx\nju@emajor\empty\relax\else
          \\(~\nju@emajor~)%
        \fi
      \end{center}}
    \par\vspace{0.9cm}
    \parbox[t][2.1cm][t]{\paperwidth-7.2cm}{
      \renewcommand{\baselinestretch}{1.2}
      \xiaosan\centering
      \begin{tabular}{rl}
        Supervisor : & \nju@esupervisor\\
        \ifx\nju@eassosupervisor\@empty
          \else Associate Supervisor : & \nju@eassosupervisor\\\fi
        \ifx\nju@ecosupervisor\@empty
          \else Cooperate Supervisor : & \nju@ecosupervisor\\\fi
      \end{tabular}}
    \parbox[t][2cm][b]{\paperwidth-7.2cm}{
    \begin{center}
      \sanhao\bfseries\sffamily\nju@edate
    \end{center}}
  \end{center}}
%    \end{macrocode}
% \end{macro}
%
% \begin{macro}{\makecover}
% 生成封面总命令。
%    \begin{macrocode}
\def\makecover{%
  \nju@setup@pdfinfo\nju@makecover}
\def\nju@setup@pdfinfo{%
  \ifnju@chinese
    \hypersetup{
      pdftitle    = \nju@ctitle,
      pdfauthor   = \nju@cauthor,
      pdfsubject  = \nju@cdegree,
      pdfkeywords = \nju@ckeywords,
    }%
  \else
    \hypersetup{
      pdftitle    = \nju@etitle,
      pdfauthor   = \nju@eauthor,
      pdfsubject  = \nju@edegree,
      pdfkeywords = \nju@ekeywords,
    }%
  \fi
  \hypersetup{
    pdfcreator={\njurepo-v\version}}}
\NewDocumentCommand{\nju@makecover}{o}{
  \phantomsection
  \pdfbookmark[-1]{\nju@ctitle}{ctitle}
  \normalsize%
  \begin{titlepage}
    \ifnju@chinese
      \nju@first@titlepage
    \else
      \nju@engcover
    \fi
    \ifnju@right\cleardoublepage\else\clearpage\fi
  \end{titlepage}
}
\newcommand{\makeabstract}{
  \normalsize
  \nju@makeabstract
  \let\@tabular\nju@tabular
}
%    \end{macrocode}
% \end{macro}
%
% \subsubsection{摘要}
% \label{sec:abstractformat}
%
% \begin{macro}{\nju@put@keywords}
% 排版关键字。
%    \begin{macrocode}
\newbox\nju@kw
\newcommand\nju@put@keywords[2]{%
  \begingroup
    \setbox\nju@kw=\hbox{#1}
    \indent%
    \box\nju@kw#2\par
  \endgroup}
%    \end{macrocode}
% \end{macro}
%
% \begin{macro}{\nju@makeabstract}
% 中文摘要部分的标题为“\textbf{摘要}”,用黑体三号字。
%    \begin{macrocode}
\newcommand{\nju@makeabstract}{%
  \clearpage
  \pagestyle{nju@plain}
  \pagenumbering{Roman}
%    \end{macrocode}
%
% 摘要内容用小四号字书写,两端对齐,汉字用宋体,外文字用 Times New Roman 体,
% 标点符号一律用中文输入状态下的标点符号。
%    \begin{macrocode}
  \ifnju@chinese
    \nju@setchinese
    \nju@chapter*[]{\cabstractname} % no tocline
    \nju@cabstract
    \vskip12bp
    \nju@put@keywords{\textbf\nju@ckeywords@title}{\nju@ckeywords}
  \else
  \nju@setenglish
    \nju@chapter*[]{\eabstractname} % no tocline
    \nju@eabstract
    \vskip12bp
    \nju@put@keywords{%
      \textbf{Key Words:\enskip}}{\nju@ekeywords}%
  \fi
  \nju@setdefaultlanguage
}
%    \end{macrocode}
% \end{macro}
%
%
%
% \subsubsection{主要符号表}
% \label{sec:denotationfmt}
% \begin{environment}{denotation}
% 主要符号对照表。
%    \begin{macrocode}
\ifnju@chinese
  \newcommand\nju@denotation@name{主要符号对照表}
\else
  \newcommand\nju@denotation@name{Nomenclature}
\fi
\newenvironment{denotation}[1][2.5cm]{%
  \nju@chapter*[]{\nju@denotation@name} % no tocline
  \vskip-30bp\xiaosi[1.6]\begin{nju@denotation}[labelwidth=#1]
}{%
  \end{nju@denotation}
}
\newlist{nju@denotation}{description}{1}
\setlist[nju@denotation]{%
  nosep,
  font=\normalfont,
  align=left,
  leftmargin=!, % sum of the following 3 lengths
  labelindent=0pt,
  labelwidth=2.5cm,
  labelsep*=0.5cm,
  itemindent=0pt,
}
%    \end{macrocode}
% \end{environment}
%
% \subsubsection{致谢与声明}
% \label{sec:ackanddeclare}
%
% \begin{environment}{acknowledgement}
% 支持扫描文件替换。
%    \begin{macrocode}
\ifnju@chinese
  \newcommand\nju@ack@name{致\hspace{\ccwd}谢}
\else
  \newcommand\nju@ack@name{Acknowledgments}
\fi
\newcommand\nju@declarename{声\hspace{\ccwd}明}
\newcommand{\nju@declaretext}{本人郑重声明:所呈交的学位论文,是本人在导师指导下
  ,独立进行研究工作所取得的成果。尽我所知,除文中已经注明引用的内容外,本学位论
  文的研究成果不包含任何他人享有著作权的内容。对本论文所涉及的研究工作做出贡献的
  其他个人和集体,均已在文中以明确方式标明。}
\newcommand{\nju@signature}{签\hspace{1em}名:}
\newcommand{\nju@backdate}{日\hspace{1em}期:}
%    \end{macrocode}
%
%  \cs{cleardoublepage},使新开章节的页码到达正确的状态。否则会因为 \cs{addcontentsline}
% 在 chapter 之前而导致目录页码错误。
% 定义致谢与声明环境。
%    \begin{macrocode}
\NewDocumentEnvironment{acknowledgement}{o}{%
    \nju@chapter*{\nju@ack@name}
  }
%    \end{macrocode}
%
% 声明部分
%    \begin{macrocode}
  {
    \ifnju@english\relax\else%
      \IfNoValueTF{#1}{%
        \nju@chapter*{\nju@declarename}
        \par{\xiaosi\parindent2em\nju@declaretext}\vskip2cm
        {\xiaosi\hfill\nju@signature\nju@underline[2.5cm]\relax%
         \nju@backdate\nju@underline[2.5cm]\relax}%
      }{%
        \includepdf[pagecommand={\thispagestyle{nju@empty}%
          \phantomsection\addcontentsline{toc}{chapter}{\nju@declarename}%
        }]{#1}%
      }%
    \fi
  }
%    \end{macrocode}
% \end{environment}
%
% \subsubsection{图表索引}
% \label{sec:threeindex}
% \begin{macro}{\listoffigures}
% \begin{macro}{\listoffigures*}
% \begin{macro}{\listoftables}
% \begin{macro}{\listoftables*}
% 定义图表以及公式目录样式。
%    \begin{macrocode}
\def\nju@starttoc#1{% #1: float type, prepend type name in \listof*** entry.
  \let\oldnumberline\numberline
  \def\numberline##1{\oldnumberline{\csname #1name\endcsname\hskip.4em ##1}}
  \@starttoc{\csname ext@#1\endcsname}
  \let\numberline\oldnumberline}
\def\nju@listof#1{% #1: float type
  \@ifstar
    {\nju@chapter*[]{\csname list#1name\endcsname}\nju@starttoc{#1}}
    {\nju@chapter*{\csname list#1name\endcsname}\nju@starttoc{#1}}}
\renewcommand\listoffigures{\nju@listof{figure}}
\renewcommand*\l@figure{\addvspace{6bp}\@dottedtocline{1}{0em}{4em}}
\renewcommand\listoftables{\nju@listof{table}}
\let\l@table\l@figure
%    \end{macrocode}
% \end{macro}
% \end{macro}
% \end{macro}
% \end{macro}
%
% \begin{macro}{\equcaption}
%   本命令只是为了生成公式列表,所以这个 caption 是假的。如果要编号最好用
%    equation 环境,如果是其它编号环境,请手动添加 \cs{equcaption}。
% 用法如下:
%
% \cs{equcaption}\marg{counter}
%
% \marg{counter} 指定出现在索引中的编号,一般取 \cs{theequation},如果你是用
%  \pkg{amsmath} 的 \cs{tag},那么默认是 \cs{tag} 的参数;除此之外可能需要你
% 手工指定。
%
%    \begin{macrocode}
\def\ext@equation{loe}
\def\equcaption#1{%
  \addcontentsline{\ext@equation}{equation}%
                  {\protect\numberline{#1}}}
%    \end{macrocode}
% \end{macro}
%
% \begin{macro}{\listofequations}
% \begin{macro}{\listofequations*}
% \LaTeX\ 默认没有公式索引,此处定义自己的 \cs{listofequations}。
%    \begin{macrocode}
\newcommand\listofequations{\nju@listof{equation}}
\let\l@equation\l@figure
%    \end{macrocode}
% \end{macro}
% \end{macro}
%
% \subsection{参考文献}
% \label{sec:ref}
%
% \begin{macro}{\inlinecite}
% 依赖于 \pkg{natbib} 宏包,修改其中的命令。 旧命令 \cs{onlinecite} 依然可用。
%    \begin{macrocode}
\newcommand\bibstyle@inline{\bibpunct{[}{]}{,}{n}{,}{,}}
\DeclareRobustCommand\inlinecite{\@inlinecite}
\def\@inlinecite#1{\begingroup\let\@cite\NAT@citenum\citep{#1}\endgroup}
\let\onlinecite\inlinecite
%    \end{macrocode}
% \end{macro}
%
% 参考文献的正文部分用五号字。
% 行距采用固定值 16 磅,段前空 3 磅,段后空 0 磅。
%
% 复用 \pkg{natbib} 的 \texttt{thebibliography} 环境,调整距离。
%    \begin{macrocode}
\renewcommand\bibsection{\nju@chapter*{\bibname}}
\renewcommand\bibfont{\wuhao[1.5]}
\setlength\bibhang{2\ccwd}
\addtolength{\bibsep}{-0.7em}
\setlength{\labelsep}{0.4em}
\def\@biblabel#1{[#1]\hfill}
%    \end{macrocode}
%
% 两种引用样式:
%    \begin{macrocode}
\expandafter\newcommand\csname bibstyle@numeric\endcsname{%
  \bibpunct{[}{]}{,}{s}{,}{\textsuperscript{,}}}
\expandafter\newcommand\csname bibstyle@author-year\endcsname{%
  \bibpunct{(}{)}{;}{a}{,}{,}}
%    \end{macrocode}
%
% 下面修改 \pkg{natbib} 的引用格式,主要是将页码写在上标位置。
% numeric 模式的 \cs{citet} 的页码:
%    \begin{macrocode}
\patchcmd\NAT@citexnum{%
  \@ifnum{\NAT@ctype=\z@}{%
    \if*#2*\else\NAT@cmt#2\fi
  }{}%
  \NAT@mbox{\NAT@@close}%
}{%
  \NAT@mbox{\NAT@@close}%
  \@ifnum{\NAT@ctype=\z@}{%
    \if*#2*\else\textsuperscript{#2}\fi
  }{}%
}{}{}
%    \end{macrocode}
%
% Numeric 模式的 \cs{citep} 的页码:
%    \begin{macrocode}
\renewcommand\NAT@citesuper[3]{\ifNAT@swa
  \if*#2*\else#2\NAT@spacechar\fi
\unskip\kern\p@\textsuperscript{\NAT@@open#1\NAT@@close\if*#3*\else#3\fi}%
   \else #1\fi\endgroup}
%    \end{macrocode}
%
% Author-year 模式的 \cs{citet} 的页码:
%    \begin{macrocode}
\patchcmd{\NAT@citex}{%
  \if*#2*\else\NAT@cmt#2\fi
  \if\relax\NAT@date\relax\else\NAT@@close\fi
}{%
  \if\relax\NAT@date\relax\else\NAT@@close\fi
  \if*#2*\else\textsuperscript{#2}\fi
}{}{}
%    \end{macrocode}
%
% Author-year 模式的 \cs{citep} 的页码:
%    \begin{macrocode}
\renewcommand\NAT@citesuper[3]{\ifNAT@swa
  \if*#2*\else#2\NAT@spacechar\fi
\unskip\kern\p@\textsuperscript{\NAT@@open#1\NAT@@close\if*#3*\else#3\fi}%
   \else #1\fi\endgroup}
%    \end{macrocode}
%
% 在顺序编码制下,\pkg{natbib} 只有在三个以上连续文献引用才会使用连接号,
% 这里修改为允许两个引用使用连接号。
%    \begin{macrocode}
\patchcmd{\NAT@citexnum}{%
  \ifx\NAT@last@yr\relax
    \def@NAT@last@yr{\@citea}%
  \else
    \def@NAT@last@yr{--\NAT@penalty}%
  \fi
}{%
  \def@NAT@last@yr{-\NAT@penalty}%
}{}{}
%    \end{macrocode}
%
% \subsection{附录}
% \label{sec:appendix}
% \begin{environment}{appendix}
% 主要给本科做外文翻译用。
%    \begin{macrocode}
\let\nju@appendix\appendix
\renewenvironment{appendix}{%
  \let\title\nju@appendix@title
  \nju@appendix}{%
  \let\title\@gobble}
%    \end{macrocode}
% \end{environment}
%
% \begin{macro}{\title}
% 本科外文翻译文章的标题,用法:\cs{title}\marg{资料标题}。这个命令只能在附录环
% 境下使用。
%    \begin{macrocode}
\let\title\@gobble
\newcommand{\nju@appendix@title}[1]{%
  \begin{center}
    \xiaosi[1.667] #1
  \end{center}}
%    \end{macrocode}
% \end{macro}
%
% \begin{environment}{translationbib}
% 外文资料的参考文用宋体五号字,取固定行距17pt,段前后3pt。
%    \begin{macrocode}
\newlist{translationbib}{enumerate}{1}
\setlist[translationbib]{label=[\arabic*],align=left,nosep,itemsep=6bp,
  leftmargin=10mm,labelsep=!,before=\vspace{0.5\baselineskip}\wuhao[1.3]}
%    \end{macrocode}
% \end{environment}
%\marginpar{这是边注}
%
%\subsection{颜色}
%    \begin{macrocode}
\RequirePackage{xcolor}
\definecolor{codegreen}{rgb}{0,0.6,0}
\definecolor{codegray}{rgb}{0.5,0.5,0.5}
\definecolor{codepurple}{rgb}{0.58,0,0.82}
\definecolor{backcolour}{rgb}{0.95,0.95,0.92}
\newcommand{\red}[1]{\textcolor{red}{#1}}
\newcommand{\redoverlay}[2]{\textcolor<#2>{red}{#1}}
\newcommand{\green}[1]{\textcolor{green}{#1}}
\newcommand{\greenoverlay}[2]{\textcolor<#2>{green}{#1}}
\newcommand{\blue}[1]{\textcolor{blue}{#1}}
\newcommand{\blueoverlay}[2]{\textcolor<#2>{blue}{#1}}
\newcommand{\purple}[1]{\textcolor{purple}{#1}}
\newcommand{\cyan}[1]{\textcolor{cyan}{#1}}
\newcommand{\teal}[1]{\textcolor{teal}{#1}}
\newcommand{\magenta}[1]{{\color{magenta}#1}}
\newcommand{\note}[2][Note]{{%
  \color{magenta}{\bfseries #1}\emph{#2}}}
%    \end{macrocode}
%
%\subsection{代码}
%    \begin{macrocode}
\RequirePackage{verbatim}
\RequirePackage{algorithm}
\RequirePackage{algpseudocode}
\newcommand{\pseduo}[2]{
\begin{algorithm}
	\caption{\textsc{#1}}
	\label{alg:#1}
	\begin{algorithmic}[1]
		#2
	\end{algorithmic}
\end{algorithm}
}
\RequirePackage{listings}
\lstdefinestyle{lstStyleBase}{%
   basicstyle=\small\ttfamily,
   aboveskip=\medskipamount,
   belowskip=\medskipamount,
   lineskip=0pt,
   boxpos=c,
   showlines=false,
   extendedchars=true,
   upquote=true,
   tabsize=2,
   showtabs=false,
   showspaces=false,
   showstringspaces=false,
   numbers=none,
   linewidth=\linewidth,
   xleftmargin=4pt,
   xrightmargin=0pt,
   resetmargins=false,
   breaklines=true,
   breakatwhitespace=false,
   breakindent=0pt,
   breakautoindent=true,
   columns=flexible,
   keepspaces=true,
   gobble=2,
   framesep=3pt,
   rulesep=1pt,
   framerule=1pt,
   backgroundcolor=\color{gray!5},
   stringstyle=\color{green!40!black!100},
   keywordstyle=\bfseries\color{blue!50!black},
   commentstyle=\slshape\color{black!60}
}

\newtcblisting{commandshell}{colback=black,colupper=white,colframe=yellow!75!black, listing only,listing options={style=tcblatex,language=sh},
every listing line={\textcolor{red}{\small\ttfamily\bfseries \$>}}}

\lstdefinestyle{lstStyleShell}{%
   style=lstStyleBase,
   frame=l,
   rulecolor=\color{purple},
   language=bash}

\lstdefinestyle{lstStyleLaTeX}{%
   style=lstStyleBase,
   frame=l,
   rulecolor=\color{violet},
   language=[LaTeX]TeX}

\lstdefinestyle{lstStylecdisplay}{%
  style=lstStyleBase,
  frame=tb,
  rulecolor=\color{cyan},
  keywordstyle=\color{magenta}\bfseries\ttfamily,
  commentstyle=\color{codegreen}\ttfamily,
	stringstyle=\color{codepurple}\ttfamily\sffamily,
	backgroundcolor=\color{backcolour},
	captionpos=b,
	numbers=left,
	numberstyle=\footnotesize\color{codegray},
	stepnumber=1,
  numbersep=5pt,
  language=C
}

\lstdefinestyle{lstStylecpseudo}{%
  style=lstStyleBase,
  frame=none,
  keywordstyle=\color{magenta}\bfseries\ttfamily,
  commentstyle=\color{codegreen}\ttfamily,
	stringstyle=\color{codepurple}\ttfamily\sffamily,
	captionpos=b,
	numbers=left,
	numberstyle=\footnotesize\color{codegray},
	stepnumber=1,
  numbersep=5pt,
  language=C
}

\lstdefinestyle{lstStylecplus}{%
  style=lstStyleBase,
  frame=l,
  rulecolor=\color{blue},
  language=C++
}

\lstdefinestyle{lstStyleverilog}{%
  style=lstStyleBase,
  frame=l,
  rulecolor=\color{brown},
  language=verilog
}

\lstdefinestyle{lstStylepython}{%
  style=lstStyleBase,
  frame=l,
  rulecolor=\color{pink},
  language=python
}

\lstnewenvironment{code}{\lstset{style=lstStyleBase}}{}
\lstnewenvironment{latex}{\lstset{style=lstStyleLaTeX}}{}
\lstnewenvironment{shell}{\lstset{style=lstStyleShell}}{}
\lstnewenvironment{cdisplay}{\lstset{style=lstStylecdisplay}}{}
\lstnewenvironment{cplus}{\lstset{style=lstStylecplus}}{}
\lstnewenvironment{verilog}{\lstset{style=lstStyleverilog}}{}
\lstnewenvironment{python}{\lstset{style=lstStylepython}}{}
\lstnewenvironment{cpseudo}{\lstset{style=lstStylecpseudo}}{}
%    \end{macrocode}
%
% \subsection{快速插入图片或图表}
%    \begin{macrocode}
\newcommand{\figpf}[2]{
	\begin{figure}[H]
		\centering
		\includegraphics[#1]{figs/#2}
	\end{figure}
}

%%%%%%%%%%%%%%%%%%%%
\newcommand{\figpfc}[3]{
	\begin{figure}[htbp]
		\centering
		\includegraphics[#1]{figs/#2}
		\caption{#3}
		\label{fig:#2}
	\end{figure}
}
%%%%%%%%%%%%%%%%%%%
\newcommand{\tabncc}[3]{
	\begin{table}[H]
		\centering
		\begin{tabular}{|*{#1}{c|}}
		\toprule
		#2\\
		\bottomrule
	\end{tabular}
	\caption{#3}
	\label{form:#3}
\end{table}}
%%%%%%%%%%%%%%%%%%%
\newcommand{\tabnc}[2]{
	\begin{table}[H]
		\centering
		\begin{tabular}{|*{#1}{c|}}
		\toprule
		#2\\
		\bottomrule
	\end{tabular}
\end{table}}
\newcommand{\tnl}{\tabularnewline\midrule}
%    \end{macrocode}
%
% \subsection{借用dtx文件代码}
%    \begin{macrocode}
\def\cmd#1{\cs{\expandafter\cmd@to@cs\string#1}}
\def\cmd@to@cs#1#2{\char\number`#2\relax}
\DeclareRobustCommand\cs[1]{\texttt{\char`\\#1}}
\newcommand*{\meta}[1]{{%
  \ensuremath{\langle}\rmfamily\itshape#1\/\ensuremath{\rangle}}}
\providecommand\marg[1]{%
  {\ttfamily\char`\{}\meta{#1}{\ttfamily\char`\}}}
\providecommand\oarg[1]{%
  {\ttfamily[}\meta{#1}{\ttfamily]}}
\providecommand\parg[1]{%
  {\ttfamily(}\meta{#1}{\ttfamily)}}
\providecommand\pkg[1]{{\sffamily#1}}
%    \end{macrocode}
% 
% \subsection{水印}
%    \begin{macrocode}
\RequirePackage{watermark}
\ifnju@draft
\AtEndOfClass{
	\watermark{% 
		\parbox[b][\paperheight]{\paperwidth}{% 
		\vfill 
		\centering% 
		\begin{tikzpicture}[remember picture,overlay] 
			\node [rotate=45,scale=10] at ($(current page.center) +(-1cm,1cm)$) 
			{\textcolor[gray]{0.8}{DRAFT}}; 
			\node [rotate=45,scale=3] at ($(current page.center) +(1cm,-1cm)$) 
			{\textcolor[gray]{0.75}{Compile time: \the\year - \the\month - \the\day}}; 
		\end{tikzpicture}% 
		\vfill 
		}
  }
}
\fi
%    \end{macrocode}
%
% \subsection{自定义代码}
%    \begin{macrocode}

\newcommand{\blankpage}{
	\clearpage
	\begin{titlepage}
		\null\vfil
		\begin{center}
			\textit{This page intentionally left blank.}
		\end{center}
	\end{titlepage}
}
\newcommand{\rmnum}[1]{\romannumeral #1}
\newcommand{\Rmnum}[1]{\expandafter\@slowromancap\romannumeral #1@}
%    \end{macrocode}
% \subsection{结束部分}
% \label{sec:finish}
%    \begin{macrocode}
\AtEndOfClass{\sloppy}
%    \end{macrocode}
%</cls> 
%
%
%
% \iffalse
%    \begin{macrocode}
%<*dtx-style>
\ProvidesPackage{dtx-style}
\RequirePackage{hypdoc}
\RequirePackage{ifthen}
\RequirePackage[UTF8,scheme=chinese]{ctex}
\RequirePackage{newpxtext}
\RequirePackage{newpxmath}
\RequirePackage[
  top=2.5cm, bottom=2.5cm,
  left=4cm, right=2cm,marginparwidth=2.6cm,marginparsep=3mm,
  headsep=3mm]{geometry}
\RequirePackage{array,longtable,booktabs}
\RequirePackage{listings}
\RequirePackage{fancyhdr}
\RequirePackage{xcolor}
\definecolor{codegreen}{rgb}{0,0.6,0}
\definecolor{codegray}{rgb}{0.5,0.5,0.5}
\definecolor{codepurple}{rgb}{0.58,0,0.82}
\definecolor{backcolour}{rgb}{0.95,0.95,0.92}
\newcommand{\red}[1]{\textcolor{red}{#1}}
\newcommand{\redoverlay}[2]{\textcolor<#2>{red}{#1}}
\newcommand{\green}[1]{\textcolor{green}{#1}}
\newcommand{\greenoverlay}[2]{\textcolor<#2>{green}{#1}}
\newcommand{\blue}[1]{\textcolor{blue}{#1}}
\newcommand{\blueoverlay}[2]{\textcolor<#2>{blue}{#1}}
\newcommand{\purple}[1]{\textcolor{purple}{#1}}
\newcommand{\cyan}[1]{\textcolor{cyan}{#1}}
\newcommand{\teal}[1]{\textcolor{teal}{#1}}
\RequirePackage{enumitem}
\RequirePackage{etoolbox}
\RequirePackage{metalogo}
\RequirePackage{mathtools}
\DeclarePairedDelimiter{\ceil}{\lceil}{\rceil}
\DeclarePairedDelimiter{\floor}{\lfloor}{\rfloor}
\DeclareMathOperator{\Hamilton}{\hat{H}} 
\ifthenelse{\equal{\@nameuse{g__ctex_fontset_tl}}{mac}}{%
  \xeCJKsetwidth{‘’“”}{1em}
}{}

\colorlet{nju@macro}{blue!60!black}
\colorlet{nju@env}{blue!70!black}
\colorlet{nju@option}{purple}
\patchcmd{\PrintMacroName}{\MacroFont}{\MacroFont\bfseries\color{nju@macro}}{}{}
\patchcmd{\PrintDescribeMacro}{\MacroFont}{\MacroFont\bfseries\color{nju@macro}}{}{}
\patchcmd{\PrintDescribeEnv}{\MacroFont}{\MacroFont\bfseries\color{nju@env}}{}{}
\patchcmd{\PrintEnvName}{\MacroFont}{\MacroFont\bfseries\color{nju@env}}{}{}

\def\DescribeOption{%
  \leavevmode\@bsphack\begingroup\MakePrivateLetters%
  \Describe@Option}
\def\Describe@Option#1{\endgroup
  \marginpar{\raggedleft\PrintDescribeOption{#1}}%
  \nju@special@index{option}{#1}\@esphack\ignorespaces}
\def\PrintDescribeOption#1{\strut \MacroFont\bfseries\sffamily\color{nju@option} #1\ }
\def\nju@special@index#1#2{\@bsphack
  \begingroup
    \HD@target
    \let\HDorg@encapchar\encapchar
    \edef\encapchar usage{%
      \HDorg@encapchar hdclindex{\the\c@HD@hypercount}{usage}%
    }%
    \index{#2\actualchar{\string\ttfamily\space#2}
           (#1)\encapchar usage}%
    \index{#1:\levelchar#2\actualchar
           {\string\ttfamily\space#2}\encapchar usage}%
  \endgroup
  \@esphack}

\lstdefinestyle{lstStyleBase}{%
   basicstyle=\small\ttfamily,
   aboveskip=\medskipamount,
   belowskip=\medskipamount,
   lineskip=0pt,
   boxpos=c,
   showlines=false,
   extendedchars=true,
   upquote=true,
   tabsize=2,
   showtabs=false,
   showspaces=false,
   showstringspaces=false,
   numbers=none,
   linewidth=\linewidth,
   xleftmargin=4pt,
   xrightmargin=0pt,
   resetmargins=false,
   breaklines=true,
   breakatwhitespace=false,
   breakindent=0pt,
   breakautoindent=true,
   columns=flexible,
   keepspaces=true,
   gobble=2,
   framesep=3pt,
   rulesep=1pt,
   framerule=1pt,
   backgroundcolor=\color{gray!5},
   stringstyle=\color{green!40!black!100},
   keywordstyle=\bfseries\color{blue!50!black},
   commentstyle=\slshape\color{black!60}}

\lstdefinestyle{lstStyleShell}{%
   style=lstStyleBase,
   frame=l,
   rulecolor=\color{purple},
   language=bash}

\lstdefinestyle{lstStyleLaTeX}{%
   style=lstStyleBase,
   frame=l,
   rulecolor=\color{violet},
   language=[LaTeX]TeX}
\lstdefinestyle{lstStylecplus}{%
   style=lstStyleBase,
   frame=l,
   rulecolor=\color{blue},
   language=C++
 }

\lstnewenvironment{latex}{\lstset{style=lstStyleLaTeX}}{}
\lstnewenvironment{shell}{\lstset{style=lstStyleShell}}{}
\lstnewenvironment{cplus}{\lstset{style=lstStylecplus}}{}

\setlist{nosep}

\DeclareDocumentCommand{\option}{m}{\textsf{#1}}
\DeclareDocumentCommand{\env}{m}{\texttt{#1}}
\DeclareDocumentCommand{\pkg}{s m}{%
  \texttt{#2}\IfBooleanF#1{\nju@special@index{package}{#2}}}
\DeclareDocumentCommand{\file}{s m}{%
  \texttt{#2}\IfBooleanF#1{\nju@special@index{file}{#2}}}
\newcommand{\myentry}[1]{%
  \marginpar{\raggedleft\color{purple}\bfseries\strut #1}}
\newcommand{\note}[2][Note]{{%
  \color{magenta}{\bfseries #1}\emph{#2}}}

\def\njurepo{\textsc{NJU}\-\textsc{repo}}
\def\thuthesis{\textsc{Thu}\-\textsc{Thesis}}
%</dtx-style>
%    \end{macrocode}
% \fi
% \Finale
