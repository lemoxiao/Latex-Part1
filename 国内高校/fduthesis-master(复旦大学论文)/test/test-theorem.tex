\chapter{定理}

\begin{proof}
道千乘之国,敬事而信,节用而爱人,使民以时。
\end{proof}

\begin{definition}
证明完毕/证讫,又写作Q.E.D.。这是拉丁词组“quod erat demonstrandum”
(这就是所要证明的)的缩写,译自希腊语“ὅπερ ἔδει δεῖξαι”
(hoper edei deixai),很多早期数学家用过,包括欧几里得和阿基米德。
“Q.E.D.”可以在证明的尾段写出,以显示证明所需的结论已经完整了。
\end{definition}

\begin{lemma}
  这是一条华丽丽的引理。
\end{lemma}

\begin{proof}[出师表]
先帝创业未半而中道崩殂,今天下三分,益州疲弊,此诚危急存亡之秋也。
\begin{equation}
  \sum_{k=0}^{\infty} \frac{1}{x^k} = \int \sin x dx
\end{equation}
\end{proof}

\begin{proof}
先帝创业未半而中道崩殂,今天下三分,益州疲弊,此诚危急存亡之秋也。
\begin{equation*}
  \sum_{k=0}^{\infty} \frac{1}{x^k} = \int \sin x dx
\end{equation*}
\end{proof}

\begin{lemma}
  这又是一条华丽丽的引理。
\end{lemma}

\chapter{定理(续)}
\newcounter{thm}

\newtheorem[style=plain,qed=\ensuremath{\sin}]{p}{平凡}
\newtheorem[style=margin]{mm}{打断}
\newtheorem[style=change]{fduc}{变革}
\newtheorem[style=break]{fdub}{平凡}
\newtheorem[style=marginbreak]{mb}{打断}
\newtheorem[style=break]{cb}{变革}
\newtheorem*[style=plain]{np}{平凡}
\newtheorem*[style=margin]{nmm}{打断}
\newtheorem*[style=change]{nfduc}{变革}
\newtheorem*[style=break,qed=]{nfdub}{平凡}
\newtheorem*[style=marginbreak,qed={}]{nmb}{打断}
\newtheorem[style=break,counter=thm]{ncb}{变革}

\newtheorem{prop}[thm]{命题}

\begin{prop}
  直角三角形。
\end{prop}
\begin{prop}[圆形]
  直角三角形。
\end{prop}
\begin{prop}
  直角三角形。
\end{prop}

\begin{p}
这是一条定理。
\[ \sum_{k=0}^{\infty} \frac{1}{x^k} = \int \sin x dx \]
\end{p}

\begin{mm}[明早]
这是一条定理。
\[ \sum_{k=0}^{\infty} \frac{1}{x^k} = \int \sin x dx \]
\end{mm}

\begin{fduc}
这是一条定理。
\[ \sum_{k=0}^{\infty} \frac{1}{x^k} = \int \sin x dx \]
\end{fduc}

\begin{fdub}
这是一条定理。
\[ \sum_{k=0}^{\infty} \frac{1}{x^k} = \int \sin x dx \]
\end{fdub}

\begin{mb}
这是一条定理。
\[ \sum_{k=0}^{\infty} \frac{1}{x^k} = \int \sin x dx \]
\end{mb}

\begin{cb}
这是一条定理。
\[ \sum_{k=0}^{\infty} \frac{1}{x^k} = \int \sin x dx \]
\end{cb}

%%%%%%%%%%%%%%%%%%%%%%%%%%%%%%%%%%%%%%%%%%%%%

\begin{np}
这是一条定理。
\[ \sum_{k=0}^{\infty} \frac{1}{x^k} = \int \sin x dx \]
\end{np}

\begin{nmm}[明天一早]
这是一条定理。
\[ \sum_{k=0}^{\infty} \frac{1}{x^k} = \int \sin x dx \]
\end{nmm}

\begin{nfduc}
这是一条定理。
\[ \sum_{k=0}^{\infty} \frac{1}{x^k} = \int \sin x dx \]
\end{nfduc}

\begin{nfdub}
这是一条定理。
\[ \sum_{k=0}^{\infty} \frac{1}{x^k} = \int \sin x dx \]
\end{nfdub}

\begin{nmb}
这是一条定理。
\[ \sum_{k=0}^{\infty} \frac{1}{x^k} = \int \sin x dx \]
\end{nmb}

\begin{ncb}
这是一条定理。
\[ \sum_{k=0}^{\infty} \frac{1}{x^k} = \int \sin x dx \]
\end{ncb}

\begin{ncb}
这是一条定理。
\[ \sum_{k=0}^{\infty} \frac{1}{x^k} = \int \sin x dx \]
\end{ncb}

\begin{ncb}
这是一条定理。
\[ \sum_{k=0}^{\infty} \frac{1}{x^k} = \int \sin x dx \]
\end{ncb}

\begin{ncb}
这是一条定理。
\[ \sum_{k=0}^{\infty} \frac{1}{x^k} = \int \sin x dx \]
\end{ncb}
