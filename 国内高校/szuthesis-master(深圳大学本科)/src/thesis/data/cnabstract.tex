\newpage


\centerline{\fangsong\bf\zihao{-2}{融合内容信息的单类协同过滤推荐算法研究}}
\addcontentsline{toc}{section}{摘要(关键词)}%加入目录


\vskip 1cm

\begin{center}
	\kaishu
	\hspace{2cm}计算机与软件学院计算机科学与技术专业 \quad 徐留成 
	\vspace{5bp}
	\newline
	学号:2012080173
\end{center}

\vskip 10bp

{
\kaishu	
\hspace{5bp}{\zihao{-4}\textbf{【摘要】}} 
对于基于隐式反馈的个性化推荐算法而言,pairwise learning是一个非常重要的技术手段。pairwise learning 算法通常基于这样一个假设:对一个用户而言,相比于未选择过的物品往往会更感兴趣于已选择过的物品。这种假设在推荐算法的学习过程中会衍生出大量的training pairs。而为了应对大规模的数据集,我们所研究的推荐算法往往都是基于均匀采样的随机梯度下降方法进行求解。不过,这种采取均匀采样的策略经常会导致算法收敛非常缓慢。在本文中首先讨论了均匀采样策略导致收敛缓慢的原因,并研究了通过在已有的BPR推荐框架中融合内容信息 改进采样策略并最终提高推荐效果的方法。实验证明,相比于均匀采样策略,通过融合内容信息的适应性采样策略的确能够有助于提高推荐效果。

\vskip 10bp

\hspace{5bp} {\zihao{-4}\textbf{【 关键词】}} 
推荐系统; 协同过滤; 适应性采样  
}