% !TEX program = xelatex
\documentclass{Ecust_Bachelor}
% \graphicspath{{aimg/}}
% \graphicspath{{oimg/}}
% \graphicspath{{timg/}}
\renewcommand{\thesistype}{}
\renewcommand{\thesistitle}{ECUST模板和简易的\LaTeX 教程 }
\hypersetup{
  pdfinfo={
    Author={Yunong Liu},
    Title={\thesistitle{}\thesistype{}},
    CreationDate={2017/5/2},
    ModDate={D:\pdfdate},
    Keywords={\LaTeX , ECUST},
    Subject={ECUST模板和简易的\LaTeX 教程 }
  }
}
\addbibresource{myref.bib}
\begin{document}
    \begin{abstractzh}{华东理工大学, 本科生毕业论文,\LaTeX 模板}
        本文是一份参考华东理工大学商学院2017年本科生毕业论文格式要求而制作的\LaTeX 模板的论文示例,涉及了模板的用法及基础的\LaTeX 教程。
    \end{abstractzh}
    \begin{abstracten}{ECUST, Bachelor Thesis, \LaTeX}
        This paper is a \LaTeX model which refer the ECUST business college 2017 bachelor thesis document.The whole article is about how to use the .cls file and the fundament of \LaTeX .
    \end{abstracten}

    \mktableofcontents

    \section{\LaTeX 简介}
    简单言之,\LaTeX 是一种基于\TeX 的文档排版系统。对于其初学者来讲,理解这个概念可能有点困难。但要使用它并不需要理解什么是\LaTeX 。下一个完整的\LaTeX 发行版,就可以愉快和\LaTeX 玩耍了。

    \subsection{\LaTeX 发行版}
    现代的\TeX 系统都是包含了引擎,编译脚本,格式转换工具,管理配置界面,成千上万宏包和文档的集合体,将这些统一打包,就是一个\TeX 发行版。

    目前,流行的\TeX 发行版有\CTeX ,MiK\TeX ,\TeX Live 三个。本文推荐\TeX Live。对于\LaTeX 新用户来说,最快组装一个能用的\TeX 系统才是最关键的。而这只需要两步。
    \begin{compactenum}
        \item 下载\TeX Live
        \item 下载任意\LaTeX 编辑器
    \end{compactenum}
    对\TeX 入门者,推荐使用TeXstudio编辑。其提供的文档一键编译,命令提示,pdf反编译等功能可以减少入门者学习\LaTeX  遇到的障碍。对于爱好传统编辑器,如Vim,Emacs等的用户,请自行寻找相关插件。

    \subsection{\LaTeX 历史}
    Knuth教授为了排版他的七卷本著作《计算机程序设计艺术》而编制了\TeX 系统的。Lamport博士基于Knuth教授开发的\TeX 系统开发了\LaTeX。作为一套开源的排版系统,\LaTeX 素来以其在数学公式上精美的排版而闻名。从\TeX 系统诞生起,其变化与电子排版体系一起发展。感兴趣的读者可自行寻找相关资料。

    \subsection{\LaTeX 功能}
    对于学术研究人员来讲,\LaTeX 系统,可以几乎满足你的任何要求,得益于强大的开源社区力量,你想做的任何事情,都有相关的包来解决。只要你有一个善于发现的眼睛,你可以使用\LaTeX 进行任何排版。
    \clearpage

    \section{华东理工大学\LaTeX 模板}
    本文的整体版式基于华东理工大学商学院2017年本科生毕业论文格式要求。由于word与\LaTeX 的区别。不保证完全一致。开题报告,文献翻译,论文的模板,请分别参考o.tex,t.tex,a.tex三个文件自行修改。本文档简要介绍\LaTeX 的部分基础概念,并引出本模板的用法。若你从未听说过\LaTeX ,请先阅读刘海洋编著的《\LaTeX 入门》\cite{_latex_2013}
    \subsection{\LaTeX 文档基础}
    每一篇\LaTeX 文档,都有一个基本的框架,进行必要的基本设置,就可以填入内容了。一个标准中文环境设置大概如下:
        \begin{lstlisting}
\documentclass{ctexart}
\title{}
\author{}
\date{}
\begin{documnet}
\maketitle
内容
\end{documnet}
        \end{lstlisting}
    在对应的大括号内填入内容,并编译后,就可以得到一份完整排版的文档了。现对其逐条进行说明:
    \begin{compactitem}
        \item 第一行是文档类,ctexart是中文文档类。
        \item 第二行到第四行声明了整个文章的标题,作者和写作日期。这些内容并不会马上出现在编译的结果内,而是要通过第六行的\\maketitle来排版。
        \item 第六行到第九行的doucument环境内是文档的主体,也是我们要填充的地方。
    \end{compactitem}
    \subsubsection{开题报告和文献翻译模板}
    参考这个基础的\LaTeX 文档结构。我们的模板与之类似。开题报告和文献翻译的模板如下:
    \begin{lstlisting}
\documentclass{Ecust_Bachelor}
\renewcommand{\thesistype}{(开题报告)}%可改为(文献翻译)
\renewcommand{\thesistitle}{}
\hypersetup{
pdfinfo={
    Author={},
    Title={\thesistitle{}\thesistype},
    CreationDate={},
    ModDate={D:\pdfdate},
    Keywords={},
    Subject={}
  }
}
\addbibresource{myref.bib}
\begin{document}
\label{title:t1}
\pdfbookmark[0]{标题}{title:t1}
\mktitle{\thesistitle}{班级(学号)姓名}
\mkabstract{摘要}{关键字}

\end{document}
    \end{lstlisting}

对其解释如下:
\begin{compactitem}
    \item 第一行指明了模板类型。用户无须修改。
    \item 第二行和第三行分别指明了文档类型和开题报告(文献翻译)的标题,需要用户自行修改。
    \item 第四行到第十三行内为生成文档的元数据。不会出现在正文中。
    \item 第十四行引入了参考文献bib文件
    \item 第十八行为作者的相关信息。
    \item 第十九行内分别需要填入文档的摘要和关键字。
\end{compactitem}
具体使用方法,用户可参考o.tex文档。
\subsubsection{正文模板}
论文正文模板如下:
\begin{lstlisting}
\documentclass{Ecust_Bachelor}
\renewcommand{\thesistype}{}
\renewcommand{\thesistitle}{}
\hypersetup{
    pdfinfo={
        Author={},
        Title={\thesistitle{}\thesistype},
        CreationDate={},
        ModDate={D:\pdfdate},
        Keywords={},
        Subject={}
      }
    }
\addbibresource{myref.bib}
\begin{document}
   \begin{abstractzh}{}

   \end{abstractzh}
   \begin{abstracten}{}

   \end{abstracten}
   \mktableofcontents
\end{document}
\end{lstlisting}
模板的用法与开题报告(文献翻译)用法类似,唯一区别在于摘要的写法,请用户自行参考a.tex(同样为本文tex原文档)
\clearpage
\section{正文内容}
\subsection{章节题目}
文章的章节,子章节,孙章节由以下三个命令提供。
\begin{lstlisting}
    \section{章节题目}
    \subsection{子章节题目}
    \subsubsection{孙章节题目}
    %\clearpage
\end{lstlisting}
在一个章节结束后输入$\backslash$clearpage可强制换页。
\subsection{致谢}
致谢请在以下环境中输入:
\begin{lstlisting}
    \begin{acknowledgement}
        致谢内容
    \end{acknowledgement}
\end{lstlisting}
\subsection{列表环境}
\subsubsection{普通列表}
\begin{lstlisting}
    \begin{compactitem}
        \item 普通列表
        \item 普通列表
    \end{compactitem}
\end{lstlisting}
效果如下:
\begin{compactitem}
    \item 普通列表
    \item 普通列表
\end{compactitem}
\subsubsection{数字列表}
\begin{lstlisting}
    \begin{compactenum}
        \item 数字列表
        \item 数字列表
    \end{compactenum}
\end{lstlisting}
效果如下:
\begin{compactenum}
    \item 数字列表
    \item 数字列表
\end{compactenum}
\subsection{图片}
    \begin{lstlisting}
        \begin{figure}[!htb]
            \centering
            \includegraphics[width=14cm]{图片全名}%图片路径
            \caption{图片标题}%标题
            \label{图片标签}%引用所需
        \end{figure}
    \end{lstlisting}
    效果如图\ref{f1}所示
    \begin{figure}[!htb]
        \centering
        \includegraphics[width=8cm]{images.png}%图片路径
        \caption{ECUST}%标题
        \label{f1}%引用所需
    \end{figure}
    \subsection{表格}
    \subsubsection{普通表格}
    \begin{lstlisting}
        \begin{table}
            \centering
            \caption{}
            \label{}
            \begin{tabular}{|c|c|c|}
                \hline
                1 & 2 & 3 \\
                \hline
                4 & 5 & 6 \\
                \hline
                7 & 8 & 9 \\
                \hline
            \end{tabular}
        \end{table}
    \end{lstlisting}
    效果如表\ref{t1}
    \begin{table}
        \centering
        \caption{普通表格}
        \label{t1}
        \begin{tabular}{|c|c|c|}
            \hline
            1 & 2 & 3 \\
            \hline
            4 & 5 & 6 \\
            \hline
            7 & 8 & 9 \\
            \hline
        \end{tabular}
    \end{table}
    \subsubsection{三线表}
    \begin{lstlisting}
        \begin{table}
            \centering
            \caption{}
            \label{}
            \begin{tabular}{ccc}
                \toprule
                1 & 2 & 3 \\
                \midrule
                4 & 5 & 6 \\
                7 & 8 & 9 \\
                \bottomrule
            \end{tabular}
        \end{table}
    \end{lstlisting}
    效果如表\ref{t2}
    \begin{table}
        \centering
        \caption{三线表}
        \label{t2}
        \begin{tabular}{ccc}
            \toprule
            1 & 2 & 3 \\
            \midrule
            4 & 5 & 6 \\
            7 & 8 & 9 \\
            \bottomrule
        \end{tabular}
    \end{table}
    \subsection{公式}
    \begin{lstlisting}
        \begin{equation}
            1+2=3
        \end{equation}
        \begin{equation}
            a+b=c
        \end{equation}
        \begin{equation}
            \frac{1}{2}+\frac{1}{2}=1
        \end{equation}
    \end{lstlisting}
结果如下:
    \begin{equation}
        1+2=3
    \end{equation}
    \begin{equation}
        a+b=c
    \end{equation}
    \begin{equation}
        \frac{1}{2}+\frac{1}{2}=1
    \end{equation}
关于\LaTeX 的数学输入,用户请自行查找amsmath宏包,参考其说明文档。
\subsection{文档内引用}
\LaTeX 内部有计数器控制标题,图片,表格的序号。用户若想引用文档的某个章节,表格,图片。只需在被引用位置输入命令$\backslash$label\{任意标签\},在引用位置位置输入$\backslash$ref\{被引位置标签\}。一个简单的例子如下:
\begin{lstlisting}
    .......
    \begin{document}
        \section{被引用章节}
        \label{l1}
        第\ref{l1}章啦啦啦啦啦啦。
    .......
    \end{document}
\end{lstlisting}
\subsection{参考文献}
本模板使用bibLaTeX来管理参考文献,用户只需将其涉及到的参考文献统一输入一个bib文件中。将其在文档头部引入,并在文档末尾输入如下命令即可:
\begin{lstlisting}
    %\nocite{*}
     \printbibliography[heading=bibliography,title=参考文献}
\end{lstlisting}
对参考文献的引用类似于文档内的引用,命令是$\backslash$cite\{bibtex引用关键词\}。

注意:参考文献中只会打印出正文中有引用的文献,若想将所有bib文献中的参考文献打印,需在打印文献命令前输入$\backslash$nocite\{*\}。

关于bib文件的制作,有以下几个方式:
\begin{compactitem}
    \item Google学术
    \item Zotero的浏览器插件
    \item 一些学术旗杆官网直接获取
\end{compactitem}
一个bib条目的示例如下:
\begin{lstlisting}
@Book{张三123−−,  %biblatex引用关键词
    Title = {论如何与傻逼相处},
    Author = {张三},
    Publisher = {SB出版社},
    Year = {2017},
    Location = {上海},
    }
\end{lstlisting}
\subsection{特殊需求}
\LaTeX 系统几乎可以实现用户排版的任何想法。由于篇幅所限,本文不是\LaTeX 的教科书,如果用户有额外的需求,可以从以下渠道获取:
\begin{compactitem}
    \item 命令行下texdoc加关键词获取说明文档
    \item TUG(\TeX 用户组织)获取\LaTeX 的教程
    \item Google,Baidu搜索引擎
    \item \CTeX 中文论坛
\end{compactitem}
\clearpage
\section{文档编译}
由于字符系统的限制,对于中文环境的编译来说,xelatex引擎是最佳选择。同时,bib文件编译的后端请使用biber。若你是\TeX studio等专业\LaTeX 编辑器的用户,在设置好引擎后可一键编译。对于使用命令行的用户,需要编译四次,编译顺序及代码如下:
(以论文正文为例)
\begin{lstlisting}
    xelatex a
    biber a
    xelatex a
    xelatex a
\end{lstlisting}
第一次编译输出论文主体,第二次编译输入参考文献,第三次和第四次输入交叉引用的部分。具体工作原理,请自行查询相关资料。
\clearpage
\printbibliography[heading=bibliography,title=参考文献]
\end{document}
