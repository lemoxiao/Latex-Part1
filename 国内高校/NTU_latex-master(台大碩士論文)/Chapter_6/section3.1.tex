%!TEX root = ../thesis.tex
\subsection{基因演算法概述}
\label{c:ch6.3.1}
基因演算法的核心理念主要可以回溯到1859年達爾文(Charles Darwin)提到「物競天擇,適者生存」的演化及淘汰概念,仿效自然界生物遺傳進化之天性,以適應該環境之生存型態,在1975年,此電腦模擬物種生存模式的演算法-基因演算法(GA, Genetic Algorithms),由\cite{holland1975adaptation}等人提出基因演算法概念。

在\shortcite{ritzel1994using}提到近幾年,基因演算法對於複雜的非線性規劃問題,具有良好的求解能力,\cite{goldberg1988genetic}表示基因演算法具有下列幾點優良的特性
\begin{enumerate}[(1)]
	\item 為了避免陷入區域最佳解,基因演算法會同時以多個點搜尋而不是單點做最佳化,因此可避免陷入區域的最佳解。
	\item 搜尋過程以機率的方式代替明確的規則,將問題引導至最佳解,故任何問題皆適用。
	\item 不需要繁瑣的數學計算,單純以適應函式作為優劣的評判。
\end{enumerate}

在本研究我們以\cite{Wu.etc}所使用的典型基因演算法中,介紹演算法的核心流程;基因演算是模擬自然界的演化程序,包括有複製(reproduction)、交配(crossover)與突變(mutation)等\dots,由於基因演算法是模擬基因的演化過程,其演化程序包括,複製、交配以及突變,從上一代到這一代的演化流程中染色體有可能會對自己進行「複製」以延續自己的序列組合,或者會與其他染色體進行「交配」,再下一代次中,有可能得到較好的配對結果成為下一代次,最後在交配以及複製期間有可能會產生「突變」的染色體,因此有可能產生比原本的基因更好的結果。
