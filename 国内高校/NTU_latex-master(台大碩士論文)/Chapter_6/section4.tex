%!TEX root = ../thesis.tex
\section{最佳化參數組合與其他業界常用組合比較}
\label{c:ch6.4}
在這個小節當中,主要分別以本研究的最佳解方法以及可行範圍內隨機抽樣的樣本與業界常用的組合,相互比較並探討研究的方法在其他解中的優劣;在前面的小節,我們從模型數值化到序列二次規劃法以及基因演算法得到模型各自的最佳解與業界常用的組合,整理合併後得到表\ref{tab:table3}以及表\ref{tab:table7},但為了與其他可能會出現在業者所制定的染整製程參數組合比較,我們則從可行區域中提供幾組解作為比較的參數組合。
\begin{table}[!htbp]
	\caption{最佳化組合與其他業界常用組合比較表}
	\center
	%!TEX root = ../thesis.tex
\begin{tabular}{cccccccccc}
\hline
\multirow{2}{*}{Method} &
\multirow{2}{*}{Items} &
\multicolumn{5}{c}{Factors} &
\multicolumn{3}{c}{\multirow{1}{*}{Result}} \\
\cline{3-10}
  & &$x_A$ & $x_B$ & $x_C$ & $x_D$ & $x_E$ & Cost & Robust & $\Delta E$ \\
\hline\hline
GA演算法 & 最小化成本 & 51 & 1.3 & 76 & 18 & 15 & 5368 & 0.49 & 0.63 \\ 
& 穩定度最佳化 & 59 & 1.2 & 86 & 19 & 20 & 6590 & 0.065 & 0.168 \\ 
& 品質最佳化 & 61 & 1.2 & 82 & 19 & 23 & 7163 & 0.115 & 0.086 \\
\\
SQP方法 & 最小化成本 & 52 & 1.2 & 76 & 16 & 14 & 5044 & 0.558 & 0.775 \\ 
& 穩定度最佳化 & 59 & 1.2 & 86 & 18 & 20 & 6561 & 0.061 & 0.225 \\ 
& 品質最佳化 & 61 & 1.2 & 78 & 19 & 25 & 7545 & 0.255 & 0.042 \\ 
\\
隨機樣本 & 樣本一 & 54 & 1.4 & 79 & 14 & 22 & 6888 & 0.432 & 0.528 \\ 
& 樣本二 & 54 & 1.4 & 83 & 14 & 22 & 6975 & 0.285 & 0.557 \\ 
& 樣本三 & 54 & 1.4 & 83 & 16 & 22 & 7033 & 0.229 & 0.426 \\
& 樣本四 & 54 & 1.6 & 79 & 14 & 22 & 6888 & 0.399 & 0.611 \\ 
& 樣本五 & 54 & 1.6 & 79 & 16 & 22 & 6946 & 0.342 & 0.468 \\ 
& 樣本六 & 54 & 1.6 & 83 & 14 & 22 & 6975 & 0.254 & 0.609 \\ 
& 樣本七 & 54 & 1.6 & 83 & 16 & 22 & 7033 & 0.198 & 0.467 \\ 
\\
長期經驗 & 業界常用 & 56 & 1.5 & 81 & 15 & 20 & 6454 & 0.288 & 0.488 \\
\hline
\end{tabular}
	\label{tab:table8}
\end{table}

本研究以隨機抽樣的方式,搭配出不同的32種參數組合,但由於必須限制在$\Delta E$低於0.8以及$K/S$介於95到105之間,所以實際可行的隨機組合只有得到部分幾組,並將隨機的製程參數組合分別帶入各個目標函式後,如表\ref{tab:table8},將樣本的結果以及表\ref{tab:table3}和表\ref{tab:table7}合併後進行比較;從表\ref{tab:table8}可以觀察到,使用序列二次規劃法以及GA演算法搜尋的最佳解,在搜尋各自的項目上與其他的製程參數組合比較,是相對優秀的,說明此兩種方法在解決染整製程最佳化問題都能找到有效的區域最佳解;那麼在比較此兩種方法中,可以看出序列二次規劃法相對應的最佳化結果,都比基因演算法的結果再好一些,而且計算上更快速一些,故在表\ref{tab:table8}中,序列二次規劃法較基因演算法好一些,不過為了探討本研究提出的方法在整體上是否可以成功的替代業界常用的組合,我們必須從業界所提供的的其他資訊以及假設作為本研究整體評估的考量。

業界常用的製程參數組合,套用到實際的大染缸中,業者表示會有15\%的染布會與目標間存在差異,所以在$\Delta E$超過0.8時染布就需要重染或報廢,因此我們假設整體的報廢成本為失敗機率的等比級數總和乘以單次染布的運作成本,則我們假設評估整體的總成本為運作成本以及平均報廢成本的總和,而業界常用組合的平均總成本約為$6454\div (1-0.15)\simeq 7593$元。

業界常用參數是由長期的經驗法則所得來的結果,而且實際紡織業者提供一天所使用的染缸約160缸,故我們假設業界常用參數組合所呈現的$\Delta E$是符合平均數為0.488,及變異數未知的常態分配假設,雖然變異數未知,但我們從業者提供的資訊知道當超過$\Delta E$為0.8時,則失敗率為0.15中可推得其變異數為$[(0.8-0.488)/1.04]^2 = 0.09$;同樣的在這裡我們也假設其他的製程參數組合也符合平均數為各自的$\Delta E$,變異數為未知的常態分配,且通常越穩定的製程參數其$\Delta E$的變異會越小,故在這裡我們假設製程參數的變異數比與穩定度是成一般反比的關係,由以上這些假設條件下,我們便可以得到各組合的總成本,如表\ref{tab:table9}所示。
\newpage
\begin{table}
	\caption{最佳化組合與其他業界常用組合估計總成本比較表}
	\center
	%!TEX root = ../thesis.tex
\begin{tabular}{cccccc}
\hline
\multirow{2}{*}{Method}&
\multirow{2}{*}{Items} &
\multicolumn{4}{c}{Result} \\
\cline{3-6}
 & & Average & Variance & Failure rate & Total cost \\
\hline\hline
GA演算法 & 最小化成本 & 0.63 & 0.153 & 33.2\% & 8036 \\ 
& 穩定度最佳化 & 0.168 & 0.020 & 0.0005\% & 6590 \\ 
& 品質最佳化 & 0.086 & 0.036 & 0.0083\% & 7164 \\ 
\\
 SQP方法 & 最小化成本 & 0.775 & 0.174 & 48.0\% & 9700 \\ 
& 穩定度最佳化 & 0.225 & 0.019 & 0.0016\% & 6561 \\ 
& 品質最佳化 & 0.042 & 0.079 & 0.36\% & 7572 \\ 
\\
隨機樣本 & 樣本一 & 0.528 & 0.135 & 22.9\% & 8933\\ 
& 樣本二 & 0.557 & 0.089 & 20.7\% & 8795 \\ 
& 樣本三 & 0.426 & 0.072 & 8.1\% & 7652 \\
& 樣本四 & 0.611 & 0.125 & 29.6\% & 9784 \\ 
& 樣本五 & 0.468 & 0.107 & 15.5\% & 8220 \\ 
& 樣本六 & 0.609 & 0.079 & 24.9\% & 9287 \\ 
& 樣本七 & 0.467 & 0.062 & 9.0\% & 7728 \\ 
\\
長期經驗 & 業界常用 & 0.488 & 0.09 & 15\% & 7593 \\
\hline
\end{tabular}
	\label{tab:table9}
\end{table}

從表\ref{tab:table9}中,雖然隨機樣本的參數組合中的樣本三對業界常用組合較為穩定以及品質較高,但單次估計總成本卻比較高,可見業界常用組合是普遍在其他隨機樣本中確實是比較好的,而本研究的穩定度最佳化參數組合,以及使用基因演算法中的穩定度最佳化組合,則在估計總成本中比業界常用成本更低;雖然基因演算法穩定度結果與本研究方法差異不大,但是從綜合成本可量下,可看出本研究的方法還是稍微好一些,故在本研究的方法估計的綜合考量下,運作成本最佳組合以及品質最佳組合,雖然以該項目作為比較依據都比基因演算法好,但綜合評估下的成本卻比較高一些,故在選擇上業者需要自行評估哪一個模型比較符合當下需求,就本研究來說,建議使用SQP得到的穩定度模型最佳解是最好的選擇。
