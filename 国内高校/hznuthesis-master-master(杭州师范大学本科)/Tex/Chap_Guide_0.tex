
\chapter{使用简介}
\label{chap:guide}

\section{数学公式、图表、参考文献等功能}

\subsection{数学公式}

比如Navier-Stokes方程:
\begin{equation} \label{eq:ns}
    \begin{cases}
        \frac{\partial \rho}{\partial t} + \nabla\cdot(\rho\Vector{V}) = 0 \\
        \frac{\partial (\rho\Vector{V})}{\partial t} + \nabla\cdot(\rho\Vector{V}\Vector{V}) = \nabla\cdot\Tensor{\sigma}\\
        \frac{\partial (\rho E)}{\partial t} + \nabla\cdot(\rho E\Vector{V}) = \nabla\cdot(k\nabla T) + \nabla\cdot(\Tensor{\sigma}\cdot\Vector{V})
    \end{cases}
\end{equation}

数学公式常用命令请见 \href{https://en.wikibooks.org/wiki/LaTeX/Mathematics}{WiKibook Mathematics}。artracom.sty中对一些常用数据类型如矢量矩阵等进行了封装,这样的好处是如有一天需要修改矢量的显示形式,只需单独修改artracom.sty中的矢量定义即可实现全文档的修改。

\subsection{表格}

请见表~\ref{tab:sample}。制表的更多范例,请见 \href{https://en.wikibooks.org/wiki/LaTeX/Tables}{WiKibook Tables}。
\begin{table}[!htbp]
    \bicaption{这是一个样表。}{This is a sample table.}
    \label{tab:sample}
    \centering
    \footnotesize% fontsize
    \setlength{\tabcolsep}{4pt}% column separation
    \renewcommand{\arraystretch}{1.2}%row space 
    \begin{tabular}{lcccccccc}
        \hline
        Row number & \multicolumn{8}{c}{This is a multicolumn} \\
        %\cline{2-9}% partial hline from column i to column j
        \hline
        Row 1 & $1$ & $2$ & $4$ & $5$ & $6$ & $7$ & $8$\\
        Row 2 & $1$ & $2$ & $4$ & $5$ & $6$ & $7$ & $8$\\
        Row 3 & $1$ & $2$ & $4$ & $5$ & $6$ & $7$ & $8$\\
        Row 4 & $1$ & $2$ & $4$ & $5$ & $6$ & $7$ & $8$\\
        \hline
    \end{tabular}
\end{table}

\subsection{图片插入}

插图的参考代码在\verb|Tex/Commands.tex|亦进行了汇集。

\subsection{算法}

如见算法~\ref{alg:euclid},详细使用方法请参见文档 \href{https://ctan.org/pkg/algorithmicx?lang=en}{algorithmicx}。

\begin{algorithm}[!htbp]
    \small
    \caption{Euclid's algorithm}\label{alg:euclid}
    \begin{algorithmic}[1]
        \Procedure{Euclid}{$a,b$}\Comment{The g.c.d. of a and b}
        \State $r\gets a\bmod b$
        \While{$r\not=0$}\Comment{We have the answer if r is 0}
        \State $a\gets b$
        \State $b\gets r$
        \State $r\gets a\bmod b$
        \EndWhile\label{euclidendwhile}
        \State \textbf{return} $b$\Comment{The gcd is b}
        \EndProcedure
    \end{algorithmic}
\end{algorithm}

\subsection{参考文献引用}

参考文献引用过程以实例进行介绍,假设需要引用名为"Document Preparation System"的文献,步骤如下:

1)使用Google Scholar搜索Document Preparation System,在目标条目下点击Cite,展开后选择Import into BibTeX打开此文章的BibTeX索引信息,将它们copy添加到ref.bib文件中(此文件位于Biblio文件夹下)。

2)索引第一行 \verb|@article{lamport1986document,|中 \verb|lamport1986document| 即为此文献的label (\textbf{中文文献也必须使用英文label},一般遵照:姓氏拼音+年份+标题第一字拼音的格式),想要在论文中索引此文献,有两种索引类型:

文本类型:\verb|\citet{lamport1986document}|。正如此处所示 \citet{MLin2016complexnet}; 

括号类型:\verb|\citep{lamport1986document}|。正如此处所示 \citep{MLin2016complexnet}。

\textbf{多文献索引用英文逗号隔开}:

\verb|\citep{lamport1986document,chen2005zhulu}|。正如此处所示 \citep{MLin2016complexnet,Ren2014WeiboEvents}

如此,即完成了文献的索引,请查看下本文档的参考文献一章,看看是不是就是这么简单呢?是的,就是这么简单!

不同文献样式和引用样式可在Thesis.tex中对artratex.sty调用实现,如:
\begin{itemize}
    \footnotesize
    \item \verb+\usepackage[numbers]{artratex}+ $\%$ 文本: Jones [1]; 括号: [1]
    \item \verb+\usepackage[super]{artratex}+ $\%$ 文本: Jones 上标[1]; 括号: 上标[1]
    \item \verb+\usepackage[authoryear]{artratex}+ $\%$ 文本: Jones (1995); 括号: (Jones, 1995)
    \item \verb+\usepackage[alpha]{artratex}+ $\%$ 文本: 不可用; 括号: [Jon95]
\end{itemize}

若在上标(super)模式下,希望在特定位置将上标改为嵌入式标,可使用

文本类型:\verb|\citetns{lamport1986document,chen2005zhulu}|。

正如此处所示\citetns{lamport1986document,chen2005zhulu}

括号类型:\verb|\citepns{lamport1986document,chen2005zhulu}|。

正如此处所示\citepns{lamport1986document,chen2005zhulu}

参考文献索引更为详细的信息,请见 \href{https://en.wikibooks.org/wiki/LaTeX/Bibliography_Management}{WiKibook Bibliography}。

\nocite{*}

\section{常见使用问题}\label{sec:qa}

\begin{enumerate}
    \item 模板每次发布前,都已在Windows,Linux,MacOS系统上测试通过。下载模板后,若编译出现错误,则请遵从 \href{https://github.com/mohuangrui/ucasthesis}{位于主页底部的用户指南}。

    \item 模板文档的编码为UTF-8编码。所有文件都必须采用UTF-8编码,否则编译后生成的文档将出现乱码文本。若出现文本编辑器无法打开文档或打开文档乱码的问题,请检查您使用的编辑器对UTF-8编码的支持。如果使用WinEdt作为文本编辑器(不推荐使用),应在其Options -> Preferences -> wrapping选项卡下将两种Wrapping Modes中的内容:
        
        TeX;HTML;ANSI;ASCII|DTX...
        
        修改为:TeX;\textbf{UTF-8|ACP;}HTML;ANSI;ASCII|DTX...
        
        同时,取消Options -> Preferences -> Unicode中的Enable ANSI Format。

    \item 推荐选择xelatex或lualatex编译引擎编译中文文档。编译脚本的默认设定为xelatex编译引擎。你也可以选择不使用脚本编译,如直接使用 \TeX{}文本编辑器编译。注:\TeX{}文本编辑器编译的默认设定为pdflatex编译引擎,若选择xelatex或lualatex编译引擎,请进入下拉菜单选择。为正确生成引用链接,需要进行全编译。
    \item Texmaker使用简介
        \begin{enumerate}
            \footnotesize
            \item 使用 Texmaker “打开” Thesis.tex。
            \item 菜单 “选项 (Options)” -> “设置当前文档为主文档 (Define as Master Document)”
            \item 菜单 “自定义 (User)” -> “自定义命令 (User Commands)” -> “编辑自定义命令 (Edit User Commands)” -> 左侧选择 “command 1”,右侧 “菜单项 (Menu Item)” 填入 Auto Build -> 点击下方“向导 (Wizard)” -> “添加 (Add)”: xelatex + bibtex + xelatex + xelatex + pdf viewer -> 点击“完成 (OK)”
            \item 使用 Auto Build 编译带有未生成引用链接的源文件,可以仅使用 xelatex 编译带有已经正确生成引用链接的源文件。
            \item 编译完成,“查看(View)” PDF,在pdf中 “ctrl+click” 可链接到相对应的源文件。
        \end{enumerate}
    
    \item 模版的设计可能地考虑了适应性。致谢等所有条目都是通过最为通用的

        \verb+\chapter{item name}+  and \verb+\section*{item name}+

        来显式实现的 (请观察Backmatter.tex),从而可以随意添加,放置,和修改,如同一般章节。对于图表目录名称则可在ucasthesis.cfg中进行修改。

    \item 设置文档样式: 在artratex.sty中搜索关键字定位相应命令,然后修改
        \begin{enumerate}
            \item 正文行距:启用和设置 \verb|\linespread{1.5}|,默认1.5倍行距。
            \item 参考文献行距:修改 \verb|\setlength{\bibsep}{0.0ex}|
            \item 目录显示subsection:修改 \verb|\setcounter{tocdepth}{2}|
            \item 文档超链接的颜色及其显示:修改 \verb|\hypersetup|
            \item 页眉页脚设定:frontmatterstyle,mainmatterstyle,和backmatterstyle分别定义前言,主要内容,和附录的页眉页脚样式。通过阅读这一部分的代码,可以轻松地理解和修改以获得自定义的样式。命令的详细解释请参见 \href{https://www.ctan.org/pkg/fancyhdr?lang=en}{fancyhdr} 的用户文档。同时可参见 \href{https://ctan.org/pkg/ctex?lang=en}{ctex} 宏包用户文档。

            \item 设置图2.3为图2-3: 设置
                {
                    \footnotesize
\begin{verbatim}
\renewcommand{\theequation}{\arabic{chapter}-\arabic{equation}}
\renewcommand{\thefigure}{\arabic{chapter}-\arabic{figure}}
\renewcommand{\thetable}{\arabic{chapter}-\arabic{table}}
\end{verbatim}
                }
        \end{enumerate}

    \item 字体控制。文档内字体切换方法:
        \begin{itemize}
            \item 宋体:飞扬跋扈~或 \textrm{飞扬跋扈}
            \item 粗宋体:{\bfseries 飞扬跋扈} 或 \textbf{飞扬跋扈}
            \item 黑体:{\sffamily 飞扬跋扈} 或 \textsf{飞扬跋扈}
            \item 粗黑体:{\bfseries\sffamily 飞扬跋扈} 或 \textsf{\bfseries 飞扬跋扈}
            \item 仿宋:{\ttfamily 飞扬跋扈} 或 \texttt{飞扬跋扈}
            \item 楷体:{\itshape 飞扬跋扈} 或 \textit{飞扬跋扈}
        \end{itemize}
        
        由于缺乏一个统一的被各个操作系统所默认携带的完备的中文字体库,\href{https://ctan.org/pkg/ctex?lang=en}{ctex} 针对不同的操作系统而调用各系统上所对应的一类中文字体库。由于很多操作系统的字库往往缺乏原生态的加粗宋体字重,有时会发生加粗宋体被黑体所替换的情形,这对封面的字体造成影响。若需要解决这个问题,可采用调用自定义的一个完备字体库的方案。若需设置字体库,请选择xelatex或lualatex编译引擎,并设置需要的字体库。
        
        如\textbf{用Times New Roman作为英文和数字字体},在artratex.sty的

        \verb|\RequirePackage{fontspec}|

        行下添加如下英文字体调用命令:

        \verb|\setmainfont{Times New Roman}| (设置英文正文字体)

        \verb|\setsansfont{Times New Roman}| (设置英文标题字体)

        Windows和Mac OS自带此字体,Linux需手动安装或使用FreeSerif字体作为替代。部分\LaTeX{}版本,若setmainfont与setsansfont调用同一字体名称,会导致字体冲突,进而导致非加粗黑体失效,影响目录及节标题的显示。解决此问题的办法为1)不设置setsansfont 或 2)调用一个名字不同的Times类字体,如setsansfont\{Times\},其中Times字体为Mac OS所带有。长图表标题的英文标题的悬挂缩进是针对Times类正文字体校准的,非Times类正文字体将存在轻微位置偏差。
                 
        如果需要调用一个\textbf{自定义的中文字体库},方法为:

        \begin{itemize}
            \item 调用 \href{https://ctan.org/pkg/ctex?lang=en}{ctex} 预定义好的备用字库: 在Thesis.tex中设置

                {
                    \small\verb|\documentclass[doublesided,fontset=fandol]{Style/ucasthesis}%|
                }

                便可调用 \href{https://ctan.org/tex-archive/fonts/fandol?lang=en}{fandol} 这一字体库。\LaTeX{}编译系统一般已携带或是能自动下载安装 \href{https://ctan.org/tex-archive/fonts/fandol?lang=en}{fandol} 字库。若不能,则请手动下载并安装链接所提供的所有字体即可。本模板使用说明文档就是采用调用 \href{https://ctan.org/tex-archive/fonts/fandol?lang=en}{fandol} 中文字库。

            \item 手动调用系统带有的中文字库: 首先需要查看系统所带有的中文字库及其名称,也可选择安装可获得的中文字库。假设系统已安装或带有名为SC的中文字库(此字库为MacOS所配备,具备原生态的加粗宋体),则可在artratex.sty的

                \verb|\RequirePackage{fontspec}|

                行下添加如下中文字体调用命令:
                {
                    \scriptsize
\begin{verbatim}
\setCJKmainfont[BoldFont=Songti SC Bold,ItalicFont=Kaiti SC]{Songti SC Light}%
\setCJKsansfont{Heiti SC}%
\end{verbatim}
                }
        \end{itemize}

        字库调用的全面解释可参见 \href{https://ctan.org/pkg/fontspec}{fontspec} (英文字体调用)和 \href{https://ctan.org/pkg/xecjk?lang=en}{xeCJK} (中文字体调用)。因为模版的设定考虑兼顾不同操作系统(Windows, Linux, Mac OS),为了模版的健壮性,上述字体设置和调用方案并未作为原始设定。

    \item 封面下划线上的文本不居中下划线,这是因为那些下划线前面还有字头,导致文本只能在页面居中和在下划线上居中二选一。当前封面采取页面居中。如需要调整文本在下划线上的位置,可用 \verb|\hspace{+/- n.0em}| 命令来插入或删除 n 个空格,进行手动调整,比如

        \verb|\advisor{\hspace{+3.0em} xxx~研究员~xxx单位}|
                
                这个解决方案是很不优雅,但问题本质还是样式的设计问题。有时下划线看上去粗细不一致,这是显示的问题,打印正常。

    \item 对于电子档的论文,在Thesis.tex的documentclass中,可使用singlesided来减少空白页。而对于打印版,可考虑printcopy选项使奇偶页的排版在打印装订后更美观。

    \item 部分所也许对格式有不同设定,因\LaTeX{}将内容与格式分离,格式的调整可独立于内容进行,并只需修改模板样式文件完成,并无风险。
\end{enumerate}


