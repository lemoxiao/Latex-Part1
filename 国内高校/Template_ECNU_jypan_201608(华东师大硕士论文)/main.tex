% 用 xelatex 编译
%\documentclass[a4paper,cs4size,hyperref,fancyhdr]{ctexbook}
\documentclass[UTF8,openright]{ctexbook}

% 论文版面要求:
% 统一按 word 格式A4纸(页面设置按word默认值)编排、打印、制作。
% 正文内容字体为宋体;字号为小4号;字符间距为标准;行距为25磅(约0.88175cm)。

%%%%% ===== 页面设置
\usepackage[a4paper,top=2.54cm,bottom=2.54cm,left=3.17cm,right=3.17cm,%
            includehead,includefoot]{geometry}
\setlength{\parindent}{2em}
\setlength{\parskip}{0.5em}

%%%%% =====章节 标题 设置
\ctexset{%
  contentsname={目\ \ 录},
  listfigurename={插\ 图\ 目\ 录},
  listtablename={表\ 格\ 目\ 录},
 % bibname={\centerline{参\ 考\ 文\ 献}},
  chapter={name={第,章},
           number=\chinese{chapter},
           nameformat={\zihao{3}\bfseries},
           titleformat={\zihao{3}\bfseries},
           beforeskip={-10pt},
           afterskip={20pt}
           },
  section={format=\raggedright,
           nameformat={\zihao{4}\bfseries},
           titleformat={\zihao{4}\bfseries},
           afterskip={1ex plus 0.2ex}
           },
  subsection={format=\raggedright,
           nameformat={\zihao{-4}\bfseries},
           titleformat={\zihao{-4}\bfseries},
           afterskip={0.5ex plus 0.1ex}
           }
}
%%%%% ===== 中英文字体
\setmainfont{Times New Roman}
%\setsansfont{Myriad Pro} % 无衬线字体 sans serif \sffamily
%\setmonofont{Consolas}   % 等宽字体 typewriter \ttfamily
%\newcommand{\Times}{\fontspec{Times New Roman}}
%% 中文字体
%\setCJKmainfont[BoldFont={Microsoft YaHei},ItalicFont={KaiTi}]{NSimSun}
%\setCJKsansfont{Microsoft YaHei}
%\setCJKmonofont{KaiTi}
%\setCJKfamilyfont{STSong}{方正小标宋_GBK}\newcommand{\STSong}{\CJKfamily{STSong}}
\setCJKfamilyfont{STSong}{STZhongsong}\newcommand{\STSong}{\CJKfamily{STSong}}

%%%%% ===== 常用宏包
\usepackage{amsmath,amssymb,amsfonts,bm}
\usepackage[amsmath,thref,thmmarks,hyperref]{ntheorem}
\usepackage{graphicx,xcolor,float}
\usepackage{fancyhdr}
\usepackage{tocloft} % 设置目录中的条目间距
\usepackage{booktabs} % toprule, midrule, bottomrule
\usepackage{varwidth} % 可变宽度的 parbox

%%%%% ===== 参考文献与链接
\usepackage[numbers,sort&compress,sectionbib]{natbib}
\usepackage[xetex,pagebackref]{hyperref}
\hypersetup{CJKbookmarks=true,colorlinks=true,citecolor=blue,%
            linkcolor=blue,urlcolor=blue,bookmarksnumbered=true,%
	        bookmarksopen=true,breaklinks=true}
\renewcommand*{\backrefalt}[4]{%
\ifcase #1 No citations.%
\or Cited on page #2.%
\else Cited on pages #2.%
%\else #1 Cited on pages #2.%
\fi
}

%%%%% ===== 浮动图表的标题
\usepackage[margin=2em,labelsep=space,skip=0.5em,font=normalfont]{caption}
\DeclareCaptionFormat{mycaption}{{\heiti\color{blue} #1}#2{\kaishu #3}}
\captionsetup{format=mycaption,tablewithin=chapter,figurewithin=chapter}

%%%%% ===== 算法
\usepackage{algorithm,algpseudocode}

%%%%% ===== 其他
\usepackage{ulem}
\def\ULthickness{1pt}

%%%%% ===== 自定义命令
\renewcommand{\C}{\mathbb{C}}
\newcommand{\R}{\mathbb{R}}


%%%%%%%%%%%%%%%%%%%%%%%%%%%%%%%%%%%%%%%%%%%%%%%%%%%%%%%%%%%%%%%%%%%%%
\begin{document}

%%%%% ===== 格式
\pagestyle{plain}                                                                          %plain格式页眉页脚
\titleformat{\section}{\centering\heiti\zihao{-2}}{实验\,\thesection: }{1em}{}              %小二号黑体居中
\titleformat{\subsection}{\heiti\zihao{-3}}{\thesubsection}{1em}{}                         %小三号黑体
\titleformat{\subsubsection}{\heiti\zihao{-4}}{\thesubsubsection}{1em}{}                   %小四号黑体
\lstset{
    basicstyle=\tt,                                                                        %行号
    numbers=left,
    rulesepcolor=\color{red!20!green!20!blue!20},
    escapeinside=``,
    xleftmargin=2em,xrightmargin=2em, aboveskip=1em,                                       %背景框
    framexleftmargin=1.5mm,
    frame=shadowbox,                                                                       %背景色
    backgroundcolor=\color[RGB]{245,245,244},                                              %样式
    keywordstyle=\color{blue}\bfseries,
    identifierstyle=\bf,
    numberstyle=\color[RGB]{0,192,192},
    commentstyle=\it\color[RGB]{96,96,96},
    stringstyle=\rmfamily\slshape\color[RGB]{128,0,0},                                     %显示空格
    showstringspaces=false
}                                                                                          %代码显示格式设置 感谢@信的灵感

\hypersetup{hidelinks}
\zihao{-4}

%%%%% ===== 中文封面信息
\graduateyear{2016} % 毕业年份
\class{O241.6} % 分类号(数值线性代数是 O241.6)
\ctitle{\uline{论文标题论文标题标题\linebreak 如果一行放不下就放两行}}
\def\cctitle{论文标题论文标题} % 在原创性声明中使用, 不能出现手工换行
\caffil{数学系}
\cmajor{XXXX} % 计算数学
\cdirection{XXXX} % 数值代数
\csupervisor{某某某\, 教授}
\cauthor{XXXX}
\def\ccauthor{\@cauthor} % 在答辩委员会中使用
\studentid{51088888888}
\cdate{20xx 年 xx 月}

%%%%% ===== 英文封面信息
\etitle{\uline{Title of Thesis Title of Thesis\linebreak
       Title Title Title of Thesis Title of\linebreak
       Thesis Title Title Title Title Title}}
\eaffil{Mathematics} 
\emajor{xxxx xxxx} % Computational Mathematics
\edirection{xxxx xxxx} % Numerical Algebra
\esupervisor{XXX Xxxxxxx (Professor)}
\eauthor{ZHANG San}
\edate{xxx, 20xx}

%%%%% ===== 生成封面 =====
\newgeometry{top=2.0cm,bottom=2.0cm,left=2.5cm,right=2.5cm}
{
\renewcommand{\baselinestretch}{1.6}
\makecover
}

%%%%% ===== 原创性声明与著作权使用声明 =====
\include{Declaration}

%%%%% ===== 答辩委员会成员 =====
\include{Committee}

\frontmatter
%\restoregeometry
%%%%% ===== 中文摘要 =====
\include{Abstract_chs}

%%%%% ===== 英文摘要 =====
\include{Abstract_eng}

%%%%% ===== 生成目录
\setcounter{tocdepth}{1}
\phantomsection
\addcontentsline{toc}{chapter}{目录}
\tableofcontents
%\listoffigures % 插图目录
%\listoftables  % 表格目录
\clearpage%{\pagestyle{empty}\cleardoublepage}

%%%%%% ===== 正文部分 ===== %%%%%
\mainmatter
\linespread{1.4}\selectfont
%\setlength{\baselineskip}{0.88175cm}

\chapter{引言}

引言部分, 介绍论文研究课题的应用背景或者问题来源,
一些基本概念, 现有成果等等.

\section{问题的提出}
问题的提出, 问题的提出, 问题的提出, 问题的提出, 问题的提出.
问题的提出, 问题的提出, 问题的提出, 问题的提出, 问题的提出.
问题的提出, 问题的提出, 问题的提出, 问题的提出, 问题的提出.

问题的提出, 问题的提出, 问题的提出, 问题的提出, 问题的提出.
问题的提出, 问题的提出, 问题的提出, 问题的提出, 问题的提出.
问题的提出, 问题的提出, 问题的提出, 问题的提出, 问题的提出.

\section{现有成果}
现有成果介绍, 现有成果介绍, 现有成果介绍, 现有成果介绍, 现有成果介绍.
现有成果介绍, 现有成果介绍, 现有成果介绍, 现有成果介绍, 现有成果介绍.
现有成果介绍, 现有成果介绍, 现有成果介绍, 现有成果介绍, 现有成果介绍.

\begin{theorem}
  这是定理, 这是定理, 这是定理, 这是定理, 这是定理, 这是定理.
  这是定理, 这是定理, 这是定理, 这是定理, 这是定理, 这是定理.
  这是定理, 这是定理, 这是定理, 这是定理, 这是定理, 这是定理.
  这是定理, 这是定理, 这是定理, 这是定理, 这是定理, 这是定理.
\end{theorem}
\begin{proof}
  这是证明, 这是证明, 这是证明, 这是证明, 这是证明, 这是证明.
  这是证明, 这是证明, 这是证明, 这是证明, 这是证明, 这是证明.
  这是证明, 这是证明, 这是证明, 这是证明, 这是证明, 这是证明.
  这是证明, 这是证明, 这是证明, 这是证明, 这是证明, 这是证明.

  这是证明, 这是证明, 这是证明, 这是证明, 这是证明, 这是证明.
  这是证明, 这是证明, 这是证明, 这是证明, 这是证明, 这是证明.
  这是证明, 这是证明, 这是证明, 这是证明, 这是证明, 这是证明.
  这是证明, 这是证明, 这是证明, 这是证明, 这是证明, 这是证明.
\end{proof}

现有成果介绍, 现有成果介绍, 现有成果介绍, 现有成果介绍, 现有成果介绍.
现有成果介绍, 现有成果介绍, 现有成果介绍, 现有成果介绍, 现有成果介绍.
现有成果介绍, 现有成果介绍, 现有成果介绍, 现有成果介绍, 现有成果介绍.
现有成果介绍, 现有成果介绍, 现有成果介绍, 现有成果介绍, 现有成果介绍.
现有成果介绍, 现有成果介绍, 现有成果介绍, 现有成果介绍, 现有成果介绍.
现有成果介绍, 现有成果介绍, 现有成果介绍, 现有成果介绍, 现有成果介绍.

\chapter{准备工作}

论文中需要用到的一些知识或工具等. 论文中需要用到的一些知识或工具等.
论文中需要用到的一些知识或工具等. 论文中需要用到的一些知识或工具等.
论文中需要用到的一些知识或工具等. 论文中需要用到的一些知识或工具等.
论文中需要用到的一些知识或工具等. 论文中需要用到的一些知识或工具等.
论文中需要用到的一些知识或工具等. 论文中需要用到的一些知识或工具等.

\section{一些定义}
这里给出一些定义, 这里给出一些定义, 这里给出一些定义,
这里给出一些定义, 这里给出一些定义, 这里给出一些定义.

\begin{definition}
  这是定义, 这是定义, 这是定义, 这是定义, 这是定义, 这是定义,
  这是定义, 这是定义, 这是定义, 这是定义, 这是定义, 这是定义,
  这是定义, 这是定义, 这是定义, 这是定义, 这是定义, 这是定义,
  这是定义, 这是定义, 这是定义, 这是定义, 这是定义, 这是定义.
\end{definition}

\section{一些引理和推论}

这里给出一些引理和推论, 这里给出一些引理和推论, 这里给出一些引理和推论,
这里给出一些引理和推论, 这里给出一些引理和推论, 这里给出一些引理和推论,
这里给出一些引理和推论, 这里给出一些引理和推论, 这里给出一些引理和推论.

\begin{lemma}
  这是引理, 这是引理, 这是引理, 这是引理, 这是引理, 这是引理,
  这是引理, 这是引理, 这是引理, 这是引理, 这是引理, 这是引理,
  这是引理, 这是引理, 这是引理, 这是引理, 这是引理, 这是引理,
  这是引理, 这是引理, 这是引理, 这是引理, 这是引理, 这是引理.
\end{lemma}

\begin{lemma}
  这是引理, 这是引理, 这是引理, 这是引理, 这是引理, 这是引理,
  这是引理, 这是引理, 这是引理, 这是引理, 这是引理, 这是引理,
  这是引理, 这是引理, 这是引理, 这是引理, 这是引理, 这是引理,
  这是引理, 这是引理, 这是引理, 这是引理, 这是引理, 这是引理.
\end{lemma}

\subsection{推论}
\begin{corollary}
  这是推论, 这是推论, 这是推论, 这是推论, 这是推论, 这是推论,
  这是推论, 这是推论, 这是推论, 这是推论, 这是推论, 这是推论,
  这是推论, 这是推论, 这是推论, 这是推论, 这是推论, 这是推论,
  这是推论, 这是推论, 这是推论, 这是推论, 这是推论, 这是推论.
\end{corollary}

\section{一些命题和性质}
现有成果介绍, 现有成果介绍, 现有成果介绍, 现有成果介绍, 现有成果介绍.
现有成果介绍, 现有成果介绍, 现有成果介绍, 现有成果介绍, 现有成果介绍.
现有成果介绍, 现有成果介绍, 现有成果介绍, 现有成果介绍, 现有成果介绍.

\begin{proposition}
  这是命题, 这是命题, 这是命题, 这是命题, 这是命题, 这是命题,
  这是命题, 这是命题, 这是命题, 这是命题, 这是命题, 这是命题,
  这是命题, 这是命题, 这是命题, 这是命题, 这是命题, 这是命题.
\end{proposition}

\begin{property}
  这是性质, 这是性质, 这是性质, 这是性质, 这是性质, 这是性质,
  这是性质, 这是性质, 这是性质, 这是性质, 这是性质, 这是性质,
  这是性质, 这是性质, 这是性质, 这是性质, 这是性质, 这是性质.
\end{property}

现有成果介绍, 现有成果介绍, 现有成果介绍, 现有成果介绍, 现有成果介绍.
现有成果介绍, 现有成果介绍, 现有成果介绍, 现有成果介绍, 现有成果介绍.
现有成果介绍, 现有成果介绍, 现有成果介绍, 现有成果介绍, 现有成果介绍.


\section{一些注记}
现有成果介绍, 现有成果介绍, 现有成果介绍, 现有成果介绍, 现有成果介绍.
现有成果介绍, 现有成果介绍, 现有成果介绍, 现有成果介绍, 现有成果介绍.
现有成果介绍, 现有成果介绍, 现有成果介绍, 现有成果介绍, 现有成果介绍.
现有成果介绍, 现有成果介绍, 现有成果介绍, 现有成果介绍, 现有成果介绍.
现有成果介绍, 现有成果介绍, 现有成果介绍, 现有成果介绍, 现有成果介绍.

\begin{remark}
  这是注记, 这是注记, 这是注记, 这是注记, 这是注记, 这是注记,
  这是注记, 这是注记, 这是注记, 这是注记, 这是注记, 这是注记,
  这是注记, 这是注记, 这是注记, 这是注记, 这是注记, 这是注记.
\end{remark}

\section{算法}
算法示例, 算法示例, 算法示例, 算法示例, 算法示例, 算法示例,
算法示例, 算法示例, 算法示例, 算法示例, 算法示例, 算法示例,
算法示例, 算法示例, 算法示例, 算法示例, 算法示例, 算法示例.

算法示例, 算法示例, 算法示例, 算法示例, 算法示例, 算法示例,
算法示例, 算法示例, 算法示例, 算法示例, 算法示例, 算法示例,
算法示例, 算法示例, 算法示例, 算法示例, 算法示例, 算法示例.

算法示例, 算法示例, 算法示例, 算法示例, 算法示例, 算法示例,
算法示例, 算法示例, 算法示例, 算法示例, 算法示例, 算法示例,
算法示例, 算法示例, 算法示例, 算法示例, 算法示例, 算法示例.

\begin{algorithm}[H]
\caption{算法示例算法示例\label{Alg:}}
  \begin{algorithmic}[1]
  \State $v_1 = r/\|r\|_2$
  \For{$j=1$ to $m$}
  \State $z = A v_j$
  \For{$i=1$ to $j$}
  \State $h_{i,j} = (v_i,z)$  \Comment{内积}
  \State $z = z - h_{i,j}v_i$
  \EndFor
  \State $h_{j+1,j}=\|z\|_2$
  \If{$h_{j+1,j}=0$}
  \State break
  \EndIf
  \State $v_{j+1}=z/h_{j+1,j}$
  \EndFor
  \end{algorithmic}
\end{algorithm}

\clearpage{\pagestyle{empty}\cleardoublepage}
\chapter{其他章节}
其他章节, 其他章节, 其他章节, 其他章节, 其他章节, 其他章节,
其他章节, 其他章节, 其他章节, 其他章节, 其他章节, 其他章节,
其他章节, 其他章节, 其他章节, 其他章节, 其他章节, 其他章节.

其他章节, 其他章节, 其他章节, 其他章节, 其他章节, 其他章节,
其他章节, 其他章节, 其他章节, 其他章节, 其他章节, 其他章节,
其他章节, 其他章节, 其他章节, 其他章节, 其他章节, 其他章节.


\clearpage{\pagestyle{empty}\cleardoublepage}
\backmatter
\linespread{1.1}\selectfont
%%%% ===== 参考文献 =====
\setlength{\bibsep}{1ex}  % 需 natbib 宏包
\begin{thebibliography}{99}
\addcontentsline{toc}{chapter}{参考文献}
\thispagestyle{plain}

% \addtolength{\itemsep}{-5pt}

\bibitem{CPZ08}
  K. C. Chang, K. Pearson and T. Zhang,
  Perron-Frobenius Theorem for nonnegative tensors,
  Commun. Math. Sci., 6 (2008), 507--520.

\bibitem{NW99}
  J. Nocedal and S. J. Wright,
  Numerical Optimization,
  Springer, New York, 1999.

\end{thebibliography}

\clearpage{\pagestyle{empty}\cleardoublepage}
\linespread{1.4}\selectfont
\chapter*{附录}
\addcontentsline{toc}{chapter}{附录}

附录部分, 附录部分, 附录部分, 附录部分, 附录部分,
附录部分, 附录部分, 附录部分, 附录部分, 附录部分,
附录部分, 附录部分, 附录部分, 附录部分, 附录部分,
附录部分, 附录部分, 附录部分, 附录部分, 附录部分.

附录部分, 附录部分, 附录部分, 附录部分, 附录部分,
附录部分, 附录部分, 附录部分, 附录部分, 附录部分,
附录部分, 附录部分, 附录部分, 附录部分, 附录部分,
附录部分, 附录部分, 附录部分, 附录部分, 附录部分.

附录部分, 附录部分, 附录部分, 附录部分, 附录部分,
附录部分, 附录部分, 附录部分, 附录部分, 附录部分,
附录部分, 附录部分, 附录部分, 附录部分, 附录部分,
附录部分, 附录部分, 附录部分, 附录部分, 附录部分.


\clearpage{\pagestyle{empty}\cleardoublepage}
\chapter*{致谢}
\addcontentsline{toc}{chapter}{致谢}

致谢部分, 致谢部分, 致谢部分, 致谢部分, 致谢部分,
致谢部分, 致谢部分, 致谢部分, 致谢部分, 致谢部分,
致谢部分, 致谢部分, 致谢部分, 致谢部分, 致谢部分,
致谢部分, 致谢部分, 致谢部分, 致谢部分, 致谢部分.

致谢部分, 致谢部分, 致谢部分, 致谢部分, 致谢部分,
致谢部分, 致谢部分, 致谢部分, 致谢部分, 致谢部分,
致谢部分, 致谢部分, 致谢部分, 致谢部分, 致谢部分,
致谢部分, 致谢部分, 致谢部分, 致谢部分, 致谢部分.

致谢部分, 致谢部分, 致谢部分, 致谢部分, 致谢部分,
致谢部分, 致谢部分, 致谢部分, 致谢部分, 致谢部分,
致谢部分, 致谢部分, 致谢部分, 致谢部分, 致谢部分,
致谢部分, 致谢部分, 致谢部分, 致谢部分, 致谢部分.

致谢部分, 致谢部分, 致谢部分, 致谢部分, 致谢部分,
致谢部分, 致谢部分, 致谢部分, 致谢部分, 致谢部分,
致谢部分, 致谢部分, 致谢部分, 致谢部分, 致谢部分,
致谢部分, 致谢部分, 致谢部分, 致谢部分, 致谢部分.

\clearpage{\pagestyle{empty}\cleardoublepage}
\chapter*{研究成果}
\addcontentsline{toc}{chapter}{研究成果}

研究生期间所取得的研究成果.

\end{document}
