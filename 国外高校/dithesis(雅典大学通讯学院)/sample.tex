%%
%% sample.tex - Documentation for dithesis Latex class
%% Copyright (C) 2011 Yannis Mantzouratos <mantzouratos@gmail.com>
%%
%% This file comes with Arial support.
%%
%% LICENSE:
%%
%% This work may be distributed and/or modified under the conditions of the
%% LaTeX Project Public License, either version 1.3 of this license or (at
%% your option) any later version.
%%
%% The latest version of this license is in:
%% http://www.latex-project.org/lppl.txt
%% and version 1.3 or later is part of all distributions of LaTeX version
%% 2005/12/01 or later.
%%
%% This work has the LPPL maintenance status "maintained".
%% The Current Maintainer of this work is Yannis Mantzouratos.
%%
%% This work consists of the source file dithesis.cls and the documentation
%% files sample.tex, samplewArial.tex, sample.pdf and samplewArial.pdf.
%% To ensure proper compilation, however, the logo of the University of Athens
%% is also distributed alongside this work, under the filename athena.jpg.
%%
%% NOTES and WARRANTY:
%%
%% This work conforms to the requirements of the Department of Informatics and
%% Telecommunications at the University of Athens regarding the preparation of
%% undergraduate theses, as of Sep 1, 2011.
%%
%% This work is distributed in the hope that it will be useful, but WITHOUT ANY
%% WARRANTY; without even the implied warranty of MERCHANTABILITY or FITNESS
%% FOR A PARTICULAR PURPOSE.
%% The entire risk as to the quality and performance of this work is with you.
%% Should this work prove defective, you assume the cost of all necessary
%% servicing, repair, or correction.
%% See the LaTeX Project Public License for more details.
%%
%% The latest official Microsoft Word(...) template can be found in
%% http://www.di.uoa.gr/lib.
%%

%% Compiled to PDF with	MikTex (interpreter: XeLaTeX) on a Windows platform
\documentclass{dithesis}

\usepackage{mathspec}
\usepackage{xgreek}
\usepackage{xunicode}

\setallmainfonts[Mapping=tex-text]{Arial}
\setallsansfonts[Mapping=tex-text]{Arial}
\setallmonofonts[Mapping=tex-text]{Arial}

\renewcommand{\university}{Εθνικό και Καποδιστριακό Πανεπιστήμιο Αθηνών}
\renewcommand{\school}{Σχολή Θετικών Επιστημών}
\renewcommand{\department}{Τμήμα Πληροφορικής και Τηλεπικοινωνιών}

\renewcommand{\thesisplace}{Αθήνα}
\renewcommand{\thesisdate}{Σεπτέμβρης 2011}

\renewcommand{\thesislabel}{Πτυχιακή Εργασία}
\renewcommand{\supervisorlabel}{Επιβλέπων}
\renewcommand{\idlabel}{Α.Μ.}

\begin{document}
\thesistitle{Η κλάσση \LaTeX{}  dithesis}
\thesisauthor{Ιωάννης Π. Μαντζουράτος}{1115200600000}
\supervisor{Αλέξης Δελής}{Καθηγητής ΕΚΠΑ}
\maketitle

\begin{thesisabstract}[Περίληψη]
  Στο παρόν έγγραφο τεκμηριώνεται η κλάσση \LaTeX{} dithesis, που μπορεί να
  χρησιμοποιηθεί για τη σύνταξη πτυχιακών εργασιών του Τμήματος Πληροφορικής
  και Τηλεπικοινωνιών του Πανεπιστημίου Αθηνών.
  Έγινε προσπάθεια η κλάσση να συμβαδίζει απολύτως με τις απαιτήσεις του 
  Αναγνωστηρίου.
  H πτυχιακή μου, στην οποία χρησιμοποιήσα την παρούσα κλάσση, έγινε αποδεκτή 
  από το Αναγνωστήριο το καλοκαίρι του 2011. 
  
  \thesiskeywords{Θεματική Περιοχή}{Τεκμηρίωση}
                 {Λέξεις Κλειδιά}{\LaTeX{}}
                                 {Κλάσσεις Εγγράφων}
                                 {Πτυχιακές Εργασίες}
                                 {Τμήμα Πληροφορικής και Τηλεπικοινωνιών}
                                 {Πανεπιστήμιο Αθηνών}
\end{thesisabstract}

\begin{thesisabstract}[Abstract]
  In this paper, we provide documentation for the \LaTeX{} document class
  dithesis, which can be used for preparing undergraduate theses at the 
  Department of Informatics and Telecommunications, University of Athens.
  The class conforms to all requirements imposed by the Library, as of September
  2011.
  My thesis, which was based on the dithesis class, was accepted by the Library
  sometime during the summer semester of 2011.

  \thesiskeywords{Subject Area}{Documentation}
                 {Keywords}{\LaTeX{}}
                           {Document Classes}
                           {Undergraduate Theses}
                           {Dept. of Informatics}
                           {University of Athens}
\end{thesisabstract}

\begin{thesisdedication}
Αφιέρωση σε κάποιους.
\end{thesisdedication}

\begin{thesisacknowledgments}[Ευχαριστίες]
  Ακολουθεί δείγμα ευχαριστιών.

  Θα ήθελα να ευχαριστήσω τον επιβλέποντα κ. Αλέξη Δελή για τη συνεργασία και τη
  βοήθεια κατά την εκπόνηση αυτής της πτυχιακής.

  Θα ήθελα επίσης να ευχαριστήσω το φίλο μου Μένιο για τις πολύτιμες
  παρατηρήσεις του σε προκαταρκτικές εκδόσεις του κειμένου.
\end{thesisacknowledgments}

\tableofcontents
\listoffigures
\listoftables

\begin{thesisprologue}[Πρόλογος]
  Το παρόν έγγραφο δημιουργήθηκε στην Αθήνα, το 2011, στα πλαίσια της 
  τεκμηρίωσης της κλάσσης \LaTeX{} dithesis.
  Η κλάσση αυτή διανέμεται με την ελπίδα ότι θα αποδειχθεί χρήσιμη, παρόλα αυτά 
  \emph{χωρίς καμιά εγγύηση}: χωρίς ούτε και την σιωπηρή εγγύηση 
  εμπορευσιμότητας ή καταλληλότητας για συγκεκριμένη χρήση.
  Για περισσότερες λεπτομέρειες, ανατρέξτε στην άδεια LaTeX Project Public 
  License.
\end{thesisprologue}

\thesissection{Εισαγωγή}
Το παρόν έγγραφο τεκμηριώνει την ανεπίσημη κλάσση \LaTeX{} dithesis, την οποία
μπορεί κανείς να χρησιμοποιήσει για να συντάξει την πτυχιακή του εργασία
σύμφωνα με τις απαιτήσεις του Τμήματος Πληροφορικής και Τηλεπικοινωνιών του
Πανεπιστημίου Αθηνών.
Η κλάσση αυτή προσπαθεί να καλύψει ένα σημαντικό κενό, μιας και το Αναγνωστήριο
Πληροφορικής και Τηλεπικοινωνιών προσφέρει πρότυπα πτυχιακών εργασιών μόνο σε 
Microsoft Word \cite{Anagnostirio}.

Η παρουσίαση των εντολών (commands) και των περιβαλλόντων (environments) που 
προσφέρει η dithesis ακολουθεί μια λογική σειρά, δηλαδή πρώτα παρουσιάζονται οι
 εντολές -- περιβάλλοντα για τις σελίδες τίτλου και αποδοχής, ύστερα για τις 
περιλήψεις, τις αφιερώσεις και τις ευχαριστίες, κατόπιν για τα περιεχόμενα, 
κ.ο.κ.
Όλες οι εντολές και τα περιβάλλοντα χρησιμοποιούνται μετά την αρχή του
εγγράφου (\textbackslash{begin}\{document\}), εκτός όπου αναφέρεται κάτι
διαφορετικό.

H πτυχιακή μου, στην οποία χρησιμοποιήσα την παρούσα κλάσση, έγινε αποδεκτή 
από το Αναγνωστήριο το καλοκαίρι του 2011.

\thesissection{Σελίδα Τίτλου και Σελίδα Αποδοχής}
Σε αυτή την ενότητα παρουσιάζονται οι εντολές που σχετίζονται με τις σελίδες 
τίτλου και αποδοχής.
Πρώτα παρουσιάζονται οι εντολές παροχής πληροφοριών και στη συνέχεια οι εντολές
παροχής πρόσβασης σε πληροφορίες.

Για τη δημιουργία των σελίδων τίτλου και αποδοχής δώστε
\textbackslash{maketitle}.
\thesissubsection{Εντολές Παροχής Πληροφοριών}
%%%%%%%%%%
\thesissubsubsection{Σταθερές Εντολές}
Οι ακόλουθες εντολές δεν απαιτούν ορίσματα.
Για να τις ανανεώσετε, χρησιμοποιείστε την εντολή \textbackslash{renewcommand}
στο preamble της εργασίας σας.
\begin{description}
\item[\textbackslash{university}]
  Το όνομα του Πανεπιστημίου. 
  Για πτυχιακές εργασίες πρέπει να είναι ``Εθνικό και Καποδιστριακό
  Πανεπιστήμιο Αθηνών''. 
\item[\textbackslash{school}]
  Το όνομα της Σχολής.
  Για πτυχιακές εργασίες πρέπει να είναι ``Σχολή Θετικών Επιστημών''.
\item[\textbackslash{department}]
  Το όνομα του Τμήματος.
  Για πτυχιακές εργασίες πρέπει να είναι ``Τμήμα Πληροφορικής και 
  Τηλεπικοινωνιών''.
\item[\textbackslash{thesislabel}]
  Το είδος της εργασίας.
  Για πτυχιακές εργασίες πρέπει να είναι ``Πτυχιακή Εργασία''.
\item[\textbackslash{supervisorlabel}]
  Η ετικέττα των επιβλεπόντων.
  Για έναν επιβλέποντα πρέπει να είναι ``Επιβλέπων'', για περισσότερους
  πρέπει να είναι ``Επιβλέποντες''.
\item[\textbackslash{idlabel}]
  Η ετικέττα του αριθμού μητρώου.
  Για πτυχιακές εργασίες πρέπει να είναι ``Α.Μ.''.
\item[\textbackslash{thesisplace}]
  Ο τόπος επιτυχούς εξέτασης της εργασίας.
  Για πτυχιακές εργασίες πρέπει να είναι ``Αθήνα''.
\item[\textbackslash{thesisdate}]
  Η ημερομηνία επιτυχούς εξέτασης της εργασίας.
  Για πτυχιακές εργασίες πρέπει να είναι ``Μήνας Έτος'', όπου ο μήνας δίνεται
  ολογράφως και το έτος αριθμητικά.
\end{description}

Παραδείγματος χάριν, για να θέσουμε την ημερομηνία εξέτασης σε ``Σεπτέμβριος
2011'', βάζουμε στο preamble της εργασίας 
\textbackslash{renewcommand}\{\textbackslash{thesisdate}\}\{Σεπτέμβριος 2011\}.
%%%%%%%%%%
\thesissubsubsection{Εντολές με Ορίσματα}
Οι ακόλουθες εντολές απαιτούν ορίσματα και πρέπει να χρησιμοποιούνται αφότου
έχει αρχίσει το έγγραφο (μετά το \textbackslash{begin}\{document\}) και προτού
δοθεί η εντολή δημιουργίας των σελίδων τίτλου και αποδοχής.
\begin{description}
\item[\textbackslash{thesistitle}]
  Η εντολή \textbackslash{thesistitle}\{\textit{title}\} παίρνει ένα 
  υποχρεωτικό όρισμα \textit{title}, το οποίο θέτει τον τίτλο της εργασίας.
  Για παράδειγμα, στην παρούσα εργασία χρησιμοποιήσαμε
  \textbackslash{thesistitle}\{Η κλάσση \textbackslash{LaTeX}\{\} dithesis\}.
\item[\textbackslash{thesisauthor}]
  Η εντολή \textbackslash{thesisauthor}\{\textit{name}\}\{\textit{id}\} παίρνει
  δύο υποχρεωτικά ορίσματα, που καθορίζουν τα στοιχεία του συγγραφέα της
  εργασίας.
  Το \textit{name} καθορίζει το όνομα, ενώ το \textit{id} τον αριθμό μητρώου.
  Για πολλαπλούς συγγραφείς, απλά χρησιμοποιείστε την εντολή επανειλημμένα.
  Για παράδειγμα, στην παρούσα εργασία χρησιμοποιήσαμε
  \textbackslash{thesisauthor}\{Ιωάννης Π. Μαντζουράτος\}\{1115200600000\}.
  Εαν θέλαμε να συμπεριλάβουμε και δεύτερο συγγραφέα, θα προσθέταμε την εντολή 
  \textbackslash{thesisauthor}\{Όν. Π. Επώνυμο\}\{1115200700000\}, κ.ο.κ.
  Οι συγγραφείς θα εμφανιστούν στις σελίδες τίτλου και αποδοχής με τη σειρά
  που συναντώνται οι εντολές \textbackslash{thesisauthor}.
\item[\textbackslash{supervisor}]
  Η εντολή \textbackslash{supervisor}\{\textit{name}\}\{\textit{profession}\} 
  παίρνει δύο υποχρεωτικά ορίσματα, που καθορίζουν τα στοιχεία του επιβλέποντα
  της εργασίας.
  Το \textit{name} καθορίζει το όνομα, ενώ το \textit{profession} τη θέση του,
  όπως Καθηγητής, Επίκουρος Καθηγητής κ.ο.κ.
  Για πολλαπλούς επιβλέποντες, απλά χρησιμοποιείστε την εντολή επανειλημμένα.
  Παραδείγματος χάριν, στην παρούσα εργασία δώσαμε
  \textbackslash{supervisor}\{Αλέξης Δελής\}\{Καθηγητής ΕΚΠΑ\}.
  Εαν θέλαμε να συμπεριλάβουμε και δεύτερο επιβλέποντα, θα προσθέταμε την 
  εντολή \textbackslash{supervisor}\{Όνομα Επώνυμο\}\{Επίκουρος Καθηγητής\}, 
  κ.ο.κ.
  Οι επιβλέποντες θα εμφανιστούν στις σελίδες τίτλου και αποδοχής με τη σειρά
  που συναντώνται οι εντολές \textbackslash{supervisor}.
\end{description}
%%%%%%%%%%
\thesissubsection{Εντολές Πρόσβασης σε Πληροφορίες}
Οι ακόλουθες σταθερές εντολές παρέχονται ώστε να δώσουν πρόσβαση σε μερικές 
πληροφορίες από τις σελίδες τίτλου και αποδοχής που ίσως χρειαστούν στο χρήστη 
αργότερα.
\begin{description}
\item[\textbackslash{thethesistitle}]
  Παρέχει τον τίτλο της εργασίας.
\item[\textbackslash{thethesisauthor}]
  Παρέχει τα ονόματα των συγγραφέων της εργασίας, χωρισμένα με κόμματα.
\end{description}

Για παράδειγμα, για να αναφερθείτε στον τίτλο της εργασίας μέσα στο κείμενο
απλά γράφετε \textbackslash{thethesistitle}.

\thesissection{Περιλήψεις και Λέξεις Κλειδιά}
Σε αυτή την ενότητα παρουσιάζονται τα περιβάλλοντα και οι εντολές που 
σχετίζονται με τις περιλήψεις και τις λέξεις κλειδιά.
%%%%%%%%%%
\thesissubsection{Περιλήψεις}
Για τη συγγραφή περιλήψεων παρέχεται ένα νέο περιβάλλον:
\begin{description}
\item[thesisabstract]
  Το περιβάλλον thesisabstract παίρνει ένα προαιρετικό όρισμα, το οποίο 
  καθορίζει την επικεφαλίδα της περίληψης.
  Εαν δε δοθεί όρισμα, χρησιμοποιείται η επικεφαλίδα που καθορίζεται από
  την εντολή \textbackslash{abstractname}.
  Μπορείτε να χρησιμοποιήσετε το περιβάλλον thesisabstract όσες φορές θέλετε.
  Π.χ., στην παρούσα εργασία το χρησιμοποιήσαμε δύο φορές:
  μία ως \textbackslash{begin}\{thesisabstract\}{[}Περίληψη{]} και μία
  ως \textbackslash{begin}\{thesisabstract\}{[Abstract]}.
\end{description}
%%%%%%%%%%
\thesissubsection{Λέξεις Κλειδιά}
Για τον ορισμό της θεματικής περιοχής και των λέξεων κλειδιών παρέχεται μια
νέα εντολή η οποία πρέπει να χρησιμοποιηθεί μέσα στο περιβάλλον thesisabstract
που παρουσιάστηκε πιο πάνω:
\begin{description}
\item[\textbackslash{thesiskeywords}]
  Η εντολή \textbackslash{thesiskeywords}\{\textit{SAlabel}\}\{\textit{SA}\}%
  \{\textit{KWlabel}\}\{\textit{K}\}\{\textit{K}\}\{\textit{K}\}%
  \{\textit{K}\}\{\textit{K}\} παίρνει 8 υποχρεωτικά ορίσματα.
  Το \textit{SAlabel} είναι η ετικέττα της θεματικής περιοχής,
  το \textit{SA} είναι η θεματική περιοχή, το \textit{KWlabel} είναι
  η ετικέττα των λέξεων κλειδιών, και καθένα από τα \textit{KW} είναι μια λέξη
  κλειδί.
  Παραδείγματος χάριν, στην παρούσα εργασία χρησιμοποιήσαμε την εντολή
  \textbackslash{thesiskeywords}\{Θεματική Περιοχή\}\{Τεκμηρίωση\}
  \{Λέξεις Κλειδιά\} \{\textbackslash{LaTeX}\{\}\} \{Κλάσσεις Εγγράφων\}
  \{Πτυχιακές Εργασίες\} \{Τμήμα Πληροφορικής και Τηλεπικοινωνιών\}
  \{Πανεπιστήμιο Αθηνών\}, και όμοια για τα αγγλικά.
\end{description}

\thesissection{Αφιερώσεις}
Για τις αφιερώσεις παρέχεται ένα νέο περιβάλλον:
\begin{description}
\item[thesisdedication]
  Το περιβάλλον thesisdedication είναι πολύ απλό και δε χρειάζεται ορίσματα.
  Παραδείγματος χάριν, στην παρούσα εργασία χρησιμοποιήσαμε τον εξής κώδικα:
  \textbackslash{begin}\{thesisdedication\} Αφιέρωση σε κάποιους.
  \textbackslash{end}\{thesisdedication\}.
\end{description}

\thesissection{Ευχαριστίες}
Για τις ευχαριστίες παρέχεται ένα νέο περιβάλλον:
\begin{description}
\item[thesisacknowledgments]
  Το περιβάλλον thesisacknowledgments παίρνει ένα προαιρετικό όρισμα, το οποίο 
  καθορίζει την επικεφαλίδα των ευχαριστιών.
  Εαν δε δοθεί όρισμα, χρησιμοποιείται η επικεφαλίδα ``Acknowledgments''.
  Παραδείγματος χάριν, στην παρούσα εργασία δώσαμε
  \textbackslash{begin}\{thesisacknowledgments\}{[Ευχαριστίες]}.
\end{description}

\thesissection{Περιεχόμενα και Κατάλογοι Σχημάτων και Πινάκων}
Για τη δημιουργία περιεχομένων, καθώς και καταλόγων σχημάτων -- πινάκων,
παρέχονται τρεις εντολές, οι οποίες πρέπει να χρησιμοποιηθούν στο κατάλληλο
σημείο (δηλαδή, μετά από τις ευχαριστίες):
\begin{description}
\item[\textbackslash{tableofcontents}]
  Παρέχει τον πίνακα περιεχομένων.
\item[\textbackslash{listoffigures}]
  Παρέχει τον κατάλογο σχημάτων.
\item[\textbackslash{listoftables}]
  Παρέχει τον κατάλογο πινάκων.
\end{description}

Αν για κάποιο λόγο πρέπει να αλλάξετε κάποια από τις επικεφαλίδες, μπορείτε να
κάνετε \textbackslash{renewcommand} στο preamble τα 
\textbackslash{contentsname}, \textbackslash{listfigurename} και 
\textbackslash{listtablename}, αντίστοιχα.
Για παράδειγμα, μπορείτε να αλλάξετε την επικεφαλίδα ``Κατάλογος Σχημάτων'' σε
``Κατάλογος Εικόνων'' με την εντολή 
\textbackslash{renewcommand}\{\textbackslash{listfigurename}\}\{Κατάλογος
Εικόνων\}.
Όμοια, εαν θέλετε να αλλάξετε τις ετικέττες των σχημάτων ή των πινάκων, μπορείτε
να κάνετε \textbackslash{renewcommand} τις εντολές \textbackslash{figurename}
και \textbackslash{tablename}.

\thesissection{Πρόλογος}
Για τον πρόλογο παρέχεται ένα νέο περιβάλλον:
\begin{description}
\item[thesisprologue]
  Το περιβάλλον thesisprologue παίρνει ένα προαιρετικό όρισμα, το οποίο 
  καθορίζει την επικεφαλίδα του προλόγου.
  Εαν δε δοθεί όρισμα, χρησιμοποιείται η επικεφαλίδα ``Prologue''.
  Παραδείγματος χάριν, στην παρούσα εργασία δώσαμε την εντολή
  \textbackslash{begin}\{thesisprologue\}{[Πρόλογος]}.
\end{description}

\thesissection{Ενότητες, Υποενότητες και Υπο-υπoενότητες}
Για την οργάνωση του κυρίως κειμένου σε ενότητες, υποενότητες και 
υπο-υποενότητες παρέχονται τρεις εντολές:
\begin{description}
\item[\textbackslash{thesissection}]
  Η εντολή \textbackslash{thesissection}\{\textit{title}\} παίρνει ένα 
  υποχρεωτικό όρισμα που καθορίζει τον τίτλο της ενότητας.
  Για παράδειγμα, μια νέα ενότητα μπορεί να δημιουργηθεί με την εντολή
  \textbackslash{thesissection}\{Ενότητα Tάδε\}.
\item[\textbackslash{thesissubsection}]
  Η εντολή \textbackslash{thesissubsection}\{\textit{title}\} παίρνει ένα 
  υποχρεωτικό όρισμα που καθορίζει τον τίτλο της υποενότητας.
  Για παράδειγμα, μια νέα υποενότητα μπορεί να δημιουργηθεί με την εντολή
  \textbackslash{thesissubsection}\{Υποενότητα Tάδε\}.
\item[\textbackslash{thesissubsubsection}]
  Η εντολή \textbackslash{thesissubsubsection}\{\textit{title}\} παίρνει ένα 
  υποχρεωτικό όρισμα που καθορίζει τον τίτλο της υπο-υποενότητας.
  Π.χ., μια νέα υπο-υποενότητα μπορεί να δημιουργηθεί με την εντολή
  \textbackslash{thesissubsubsection}\{Υπο-υποενότητα Tάδε\}.
\end{description}

Και για τις τρεις εντολές υπάρχουν οι αντίστοιχες εκδόσεις με αστερίσκο,
δηλαδή της φόρμας \textbackslash{thesissection*}\{title\} κ.ο.κ., οι οποίες
δημιουργούν μια νέα ενότητα (ή υποενότητα ή υπο-υποενότητα) χωρίς παρεμβολή
στην αρίθμηση.


\thesissection{Πίνακας Ορολογίας}
Για τη δημιουργία της ενότητας της ορολογίας παρέχεται ένα νέο περιβάλλον:
\begin{description}
\item[thesisterminology]
  Το περιβάλλον thesisterminology παίρνει ένα προαιρετικό όρισμα, το οποίο 
  καθορίζει την επικεφαλίδα της ενότητας της ορολογίας.
  Εαν δε δοθεί όρισμα, χρησιμοποιείται η επικεφαλίδα ``Table of Terminology''.
  Π.χ., στην παρούσα εργασία δώσαμε την εντολή
  \textbackslash{begin}\{thesisterminology\}{[Πίνακας Ορολογίας]}.
\end{description}

\thesissection{Ακρωνύμια}
Για τη δημιουργία της ενότητας των ακρωνυμίων παρέχεται ένα νέο περιβάλλον:
\begin{description}
\item[thesisabbreviations]
  Το περιβάλλον thesisabbreviations παίρνει ένα προαιρετικό όρισμα, το οποίο 
  καθορίζει την επικεφαλίδα της ενότητας των ακρωνυμίων.
  Εαν δε δοθεί όρισμα, χρησιμοποιείται η επικεφαλίδα ``Abbreviations, Initials
  and Acronyms''.
  Π.χ., στην παρούσα εργασία δώσαμε την εντολή
  \textbackslash{begin}\{thesisabbreviations\}{[Συντμήσεις, Αρκτικόλεξα και
  Ακρωνύμια]}.
\end{description}

\thesissection{Βιβλιογραφία}
Για τη δημιουργία της βιβλιογραφίας παρέχεται ένα νέο περιβάλλον που
βασίζεται στο περιβάλλον thebibliography:
\begin{description}
\item[thesisbibliography]
  Το περιβάλλον thesisbibliography παίρνει δύο ορίσματα, το ένα εκ των οποίων
  προαιρετικό και το άλλο υποχρεωτικό.
  Το προαιρετικό όρισμα καθορίζει την επικεφαλίδα της βιβλιογραφίας.
  Εαν δε δοθεί όρισμα, χρησιμοποιείται η επικεφαλίδα ``References''.
  Το δεύτερο όρισμα καθορίζει το μέγιστο επιτρεπόμενο αριθμό αναφορών, ώστε
  να γίνει ομοιόμορφη στοίχιση.
  Για παράδειγμα, στην παρούσα εργασία δώσαμε
  \textbackslash{begin}\{thesisbibliography\}{[Αναφορές]}\{99\}.
\end{description}

Για τη δημιουργία των αναφορών μέσα στο περιβάλλον της βιβλιογραφίας 
χρησιμοποιείστε την εντολή \textbackslash{bibitem}\{\textit{reference id}\} και
στη συνέχεια δώστε το κυρίως κείμενο της αναφοράς.
Για να αναφερθείτε σε κάποια συγκεκριμένη βιβλιογραφική αναφορά στο κείμενο
χρησιμοποιείστε την εντολή \textbackslash{cite}\{\textit{reference id}\}.
Για περισσότερα δείτε και την τεκμηρίωση του περιβάλλοντος thebibliography
\cite{Bibliography}.

\thesissection{Γραμματοσειρά}
Το Αναγνωστήριο απαιτεί να χρησιμοποιηθεί η γραμματοσειρά Arial στην πτυχιακή,
η οποία όμως δεν είναι από default εγκατεστημένη σε Linux διανομές.
Προσωπικά, νομίζω πως ο πιο εύκολος τρόπος για να φορμάρετε την πτυχιακή σε
Arial είναι να κάνετε το τελικό compilation σε Windows με το MikTex 
\cite{Miktex}.
Χρησιμοποιείστε τον interpreter XeLaTeX, και προσθέστε τις παρακάτω
εντολές στο preamble:
\begin{enumerate}
\item
  \textbackslash{usepackage}\{mathspec\}
\item
  \textbackslash{setallmainfonts}{[Mapping=tex-text]}\{Arial\}
\item
  \textbackslash{setallsansfonts}{[Mapping=tex-text]}\{Arial\}
\item
  \textbackslash{setallmonofonts}{[Mapping=tex-text]}\{Arial\}
\end{enumerate}

Σημειώνεται ότι το MikTex θα εντοπίσει και θα εγκαταστήσει αυτόματα όσα πακέτα 
λείπουν.

\begin{thesisterminology}[Πίνακας Ορολογίας]
Ακολουθεί δείγμα πίνακα ορολογίας.

\begin{tabularx}{\textwidth}{|X|X|}
  \hline
  κλάσση     & class \\
  \hline
  εντολή     & command \\
  \hline
  περιβάλλον & environment \\
  \hline
\end{tabularx}

\end{thesisterminology}

\begin{thesisabbreviations}[Συντμήσεις, Αρκτικόλεξα και Ακρωνύμια]
Ακολουθεί δείγμα συντμήσεων, αρκτικολέξων και ακρωνυμίων.

\begin{tabularx}{\textwidth}{|X|X|}
  \hline
  NP & Non-deterministic polynomial time \\
  \hline
\end{tabularx}

\end{thesisabbreviations}

\begin{thesisbibliography}[Αναφορές]{99}
\bibitem{Anagnostirio}
  ``Ypodeigmata.Ergasion.zip,''
  \textit{Βιβλιοθήκη Θετικών Επιστημών -- Αναγνωστήριο Πληροφορικής,}.
  [Online].
  Available: \url{http://www.di.uoa.gr/lib}.
  [Accessed: Aug 18, 2011].
\bibitem{Bibliography}
  ``Bibliography Management,''
  \textit{WikiBooks on Latex},
  [Online].
  Available: \url{http://en.wikibooks.org/wiki/LaTeX/Bibliography_Management}.
  [Accessed: Aug 18, 2011].
\bibitem{Miktex}
  ``MikTex,''
  [Computer Program]. Availabe: \url{http://www.miktex.org/}.
  [Accessed: Aug 18, 2011].
\end{thesisbibliography}
\end{document}
