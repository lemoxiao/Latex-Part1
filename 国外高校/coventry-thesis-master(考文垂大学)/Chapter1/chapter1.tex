%!TEX root = ../thesis.tex
%*******************************************************************************
%*********************************** First Chapter *****************************
%*******************************************************************************

\chapter{Introduction to CULTT}  %Title of the First Chapter

\ifpdf
    \graphicspath{{Chapter1/Figs/Raster/}{Chapter1/Figs/PDF/}{Chapter1/Figs/}}
\else
    \graphicspath{{Chapter1/Figs/Vector/}{Chapter1/Figs/}}
\fi

%********************************** First Section ******************************
\section{What is CULTT?} %Section - 1.1 

The Coventry University \LaTeX{} Thesis Template (CULTT) is designed as a starting point for writing and formatting a Coventry University (CU) Thesis prior to submission for examination and printing. This template is not official and it is up to the thesis author to ensure that the format, requirements and contents of the thesis are correct. However, this template should prove useful for those using \LaTeX{} to prepare their CU Thesis.

CULTT was created as a CU Thesis template for use on ShareLaTeX, the online service for producing documents with \LaTeX{}. ShareLaTeX also supports collaborative working, see \url{https://www.sharelatex.com/} for futher information. Another online \LaTeX{} editor and collaboration service is Overleaf, \url{https://www.overleaf.com/}.

Rather than reinvent the wheel CULTT was derived from an existing thesis template. There are plenty of examples available. For CULTT the basis was the Cambridge University Engineering Department's (CUED) thesis template.

\nomenclature[z-cif]{CU}{Coventry University}
\nomenclature[z-cif]{CULTT}{Coventry University \LaTeX{} Thesis Template}
\nomenclature[z-cif]{CUED}{University of Cambridge Department of Engineering thesis template}

If you are not using \LaTeX{} you will not need CULTT. However, the use of \LaTeX{} for producing all academic documents (not only a thesis) is recommended. Although \LaTeX{} may appear strange to those who normally use word processing software, the benefits are immense once you get use to it. ShareLaTeX has some great documentation to help get started, see \url{https://www.sharelatex.com/learn/}. The \LaTeX{} Project website is at \url{https://www.latex-project.org/}. If not using an online \LaTeX{} service then install a \LaTeX{} environment onto your computer. MiKTeX is a good package which can be installed on Windows, Mac, and Linux, see \url{https://miktex.org/} (also available from the CU Cloudpaging Player Apps service). MiKTeX includes the TeXworks editor, \url{https://www.tug.org/texworks/}. A \emph{\LaTeX{} for Beginners} document is available at \url{http://www.docs.is.ed.ac.uk/skills/documents/3722/3722-2014.pdf}.

%********************************** Second Section  ****************************
\section{Using CULTT} %Section - 1.2

CULTT can be used on a local computer or uploaded to a online \LaTeX{} service. The thesis template is divided into sub-folders for ease of organisation. The template provides lots of examples of textual constructs that may be required in the thesis. All of the the textual content (and constructs) in the various \LaTeX{} files (e.g. this Introduction chapter) will be replaced with the authors own content. The content stored in the \emph{tex} files is brought together in the top level (root) \emph{thesis.tex}, which includes the other tex files in the sub-folders as required. This allows the thesis to be divided into smaller chunks for better management and safety. Working on smaller sections of the thesis reduces the chance of accidentally destroying large parts of text. (Always take regular backups of all work produced.)

When your final CU Ethics PDF is available (\url{https://ethics.coventry.ac.uk/}) replace the PDF in the \emph{Ethics} folder.

\subsection{Coventry University Thesis Writing Resources}

There are a several requirements to meet for a correct CU Thesis submission. They are covered by the CU Academic Regulations\index{Academic Regulations}, visit \url{http://www.coventry.ac.uk/life-on-campus/the-university/key-information/registry/academic-regulations/} and download section 8 to read \textbf{8.12 The higher degree thesis}. 

The CU Student Portal, accessed via the \emph{Portals} link on the CU website, \url{https://share.coventry.ac.uk/students}, has links to relevant documents in the \emph{Doctoral College} section (under \emph{Study at CU}). See the \emph{Preparing for your PRP and Viva} section to find the \textbf{Thesis Information PDF} (called \emph{Thesis Requirements.pdf}).

Further CU help on thesis writing can be obtained from Coventry University's Centre for Academic Writing (CAW), \url{https://cawbookings.coventry.ac.uk/}. They also look after the CU Harvard Referencing Style (short linked at \url{https://goo.gl/9jxdPX}).

