\chapter[Situering]{Situering}
\label{chap_situering}

Dit is het sjabloon voor bachelorproef scripties. 
Lees deze tekst grondig na voor je aan het schrijven van uw scriptie begint. 
Deze tekst bevat o.a. een algemene uitleg van de structuur en de inhoud die we verwachten van een scriptie en nuttige tips i.v.m. het gebruik van \LaTeX.

De gebruiker van dit sjabloon kan best de .tex bronbestanden en het .pdf bestand samen doornemen om goed te begrijpen hoe je met \LaTeX moet werken.

Alle velden die tussen [..] zijn vermeld dienen door de student te worden vervangen door zijn of haar persoonlijke gegevens.

In dit document zijn enkele vaak voorkomende voorbeelden opgenomen. 

Het is zeker niet onze bedoeling om het wiel opnieuw uit te vinden, daarom verwijzen we u voor meer diepgaande informatie i.v.m. \LaTeX naar volgende websites:

\begin{itemize}
\item \url{http://nl.wikibooks.org/wiki/LaTeX}
\item \url{http://latex.ugent.be/}
\end{itemize}


\section{Structuur scriptie}

De structuur van de scriptie wordt opgelegd en wordt niet aangepast door de student.

De scriptie bestaat uit meerdere hoofdstukken. 
In het begin van ieder hoofdstuk staat in dit sjabloon een toelichting met een uitleg wat we juist verwachten van dat hoofdstuk.

Op de tweede en op de laatste pagina werden twee witte bladzijden ingevoegd om eventueel doorschijnen bij afdrukken op dun papier te voorkomen.


\section{Inhoud hoofdstuk ``Situering''}

Omschrijf de context van uw project: wat was de aanleiding, wie werkt er nog mee aan het project, waarom is het project nuttig.


%\TODO[nog aanvullen]
