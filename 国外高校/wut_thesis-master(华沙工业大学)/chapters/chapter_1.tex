\chapter{Wstęp}

Witaj, to szablon pracy dyplomowej \textbf{częściowo} przystosowany do nowych wymagań edycyjnych PW\footnote{Niektóre elementy z Zarządzenia Rektora nr 43/2016 są bardzo trudne do zrealizowania w \LaTeX-u.}. Zapoznaj się z kodem źródłowym przed przystąpieniem do pracy. Na początek wymienię kilka ważnych uwag:

\begin{itemize}
	\item[--] ustawienia dokumentu znajdują się w pliku \texttt{config.tex},
	\item[--] niestety ze względu na użycie płatnych fontów w stronie tytułowej aktualna wersja używa plików .png\footnote{Jeśli uda się dostać nagłówek i rodzaj pracy w krzywych, to zostaną one podmienione.},
	\item[--] bibliografia używa stylu zbliżonego do stylu harwardzkiego\parencite{ogata2010modern}, a cytowania są w większości zgodne z zaleceniami BG PW \parencite{linh2002line}; można to zmienić w pliku konfiguracyjnym\parencite[Some kind of post note]{DUMMY:1} na cytowania numeryczne,
	\item[--] do kompilacji polecam TeXstudio 2.10.8 + MiKTeX w sekwencji pdflatex + biber + pdflatex + pdflatex\footnote{By dodać w TeXstudio przejdź do Opcje > Build > User commands i dodaj: \texttt{txs:///pdflatex | txs:///biber | txs:///pdflatex | txs:///pdflatex}} (czasem warto powtórzyć 2-3 razy, by zaktualizowała się bibliografia); do podglądu zaś \href{https://www.sumatrapdfreader.org/}{Sumatra PDF},
	\item do zarządzania bibliografią polecam oprogramowanie \href{https://www.zotero.org/}{Zotero} i opcja Better BibTeX.
\end{itemize}

\begin{flushright}
	Pozdrawiam, Dominik Roszkowski.
\end{flushright}

\begin{figure}[h] % h means here
\centering
\includegraphics[width=0.6\linewidth]{img/PW_logo}
\caption[Krótki podpis grafiki]{Długi podpis grafiki opisujący na przykład, co znajduje się na niej po lewej a co po prawej stronie rozciągający się na więcej niż jedną linijkę}
\label{fig:PW_logo}
\end{figure}

\section{Motywacja do pracy}
\lipsum[1]

\begin{figure}[h]
	\centering
	\begin{subfigure}{.5\textwidth}
		\centering
		\includegraphics[width=.6\textwidth]{img/PW_logo}
		\caption{A subfigure 1}
		\label{fig:sub1}
	\end{subfigure}%
	\begin{subfigure}{.5\textwidth}
		\centering
		\includegraphics[width=.6\textwidth]{img/PW_logo}
		\caption{A subfigure 2}
		\label{fig:sub2}
	\end{subfigure}
	\caption{A figure with two subfigures that you can reference easily e.g. in Figure \ref{fig:sub1} and Figure \ref{fig:sub2}}
	\label{fig:test}
\end{figure}

\lipsum[2]

\begin{table}
\centering
\caption{My caption of this table}
\label{tab:table1}
\begin{tabular}{cSSSSSSS}\toprule % S is for the SI units
    L.p. & {$\Re\{\underline{\mathfrak{X}}(m)\}$} & {$-\Im\{\underline{\mathfrak{X}}(m)\}$} & {$\mathfrak{X}(m)$} & {$\frac{\mathfrak{X}(m)}{23}$} & {$A_m$} & {$\varphi(m)\ /\ ^{\circ}$} & {$\varphi_m\ /\ ^{\circ}$} \\ \midrule
    1  & 16.128 & +8.872 & 16.128 & 1.402 & 1.373 & -146.6 & -137.6 \\ 
    2  & 1.29   & +0.099 & 1.29   & 0.112 & 0.097 & -175.6 & -114.7 \\ 
    3  & 16.128 & +8.872 & 16.128 & 1.402 & 1.373 & -146.6 & -137.6 \\ 
    4  & 1.29   & +0.099 & 1.29   & 0.112 & 0.097 & -175.6 & -114.7 \\
    5  & 16.128 & +8.872 & 16.128 & 1.402 & 1.373 & -146.6 & -137.6 \\ 
    6  & 1.29   & +0.099 & 1.29   & 0.112 & 0.097 & -175.6 & -114.7 \\  
    7  & 0.641  & -0.466 & 0.641  & 0.056 & 0.045 & 133.3  & -106.3 \\ \bottomrule
\end{tabular}
\end{table}
 % polecam inputować tabele jako oddzielne pliki

\lipsum[5]

\subsection{Przykłady innych prac}
\lipsum[1-3]

\section{Metodyka pracy}
\lipsum[2-4]

\section{Plan pracy}
\lipsum[1]