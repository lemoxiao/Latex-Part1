% ----------------------------------------------------------------------------
% User Manual
% ----------------------------------------------------------------------------
\documentclass[thesis.tex]{subfiles}
\begin{document}

\chapter[\vspace{-2\baselineskip}]{QUThesis User Manual}
\begin{quote}[Lewis Carroll, Alice in Wonderland][flushright]
``Begin at the beginning and go on till you come to the end, then stop.''
\end{quote}

\lettrine{T}{he quthesis style package} is designed for Doctorate of Philosophy
dissertations at the Queensland University of Technology (QUT), but may also be
useful at many other institutions. The goals of the package are threefold:
(i)~to comply with QUT's thesis submission guidelines, (ii)~to provide
comprehensive base-level functionality, structure and version control so you
can focus on content, and (iii)~to produce beautiful thesis documents that are
a pleasure to read. To this end, quthesis provides a number of builtin commands
and options, which are described in this manual.
\subsection*{Package Options}
The quthesis package is designed to be used with the \code{book} documentclass.
The package is initialized in the typical manner,

\begin{lstlisting}[language=tex]
    \documentclass[11pt,a4paper]{book}
    \usepackage[options]{quthesis}
\end{lstlisting}

\noindent The package accepts the following options:

\noindent\begin{longtable}{l p{8.5cm}}
\code|print|    & Print binding offsets of 0.5in are added to the inner edge, and
                  hyperlink coloring is disabled \\
\code|onehalfspacing| & Increase line spacing in the content, while preserving
                  header spacing \\
\code|strict|   & Comply with the QUT style guidelines (default)\\
\code|relaxed|  & Remove material that isn't of interest to the reader;
                  breaks compliance with the official QUT guidelines \\
\code|dropcaps| & Initializes the lettrine package so that paragraph beginning
                  words can be formatted using dropped capitals (dropcaps), as
                  per the first character of this manual \\
\code|calendas| & Use the Calendas font for headings, or a similar fallback if
                  Calendas is not available \\
\code|nonumber| & Remove numbering from sections and subsections (more suitable
                  for historical/literature theses)
\end{longtable}


\subsection*{Features}
\begin{tightemize}
\item Automatically generated cover page, and templates for front matter
      (dedication, acknowledgements, abstract)
\item Margins and font size optimized for readability
\item Simple and elegant headers and footers
\item Explicit breaking of chapter titles without affecting headers
\item Customized \code{listings} environment to better match the prose
      formatting
\item Better default settings for \code{hyperref}, including link colors, PDF
      metadata and references
\item Adjusted footnote size and spacing from main prose
\end{tightemize}


\subsection*{Provided Environments}
\begin{itemize}
\item \code|\begin{quote}[author][alignment] ... \end{quote}| \\ Format a
quote, like the one at the beginning of the manual. The command has two
optional arguments. The \code{author} formats a right-justified attribution
after the quote. The \code{alignment} argument specifies the horizontal
alignment of the quote on the page. It may be one of
\code{flushleft|center|flushright}.
\item \code|\begin{verticenter} ... \end{verticenter}| \\ Vertically center a
block of content with optical adjustment (the content is actually positioned
slightly above the true center).
\item \code|\begin{tightemize} ... \end{tightemize}| \\ A renewed itemize
environment with tighter spacing between items.
\item \code|\begin{biography}[path/to/portrait.png] ... \end{biography}| \\ An
environment for providing a short author biography at the end of the thesis.
Text is wrapped around the optional portrait figure. The environment opens left
instead of the usual right so that it appears on the back page as per novels.
\item \code|\begin{verse} ... \end{verse}| \\ A renewed verse environment that
provides two extra features over the regular verse environment: (1) lines can
be input verbatim, with blank lines indicating stanzas, and (2) page breaks are
disallowed mid-stanza.

\end{itemize}


\subsection*{Provided Commands}
\begin{itemize}
\item \code|\chapter[Alpha Name]{Chapter Name}| \\
A custom chapter command that, along with behaving like a normal chapter, also
allows unnumbered chapters and chapters with alpha names. This is useful for
introductory chapters such as \\\code|\chapter[Introduction]{The State of the Union}|.
\item \code|\chap, \sec, \subsec, \fig, \eqn| \\
These commands wrap \code|\ref| to provide two utilities: (i) enforce
namespacing in the use of labels, and (ii) enforce consistency of referring to
labels. A figure with the label \code|\label{fig:results}| for example, can be
referred to by \code|\fig{results}| which will evaluate to \code{Figure 2}. In
the five variants, the namespace in the label is always the command name
followed by a colon.
\item \code|\code{inline code}| \\
An alias of the \code{lstinline} command.
\item \code|\ie, \eg, \cf, \etc, \wrt, \dof, \etal| \\
Italicized abbreviations which respect the spacing and deduplication (for
sentence ending abbreviations) of the final period: \ie, \eg, \cf, \etc, \wrt,
\dof, \etal.
\end{itemize}

\end{document}
