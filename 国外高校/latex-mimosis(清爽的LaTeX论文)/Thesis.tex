\documentclass{mimosis}

\usepackage{metalogo}

%%%%%%%%%%%%%%%%%%%%%%%%%%%%%%%%%%%%%%%%%%%%%%%%%%%%%%%%%%%%%%%%%%%%%%%%
% Some of my favourite personal adjustments
%%%%%%%%%%%%%%%%%%%%%%%%%%%%%%%%%%%%%%%%%%%%%%%%%%%%%%%%%%%%%%%%%%%%%%%%
%
% These are the adjustments that I consider necessary for typesetting
% a nice thesis. However, they are *not* included in the template, as
% I do not want to force you to use them.

% This ensures that I am able to typeset bold font in table while still aligning the numbers
% correctly.
\usepackage{etoolbox}

\usepackage[binary-units=true]{siunitx}
\DeclareSIUnit\px{px}

\sisetup{%
  detect-all           = true,
  detect-family        = true,
  detect-mode          = true,
  detect-shape         = true,
  detect-weight        = true,
  detect-inline-weight = math,
}

%%%%%%%%%%%%%%%%%%%%%%%%%%%%%%%%%%%%%%%%%%%%%%%%%%%%%%%%%%%%%%%%%%%%%%%%
% Hyperlinks & bookmarks
%%%%%%%%%%%%%%%%%%%%%%%%%%%%%%%%%%%%%%%%%%%%%%%%%%%%%%%%%%%%%%%%%%%%%%%%

\usepackage[%
  colorlinks = true,
  citecolor  = RoyalBlue,
  linkcolor  = RoyalBlue,
  urlcolor   = RoyalBlue,
  ]{hyperref}

\usepackage{bookmark}

%%%%%%%%%%%%%%%%%%%%%%%%%%%%%%%%%%%%%%%%%%%%%%%%%%%%%%%%%%%%%%%%%%%%%%%%
% Bibliography
%%%%%%%%%%%%%%%%%%%%%%%%%%%%%%%%%%%%%%%%%%%%%%%%%%%%%%%%%%%%%%%%%%%%%%%%
%
% I like the bibliography to be extremely plain, showing only a numeric
% identifier and citing everything in simple brackets. The first names,
% if present, will be initialized. DOIs and URLs will be preserved.

\usepackage[%
  autocite     = plain,
  backend      = bibtex,
  doi          = true,
  url          = true,
  giveninits   = true,
  hyperref     = true,
  maxbibnames  = 99,
  maxcitenames = 99,
  sortcites    = true,
  style        = numeric,
  ]{biblatex}

%%%%%%%%%%%%%%%%%%%%%%%%%%%%%%%%%%%%%%%%%%%%%%%%%%%%%%%%%%%%%%%%%%%%%%%%
% Some adjustments to make the bibliography more clean
%%%%%%%%%%%%%%%%%%%%%%%%%%%%%%%%%%%%%%%%%%%%%%%%%%%%%%%%%%%%%%%%%%%%%%%%
%
% The subsequent commands do the following:
%  - Removing the month field from the bibliography
%  - Fixing the Oxford commma
%  - Suppress the "in" for journal articles
%  - Remove the parentheses of the year in an article
%  - Delimit volume and issue of an article by a colon ":" instead of
%    a dot ""
%  - Use commas to separate the location of publishers from their name
%  - Remove the abbreviation for technical reports
%  - Display the label of bibliographic entries without brackets in the
%    bibliography
%  - Ensure that DOIs are followed by a non-breakable space
%  - Use hair spaces between initials of authors
%  - Make the font size of citations smaller
%  - Fixing ordinal numbers (1st, 2nd, 3rd, and so) on by using
%    superscripts

% Remove the month field from the bibliography. It does not serve a good
% purpose, I guess. And often, it cannot be used because the journals
% have some crazy issue policies.
\AtEveryBibitem{\clearfield{month}}
\AtEveryCitekey{\clearfield{month}}

% Fixing the Oxford comma. Not sure whether this is the proper solution.
% More information is available under [1] and [2].
%
% [1] http://tex.stackexchange.com/questions/97712/biblatex-apa-style-is-missing-a-comma-in-the-references-why
% [2] http://tex.stackexchange.com/questions/44048/use-et-al-in-biblatex-custom-style
%
\AtBeginBibliography{%
  \renewcommand*{\finalnamedelim}{%
    \ifthenelse{\value{listcount} > 2}{%
      \addcomma
      \addspace
      \bibstring{and}%
    }{%
      \addspace
      \bibstring{and}%
    }
  }
}

% Suppress "in" for journal articles. This is unnecessary in my opinion
% because the journal title is typeset in italics anyway.
\renewbibmacro{in:}{%
  \ifentrytype{article}
  {%
  }%
  % else
  {%
    \printtext{\bibstring{in}\intitlepunct}%
  }%
}

% Remove the parentheses for the year in an article. This removes a lot
% of undesired parentheses in the bibliography, thereby improving the
% readability. Moreover, it makes the look of the bibliography more
% consistent.
\renewbibmacro*{issue+date}{%
  \setunit{\addcomma\space}
    \iffieldundef{issue}
      {\usebibmacro{date}}
      {\printfield{issue}%
       \setunit*{\addspace}%
       \usebibmacro{date}}%
  \newunit}

% Delimit the volume and the number of an article by a colon instead of
% by a dot, which I consider to be more readable.
\renewbibmacro*{volume+number+eid}{%
  \printfield{volume}%
  \setunit*{\addcolon}%
  \printfield{number}%
  \setunit{\addcomma\space}%
  \printfield{eid}%
}

% Do not use a colon for the publisher location. Instead, connect
% publisher, location, and date via commas.
\renewbibmacro*{publisher+location+date}{%
  \printlist{publisher}%
  \setunit*{\addcomma\space}%
  \printlist{location}%
  \setunit*{\addcomma\space}%
  \usebibmacro{date}%
  \newunit%
}

% Ditto for other entry types.
\renewbibmacro*{organization+location+date}{%
  \printlist{location}%
  \setunit*{\addcomma\space}%
  \printlist{organization}%
  \setunit*{\addcomma\space}%
  \usebibmacro{date}%
  \newunit%
}

% Do not abbreviate "technical report".
\DefineBibliographyStrings{english}{%
  techreport = {technical report},
}

% Display the label of a bibliographic entry in bare style, without any
% brackets. I like this more than the default.
%
% Note that this is *really* the proper and official way of doing this.
\DeclareFieldFormat{labelnumberwidth}{#1\adddot}

% Ensure that DOIs are followed by a non-breakable space.
\DeclareFieldFormat{doi}{%
  \mkbibacro{DOI}\addcolon\addnbspace
    \ifhyperref
      {\href{http://dx.doi.org/#1}{\nolinkurl{#1}}}
      %
      {\nolinkurl{#1}}
}

% Use proper hair spaces between initials as suggested by Bringhurst and
% others.
\renewcommand*\bibinitdelim {\addnbthinspace}
\renewcommand*\bibnamedelima{\addnbthinspace}
\renewcommand*\bibnamedelimb{\addnbthinspace}
\renewcommand*\bibnamedelimi{\addnbthinspace}

% Make the font size of citations smaller. Depending on your selected
% font, you might not need this.
\renewcommand*{\citesetup}{%
  \biburlsetup
  \small
}

\DeclareLanguageMapping{british}{bibliography-correct-ordinals}
\DeclareLanguageMapping{english}{bibliography-correct-ordinals}

\bibliography{Thesis}

%%%%%%%%%%%%%%%%%%%%%%%%%%%%%%%%%%%%%%%%%%%%%%%%%%%%%%%%%%%%%%%%%%%%%%%%
% Fonts
%%%%%%%%%%%%%%%%%%%%%%%%%%%%%%%%%%%%%%%%%%%%%%%%%%%%%%%%%%%%%%%%%%%%%%%%

\ifxetexorluatex
  \setmainfont{Minion Pro}
\else
  \usepackage[lf]{ebgaramond}
  \usepackage[oldstyle,scale=0.7]{sourcecodepro}
  \singlespacing
\fi

\renewcommand{\th}{\textsuperscript{\textup{th}}\xspace}

\newacronym[description={Principal component analysis}]{PCA}{PCA}{principal component analysis}
\newacronym                                            {SNF}{SNF}{Smith normal form}
\newacronym[description={Topological data analysis}]   {TDA}{TDA}{topological data analysis}

\newglossaryentry{LaTeX}{%
  name        = {\LaTeX},
  description = {A document preparation system},
  sort        = {LaTeX},
}

\newglossaryentry{Real numbers}{%
  name        = {$\real$},
  description = {The set of real numbers},
  sort        = {Real numbers},
}

\makeindex
\makeglossaries

%%%%%%%%%%%%%%%%%%%%%%%%%%%%%%%%%%%%%%%%%%%%%%%%%%%%%%%%%%%%%%%%%%%%%%%%
% Incipit
%%%%%%%%%%%%%%%%%%%%%%%%%%%%%%%%%%%%%%%%%%%%%%%%%%%%%%%%%%%%%%%%%%%%%%%%

\title{\texttt{latex-mimosis}}
\subtitle{A minimal, modern \LaTeX{} package for typesetting your thesis}
\author{Bastian Rieck}

\begin{document}

\frontmatter
  \begin{titlepage}
  \vspace*{5cm}
  \makeatletter
  \begin{center}
    \begin{Huge}
      \@title
    \end{Huge}\\[0.1cm]
    %
    \begin{Large}
      \@subtitle
    \end{Large}\\
    %
    \emph{by}\\
    \@author
    %
    \vfill
    A document submitted in partial fulfillment
    of the requirements for the degree of\\
    \emph{Technical Report}\\
    at\\
    \textsc{Miskatonic University}
  \end{center}
  \makeatother
\end{titlepage}

\newpage
\null
\thispagestyle{empty}
\newpage

  %*******************************************************
% Abstract
%*******************************************************
%\renewcommand{\abstractname}{Abstract}
\addcontentsline{toc}{chapter}{\abstractname}

\pdfbookmark[1]{Abstract}{Abstract}
\begingroup
\let\clearpage\relax
\let\cleardoublepage\relax
\let\cleardoublepage\relax

\chapter*{Abstract}
An abstract is a brief of a research article, thesis, review,
conference proceeding or any in-depth analysis of a particular
subject or discipline, and is often used to help the reader
quickly ascertain the paper's purpose. When used, an abstract
always appears at the beginning of a manuscript or typescript,
acting as the point-of-entry for any given academic paper or
patent application. Abstracting and indexing services for various
academic disciplines are aimed at compiling a body of literature
for that particular subject.

Max 2200 characters, spaces included.

\vfill
\newpage
\pdfbookmark[1]{Sommario}{Sommario}
\chapter*{Sommario}
Per abstract si intende il sommario di un documento, senza l'aggiunta di interpretazioni e valutazioni. L'abstract si limita a riassumere, in un determinato numero di parole, gli aspetti fondamentali del documento esaminato. Solitamente ha forma "indicativo-schematica"; presenta cioé notizie sulla struttura del testo e sul percorso elaborativo dell'autore.

Max 2200 caratteri compresi gli spazi.

\endgroup

  \tableofcontents

\mainmatter

  \defaultfont

\BiChapter{����}{Introduction}
\label{Introduction}

\BiSection{���ⱳ��������}{The Background and Significance}
\label{Introduction:background}
\LaTeX~���ھ����Ű����ۡ��Թ�ʽ��ͼ���Ĵ�������ǿ���Լ���ƽ̨ͨ����ǿ�����ƣ�
ʹ�����ڿƼ��Ű��е�Ӧ��Խ��Խ�㷺��

\BiSection{����֪ʶ}{Mastering \LaTeX{}}
\label{sec:learningknowledge}
���ǵ�����ͬѧû�нӴ���~\LaTeX{}��Ϊ��������·���������֣�����ѧ��~\LaTeX{}�Ļ���ʹ�÷�����
�Ӷ��Ѹ����ʱ��Ͷ�뵽���ĵ�д�������У���רע�����ĵ����ݣ�������һ�ڽ���~\LaTeX �Ļ���֪ʶ��
�Ƽ�һЩ�ĵ����ϣ������õı༭���ɵ����ݡ�

\BiSubsection{ʲô��\LaTeX{}}{What is \LaTeX}
\label{sec:whatislatex}
        \TeX/\LaTeX ��һ�׹���ǿ���Ű������Ŀ���Դ�������Ѱ칫�Ű�������

        �Զ��ֲ���ϵͳ������~Microsoft Windows��\CJKglue Unix��~���磺Solaris��\CJKglue Linux �ȣ���\CJKglue
�Լ�~Mac OS X ��������Ӧ�����а汾��������Ҳ������ͬ���䷢չ���������ڻ���~linux ��
�˵��ڶ�~linux ����ϵͳ�ķ�չ���̡�

        ��~Windows ����õ���~\href{http://www.miktex.org}{MikTeX} ����������������װ��Linux ��������ò��ҳ������µ���
        TeXlive����ƽ̨��ijЩ�汾Ҳ������windows�£�����һ������ϵͳ~teTeX ���ֹͣ��ά����

        ��MikTeX�����ϣ�\href{http://www.ctex.org}{CTeX} ��~Aloft վ�������������������֧�֣�������~CTeX ������
װ����װ���ã���ȥ���û�������֮�࣬�Ƽ������û�����ʹ�á�

        ������������~��WYSIWYG��What You See Is What You Get�� ��~Microsoft Office
������ȣ������ص��ǣ�
   \begin{itemize}
      \item ���뼴����~��WYSTWYG��What You Think Is What You Get���������רע
�����ĵ�˼·��ͨ�����Ƿ��ӵĸ�ʽҪ�󣬸��ʺ��Ű�Ƽ����ģ�
      \item ���Ƹ�ʽ����,���������ݣ���ѧ��ʽ�����Ű淽��,���������
      \item ���ı��ļ�����������~MSWord �ĸ��ָ�ʽ�ױ䡢�ĵ��𻵡���ʽ�޷��༭�Ȳ��ȶ�����
      Ҳ�������ڰ汾���ƣ�
      \item �����~PDF �ļ��ǹ����ĵ���׼��������Ҫ��˶��ʿ��ҵ�����ύ��Ҳ��~PDF ��ʽ��
      \item �����ڿ�������һ�㶼�ṩ~\LaTeX ����ģ��,ʹ����Ͷ���Ű�����ף�
      \item Ŀǰ�����ⲻ�ٸ�УҲ������~\LaTeX{}ѧλ����д��ģ�壬ʹд��ѧλ���ĵ��Ű治����ʹ�࣬
      ����һ�����ܡ�������Ƚϳ���ļ�Ϊ��ģ�壻
      \item �����õ�Ƭ��~LaTeX ���~beamer���Ű湫ʽ����������һ�����㣬û��~PowerPoint
�����ַ�����ʽ��ͼƬλ�õ������ڶ��Ĭ��ģ�湩ѡ��һ��������Ϳ��л����ûõ�
Ƭ���������ɡ�רҵ��Ư����
      \item  �ڶ���ĵ���ͺ��֧�֣�����ĸо���``û�����������ģ�ֻ�����벻����''��
      \item  ��������ʵ��~TeXer ���ԣ�\TeX/\LaTeX �Ѿ���������һ���Ű�����������Ϊһ��������
      ��Ϊ���ĵ������䷢չ��������һ�����������Ĵ��档
    \end{itemize}

    �ܶ��˶���~\LaTeX{} �������ܣ�����Դ��϶����TeX����Ϊ~ 1 �����������ϼ�ҳ������У�ڵ�
    \url{ftp://202.118.224.241/software/Science/TeX&LaTeX/TeX\%20documents} ��Ҳ�м����õ�Ƭ���������˽��ܡ�
    �������߷����޸Ķ�Σ��������Ƽ���վ~\href{http://zzg34b.w3.c361.com/index.htm}{LaTeX�༭��} �ϵļ�ƪ���£�����ȫ���˽�����
   \begin{itemize}
     \item  \href{http://learn.tsinghua.edu.cn:8080/2001315450/tex_frame.html}{TeX���}������ʹ�õ�~\LaTeX ϵͳ�Ļ�����
     \item \href{http://zzg34b.w3.c361.com/homepage/TeXvirtue.htm}{TeX����ȱ��}: ���������������ǻ�����ʮȫʮ����
     \item \href{http://zzg34b.w3.c361.com/homepage/LaTeXbring.htm}{LaTeX�IJ���}�������ճ��Ӵ�����������
     \item \href{http://zzg34b.w3.c361.com/homepage/compareWord.htm}{LaTeXӦ�����}:���������������ڹ�����ʹ����Խ��Խ�ࡣ
     \item \href{http://zzg34b.w3.c361.com/homepage/compareWord.htm}{��Word��Ƚ�}:���������õ�����~word��~LaTeX�����߸����ŵ��ȱ�㣬��Ҫ�����ƫִ�Ĺ۵��У�����һƪ�ȽϿ͹۵����¡�
     \item \href{http://zzg34b.w3.c361.com/homepage/KnuthResume.htm}{Knuth ���ڼ���}:�ײ��ں�~ \TeX ������~ Donald Knuth (�ߵ���)�Ľ��ܣ����д���ɫ�ʣ�
     \item \href{http://zzg34b.w3.c361.com/homepage/LamportResume.htm}{Lamport ��ʿ����}�� \LaTeX ������~Lamport����������Ŭ��������~\LaTeX ʹ�ü򵥺ܶ࣬���ҿƼ��磻
   \end{itemize}

\BiSubsection{�Ƽ�������������}{LaTeX Software}
\label{sec:latexsoftware}

�������֣�Windows ��ֻ�Ƽ���ѵ�~CTeX ��װ����Ϊ˿������Ҫ�Լ����ã���װ��Ϳ���ʹ�á�
������У԰���û����Դ�~ \url{ftp://202.118.224.241/software/Science/TeX&LaTeX/CTex/} ����~
CTeX-2.4.5-8-Full.exe ��~ CTeX-Fonts-2.4.4.exe���Ȱ�װ��װϵͳ��Ȼ��װ���塣��У԰���û�
���Դ�~ \href{http://www.ctex.org}{CTeX�Ĺٷ���վ} �����������ļ���

\BiSubsection{�Ƽ�����������}{LaTeX Documents}

������ǵ�һ�νӴ�~ LaTeX����ô��װ~ CTeX ֮�󣬲�Ҫֱ�Ӵ򿪱༭����~WinEdt ���в�������Ϊ���������������˽⻹���٣��������ʴӡ����~ Windows
ϵͳ�Ŀ�ʼ~ $\rightarrow$ ����~ $\rightarrow$ ���� TeX ��װ~ $\rightarrow$ help $\rightarrow$
�����ĵ��˰ɣ����ǽ����˳���ǣ����ȴ�~ CTeX FAQ����ͷ����''��������''��һ�ڣ�Ȼ��
��ͬһĿ¼�´�~ LShort-cn �ļ�����ͷ��β�������һ�飬���ó��Լ�ס���е����
ֻҪ�˽�~ LaTeX ���ص㣬��ÿһ������һ��������ʶ�Ϳ��ԡ��Ժ��㻹���Ի�ͷ������������
Ȼ��� ~CTeX FAQ �����ϣ�����������У����Գ���ȥ��ϰ�Ű�һЩ�ĵ����鿴����Ч��������~ WinEdt ����
�����뿴����Ľ� \ref{sec:winedttricks}����

������⵽���������ĵ�~PDF: latex2e ��ͼָ�Ϻ�~mathematics (LaTeX ���� The LaTeX Companion ��~chapter8)��
�ֱ��ǽ���ͼ֪ʶ�͹�ʽ���뷽���ģ�������ϸ������Ҳ���һ�¡����ڹ�ʽ���룬����һ���ر�ֵ���Ƽ�����
\href{http://www.tug.org/tex-archive/info/math/voss/mathmode/}{mathmode 2.0}��У԰���û������Դ�
\href{ftp://202.118.224.241/software/Science/TeX&LaTeX/TeX documents/}{У��~FTP TeX ����Ŀ¼}���ء�

\BiSubsection{WinEdt�ı��뼰��������}{Winedt Tricks}
\label{sec:winedttricks}
~
��һ�ĵ�~ WinEdt\_LaTeX\_guide.doc �򵥽�����~ WinEdt �ļ��ĵ��ı��뷽�������Ե��
\url{http://bbs.hit.edu.cn/bbscon.php?bid=296&id=1887&ap=719} �õ���

������ϸ���ܱ��밴ť�ĺ��壬����һ�������ر���棬
��ע������Ľ���˳����~ WinEdt ��Ĭ������˳��
\begin{hitlist}
  \item TeX: ��������ʹ��~ TeX ����д���ĵ����ǵײ�ı���ϵͳ��
  \item LaTeX: ��������ʹ��~ LaTeX ����д���ĵ�����Ŀǰ����ʹ������~ LaTeX2e �ĵ�����ϵͳ������~ dvi �ļ���
  \item cct\& LaTeX: cct �ǹ��ڵ����ֲ��о�Ա������һ��ʹ��~ LaTeX �����������ĵ��Ľӿ�ϵͳ��
  ���Ȱ�~cct ���ĵ���~.ctx ת����~.tex ��ʽ��Ȼ����ñ�׼��~ LaTeX ����������dvi�ļ���
  \item PDFLaTeX: ����������~LaTeX������һ�ֱ���ϵͳ��ֱ������~pdf �ļ���֧�ָ����~ pdf �ļ���Ч������Ӧ��Խ��Խ�㷺���������Ļõ�Ƭ��
  \item BibTeX: ��������������ο����׵����ͨ��������һ�������ο�������Ŀ���б�~ bbl �ļ����Ű�ʹ�á�
  \item Make Index: ������������ĵ���������
  \item TeXify: ���Ǽ�����������ĺϼ������Զ�����~ LaTeX����pdflatex����MakeIndex ��~ BibTeX ��������Ҫ�Ĵ���������һ��
  ��������������б��ͽ������õ�~ dvi��pdf���ļ�������~ dvi��pdf���ļ������ɹ��̡�
  \item CTeXify: �����~ CTeX ��װ����������֧�ֵ�~TeXify ��������������ĵ�~ dvi(pdf) �ĵ���
\end{hitlist}

����ʹ���ĸ����밴ť��������ĵ����ͼ������������й�ϵ����Ϊ��ͬ�ı�����������ĵ��е�Ԫ��Ҫ��һ����
���磬��������õ���~eps ͼ�Σ�Ӧ����~latex�����룬����������~pdf ͼ�Σ�Ӧ����~pdflatex �����롣
�����������ǰ����ĵ���Ӧ��Ҳ�Ѿ������ˡ�

\BiSubsubsection{��ʾ�ĵ��ṹͼ}{File Structure Display}

WinEdt�е�~gather�����ռ��½ڱ��⣬�γ�~TOC �б�������������~word �е��ĵ��ṹͼ����
д���ĵ���ʱ��������ܷdz����ã����������ǵ�~Pluto ģ���У��Զ�����һЩ�½ڱ��⣬
��Щ�Զ������ȱʡ��~gather �Dz�ʶ��ġ�TeX@lilac �ṩ��һ�ַ�������~WinEdt.gdi�ж�����������
��Ч��ϣ����ʹ��������ܵ����ѿ����Լ������޸ģ�tools �ļ����������޸Ĺ���~WinEdt.gdi��
������Ҳ����ֱ��ʹ�ã��ŵ�~winedt Ŀ¼���滻ͬ���ļ����ɡ�

winEdt ��~ tree interfacezho ��Ҳ��~ TOC ��һ��������ͨ���޸� ~WinedtĿ¼�µ�~WinEdtEx.iniʵ�֡���~tools�ļ�����
��~WinEdtEx.ini �滻��ͬ���ļ��Ϳ��ԡ�

������Щ�Զ���������ڱ༭״̬�²�����ȱʡ����һ������������ܸ����ͺ��ˣ�TeX ͬ���ҵ����Լ�����ķ�����
��~winedt �˵�~ option/highlighting/switches �޸ģ�tools �ļ���~ Switches.dat ������
����õģ������������˵�λ��ʹ�öԻ���������~``Load from'' ��ť���ء�

\BiSubsection{����ͼ�ij�����ʽ}{Figure Generating}

��~latex �ĵ���д�����У����õ�ͼ�θ�ʽ��~ eps ��~ pdf ��

pdf �ļ����ɣ������úܶ��������ɣ�����~adobe acrobat ��pdffactory��pdf xchange �ȡ�
�����Ƽ�~ acrobat (ע�ⲻ��~ acrobat  reader)����Ϊ����װ������һ��~ pdf ��ӡ����
�κ�һ���ĵ�������ͨ�������ӡ������~pdf �ļ�������Ҫ�Ĺ����Ƕ�~ pdf �ļ��Ķ��༭��
pdf �ļ�������~acrobat ����вü�~ (documents,crop pages...)��ȡ������ҳ�档����֧��
ֱ������Ϊ~ eps �ļ��Ĺ��ܡ����ԣ�����~acrobat �������������е�ͼ�δ������ⶼ���Խ����
���������������ҵ������

������~latex ����ͼ�����Ƚ϶࣬���е���ͼ������~visio��~ coraldraw��~ photoshop��~ gnuplot ���ɵ�ͼ
������ͨ������ķ���ת����~eps ��~pdf �ļ������⣬���кܶ�ר��Ϊ~latex ��������ͼ������
��ͨ���������ʽ����ͼ�ģ�����~metapost��~pstricks��~asymptote��~pgf/tikz �ȣ�Ҳ�м򵥵�
������ͼ���������������~Dia��~winfig��~gclc �ȡ�  ��������������Щ����������ԱȽϼ򵥣�������ͼҲ����Ư�������Ƕ�û��΢����
~visio ��������ǿ�����Զ����������������~visio ����ͼ��

���ڲ�ͼ��������ڵ����İ��ͼָ���Ѿ�����ϸ�ˣ����ﲻ�ٽ��ܣ��뷭�ĸ��顣

\BiSubsection{�������д��ѧ��ʽ}{Math Inputting}

����һ��о���~LaTeX д��ʽ�Ƚ��鷳���Ӷ�����ȴ������ʵ������Դ�~mathtype����ҵ������
�������~ word �ﲻ�����Ļ�������������ѵ�~TeX ��ʽ����~TeXaids��ת���ķ�����
��~mathtype ��ʽ��������£�ѡ��˵��� ~``preference��translators...,translate to other language,
ѡ��~ TeX--LaTeX2.09 and later (��׼��LaTeX����)������~ TeX--AMS-LaTeX(��Ҫ~amsmaht ���֧��) ,ȷ�� ''
��Ȼ�����빫ʽ��ѡ�У����ƣ�ճ������ı༭����������빫ʽ�ĵط�����ῴ���������
��Ҫ�Ĺ�ʽ��~ LaTeX  ���룬�����Ϳ������������ˣ���ʱ������Ҫ����һ����ʹ�õ���ѧ�������ܴﵽ
���Ҫ��

�����������ѧ��ʽ��ʱ����������һ����ѧ���ţ����Ե������������ͷ���ͼ�꣬����ʾ
��ѧ���Ź�������Ȼ��������Ҫ����ѧ���ţ��Ͳ��뵽���ĵ��С�����һ��ר�ŵ�~symbol.pdf �ĵ���ctex
��װ��Ҳ�Ѿ��䱸���������еķ��ţ�����һЩϡ��Ź֣�������û�������ķ��ţ��������������ҵ���

ǰ���ᵽ��ctex ��װ��~help �ļ����Ѵ���~ch8.pdf ��~Mathmode.pdf �ļ��Թ�ʽ��д���ܵĺ���ϸ�����ָ����Ĺ�ʽ
������ͨ����ͬ����ѧ�����õ�������д��ʽʱ���ġ�

\BiSubsection{���ٲ���ͼ��}{Inserting Figures and Tables}

����ͼ���Ȼ����IJ��룬WinEdt �ṩ�˺������ķ�����ֻҪѡ�񹤾�����ͼƬ�ͱ���ť���Ϳ���
����һ��������ͼ���������Ҫ��ֻ�ǰ����е��ǺŻ����Լ��Ķ����Ϳ����ˡ�
WinEdt���ṩ��һ���꣬GUI ��ʽ���ͼ�IJ�����̣�ѡ����࣬Ҳ�����㡣
��ʱ�򣬸о�����Ƚϸ��ӣ���������~LaTeX д������Գ���һ�� ~xl2latex 2.0 ���~ excel
�������Ϊ~LaTeX ����ĺꡣ������~excel �����ɱ���Ȼ������һ�º꣬�������˱����~LaTeX ���롣
ģ���~edittools Ŀ¼���ṩ����һ�ļ���

\BiSubsection{���༼�ɼ����������ĵ�}{More Tricks and Others' Thoughts}

��������ż��ɣ����Բο��϶���TeX����õ׵�����~\href{http://bbs.hit.edu.cn/bbscon.php?board=TeX&id=2038&ftype=11}{���沿�����⵼��(��Ҫ����������)}��


% This ensures that the subsequent sections are being included as root
% items in the bookmark structure of your PDF reader.
\bookmarksetup{startatroot}
\backmatter

  \begingroup
    \let\clearpage\relax
    \glsaddall
    \printglossary[type=\acronymtype]
    \newpage
    \printglossary
  \endgroup

  \printindex
  \printbibliography

\end{document}
