\thispagestyle{empty}

\begin{center}
\bfseries \abstractname 
\end{center}

\emph{Formal methods} are structured methodologies that support the development of critical systems---whose safety and reliability are fundamental requirements---with the aim of 
establishing system correctness with mathematical rigor,
providing effective verification techniques and tools, and reducing verification time while simultaneously increasing coverage.

\emph{Model checking} (MC) is a family of formal methods that have been accepted by industry and are becoming integral part of standards and of system development cycles. 
In MC, some properties of a finite-state transition system are expressed in suitable specification languages and then verified over a model of the system itself (usually a finite Kripke structure) through \emph{exhaustive enumeration of all the reachable states}. This technique is \emph{fully automatic} and every time the design violates a desired property, a \emph{counterexample} is produced, which illustrates a behavior falsifying such a property: this is extremely useful for debugging.

The most famous MC techniques---just to mention a few, \emph{partial order reduction}, \emph{symbolic} and \emph{bounded} MC---were developed starting from the late 80s, bearing in mind the well-known ``point-based'' temporal logics \LTL\ and \CTL. However, while the expressiveness of such logics is beyond doubt,
there are some properties we may want to check that are inherently ``interval-based'' and thus cannot be expressed by point-based temporal logics, e.g., ``the proposition $p$ has to hold in at least an average number of system states in a given computation sector''. Here \emph{interval temporal logics} (ITLs) come into play, providing an alternative setting for 
reasoning about time. Such logics deal with intervals, instead of points, as their primitive entities: this feature gives them the ability of 
expressing temporal properties, such as actions with duration, accomplishments, and temporal aggregations, which cannot be dealt with in standard point-based logics.

The \emph{Halpern
and Shoham's modal logic of time intervals} (\HS, for short) is one of the most famous ITLs:
it features one modality for each of the 13 possible ordering
relations between pairs of intervals, apart from equality.
In this thesis we focus our attention on MC based on \HS , in the role of property specification language,
for which a little work has been done if compared to MC for point-based temporal logics.
The idea is to evaluate \HS\ formulas on finite Kripke structures, making it
possible to check the correctness of the behavior of systems with respect to 
meaningful interval properties.
To this end, we interpret each one of the (possibly infinitely many) finite paths of a Kripke structure as an interval, 
and we define its atomic properties on the basis of the properties of the states composing it,
at first assuming the \emph{homogeneity principle}: the latter enforces an atomic property to hold over an interval if and only if it holds over all its subintervals.
%
% Formally, we will show that finite Kripke structures can be suitably mapped into
% interval-based structures, called \emph{abstract interval models}, over which 
% HS formulas can be interpreted. Such models have in general an \emph{infinite} domain, 
% because finite Kripke structures may have loops and thus infinitely many tracks.
% In order to devise a model checking procedure for HS over finite Kripke
% structures, we prove a \emph{small model theorem} showing that, given an HS 
% formula $\psi$ and a finite Kripke structure $\mathpzc{K}$, there exists a \emph{finite}
% interval model which is equivalent to the one induced by $\mathpzc{K}$ with respect 
% to the satisfiability of $\psi$.
We prove that MC for \HS\ interpreted over finite Kripke structures is a \emph{decidable} problem (whose computational complexity has a nonelementary upper bound), and
then we show it to be \EXPSPACE-hard.

Since the problem provably admits no polynomial-time decision procedure, we also focus on \HS\ fragments, which 
feature considerably better complexities---from \EXPSPACE, down to \Psp\ and to low levels of the polynomial hierarchy---yet retaining the ability to capture meaningful interval properties of state transition systems.
Several MC algorithms are presented, tailored to the specific fragments being considered, and founded on concepts and techniques different from each other.

Moreover, we study the expressive power of $\HS$ in MC, in comparison with that of the standard point-based logics $\LTL$, $\CTL$ and $\CTLStar$, still under the homogeneity principle, which is then relaxed showing how this impacts on the complexity of MC for \HS\ and its fragments, and on the expressiveness of the logic.

Finally, we consider a possible replacement of Kripke structures by a more expressive model, which allows us to directly  describe systems in terms of their \emph{interval-based behaviour and properties}, thus paving the way for a more general interval-based MC.

\bigskip
\noindent\hrulefill
\bigskip

\begin{center}
\begin{tabular}{rl}
    \textbf{Keywords:} & Model checking, interval temporal logics, \\
    & timelines, computational complexity \\ 
    \rule[-1ex]{0pt}{4.5ex} \textbf{2010 MSC:} & 03B70, 68Q60 \\ 
    \rule[-1ex]{0pt}{4.5ex} \textbf{ACM classes:} & F.4.1, D.2.4 \\ 
\end{tabular}
\end{center}

\cleardoublepage
\thispagestyle{empty}
%\enlargethispage{2\baselineskip}
\selectlanguage{italian}
\begin{center}
\bfseries \abstractname 
\end{center}

I \emph{metodi formali} sono metodologie strutturate che supportano lo sviluppo di sistemi critici---la sicurezza ed affidabilità dei quali sono requisiti fondamentali---allo scopo di dimostrare la correttezza di tali sistemi con rigore matematico, fornendo tecniche e strumenti di verifica efficaci, e riducendo il tempo del processo di verifica, aumentando contemporaneamente il grado di copertura.

Il \emph{model checking} (MC)
è una famiglia di metodi formali che sono stati accettati dal mondo dell'industria e stanno diventando parte integrante di standard e dei cicli di sviluppo dei sistemi.
Nel contesto del MC, alcune proprietà di un sistema di transizione a stati finiti vengono espresse mediante linguaggi di specifica e, successivamente, queste sono verificate su un modello del sistema stesso (di solito una struttura di Kripke finita), tramite \emph{l'enumerazione completa di tutti gli stati raggiungibili}.
Questa tecnica è \emph{totalmente automatica} ed ogni volta che viene violata una proprietà desiderata, viene fornito un \emph{controesempio} che illustra un comportamento che falsifica tale proprietà: ciò è estremamente utile per il processo di debugging.

Le più famose tecniche di MC, come la \emph{partial order reduction}, il \emph{symbolic} ed il \emph{bounded} MC,
furono sviluppate verso la fine degli anni 80 prendendo in considerazione le famose logiche temporali \LTL\ e \CTL, che sono basate su punti. Tuttavia, nonostante l'indubbia espressività di tali logiche, esistono alcune proprietà che potremmo voler verificare che hanno inerentemente una semantica intervallare e quindi non possono essere espresse da logiche puntuali, per esempio: ``la proposizione $p$ deve valere in almeno un dato numero medio di stati del sistema, in un settore di computazione specifico''.
%
Le logiche temporali intervallari entrano in gioco in questi casi, permettendoci di ragionare su aspetti temporali in modo diverso: esse adottano gli intervalli, invece dei punti, come loro entità primitive.
Questa caratteristica dà loro l'abilità di esprimere proprietà intervallari, come azioni con durata, conseguimenti di obiettivi e aggregazioni temporali, che non possono essere trattate nelle logiche puntuali standard.

La \emph{logica modale degli intervalli temporali di Halpern e Shoham} (\HS\  in breve) è una delle più famose logiche intervallari: essa possiede una modalità per ognuna delle 13 possibili relazioni di ordinamento fra coppie di intervalli, eccetto l'uguaglianza.
In questa tesi viene considerato il problema del MC basato su \HS , come linguaggio di specifica delle proprietà, il quale ha ricevuto ben poca attenzione in letteratura in confronto al MC per logiche temporali puntuali.
%
L'idea è quella di valutare formule di \HS\  su strutture di Kripke finite, per riuscire a verificare la correttezza del comportamento di un sistema rispetto a proprietà intervallari.
A questo scopo, ognuno dei percorsi finiti di una struttura di Kripke (i quali possono essere presenti in quantità infinita) è interpretato come un intervallo, e le proprietà atomiche che valgono su quest'ultimo sono definite sulla base di quelle degli stati che lo costituiscono, inizialmente secondo il \emph{principio di omogeneità}: esso prevede che una proprietà atomica valga su un intervallo se e solo se vale su tutti i suoi sottointervalli. 
% Mostreremo infatti che le strutture di Kripke possono essere mappate in certe strutture intervallari, chiamate \emph{abstract interval models}, sulle quali le formule di HS vengono interpretate; esse hanno in generale un dominio \emph{infinito}, perché le strutture di Kripke possono avere cicli e quindi infinite tracce. Al fine di sviluppare una procedura di model checking per HS su strutture di Kripke finite, proviamo uno \emph{small model theorem} che dimostra che data una formula di HS $\psi$ e una struttura di Kripke finita $\mathpzc{K}$, esiste un interval model \emph{finito} che è equivalente a quello indotto da $\mathpzc{K}$ rispetto alla soddisfacibilità di $\psi$.
Dimostriamo innanzitutto che il MC per \HS\  interpretata su strutture di Kripke finite è un problema \emph{decidibile} (la sua complessità computazionale ha un upper bound non-elementare); poi mostriamo che esso è \EXPSPACE-hard.

Poiché il problema non ammette procedure di decisione di complessità polinomiale, 
consideriamo anche frammenti di \HS, i quali si caratterizzano per complessità notevolmente migliori---da
\EXPSPACE, giù fino a \Psp\ e a livelli bassi della gerarchia polinomiale---pur tuttavia mantenendo l'abilità di esprimere
proprietà intervallari significative dei sistemi di transizione. Presentiamo svariati algoritmi di MC, costruiti ad-hoc per gli specifici frammenti considerati, e fondati su concetti e tecniche diversi fra loro.

Inoltre studiamo il potere espressivo di \HS\ in confronto a quello delle logiche puntuali standard $\LTL$, $\CTL$ e $\CTLStar$, sempre sotto l'ipotesi di omogeneità, la quale viene poi rilassata mostrando quali implicazioni ha questo sulla complessità del MC per \HS\ ed i suoi frammenti, e sull'espressività della logica stessa.

Infine consideriamo una possibile alternativa alle strutture di Kripke: studiamo un modello di sistemi più espressivo, che ci permette di descrivere gli stessi direttamente in termini delle loro proprietà intervallari. Questo apre la strada a un MC basato su intervalli più generale.

\bigskip
\noindent\hrulefill
\bigskip

\begin{center}
\begin{tabular}{rl}
    \textbf{Parole chiave:} & Model checking, logiche temporali intervallari, \\
    & timeline, complessità computazionale \\ 
    \rule[-1ex]{0pt}{4.5ex} \textbf{2010 MSC:} & 03B70, 68Q60 \\ 
    \rule[-1ex]{0pt}{4.5ex} \textbf{Classi ACM:} & F.4.1, D.2.4 \\ 
\end{tabular}
\end{center}

\selectlanguage{english}
