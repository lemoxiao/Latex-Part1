\let\openbox\undefined

\usepackage{amsmath,amsthm,amssymb}%amsfonts,
\usepackage{mathtools}

\allowdisplaybreaks

\DeclareMathAlphabet{\mathpzc}{OT1}{pzc}{m}{it}

\theoremstyle{plain}
    \newtheorem{proposition}{Proposition}[section]
    \newtheorem*{proposition*}{Proposition}
    \newtheorem{theorem}[proposition]{Theorem}
    \newtheorem*{theorem*}{Theorem}
    \newtheorem{lemma}[proposition]{Lemma}
    \newtheorem*{lemma*}{Lemma}
    \newtheorem{corollary}[proposition]{Corollary}

\theoremstyle{definition}
    \newtheorem{definition}[proposition]{Definition}
    \newtheorem{example}[proposition]{Example}
    \newtheorem{property}[proposition]{Property}

\theoremstyle{remark}
    \newtheorem{remark}[proposition]{Remark}
    \newtheorem{claim}[proposition]{Claim}


\tcolorboxenvironment{proposition}{enhanced jigsaw,colback=gray!30!white,colframe=white,arc=0mm,breakable,before skip=5pt plus 0.2em,after skip=5pt plus 0.2em}
\tcolorboxenvironment{proposition*}{enhanced jigsaw,colback=gray!30!white,colframe=white,arc=0mm,breakable,before skip=5pt plus 0.2em,after skip=5pt plus 0.2em}
\tcolorboxenvironment{theorem}{enhanced jigsaw,colback=gray!30!white,colframe=white,arc=0mm,breakable,before skip=5pt plus 0.2em,after skip=5pt plus 0.2em}
\tcolorboxenvironment{theorem*}{enhanced jigsaw,colback=gray!30!white,colframe=white,arc=0mm,breakable,before skip=5pt plus 0.2em,after skip=5pt plus 0.2em}
\tcolorboxenvironment{lemma}{enhanced jigsaw,colback=gray!30!white,colframe=white,arc=0mm,breakable,before skip=5pt plus 0.2em,after skip=5pt plus 0.2em}
\tcolorboxenvironment{lemma*}{enhanced jigsaw,colback=gray!30!white,colframe=white,arc=0mm,breakable,before skip=5pt plus 0.2em,after skip=5pt plus 0.2em}
\tcolorboxenvironment{corollary}{enhanced jigsaw,colback=gray!30!white,colframe=white,arc=0mm,breakable,before skip=5pt plus 0.2em,after skip=5pt plus 0.2em}
\tcolorboxenvironment{definition}{enhanced jigsaw,colback=gray!30!white,colframe=white,arc=0mm,breakable,before skip=5pt plus 0.2em,after skip=5pt plus 0.2em}
\tcolorboxenvironment{example}{enhanced jigsaw,colback=white,colframe=black,boxrule=0.5pt,arc=1mm,breakable,before skip=8pt plus 0.2em,after skip=8pt plus 0.2em}
\tcolorboxenvironment{property}{enhanced jigsaw,colback=gray!30!white,colframe=white,arc=0mm,breakable,before skip=5pt plus 0.2em,after skip=5pt plus 0.2em}
\tcolorboxenvironment{remark}{enhanced jigsaw,colback=gray!30!white,colframe=white,arc=0mm,breakable,before skip=5pt plus 0.2em,after skip=5pt plus 0.2em}
\tcolorboxenvironment{claim}{enhanced jigsaw,colback=gray!30!white,colframe=white,arc=0mm,breakable,before skip=5pt plus 0.2em,after skip=5pt plus 0.2em}

%RIDEFINISCO PROOF
\makeatletter
\renewenvironment{proof}[1][\proofname]{%
\par
\pushQED{\qed}%
\normalfont %\topsep6\p@\@plus6\p@\relax
%\trivlist
%\item\relax
\noindent{\bfseries #1\@addpunct{.}}\hspace\labelsep\ignorespaces
}{%
\popQED%\endtrivlist
\@endpefalse
\par\smallskip
}
\makeatother

\renewcommand{\qedsymbol}{$\square$}

%DEFs
\DeclareMathAlphabet{\mathpzc}{OT1}{pzc}{m}{it}

\newcommand{\Ku}{\ensuremath{\mathpzc{K}}}
\newcommand{\Prop}{\ensuremath{\mathpzc{AP}}}
\newcommand{\AP}{\Prop}%alias
\newcommand{\States}{S}
\newcommand{\Edges}{\ensuremath{\textit{R}}}
\newcommand{\sinit}{s_0}
\newcommand{\Lab}{\mu}
\newcommand{\KuDef}{(\Prop,\States,\Edges,\Lab,\sinit)}

\newcommand{\Nat}{\mathbb{N}}

\DeclareMathOperator{\true}{\top}
\DeclareMathOperator{\false}{\bot}

\newcommand{\Length}{\textit{length}}

\newcommand{\tpl}[1]{(#1)}
\newcommand{\tupleof}[1]{(#1)}%alias

\DeclareMathOperator{\Trk}{Trc}
\DeclareMathOperator{\Pref}{Pref}
\DeclareMathOperator{\Suff}{Suff}
\DeclareMathOperator{\states}{states}
\DeclareMathOperator{\intstates}{intstates}
\DeclareMathOperator{\lst}{lst}
\DeclareMathOperator{\fst}{fst}

\DeclareMathOperator{\nestbe}{\mathsf{Nest}_{BE}}
\DeclareMathOperator{\nestb}{\mathsf{Nest}_B}
\DeclareMathOperator{\neste}{\mathsf{Nest}_E}

\newcommand{\nnf}{\textsf{NNF}}
\newcommand{\NNF}{\nnf}

%Descrittori
\DeclareMathOperator{\Root}{root}
\DeclareMathOperator{\DV}{\mathcal{V}}
\DeclareMathOperator{\DE}{\mathcal{E}}

\newcommand{\RE}{\mathsf{RE}}
\newcommand{\Lang}{{\mathpzc{L}}}
\newcommand{\lang}{\Lang}%alias
\newcommand{\SPEC}{\textsf{Spec}}
\newcommand{\NFA}{\text{\sffamily NFA}}
\newcommand{\DFA}{\text{\sffamily DFA}}
\newcommand{\Au}{\ensuremath{\mathcal{A}}}
\newcommand{\Du}{\ensuremath{\mathcal{D}}}

%OPERATORI HS
\DeclareMathOperator{\hsA}{\langle A\rangle}
\DeclareMathOperator{\hsL}{\langle L\rangle}
\DeclareMathOperator{\hsB}{\langle B\rangle}
\DeclareMathOperator{\hsE}{\langle E\rangle}
\DeclareMathOperator{\hsD}{\langle D\rangle}
\DeclareMathOperator{\hsO}{\langle O\rangle}
\DeclareMathOperator{\hsX}{\langle X\rangle}
\DeclareMathOperator{\hsG}{\langle G\rangle}
\DeclareMathOperator{\hsAt}{\langle \overline{A}\rangle}
\DeclareMathOperator{\hsLt}{\langle \overline{L}\rangle}
\DeclareMathOperator{\hsBt}{\langle \overline{B}\rangle}
\DeclareMathOperator{\hsEt}{\langle \overline{E}\rangle}
\DeclareMathOperator{\hsDt}{\langle \overline{D}\rangle}
\DeclareMathOperator{\hsOt}{\langle \overline{O}\rangle}
\DeclareMathOperator{\hsXt}{\langle \overline{X}\rangle}
\DeclareMathOperator{\hsAu}{\mathopen[ A \mathclose]}
\DeclareMathOperator{\hsLu}{\mathopen[ L \mathclose]}
\DeclareMathOperator{\hsBu}{\mathopen[ B \mathclose]}
\DeclareMathOperator{\hsEu}{\mathopen[ E \mathclose]}
\DeclareMathOperator{\hsDu}{\mathopen[ D \mathclose]}
\DeclareMathOperator{\hsOu}{\mathopen[ O \mathclose]}
\DeclareMathOperator{\hsXu}{\mathopen[ X \mathclose]}
\DeclareMathOperator{\hsGu}{\mathopen[ G \mathclose]}
\DeclareMathOperator{\hsAtu}{\mathopen[ \overline{A} \mathclose]}
\DeclareMathOperator{\hsLtu}{\mathopen[ \overline{L} \mathclose]}
\DeclareMathOperator{\hsBtu}{\mathopen[ \overline{B} \mathclose]}
\DeclareMathOperator{\hsEtu}{\mathopen[ \overline{E} \mathclose]}
\DeclareMathOperator{\hsDtu}{\mathopen[ \overline{D} \mathclose]}
\DeclareMathOperator{\hsOtu}{\mathopen[ \overline{O} \mathclose]}
\DeclareMathOperator{\hsXtu}{\mathopen[ \overline{X} \mathclose]}

%FRAMMENTI HS
\newcommand{\A}{\mathsf{A}}
\renewcommand{\AE}{\mathsf{AE}}
\newcommand{\AB}{\mathsf{AB}}
\newcommand{\Abar}{\mathsf{\overline{A}}}
\newcommand{\AAbar}{\mathsf{A\overline{A}}}
\newcommand{\AAbarB}{\mathsf{A\overline{A}B}}
\newcommand{\AbarB}{\mathsf{\overline{A}B}}
\newcommand{\AAbarE}{\mathsf{A\overline{A}E}}
\newcommand{\AAbarBbarEbar}{\mathsf{A\overline{A}\overline{B}\overline{E}}}
\newcommand{\AAbarBBbar}{\mathsf{A\overline{A}B\overline{B}}}
\newcommand{\AAbarEEbar}{\mathsf{A\overline{A}E\overline{E}}}
\newcommand{\ABbar}{\mathsf{A\overline{B}}}
\newcommand{\Bbar}{\mathsf{\overline{B}}}
\newcommand{\Ebar}{\mathsf{\overline{E}}}
\newcommand{\B}{\mathsf{B}}
\newcommand{\E}{\mathsf{E}}
\renewcommand{\L}{\mathsf{L}}
\newcommand{\D}{\mathsf{D}}
\newcommand{\BE}{\mathsf{BE}}
\newcommand{\AbarE}{\mathsf{\overline{A}E}}
\newcommand{\BEbar}{\mathsf{B\overline{E}}}
% \newcommand{\HSexi}{\mathsf{\exists A\overline{A}BE}}
% \newcommand{\HSforall}{\mathsf{\forall A\overline{A}BE}}
\newcommand{\ABBbar}{\mathsf{AB\overline{B}}}
\newcommand{\AAbarBBbarEbar}{\mathsf{A\overline{A}B\overline{B}\overline{E}}}
\newcommand{\BBbarEbar}{\mathsf{A\overline{A}B\overline{B}\overline{E}}}
\newcommand{\AAbarEBbarEbar}{\mathsf{A\overline{A}E\overline{B}\overline{E}}}
\newcommand{\AAbarBE}{\mathsf{A\overline{A}BE}}
\newcommand{\HSprop}{\mathsf{Prop}}

%LOGICHE
\newcommand{\HS}{\text{\sffamily HS}}
\newcommand{\PITL}{\text{\sffamily PITL}}
\newcommand{\DC}{\text{\sffamily DC}}
\newcommand{\CDT}{\text{\sffamily CDT}}
\newcommand{\FO}{\text{\sffamily FO}}
\newcommand{\CTLStar}{\text{\sffamily CTL$^{*}$}}
\newcommand{\CTLStarLP}{\text{\sffamily CTL$^{*}_{lp}$}}
\newcommand{\LTL}{\text{\sffamily LTL}}
\newcommand{\LTLP}{\text{\sffamily LTL$_p$}}
\newcommand{\CTL}{\text{\sffamily CTL}}
\newcommand{\Downarrowx}{\text{$\downarrow$$x$}}
\newcommand{\Downarrowy}{\text{$\downarrow$$y$}}
\newcommand{\until}{\textsf{U}}
\newcommand{\Next}{\textsf{X}}
\newcommand{\Always}{\textsf{G}}
\newcommand{\Eventually}{\textsf{F}}
\newcommand{\EQ}{\exists}
\newcommand{\EQF}{\exists_f}
\newcommand{\AQ}{\forall}
\newcommand{\EQSubf}{\exists\textit{SubF}}
\newcommand{\present}{\textit{present}}

%CLASSI COMPLEX
\newcommand{\PTIME}{\textbf{P}}
\newcommand{\NP}{\textbf{NP}}
\newcommand{\PSPACE}{\textbf{PSPACE}}
\newcommand{\Psp}{\PSPACE}%alias
\newcommand{\NPsp}{\textbf{NPSPACE}}
\newcommand{\LOGSPACE}{\textbf{LOGSPACE}}
\newcommand{\NLOGSPACE}{\textbf{NLOGSPACE}}
\newcommand{\NLOGSP}{\NLOGSPACE}%alias
\newcommand{\EXPSPACE}{\textbf{EXPSPACE}}
\newcommand{\NEXPTIME}{\textbf{NEXPTIME}}
\newcommand{\EXPTIME}{\textbf{EXPTIME}}
\newcommand{\AEXP}{\textbf{AEXP}}
\newcommand{\LINAEXPTIME}{\mathbf{AEXP_{pol}}}
\newcommand{\PH}{\textbf{PH}}
\newcommand{\co}{\textbf{co-}}
\newcommand{\Th}{\PTIME^{\NP[O(\log n)]}}
\newcommand{\Thsq}{\PTIME^{\NP[O(\log^2 n)]}}
\newcommand{\Thpar}{\PTIME^{\NP}_{\parallel}}
\newcommand{\Thparlogn}{\PTIME^{\NP}_{\parallel O(\log n)}}
\newcommand{\Tower}{\mathsf{Tower}}
\newcommand{\TOWER}{\textbf{TOWER}}
%\newcommand{\ACK}{\textbf{ACK}} 