\section{Preliminaries}\label{sec:backgr}

In the following, let $\Sigma$ be an alphabet and $w$ be a non-empty finite or infinite word over $\Sigma$. We denote by $|w|$ the length of $w$ ($|w|=\infty$ if $w$ is infinite). For all  $i,j\in\Nat$ with $i\leq j$, $w(i)$ denotes the
$i$-th letter of $w$, while $w(i,j)$ denotes the finite subword of $w$ given by $w(i)\cdots w(j)$.
The set of all the finite words over $\Sigma$ is denoted by $\Sigma^*$, and $\Sigma^+$ represents $\Sigma^*\setminus\{\varepsilon\}$, where $\varepsilon$ is the empty word.

Clearly, a trace $\rho$ of a Kripke structure $\Ku=\KuDef$ can be considered as a finite word over $\States$, where $(\rho(i),\rho(i+1))\in\Edges$ for all $0\leq i<|\rho|-1$.
In addition to traces, we define an \emph{infinite path} $\pi$ of $\Ku$ as an infinite word over $\States$ such that $(\pi(i),\pi(i+1))\in \Edges$ for all $i\geq 0$.
Here, unlike the previous chapters, we assume traces and paths to be 0-based (and not 1-based) for notational convenience.

\subsection{Standard point-based temporal logics}\label{sect:PTL}

We start now by recalling the standard propositional temporal logics $\CTLStar$, $\CTL$  and $\LTL$~\cite{EH86,Pnu77}.
%
Given a set of proposition letters $\Prop$, the formulas $\varphi$ of
$\CTLStar$ are defined as follows:
%
\[
\varphi ::= \top \mid p \mid \neg \varphi \mid \varphi \wedge \varphi \mid \Next \varphi \mid \varphi \until \varphi \mid \EQ  \varphi ,
\]
%
where $p\in \Prop$, $\Next$ and $\until$ are the
``next'' and ``until'' temporal modalities,   and $\EQ$ is the
 existential path quantifier.%
\footnote{Hereafter, we denote by $\exists/\forall$ the existential/universal path quantifiers (instead of by the usual E/A), in order not to confuse them with the $\HS$ modalities $\mathsf{E}/\mathsf{A}$.}
 We also use the standard shorthands $\AQ\varphi=\neg\EQ\neg\varphi$ (``universal path quantifier''),
$\Eventually\varphi= \top \until \varphi$ (``eventually'' or ``in the future'') and its
dual $\Always \varphi=\neg \Eventually\neg\varphi$ (``always'' or ``globally'').
We denote by $|\varphi|$ the size of $\varphi$, that is, the number of its subformulas.
%
The logic $\CTL$ is the  fragment of $\CTLStar$ where each temporal modality is immediately preceded by a path quantifier, whereas  $\LTL$ corresponds to the path-quantifier-free  fragment of $\CTLStar$.

Given a Kripke
structure $\Ku=\KuDef$, an infinite path $\pi$ of
$\Ku$, and a position $i\geq 0$ along $\pi$, the
satisfaction relation $\Ku,\pi,i \models \varphi$ for
$\CTLStar$, written simply $\pi,i \models \varphi$ when $\Ku$ is clear from the context, is defined as follows (Boolean connectives are treated as usual):%in the standard way):%
\[ \begin{array}{ll}
\pi, i \models p  &  \Leftrightarrow  p \in \Lab(\pi(i)),\\
\pi, i \models \Next \varphi  & \Leftrightarrow   \pi, i+1 \models \varphi ,\\
\pi, i \models \varphi_1\until \varphi_2  &
  \Leftrightarrow  \text{for some $j\geq i$}: \pi, j
  \models \varphi_2,
  \text{ and }  \pi, k \models  \varphi_1 \text{ for all }i\leq k<j,\\
\pi, i \models \EQ \varphi  & \Leftrightarrow \text{for some infinite path } \pi'  \text{ starting from $\pi(i)$, }  \pi', 0 \models \varphi .
\end{array} \]
%
The MC problem is defined as follows: $\Ku$ is a model of $\varphi$, written $\Ku\models \varphi$, if for all initial infinite paths $\pi$ of $\Ku$, it holds that $\Ku,\pi, 0 \models\varphi$.

We also consider a variant of $\CTLStar$, called \emph{finitary} $\CTLStar$, where the path quantifier $\EQ$ of $\CTLStar$ is replaced by the finitary path
quantifier $\EQF$. In this setting, path quantification ranges over the traces starting from the current state.
The satisfaction relation $\rho, i \models \varphi$, where $\rho$ is a trace and $i$ is a position along $\rho$, is similar to that given for $\CTLStar$ with the only difference of finiteness of paths, and the fact that, for a formula $\Next\varphi$, we have  $\rho, i \models \Next\varphi$ if and only if $i+1<|\rho|$ and $\rho, i+1 \models \varphi$.    A Kripke structure $\Ku$  is a model of a finitary $\CTLStar$ formula if, for each initial trace $\rho$ of  $\Ku$,  it holds that $\Ku,\rho, 0 \models\varphi$.
  
The MC problem for both $\CTLStar$ and $\LTL$ is $\PSPACE$-complete~\cite{DBLP:conf/popl/EmersonL85,Sistla:1985}. It is not difficult to show that, as it happens with finitary $\LTL$~\cite{DBLP:conf/ijcai/GiacomoV13}, MC for finitary $\CTLStar$ is $\PSPACE$-complete as well.


\subsection{Three semantic variants of $\HS$ for MC}\label{sect:3sem}%{Finite model checking against \HS.}
In this section we formally define the three presented variants of $\HS$ semantics $\HS_\stat$ (state-based), $\HS_\LinearPast$ (computation-tree-based), and $\HS_\LinearTime$ (trace-based) for model checking $\HS$ formulas against Kripke structures.
For each variant, the related (finite) MC problem consists in deciding whether or not a finite Kripke structure is a model of an $\HS$ formula under such a semantic variant, as defined in the following.

\paragraph*{State-based variant $\HS_\stat$.} 
Let us recall the state-based variant,
which is basically the one we have been assuming so far,
where 
%the authors define a mapping from Kripke structures to abstract interval models.
an abstract interval model  (Definition~\ref{def:AIM})
%(over $\Trk_\Ku$) 
is naturally associated with a given Kripke structure $\Ku$ %, 
by considering the set of intervals as the set 
$\Trk_\Ku$
of traces of $\Ku$. 

\begin{definition}[Abstract interval model induced by a Kripke structure, Definition~\ref{def:inducedmodel}]
The abstract interval model induced by a finite Kripke structure $\Ku=\KuDef$ is
$\mathpzc{A}_\Ku=(\Prop,\mathbb{I},B_\mathbb{I},E_\mathbb{I},\sigma)$, where
\begin{itemize}
    \item $\mathbb{I}=\Trk_\Ku$,
    \item $B_\mathbb{I}=\{(\rho,\rho')\in\mathbb{I}\times\mathbb{I}\mid \rho'\in\Pref(\rho)\}$,
    \item $E_\mathbb{I}=\{(\rho,\rho')\in\mathbb{I}\times\mathbb{I}\mid \rho'\in\Suff(\rho)\}$, and
    \item $\sigma:\mathbb{I}\to 2^\Prop$ such that, for all $\rho\in\mathbb{I}$, 
        \begin{equation*}
            \sigma(\rho)=\bigcap_{s\in\states(\rho)}\Lab(s).
        \end{equation*}
\end{itemize}
\end{definition}

\begin{definition}[State-based $\HS$---$\HS_\stat$, Definition~\ref{def:satkripke} and~\ref{def:MCkripke}]
Let $\Ku$ be a finite Kripke structure and
$\psi$ be an $\HS$ formula.
A trace $\rho\in\Trk_\Ku$ satisfies $\psi$ under the state-based semantic variant,
denoted as $\Ku,\rho\models_\stat \psi$, if and only if $\mathpzc{A}_\Ku,\rho\models \psi$
(Definition~\ref{def:satisfaction}).

Moreover,
$\Ku$ is a model of $\psi$ under the state-based semantic variant, denoted as $\Ku\models_\stat \psi$, if and only if,
for all \emph{initial} traces $\rho\in\Trk_\Ku$, it holds that $\Ku,\rho\models_\stat \psi$.
\end{definition}

\paragraph*{Computation-tree-based semantic variant $\HS_\LinearPast$.}%
We now describe the com\-putation-tree-based semantic variant: to the aim, we consider the abstract interval model \emph{induced by the computation tree} of a Kripke structure (this will be formally defined in the following), and basically  proceed as in the previous case. 

We start by introducing the notion of \emph{$D$-tree structure}, namely, an infinite tree-shaped Kripke structure with branches over a set $D$ of directions.

\begin{definition}[$D$-tree structure] Given a set $D$ of directions,
a \emph{$D$-tree structure} over a set of proposition letters $\Prop$ is a Kripke structure $\Ku=\KuDef$
such that $\sinit\in D$, $\States$ is a prefix closed subset of $D^{+}$, and $\Edges$ is the set of pairs $(s,s')\in \States\times \States$ such that there exists $d\in D$
for which $s'=s\cdot d$ (note that $\Edges$ is completely specified by $\States$). The states of a $D$-tree structure are called \emph{nodes}.
\end{definition}

A Kripke structure $\Ku=\KuDef$ induces an $\States$-tree structure, called the \emph{computation tree of $\Ku$}, denoted by
$\mathpzc{C}(\Ku)$, which is obtained by unwinding $\Ku$ %starting 
from the initial state (note that the directions are the set of states of $\Ku$).
Formally, $\mathpzc{C}(\Ku)= (\Prop,\Trk_\Ku^{0}, \Edges',\Lab',\sinit)$, where the set of nodes is the set of initial traces of
$\Ku$---hereafter denoted as $\Trk_\Ku^0$---and, for all $\rho,\rho'\in \Trk_\Ku^{0}$, $\Lab'(\rho)=\Lab(\lst(\rho))$ and $(\rho,\rho')\in \Edges'$ if and only if
$\rho'=\rho\cdot s$ for some $s\in \States$.
See Figure~\ref{fig:unr} for an example.

\begin{figure}[tp]
    \centering
\begin{tikzpicture}[->,>=stealth,thick,shorten >=1pt,auto,node distance=1.6cm,every node/.style={ellipse,draw}]
			\node[double] (v0) {$s_0$};
			\node (v1) at (0,-1) {$s_0s_1$};
			\node (v01) at (-1,-2) {$s_0s_1s_0$};
			\node (v11) at (1,-2) {$s_0s_1s_1$};
			\node (v02) at (-1,-3) {$s_0s_1s_0s_1$};
			\node (v022) at (1,-3) {$s_0s_1s_1s_0$};
			\node (v12) at (3,-3) {$s_0s_1s_1s_1$};
% 			\node (v03) [below of=v02] {$s_1$};
% 			\node (v53) [left of=v03] {$s_0$};
% 			\node (v023) at (v03 -| v11) {$s_1$};
			
% 			\node (v92) [below of=v12] {$s_0$};
% 			\node (v93) [right of=v92] {$s_1$};
			
			
			\node (v30) [draw=none, below =0.2cm of v02] {\Large $\cdots$};
			\node (v31) [draw=none, below left =0.21cm and 0.09cm of v12] {\Large $\cdots$};
			
			
			\draw (v0) to (v1);
			\draw (v1) to (v01); \draw (v1) to (v11);
			\draw (v01) to (v02);
			\draw (v11) to (v022);
% 			\draw (v02) to (v03);
% 			\draw (v02) to (v53);
% 			\draw (v022) to (v023);
			\draw (v11) to (v12);
% 			\draw (v12) to (v92);
% 			\draw (v12) to (v93);
		\end{tikzpicture}
	\vspace{-0.3cm}
    \caption{Computation tree $\mathpzc{C}(\mathpzc{K})$ of a finite Kripke structure $\Ku=\KuDef$ with $\States=\{s_0,s_1\}$ and $\Edges=\{(s_0,s_1),(s_1,s_1),(s_1,s_0)\}$.}
    \label{fig:unr}
\end{figure}



Notice that since each state in a computation tree has a unique predecessor (with the exception of the initial state), this $\HS$ variant enforces a linear reference in the past.

\begin{definition}[Computation-tree-based $\HS$---$\HS_\LinearPast$]
A finite Kripke structure $\Ku$ is a model of an $\HS$ formula $\psi$ under the computation-tree-based semantic variant, denoted as $\Ku\models_\LinearPast \psi$, if and only if
$\mathpzc{C}(\Ku)\models_\stat \psi$.
%The \emph{computation-tree-based model checking problem} for $\HS$ over finite Kripke structures is
%checking whether $\Ku\models_\LinearPast \psi$.
\end{definition}

\paragraph*{Trace-based semantic variant $\HS_\LinearTime$.}
Finally, we introduce the trace-based semantic variant, which exploits the interval structures induced by the infinite paths of a Kripke structure, as defined in the following.

We recall that,
given a strict partial ordering $\mathbb{S}=(\mathpzc{S},<)$, an \emph{interval} in $\mathbb{S}$ is an ordered pair
$[x,y]$, such that $x,y\in \mathpzc{S}$ and $x\leq y$, which represents the subset of $\mathpzc{S}$ given by all points $z\in \mathpzc{S}$ such
that $x\leq z\leq y$. We denote by $\mathbb{I}(\mathbb{S})$ the set of intervals in $\mathbb{S}$.

The following notion has already been presented in Section~\ref{sec:preliminaries}, but here we use (for technical convenience) a different notation.
\begin{definition}[Interval structure]
An \emph{interval structure (or interval model)} $\mathpzc{IS}$ over  $\Prop$ is a pair $\mathpzc{IS}=(\mathbb{S},\sigma)$ such that $\mathbb{S}=(\mathpzc{S},<)$ is a strict partial ordering
and $\sigma: \mathbb{I}(\mathbb{S}) \to 2^\Prop$ is a labeling function assigning a set of proposition letters
to each interval in $\mathbb{S}$.
\end{definition}

The next definition shows how to derive an abstract interval model from an interval structure,
allowing us to interpret $\HS$ over interval structures.

\begin{definition}[Abstract interval model induced by an interval structure] 
An interval structure 
$\mathpzc{IS}=(\mathbb{S},\sigma)$ over $\Prop$, where $\mathbb{S}=(\mathpzc{S},<)$ is a strict partial ordering, induces the abstract interval model
\[
\mathpzc{A}_{\mathpzc{IS}}=(\Prop,\mathbb{I}(\mathbb{S}), B_{\mathbb{I}(\mathbb{S})},E_{\mathbb{I}(\mathbb{S})},\sigma),
\]
where
%\begin{compactitem}
 $[x,y] \, B_{\mathbb{I}(\mathbb{S})}\, [v,z]$ if and only if $x=v$ and $z<y$, and
 $[x,y] \, E_{\mathbb{I}(\mathbb{S})}\, [v,z]$ if and only if $y=z$ and $x<v$.
%\end{compactitem}
\end{definition}
%
Given an interval $I$ and an $\HS$ formula $\psi$, we write $\mathpzc{IS},I\models \psi$ to mean that $\mathpzc{A}_{\mathpzc{IS}},I\models \psi$.
%
We now interpret $\HS$ over the infinite paths of a Kripke structure by mapping them into interval structures.
\begin{definition}[Interval structure induced by an infinite path]\label{def:inducedPathIntervalStructure}
For a finite  Kripke structure $\Ku=\KuDef$ and an infinite path $\pi=\pi(0)\pi(1)\cdots$ of $\Ku$,
the \emph{interval structure induced by $\pi$} is
$\mathpzc{IS}_{\Ku,\pi}=((\Nat,<),\sigma)$, where
 for each interval $[i,j]$, we have  $\sigma([i,j])=\bigcap_{h=i}^{j}\Lab(\pi(h))$.
\end{definition}

\begin{definition}[Trace-based $\HS$---$\HS_\LinearTime$]
A finite Kripke structure $\Ku$ is a model of an $\HS$ formula $\psi$ under the trace-based semantic variant, denoted as $\Ku\models_\LinearTime \psi$, if and only if,
for each initial infinite path $\pi$  and for each initial interval $[0,i]$, it holds that  $\mathpzc{IS}_{\Ku,\pi},[0,i]\models \psi$. 
\end{definition}
