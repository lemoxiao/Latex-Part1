\section{Conclusions}
In this chapter we have compared $\HS$ MC to MC for point-based temporal logics as far as it concerns expressiveness (and succinctness). To this end, we have taken into consideration three semantic variants  of $\HS$, namely, $\HS_\stat$, $\HS_\LinearPast$, and $\HS_\LinearTime$, under the homogeneity assumption. We have investigated their expressiveness and contrasted them with the point-based temporal logics $\LTL$, $\CTL$, finitary $\CTLStar$, and $\CTLStar$.

The resulting picture is as follows: $\HS_\LinearTime$ and $\HS_\LinearPast$ turn out to be as expressive as $\LTL$ and finitary $\CTLStar$, respectively. Moreover, $\HS_\LinearTime$ is at least exponentially more succinct than $\LTL$.
$\HS_\stat$ is expressively incomparable with $\HS_\LinearTime$/$\LTL$, $\CTL$, and $\CTLStar$, but it is strictly more expressive than $\HS_\LinearPast$/finitary $\CTLStar$.
%
We believe it possible to fill the expressiveness gap between $\HS_\LinearPast$ and $\CTLStar$ by considering abstract interval models, induced by Kripke structures, featuring worlds also for infinite paths/intervals, and extending the semantics of $\HS$ modalities accordingly. %Such an extension will be investigated in future research.

It is finally worth noting that the decidability of the MC problem for (full) $\HS_\LinearPast$ and $\HS_\LinearTime$ immediately follows, as a byproduct, from the proved results. 