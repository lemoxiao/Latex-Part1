\subsection{An example: a vending machine}\label{subs:vendingMach}
In this section, we give an example highlighting the differences among the $\HS$ semantic variants $\HS_\stat$, $\HS_\LinearPast$, and $\HS_\LinearTime$.

\begin{figure}[tp]
\centering
\resizebox{\textwidth}{!}{
\begin{tikzpicture}[->,>=stealth',shorten >=1pt,auto,node distance=2.8cm,semithick]
  \tikzstyle{every state}=[inner sep=1pt, outer sep=0pt, minimum size = 40pt]

  \node[accepting,state] (A0)                    {$\stackrel{s_0}{p_\text{\$=0}}$};
  \node[state]         (B2) [below of=A0] {$\stackrel{s_2}{p_\text{\$=2}}$};
  \node[state]         (B1) [left of=B2] {$\stackrel{s_1}{p_\text{\$=1}}$};
  \node[state]         (B05) [right of=B2] {$\stackrel{s_3}{p_\text{\$=0.50}}$};
  \node[state]         (C1) [below of=B1] {$\stackrel{s_4}{p_\text{candy}}$};
  \node[state]         (C2) [below of=B2] {$\stackrel{s_5}{p_\text{hotdog}}$};
  \node[state]         (C05) [below of=B05] {$\stackrel{s_6}{p_\text{water}}$};
  \node[state]         (D) [below of=C2] {$\stackrel{s_7}{p_\text{change}}$};
  
  \node[state]         (M1) [right of=C05] {$\stackrel{s_8}{p_\text{maint}}$};
  \node[state]         (M2) [right of=B05] {$\stackrel{s_9}{p_\text{maint\_end}}$};
  
  \path (A0) edge  [swap]  node {ins\_\$2} (B2)
            edge  [bend right,swap]  node {ins\_\$1} (B1)
            edge  [bend left,swap]  node {ins\_\$0.50} (B05)
        (B2) edge  [near start]  node {sel} (C2)
             edge  [near end]  node {sel} (C1)
             edge   node {sel} (C05)
        (B1) edge   node {sel} (C1)
             edge  [very near start]  node {sel} (C05)
        (B05) edge  [near end]  node {sel} (C05)
        (C1) edge   [bend right,very near start]  node {dispensed} (D)
        (C2) edge  [swap]  node {dispensed} (D)
        (C05) edge  [bend left,swap]  node {dispensed} (D)
        (D) edge   [bend right,swap]  node {change\_given} (M1)
        (D) edge   [out=200,in=160,looseness=1.7]  node {change\_given} (A0)
        (M1) edge  [near end]  node {maint\_ongoing} (M2)
        (M2) edge [bend left] node {maint\_failed} (M1)
        (M2) edge   [bend right,swap]  node {maint\_success} (A0);
        
\draw (-3.6,1) [dashed] rectangle (3.6,-9.5);
\draw (4.5,-1.5) [dashed] rectangle (6.8,-7);
\node (P1) at (4.3,1) {$p_\text{operative}$};
\node (P2) at (7,-1.3) {$\neg p_\text{operative}$};
\end{tikzpicture}
}
\vspace{-1cm}
\caption{Kripke structure representing a vending machine.}\label{fig:vending}
\end{figure}

The Kripke structure of Figure~\ref{fig:vending} represents a \emph{vending machine}, which can dispense water, hot dogs, and candies.
In state $s_0$ (the initial one), no coin has been inserted into the machine (hence, the proposition letter $p_\text{\$=0}$ holds there).
Three edges, labelled by \lq\lq ins\_\$1\rq\rq, \lq\lq ins\_\$2\rq\rq, and \lq\lq ins\_\$0.50\rq\rq{}, connect $s_0$ to $s_1$, $s_2$, and $s_3$, respectively. 
Edge labels do not convey semantic value (they are neither part of the structure definition nor associated with proposition letters) and are simply used for an easy reference to edges. 
In $s_1$ (resp., $s_2$, $s_3$) the proposition letter $p_\text{\$=1}$ (resp., $p_\text{\$=2}$, $p_\text{\$=0.50}$) holds, representing the fact that 1 Dollar (resp., 2, 0.50 Dollars) has been inserted into the machine.
The cost of a bottle of water (resp., a candy, a hot dog) is \$0.50 (resp., \$1, \$2). A state $s_i$, for $i=1,2,3$, is connected to a state $s_j$, for $j=4,5,6$, only if the available credit allows one to buy the corresponding item.
Then, edges labelled by \lq\lq dispensed\rq\rq{} connect $s_4,s_5$, and $s_6$ to $s_7$. In $s_7$, the machine gives change, and  can nondeterministically move back to $s_0$ (ready for dispensing another item), or to $s_8$, where it begins an automatic maintenance activity ($p_\text{maint}$ holds there). Afterwards, state $s_9$ is reached, where maintenance ends. From there, if the maintenance activity fails (edge \lq\lq maint\_failed\rq\rq ), $s_8$ is reached again (another maintenance cycle is attempted); otherwise, maintenance concludes successfully (\lq\lq maint\_success\rq\rq ) and $s_0$ is reached. Since the machine is operating in states $s_0,\ldots,s_7$, and under maintenance in $s_8$ and $s_9$, $p_\text{operative}$ holds over the former, and it does not on the latter.

In the following, we will make use of the $\B$ formulas $\Length_{n}$, with $n\geq 1$, presented in Example~\ref{example:length}.

We now give some examples of properties we can formalize under all, or some, of the $\HS$ semantic variants $\HS_\stat$, $\HS_\LinearPast$, and $\HS_\LinearTime$.

\begin{itemize}
    \item In any run of length 50, during which the machine never enters maintenance mode, it dispenses at least a hotdog, a bottle of water and a candy. 
    \begin{multline*}
        \Ku\not\models (p_\text{operative} \wedge \Length_{50})\longrightarrow\\ \big((\hsB\hsE p_\text{hotdog}) \wedge  (\hsB\hsE p_\text{water}) \wedge (\hsB\hsE p_\text{candy}) \big)
    \end{multline*}
    Clearly this property is false, as the machine can possibly dispense only one or two kinds of items.
    We start by observing that the above formula is equivalent in all of the three semantic variants of $\HS$: since modalities $\hsB$ and $\hsE$ only allow one to \lq\lq move\rq\rq{} from an interval to its subintervals, $\B\E_\LinearTime$, $\B\E_\stat$, and $\B\E_\LinearPast$ coincide (for this reason, we have omitted the subscript from the symbol $\models$). 
    %
    Homogeneity plays a fundamental role here: asking $p_\text{operative}$ to be true implies that such a letter is true along the whole trace (thus $s_8$ and $s_9$ are always avoided).
    
    It is worth observing that the same property can be expressed in $\LTL$, for instance as follows:
    \[
    \smashoperator{\bigwedge_{i\in\{0,\ldots,49\}}}\Next^i p_\text{operative} \wedge\smashoperator[r]{\bigvee_{i,j,k\in\{1,\ldots,48\}, i\neq j\neq k\neq i}} (\Next^i p_\text{hotdog}) \wedge (\Next^j p_\text{water}) \wedge (\Next^k p_\text{candy}).
    \]
    The length of this $\LTL$ formula is \emph{exponential} in the number of items (in this case, 3), whereas the length of the above $\HS$ one is only linear. As a matter of fact, we will prove (Theorem~\ref{theo:succinctnessHSlin}) that $\B\E$ is at least exponentially more succinct than $\LTL$.
    
    \item If the credit is \$0.50, then no hot dog or candy may be provided.
    \[
        \Ku\models (\hsE p_\text{\$=0.50})\longrightarrow \neg \hsA (\Length_{2} \wedge \hsE (p_\text{hotdog} \vee p_\text{candy})) 
    \]
    We observe that a trace satisfies $\hsE p_\text{\$=0.50}$ if and only if it ends in $s_3$.
    This property is satisfied under all of the three semantic variants, even though the nature of future differs among them (recall Figure~\ref{fig:ST}, \ref{fig:CT}, and \ref{fig:LN}). As we have already mentioned, a linear setting (rather than branching) is suitable for the specification  of dynamic behaviors, because it considers states \emph{of a computation}; conversely, a branching approach focuses on machine states (and thus on the structure of a system).
    
    In this case, only the state $s_6$ can be reached from $s_3$, regardless of the nature of future. For this reason, $\HS_\stat$, $\HS_\LinearPast$, and $\HS_\LinearTime$ behave in the same way. 
    
    \item Let us exemplify now a difference between $\HS_\stat$ (and $\HS_\LinearPast$) and $\HS_\LinearTime$.
    \[
        \begin{array}{l}
            \Ku\models_\stat \\
            \Ku\models_\LinearPast  \\
            \Ku\not\models_\LinearTime
        \end{array}
        (\hsE p_\text{maint\_end})\longrightarrow \hsA\hsE p_\text{operative}
    \]
    This is a structural property, requiring that when the machine enters state $s_9$ (where maintenance ends), it can become again operative reaching state $s_0$ ($s_9$ is not a lock state for the system). This is clearly true when future is branching and it is not when future is linear: $\HS_\LinearTime$ refers to system computations, and some of these may ultimately loop between $s_8$ and $s_9$.
    
    \item 
    Conversely, some properties make sense only if they are predicated over computations. This is the case, for instance, of fairness.
    \[
        \begin{array}{l}
            \Ku\models_\stat \\
            \Ku\models_\LinearPast  \\
            \Ku\not\models_\LinearTime
        \end{array}
        (\hsAu\hsA\hsE p_\text{maint})\longrightarrow \hsAu\hsA\hsE p_\text{operative}
    \]
    Assuming the trace-based semantics, the property requires that if a system computation enters infinitely often into maintenance mode, it will infinitely often enter operation mode.
    Again, this is not true, as some system computations may ultimately loop between $s_8$ and $s_9$ (hence, they are not fair). On the contrary, such a property is trivially true under $\HS_\stat$ or $\HS_\LinearPast$, as, for any initial trace $\rho$, it holds that $\Ku,\rho\models \hsA\hsE p_\text{operative}$.
    
    \item We conclude with a property showing the difference between linear and branching \emph{past}, that is, between $\HS_\stat$ and $\HS_\LinearTime$ (and $\HS_\LinearPast$).
    The requirement is the following: the machine may dispense water with any amount of (positive) credit.
    \[
        \begin{array}{l}
            \Ku\models_\stat \\
            \Ku\not\models_\LinearPast  \\
            \Ku\not\models_\LinearTime
        \end{array}
         (\hsE p_\text{water})\longrightarrow \hsE\big(p_\text{water}\wedge\smashoperator[r]{\bigwedge_{p\in\{p_\text{\$=2},p_\text{\$=1},p_\text{\$=0.50}\}}}\hsAt(\Length_{2} \wedge\hsB p)\big)
    \]
    Again, this one is a structural property, that cannot be expressed in $\HS_\LinearTime$ or $\HS_\LinearPast$, as these refer to a specific computation in the past. Conversely, it is true under $\HS_\stat$, since $s_6$ is backward reachable in one step by $s_1$, $s_2$, and $s_3$.
\end{itemize}