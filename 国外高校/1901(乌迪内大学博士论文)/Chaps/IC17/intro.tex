\lettrine[lines=3]{I}{n this chapter,} we study the MC problem for some of the sub-fragments of $\AAbarBBbar$ and of $\AAbarEEbar$---removing $\Bbar$ and $\Ebar$ which lead to $\Psp$-hardness---namely, $\A$, $\Abar$, $\AAbar$, $\AB$, $\AbarB$, $\AE$, $\AbarE$, $\AAbarB$, and  $\AAbarE$.
%What they have in common is the 
All these have a similar computational complexity, as their MC settles in one of the lowest levels of the \emph{polynomial-time hierarchy}, $\PTIME^{\NP}$, or even below. 
We recall that $\PTIME^{\NP}$ is the set of problems decided by a deterministic polynomial-time Turing machine, endowed with an oracle for the class $\NP$
which decides, in one computation step, whether an instance of a problem belonging to $\NP$ is positive or not. 
$\PTIME^{\NP}$ is also referred to as $\PTIME$ \emph{relative to} $\NP$.

In order to solve their MC, $\AAbarB$, $\AAbarE$, $\AB$, and $\AbarE$ require the $\PTIME$ Turing machine to perform $O(n^k)$
queries to the $\NP$ oracle, for an input of size $n$ and for some constant $k\geq 0$.
Conversely, $\A$, $\Abar$, $\AAbar$, $\AbarB$, and $\AE$ are \lq\lq easier\rq\rq{}
since, for them, $O(\log^2 n)$ queries are always enough:
the MC problem for the latter fragments witnesses
a \lq\lq non-standard\rq\rq{} complexity class in the polynomial-time hierarchy, called \emph{bounded-query} class and denoted as $\Thsq$, that will be presented in more detail below.

More formally, we first prove that MC for $\AB$, $\AbarE$, $\AAbarB$, and $\AAbarE$ is a $\PTIME^{\NP}$-complete problem.
To this end, we design a $\PTIME^{\NP}$ algorithm exploiting the polynomial small-model property of Section~\ref{sec:AAbarEEbar} and we prove a matching complexity lower bound.%
%\footnote{In this paper, we use $\LOGSPACE$ many-one reductions for the proofs of hardness.} 
%
Next, we devise another MC algorithm for all the remaining fragments, $\A$, $\Abar$, $\AAbar$, $\AbarB$, and $\AE$, via a reduction \emph{to} the problem \TBSATM~\cite{schnoebelen2003} (a problem complete for the aforementioned bounded-query class), whose instances are complex circuits in which some of the gates feature $\NP$ oracles.
Finally, we prove a lower bound showing that at least $\log n$ queries are needed to solve MC; unfortunately, it does not match the upper bound, leaving open the question whether the problem can be solved by $o(\log^2 n)$ (i.e., strictly less than $O(\log^2 n)$) queries to an $\NP$ oracle, or a tighter lower bound can be proved (or both).
%
Figure~\ref{fig:lattice} depicts the scenario of complexity of MC for $\AAbarBBbar$ and all its sub-fragments.
\begin{sidewaysfigure}
    \centering
    \begin{tikzpicture}
\tikzset{
    every node/.style={
        draw
    },
    every edge/.style={
        draw, gray
    }
}

\tikzstyle{coNPstyle}=[draw,densely dashed,circle]
\tikzstyle{PNPlogstyle}=[draw,densely dotted,thick,rectangle, minimum width=1.2cm]
\tikzstyle{PNPstyle}=[draw,regular polygon,regular polygon sides=9,inner sep=0.5pt]
\tikzstyle{PSPstyle}=[draw,dash dot dot,thick,ellipse, minimum width=1.2cm]


\node [style=coNPstyle] (v2) at (-3.5,-3) {$\B$};
\node [style=PNPlogstyle] (v3) at (-0.5,-3) {$\A$};
\node [style=PNPlogstyle] (v4) at (2.5,-3) {$\Abar$};
\node [style=PSPstyle] (v5) at (5.5,-3) {$\Bbar$};

\node [style=PNPstyle](v6) at (-6.5,-1.5) {$\AB$};
\node [style=PNPlogstyle](v9) at (-3.5,-1.5) {$\AAbar$};
\node [style=PNPlogstyle] (v7) at (-0.5,-1.5) {$\AbarB$};
\node [style=PSPstyle] (v10) at (2.5,-1.5) {$\ABbar$};
\node [style=PSPstyle] (v11) at (5.5,-1.5) {$\mathsf{\overline{AB}}$};
\node [style=PSPstyle] (v8) at (8.5,-1.5) {$\mathsf{B\overline{B}}$};

\node [style=PNPstyle] (v12) at (-3.5,0) {$\AAbarB$};
\node [style=PSPstyle] (v13) at (-0.5,0) {$\ABBbar$};
\node [style=PSPstyle] (v14) at (2.5,0) {$\mathsf{A\overline{AB}}$};
\node [style=PSPstyle] (v15) at (5.5,0) {$\mathsf{\overline{A}B\overline{B}}$};


\node [style=PSPstyle] (v16) at (1,1.5) {$\AAbarBBbar$};
\node [style=coNPstyle] (v1) at (1,-4.5) {$\HSprop$};

\draw  (v1) edge (v2);
\draw  (v1) edge (v3);
\draw  (v1) edge (v4);
\draw  (v1) edge (v5);
\draw  (v2) edge (v6);
\draw  (v2) edge (v7);
\draw  (v2) edge (v8);
\draw  (v3) edge (v6);
\draw  (v3) edge (v9);

\draw  (v3) edge (v10);
\draw  (v4) edge (v9);
\draw  (v4) edge (v7);
\draw  (v4) edge (v11);
\draw  (v5) edge (v8);
\draw  (v5) edge (v11);
\draw  (v5) edge (v10);
\draw  (v6) edge (v12);
\draw  (v6) edge (v13);
\draw  (v9) edge (v12);
\draw  (v9) edge (v14);
\draw  (v7) edge (v12);
\draw  (v7) edge (v15);
\draw  (v10) edge (v13);
\draw  (v10) edge (v14);
\draw  (v11) edge (v14);
\draw  (v11) edge (v15);
\draw  (v12) edge (v16);
\draw  (v13) edge (v16);
\draw  (v14) edge (v16);
\draw  (v15) edge (v16);

\draw  (v8) edge (v15);
\draw  (v8) edge (v13);


\node [style=coNPstyle] at (-6.5,-5.5) {\scriptsize $\co\NP$};
\node [style=PNPlogstyle] at (-4.4,-5.5) {\scriptsize $\Thsq$};
\node [style=PNPstyle] at (-2.5,-5.5) {\scriptsize $\PTIME^{\NP}$};
\node [style=PSPstyle] at (-0.5,-5.5) {\scriptsize $\PSPACE$};

\end{tikzpicture}
    \caption{The computational complexity of MC for the sub-fragments of $\AAbarBBbar$.}
    \label{fig:lattice}
\end{sidewaysfigure}

Before moving on, we want to intuitively explain why, for instance, $\AbarB$ is \lq\lq easier\rq\rq{} than $\AB$ (the same holds for the symmetric fragments $\AE$ and $\AbarE$); this is justified by the different expressiveness of the two fragments.
%To have an intuition about the difference in the complexity of the two fragments, 
Let us consider an $\A\B$ formula of the form $\hsB \hsA \theta$. A trace $\rho$ satisfies $\hsB \hsA \theta$ if there exists a prefix $\tilde{\rho}$ of $\rho$ from which a branch, i.e., a trace starting from $\lst(\tilde{\rho})$, satisfying $\theta$ departs. 
%This amounts to say that 
Hence, $\A\B$ allows one to state specific properties of the branches departing from a state occurring in a given path. Such an ability will be exploited in Section~\ref{sec:ABhard} to prove the $\PTIME^{\NP}$-hardness of $\AB$.
%
This kind of properties cannot be expressed in the fragment $\AbarB$. Indeed, for any given trace $\rho$, modality $\hsAt$ only allows one to \lq\lq select\rq\rq{} traces leading to the first state of $\rho$, and modality $\hsB$ is of no help: if we consider any prefix $\tilde{\rho}$ of $\rho$, the set of traces leading to its first state is exactly the same as the set of those leading to the first state of $\rho$, as $\fst(\tilde{\rho}) = \fst(\rho)$. Therefore, pairing $\hsAt$ and $\hsB$ does not give any advantage in terms of expressiveness. 
%Such a weakness of $\AbarB$ represents the reason why $\AbarB$ formulas can be checked in time $\Thsq$, instead of $\PTIME^{\NP}$.
%
Finally, since $\A$, $\Abar$, and $\AAbar$ are devoid of modalities for prefixes (and suffixes), they analogously belong to $\Thsq$.

\paragraph*{Organization of the chapter.}
\begin{itemize}
	\item In the next section, we start by recalling some complexity classes that come into play.
	\item In Section~\ref{sec:AAbarBalgo}, we describe a $\PTIME^{\NP}$ MC algorithm for $\AAbarB$ (and $\AAbarE$). 
	\item In Section~\ref{sect:AAbarAlg}, we provide a different MC algorithm for the fragments $\AAbar$, $\AbarB$, and $\AE$, whose computational complexity is lower, 
as it requires only $O(\log^2 n)$ queries to a $\NP$ oracle.
	\item In Section~\ref{sec:ABhard}, we prove a $\PTIME^{\NP}$ lower bound for $\AB$ (resp., $\AbarE$) MC. The bound immediately propagates to $\AAbarB$ (resp., $\AAbarE$), closing the complexity gap and proving that the MC problem for  $\AB$, $\AbarE$, $\AAbarB$, and $\AAbarE$ is $\PTIME^{\NP}$-complete. 
	\item In Section~\ref{sect:AHard}, we prove that MC for $\A$ and $\Abar$ formulas requires at least $\log n$ queries to a $\NP$ oracle: this bound propagates to $\AAbar$, $\AbarB$, and $\AE$.
\end{itemize}
