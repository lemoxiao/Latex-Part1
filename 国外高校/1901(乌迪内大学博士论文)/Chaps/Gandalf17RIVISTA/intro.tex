\lettrine[lines=3]{A}{\ good balancing} of \emph{expressiveness} and \emph{complexity} in the choice 
of the system 
model and of the specification language is a key factor for the effective 
exploitation of MC. 
Various improvements to both of them have been proposed in the literature. As 
for the latter we recall in particular the extensions of $\LTL$ with 
promptness, that make it 
possible to bound the delay with which a liveness request is fulfilled (see, 
e.g.,~\cite{DBLP:journals/fmsd/KupfermanPV09}). 
Another possible direction is
\emph{adding regular expressions}, that allow one to enrich the expressive 
power of existing logics. This has been investigated, for instance, in the 
cases of $\LTL$~\cite{Leucker2007} and $\CTL$~\cite{MATEESCU20112854}.
%
In this chapter, we study the MC problem for \emph{$\HS$ extended with regular 
expressions}, which are exploited as a means for \emph{relaxing the homogeneity 
assumption}, that otherwise limits the expressive power of $\HS$. 

As we have already said (Section~\ref{sec:LOMrelated}), in~\cite{lm16} %the first meaningful attempt to relax the homogeneity assumption can be found, where 
Lomuscio and Michaliszyn propose to use regular expressions to define the 
labeling of proposition letters over intervals in terms of the component 
states---thus relaxing homogeneity, that can be trivially \lq\lq 
encoded\rq\rq{} by regular expressions, as shown later. In that work the 
authors prove decidability of MC with regular expressions for some very 
restricted fragments of epistemic $\HS$, giving some rough upper bounds to its 
computational complexity
(see the fourth column of Table~\ref{fig:overv}).
%
In this chapter, 
we define interval labeling via regular expressions in a way that 
can be shown to be equivalent to that of~\cite{lm16}, and
we give a detailed picture of decidability and complexity for $\HS$ with 
regular expressions, which was still missing. The results are summarized in the 
third column of Table~\ref{fig:overv}. 

\begin{table}
	\centering
	\caption{Complexity of MC for $\HS$ and its fragments ($^\dagger$local MC). 
	In red, the new results proved in this chapter.}\label{fig:overv}
	\resizebox{\textwidth}{!}{
		\begingroup
\renewcommand*{\arraystretch}{1.3}
%
\begin{tabular}{|@{\ }c@{\ }|@{\ }c@{\ }|@{\ }c@{\ }||@{\ }c@{\ }|}
\hline 
 & Homogeneity & Regular expressions & Endpoints~\cite{LM13,LM14,lm16}\\
\hline \hline 
\multirow{2}{*}{Full $\HS$, $\B\E$} & nonelementary  & \textcolor{red}{nonelementary}  & $\B\E$+$\epist^\dagger$: $\Psp$ \\
%\cline{2-4} 
 & $\EXPSPACE$-hard & $\EXPSPACE$-hard & $\B\E^\dagger$: $\PTIME$\\
\hline 
\multirow{2}{*}{$\A\Abar\B\Bbar\Ebar,\A\Abar\E\Bbar\Ebar$} & $\in\EXPSPACE$ \textcolor{red}{$\in\LINAEXPTIME$}& nonelem.\  $\Psp$-hard &\\
%\cline{2-4} 
 & $\Psp$-hard & \textcolor{red}{$\LINAEXPTIME$-complete}& \\
\hline 
\multirow{2}{*}{$\AAbar\Bbar\Ebar$} & \multirow{2}{*}{$\Psp$-complete} & \textcolor{red}{$\in\LINAEXPTIME$} & \\
 & & $\Psp$-hard & \\
\hline
$\AAbarBBbar,\B\Bbar,\Bbar,$ & \multirow{2}{*}{$\Psp$-complete} & \multirow{2}{*}{\textcolor{red}{$\Psp$-complete}} & \multirow{2}{*}{$\A\Bbar$+$\epist$: nonelementary}\\
$\AAbarEEbar,\E\Ebar,\Ebar$ & & & \\
\hline 
$\AAbar\B,\AAbar\E,\A\B,\Abar\E$ & $\PTIME^{\NP}\!$-complete & \textcolor{red}{$\Psp$-complete} & \\
\hline 
\multirow{2}{*}{$\A\Abar,\Abar\B,\A\E,\A,\Abar$} & $\in\Thsq$ & \multirow{2}{*}{\textcolor{red}{$\Psp$-complete}} & \\
%\cline{2-2} \cline{4-4} 
 & $\Th$-hard &  & \\
\hline 
$\HSprop, \B,\E$ & $\co\NP$-complete & \textcolor{red}{$\Psp$-complete} & \\
\hline 
\end{tabular}

\endgroup
	}
	%\vspace*{-0.2cm}
\end{table}

It is interesting to compare the complexity of MC for $\HS$ fragments extended with regular expressions with the same fragments under the homogeneity assumption. The rich spectrum of complexities for the less expressive fragments of $\HS$ under homogeneity (last four rows in the table) collapses to $\Psp$-completeness in the case of the corresponding fragments with regular expressions, witnessing that using regular expressions increases the expressive power of (syntactically) small fragments of $\HS$. Whether or not there exists an elementary algorithm for full $\HS$ remains an open issue, just like in the case of full $\HS$ under homogeneity. The main results of the chapter are summarized in the following account of the contents of the next sections.

\paragraph*{Organization of the chapter.}
\begin{itemize}

\item
In Section~\ref{sect:backgrRegex}, we start by recalling the notions of regular expression and finite-state automaton, and then give syntax and semantics of $\HS$ with regular expressions.

\item
In Section \ref{sect:fullHS}, we prove that MC for (full) $\HS$ extended with regular expressions (under the state-based semantics) is decidable,
by exploiting an automata-theoretic approach and the notion of $\Ku$-\NFA, a particular version of finite-state automaton. 
Moreover, the problem can be shown to be in $\PTIME$ when it is restricted to 
system models, assuming the formula to be of constant length. 

\item
Then, in Section~\ref{sec:AAbarBBbarEbarRegex},
we study the problems of MC for the two (syntactically) maximal (symmetric) 
fragments $\A\Abar\B\Bbar\Ebar$ and $\A\Abar\E\Bbar\Ebar$ with regular 
expressions, proving that both problems are $\LINAEXPTIME$-complete. 
$\LINAEXPTIME$ denotes the complexity class of problems decided by   
\emph{exponential-time bounded} alternating Turing machines making a 
\emph{polynomially 
bounded number of alternations}; such a class captures the exact complexity of 
some relevant problems~\cite{tcs15l,FR75}, such as, for instance, the 
first-order 
theory of real addition with order~\cite{FR75}.
First, we recall (Chapter~\ref{chap:TCS17}) that settling the exact complexity 
of these fragments under the homogeneity assumption is a difficult open 
question. Moreover,  considering that $\LINAEXPTIME \subseteq \AEXP = \EXPSPACE$ and 
that $\HS$ under homogeneity  is subsumed by $\HS$ with regular expressions (as 
we will see later), 
the results proved in this chapter improve  the complexity upper bounds for the 
fragments  
$\A\Abar\B\Bbar\Ebar$ and $\A\Abar\E\Bbar\Ebar$ given in 
Section~\ref{sec:AAbarBBbarEbar}. 
%
More in detail, we preliminarily establish an exponential small-model property 
for $\A\Abar\B\Bbar\Ebar$ (Section~\ref{sec:AAbarBBbarEbarTrackProperty}): for 
each interval (trace),  it is possible to find an interval of bounded 
exponential 
length that is indistinguishable with respect to the fulfillment of 
$\A\Abar\B\Bbar\Ebar$ formulas (respectively, $\A\Abar\E\Bbar\Ebar$ formulas).
Such a property allows us to devise an MC procedure belonging to the class $\LINAEXPTIME$ (Section~\ref{sec:UpperBound}). 
Finally, the %hardness of the problem 
matching lower bounds are obtained % proved for the fragment $\B\Ebar$ of $\A\Abar\B\Bbar\Ebar$ 
by polynomial-time reductions from the so-called \emph{alternating multi-tiling 
problem}, showing that they already hold for the fragments $\B\Ebar$ and 
$\E\Bbar$ of $\A\Abar\B\Bbar\Ebar$ and $\A\Abar\E\Bbar\Ebar$, respectively 
(Section~\ref{sec:LowerBound}). 

\item
Finally, in Section~\ref{sec:AABB}, we show that 
formulas of $\HS$ fragments featuring (any subset of) $\HS$ modalities for the Allen's relations \emph{meets, met-by, started-by}, and \emph{starts} ($\AAbarBBbar$) can be checked in polynomial working space (MC for all these is $\Psp$-complete). 
%
In particular, in Section~\ref{subsec:AAbarEEbar} we prove a small-model 
theorem for $\AAbarBBbar$ (and the symmetric $\AAbarEEbar$) with regular 
expressions, which is then 
exploited in Sections~\ref{sect:PspAlgo} and \ref{sect:genResult} to devise a 
$\Psp$ MC algorithm for $\AAbarBBbar$ (and $\AAbarEEbar$). Moreover, in 
Section~\ref{sect:genResult}, we prove that MC for the purely propositional 
fragment of $\HS$, denoted as $\HSprop$, is hard for $\Psp$, which is enough to 
conclude that MC for any sub-fragment of $\AAbarBBbar$ or $\AAbarEEbar$ is 
complete for $\Psp$.
%
Hence, relaxing the homogeneity assumption via regular expressions comes at no 
cost for $\AAbarBBbar$, $\AAbarEEbar$, $\B\Bbar$, $\E\Ebar$, $\Bbar$ and 
$\Ebar$---that remain in $\Psp$---while $\AAbarB$ and $\A\Abar\E$ and their 
sub-fragments
%$\B$, $\E$, $\AAbar$, $\AAbarB$, $\A\Abar\E$, \dots{} 
increase their complexity to $\Psp$ (see Table~\ref{fig:overv} once more). 
\end{itemize}