\chapter{Proofs and complements of Chapter~\ref{chap:TCS17}}
\minitoc\mtcskip

\section{Proof of Lemma~\ref{lemmamdc}}\label{proof:lemmamdc}

\begin{lemma*}[\ref{lemmamdc}]
Let $\psi$ be an $\AAbarEEbar$ formula, $\Ku$ be a finite Kripke structure, and $\sigma\in\Trk_\Ku$. Then, $\texttt{Check}(\Ku, \psi,\sigma)=1$ if and only if $\Ku,\sigma\models \psi$.
\end{lemma*}

\begin{proof}
The proof is by induction on the structure of $\psi$.
 The base case where $\psi=p$, for some $p\in\Prop$, directly follows from the definition (line 2 of Algorithm~\ref{Chk2}).
  The cases in which $\psi=\neg\varphi$ and $\psi=\varphi_1\wedge\varphi_2$ are also trivial and thus omitted. We focus on the remaining cases.
\begin{itemize}
    \item $\psi=\hsA\varphi$. If $\Ku,\sigma\models \psi$, then there exists a trace $\rho\in \Trk_\Ku$ such that $\lst(\sigma)=\fst(\rho)$ and $\Ku,\rho\models \varphi$.
    By Theorem~\ref{theorem:polynomialSizeModelProperty}, there exists a trace $\pi\in\Trk_\Ku$, with $|\pi|\leq |\States|\cdot(|\varphi'|+1)^2$ and $\fst(\pi)=\fst(\rho)(=\lst(\sigma))$, such that $\Ku,\pi\models\varphi'$, where $\varphi'$ is the \nnf{} of $\varphi$. Thus, being $|\pi|\leq |\States|\cdot(2|\varphi|+1)^2$, such trace $\pi$ is considered in the for-loop at line 12. By the inductive hypothesis, $\texttt{Check}(\Ku,\varphi,\pi)=1$ and thus \texttt{Check}$(\Ku,\psi,\sigma)=1$.

    Vice versa, if \texttt{Check}$(\Ku,\psi,\sigma)=1$, then there exists a trace $\pi\in \Trk_\Ku$, with $\lst(\sigma)=\fst(\pi)$, such that \texttt{Check}$(\Ku,\varphi,\pi)=1$. By the inductive hypothesis, $\Ku,\pi\models \varphi$, hence $\Ku,\sigma\models \psi$.

    \item $\psi=\hsAt\varphi$ is analogous to the previous case.
    \item $\psi=\hsE\varphi$. If $\Ku,\sigma\models \psi$, there exists a trace $\pi\in\Suff(\sigma)$ such that $\Ku,\pi\models \varphi$. By the inductive hypothesis, \texttt{Check}$(\Ku,\varphi,\pi)=1$. Since all the proper suffixes of $\sigma$ are checked (line 17), \texttt{Check}$(\Ku,\psi,\sigma)=1$.

    Vice versa, if \texttt{Check}$(\Ku,\psi,\sigma)=1$, then for some $\pi\in\Suff(\sigma)$, it holds that \texttt{Check}$(\Ku,\varphi,\pi)=1$. By the inductive hypothesis $\Ku,\pi\models \varphi$ implying that $\Ku,\sigma\models \psi$.
    \item $\psi=\hsEt\varphi$. If $\Ku,\sigma\models \psi$, then there exists a trace $\rho\in \Trk_\Ku$, with $|\rho|\geq 2$, such that $\Ku,\rho\star\sigma\models \varphi$.
    By Theorem~\ref{theorem:polynomialSizeModelProperty}, there exists a trace $\pi\in\Trk_\Ku$ induced by $\rho$, with $|\pi|\leq |\States|\cdot(|\varphi'|+1)^2$, such that $\Ku,\pi\star\sigma\models\varphi'$, where $\varphi'$ is the \nnf{} of $\varphi$. Such trace $\pi$ is considered in the for-loop at line 22, since $|\pi|\leq |\States|\cdot(2|\varphi|+1)^2$ and $|\pi|\geq 2$ as it is induced by $\rho$. By the inductive hypothesis, \texttt{Check}$(\Ku,\varphi,\pi\star\sigma)=1$ implying that  \texttt{Check}$(\Ku,\psi,\sigma)=1$.

    Vice versa, if \texttt{Check}$(\Ku,\psi,\sigma)=1$, then there exists a trace $\pi \in \Trk_\Ku$, with $|\pi|\geq 2$, such that \texttt{Check}$(\Ku,\varphi,\pi\star\sigma)=1$. By the inductive hypothesis, $\Ku,\pi\star\sigma\models \varphi$, hence $\Ku,\sigma\models \psi$.\qedhere
\end{itemize}
\end{proof}


\section{Proof of Lemma~\ref{ThCorrComplMC}}\label{proof:ThCorrComplMC}

\begin{theorem*}[\ref{ThCorrComplMC}]
Let $\psi$ be an $\AAbarEEbar$ formula and $\Ku$ be a finite Kripke structure. Then, \texttt{ModCheck}$(\Ku,\psi)=1$ if and only if $\Ku\models \psi$.
\end{theorem*}

\begin{proof}
	($\Leftarrow$)
	If $\Ku\models \psi$ then, for all initial traces $\rho\in\Trk_\Ku$, we have that $\Ku,\rho\models \psi$.
	By Lemma~\ref{lemmamdc},
it follows that $\texttt{Check}(\Ku,\psi,\rho)=1$. Now, since the for-loop at line 1 considers a subset of the initial traces, it holds that \texttt{ModCheck}$(\Ku,\psi)=1$.

    ($\Rightarrow$)
	If \texttt{ModCheck}$(\Ku,\psi)=1$, then, for any initial trace $\rho$ considered by the for-loop at line 1, that is, with $|\rho|\leq |\States|\cdot (2|\psi|+3)^2$, it holds that $\texttt{Check}(\Ku,\psi,\rho)=1$.
	%
	Let us assume by contradiction that $\Ku\not\models\psi$, that is, there exists an initial trace $\rho'\in\Trk_\Ku$ such that $\Ku,\rho'\models\neg\psi$, or, equivalently, $\Ku,\rho'\models\overline{\psi}$, where $\overline{\psi}$ is the \nnf{} of $\neg\psi$. Thus, by Theorem~\ref{theorem:polynomialSizeModelProperty}, there exists an initial trace $\sigma$, with $|\sigma|\leq |\States|\cdot (|\overline{\psi}|+1)^2\leq |\States|\cdot (2|\psi|+3)^2$, such that $\Ku,\sigma\models \overline{\psi}$, namely, $\Ku,\sigma\not\models\psi$. By Lemma~\ref{lemmamdc}, it holds that $\texttt{Check}(\Ku,\psi,\sigma)=0$, leading to a contradiction and proving that $\Ku\models \psi$.
\end{proof}


\section{$\Psp$-hardness of MC for $\Bbar$ and $\Ebar$}\label{sect:BbarHard}
In this section, we prove that the MC problem for formulas of $\Bbar$ and of $\Ebar$, over finite Kripke structures, is $\Psp$-hard by means of a reduction from 
the QBF problem, that is, the problem of determining the truth of a \emph{fully-quantified} Boolean formula in \emph{prenex normal form}, which is known to be $\Psp$-complete (see, for example, \cite{Sip12}). The proof for  $\Bbar$ can easily be modified to show the $\Psp$-hardness of the symmetric fragment $\Ebar$.
%---to the model checking problem for $\ABbar$ formulas over finite Kripke structures. 

Let $\psi=Q_n x_n Q_{n-1} x_{n-1}$ $\cdots Q_1 x_1 \phi(x_n,x_{n-1},\ldots ,x_1)$ be a quantified Boolean formula where, for $i=1,\ldots ,n,$ $Q_i\in \{\exists, \forall\}$ and $\phi(x_n,x_{n-1},\ldots ,x_1)$ is a quantifier-free Boolean formula over the set of variables $Var = \{x_n,\ldots ,x_1\}$. We define a Kripke structure $\Ku_{QBF}^{Var}$, whose initial traces represent all the possible assignments to the variables of $Var$. For each variable $x \in Var$, $\Ku_{QBF}^{Var}$ features a pair of states $s_x^{\top}$ and $s_x^{\bot}$, that represent a $\top$ and $\bot$ truth assignment to $x$, respectively. An example of $\Ku_{QBF}^{Var}$, with $Var=\{x,y,z\}$, is given in Figure~\ref{Kqbf}.

Formally, let $\Ku_{QBF}^{Var}=\KuDef$, where:
\begin{itemize}
    \item $\Prop= Var \cup \{t\} \cup \{ \tilde{x_i} \mid 1\leq i\leq n\}$;
    \item $\States= \{s_{x_i}^\ell \mid 1\leq i\leq n,\; \ell \in \{\bot,\top\}\} \cup \{s_0,sink\}$;
    \item if $n=0$, $\Edges=\{(s_0,sink),(sink,sink)\}$;\newline
            if $n>0$,  
            $\Edges = \{(s_0,s_{x_n}^{\top}),(s_0,s_{x_n}^{\bot})\}\cup 
            \{(s_{x_i}^\ell,s_{x_{i-1}}^m) \mid \ell \, , m \in \{\bot,\top\},\: 2 \leq i \leq n\} \cup 
            \{(s_{x_1}^{\top},sink),(s_{x_1}^{\bot},sink),(sink,sink) \}$. 
    \item $\Lab(s_0) = Var \cup \{t\}\cup \{ \tilde{x_i} \mid x_i \in Var\}$; \newline
            for all $1\leq i\leq n$, $\Lab(s_{x_i}^\top) = Var \cup \{\tilde{x_j} \mid 1\leq j\leq i\}$
            and $\Lab(s_{x_i}^\bot) = (Var \setminus \{x_i\}) \cup \{\tilde{x_j} \mid 1\leq j\leq i\}$; \newline
            $\Lab(sink) = Var$.
\end{itemize}

\begin{figure}[tb]
\centering
%\resizebox{\linewidth}{!}{
\begin{tikzpicture}[->,>=stealth',shorten >=1pt,node distance=2.2cm,semithick,every node/.style={circle,draw,inner sep=-2pt,outer sep=0},every loop/.style={max distance=8mm}]  
    \node (0) [double] {$\begin{array}{c} s_0 \\ x,y,z \\ t,\tilde{x},\tilde{y},\tilde{z} \end{array}$};
    \node (1a) [above right of= 0] {$\begin{array}{c} s_x^{\top} \\ x,y,z \\ \tilde{x},\tilde{y},\tilde{z} \end{array}$};%below right=0.5cm and 3.7cm
    \node (1b) [below right of= 0] {$\begin{array}{c} s_x^{\bot} \\ y,z \\ \tilde{x},\tilde{y},\tilde{z} \end{array}$};
    
    \node (2a) [right of=1a] {$\begin{array}{c} s_y^{\top} \\ x,y,z \\ \tilde{y},\tilde{z} \end{array}$};
    \node (2b) [right of=1b] {$\begin{array}{c} s_y^{\bot} \\ x,z \\ \tilde{y},\tilde{z} \end{array}$};

    \node (3a) [right of=2a] {$\begin{array}{c} s_z^{\top} \\ x,y,z \\ \tilde{z} \end{array}$};
    \node (3b) [right of=2b] {$\begin{array}{c} s_z^{\bot} \\ x,y \\ \tilde{z} \end{array}$};
    
    \node (pit) [below right of=3a] {$\begin{array}{c} sink \\ x,y,z \\  \end{array}$};
  
  \path
    (0) edge (1a)
        edge (1b)
        
    (1a) edge (2a)
        edge (2b)
    (1b) edge (2a)
        edge (2b)

    (2a) edge (3a)
        edge (3b)
    (2b) edge (3a)
        edge (3b)

    (3a) edge (pit)
    (3b) edge (pit)
    (pit) edge [loop above] (pit)
    ;
\end{tikzpicture}%}
%\vspace{-0.6cm}
\caption{The finite Kripke structure $\Ku_{QBF}^{x,y,z}$.}\label{Kqbf}
\end{figure}

The QBF formula $\psi$ is reduced to the $\Bbar$ formula $\xi=t\rightarrow \xi_n$, where
\begin{equation*}
\xi_i=
\begin{cases}
\phi(x_n,x_{n-1},\ldots ,x_1) & \text{ if } i=0\\
\hsBt\big(\tilde{x_i} \wedge \xi_{i-1}\big) & \text{ if } i>0 \text{ and }  Q_i=\exists\\
\hsBtu\big(\tilde{x_i} \rightarrow \xi_{i-1}\big) & \text{ if } i>0 \text{ and }  Q_i=\forall
\end{cases}.
\end{equation*}
%Clearly, 
Note that both $\Ku_{QBF}^{Var}$ and $\xi$ can be built in polynomial time. 
%
In Theorem~\ref{th:ABbarHard}, we shall show the correctness of the reduction, i.e., that $\psi$
%=Q_n x_n Q_{n-1} x_{n-1} \cdots Q_1 x_1 \phi(x_n,x_{n-1},\cdots x_1)$ 
is true iff $\Ku_{QBF}^{Var}\models\xi$.
%
As a preliminary step, we introduce some technical definitions and an auxiliary lemma.

Given a Kripke structure $\Ku=\KuDef$ and a $\Bbar$ formula $\chi$, we denote by $p\ell(\chi)$ the set of proposition letters occurring in $\chi$ and by $\Ku_{\,|p\ell(\chi)}$ the 
%Kripke 
structure obtained from $\Ku$ by restricting the labelling of each state to $p\ell(\chi)$, namely, the 
Kripke 
structure $(\overline{\Prop},\States, \Edges,\overline{\Lab},s_0)$, where $\overline{\Prop}=\Prop\cap p\ell(\chi)$ and $\overline{\Lab}(s)=\Lab(s)\cap p\ell(\chi)$, for all $s\in \States$.

Moreover, for $v\in \States$, we denote by $reach(\Ku,v)$ the subgraph of $\Ku$ induced by the states reachable from $v$, namely, the 
Kripke 
structure $(\Prop,\States',\Edges',\Lab',v)$, where $\States'=\{s\in \States \mid \text{ there exists } \rho\in \Trk_\Ku \text{ with } \fst(\rho)=v \text{ and } \lst(\rho)=s\}$, $\Edges'=\Edges \cap (\States'\times \States')$, and $\Lab'(s)=\Lab(s)$, for all $s\in \States'$. %$reach(\Ku,v)$ is thus  Notice that the initial state of $reach(\Ku,v)$ is $v$. 

As usual, we say that two Kripke structures $\Ku=\KuDef$ and $\Ku'=(\Prop',\States', \Edges',\Lab',\sinit')$ are \emph{isomorphic} (written $\Ku\sim \Ku'$) if and only if there is a \emph{bijection} $f:\States\to \States'$ such that 
\begin{itemize}
    \item $f(s_0)=s_0'$;
    \item for all $u,v\in \States$, $(u,v)\in \Edges \iff (f(u),f(v))\in\Edges'$;
    \item for all $v\in \States$, $\Lab(v)=\Lab'(f(v))$.
\end{itemize}

Finally, if  $\mathpzc{A}_\Ku=(\Prop,\mathbb{I},A_\mathbb{I},B_\mathbb{I},E_\mathbb{I},\sigma)$ is the abstract interval model induced by a Kripke structure  $\Ku$ and $\rho\in\Trk_{\Ku}$,
 %is a trace, 
 we denote $\sigma(\rho)$ by $\mathpzc{L}(\Ku,\rho)$.

Let $\Ku$ and $\Ku'$ be two Kripke structures. The following lemma, which is an immediate consequence of Lemma~1 of \cite{MMP15B}, states that, for any $\Bbar$ formula $\psi$, if the same set of proposition letters, restricted to $p\ell(\psi)$, holds over two traces $\rho \in \Trk_{\Ku}$ and $\rho' \in \Trk_{\Ku'}$, and the subgraphs consisting of the states reachable from, respectively, $\lst(\rho)$ and $\lst(\rho')$ are isomorphic, then $\rho$ and $\rho'$ are equivalent with respect to $\psi$.

\begin{lemma}\label{lemmaBbar}
Given a $\Bbar$ formula $\psi$, two finite Kripke structures \[\Ku=\KuDef \text{ and } \Ku'=(\Prop',\States', \Edges',\Lab',\sinit'),\] and two traces $\rho\in\Trk_\Ku,\, \rho'\in \Trk_{\Ku'}$ such that  \[\mathpzc{L}(\Ku_{\,|p\ell(\psi)},\rho)=\mathpzc{L}(\Ku'_{\,|p\ell(\psi)},\rho')\] and 
\[reach(\Ku_{\,|p\ell(\psi)},\lst(\rho))\sim reach(\Ku'_{\,|p\ell(\psi)},\lst(\rho')),\]
it holds that $\Ku,\rho\models\psi\iff \Ku',\rho'\models\psi$.
\end{lemma}
%The proof is omitted (see Lemma~1 of \cite{MMP15B}).

\begin{theorem}\label{th:ABbarHard}
The MC problem for $\Bbar$ formulas over finite Kripke structures is $\PSPACE$-hard (under polynomial-time reductions).
\end{theorem}
\begin{proof}
Let $\psi=Q_n x_n$ $Q_{n-1} x_{n-1} \cdots Q_1 x_1 \phi(x_n,x_{n-1},\ldots ,x_1)$.
We prove by induction on the number $n$ of variables of $\psi$ that $\psi$ is true 
%We prove that the quantified Boolean formula $\psi=Q_n x_n Q_{n-1} x_{n-1} \cdots Q_1 x_1 \phi(x_n,x_{n-1},\cdots ,x_1)$ is true 
if and only if $\Ku_{QBF}^{x_n,\cdots , x_1}\models\xi$. In the following, $\phi(x_n,x_{n-1},\ldots , x_1)\{x_i/\upsilon\}$, with $\upsilon\in\{\top ,\bot\}$, denotes the formula obtained from $\phi(x_n,x_{n-1},\ldots ,x_1)$ by replacing all 
occurrences of $x_i$ by $\upsilon$. 
Notice that $\Ku_{QBF}^{x_n,x_{n-1},\cdots , x_1}$ and $\Ku_{QBF}^{x_{n-1},\cdots , x_1}$ are isomorphic if restricted to the states 
$s_{x_{n-1}}^\top$, $s_{x_{n-1}}^\bot$, $\cdots$, $s_{x_1}^\top$, $s_{x_1}^\bot$, $sink$ (i.e., the initial parts of both Kripke structures are eliminated), and the labelling of states is suitably restricted. Moreover, notice that only the trace $s_0$ satisfies the proposition letter $t$ in $\xi$.

Base case ($n=0$). In this case, $\psi=\phi(\emptyset)$ is a Boolean formula devoid of variables. 
If $\psi$ is true, then in particular $\Ku_{QBF}^{\emptyset},s_0\models\phi(\emptyset)$ and thus $\Ku_{QBF}^{\emptyset}\models s\rightarrow \phi(\emptyset)(=\xi)$. Conversely, if $\Ku_{QBF}^{\emptyset}\models s\rightarrow \phi(\emptyset)$, then $\Ku_{QBF}^{\emptyset},s_0\models\phi(\emptyset)$, and since $\phi(\emptyset)$ has no variables, it must be true.

%\medskip
Case $n\geq 1$.
We first prove that if $\psi\!=\!Q_n x_n Q_{n-1} x_{n-1}\!\cdots Q_1 x_1 \phi(x_n,x_{n-1},\ldots\! , x_1)$ is true, then $\Ku_{QBF}^{x_n,\cdots , x_1}\models\xi$. 

If
%$\psi=Q_n x_n Q_{n-1} x_{n-1} \cdots Q_1 x_1 \phi(x_n,x_{n-1},\cdots , x_1)$ is true and 
$Q_n=\exists$, one possibility is that $Q_{n-1} x_{n-1} \cdots Q_1 x_1 \phi'(x_{n-1},\ldots , x_1)$ is true, where $\phi'(x_{n-1},\ldots , x_1)=\phi(x_n,x_{n-1},\ldots , x_1)\{x_n/ \top\}$. By the inductive hypothesis, it holds that $\Ku_{QBF}^{x_{n-1},\cdots , x_1}\models\xi'$, where $\xi'=t\rightarrow\xi_{n-1}'$ and $\xi_{n-1}'=\xi_{n-1}\{x_n/\top\}$. Thus $\Ku_{QBF}^{x_{n-1},\cdots , x_1},s_0'\models \xi_{n-1}'$ ($s_0'$ is the initial state of the structure $\Ku_{QBF}^{x_{n-1},\cdots , x_1}$). By Lemma~\ref{lemmaBbar}, $\Ku_{QBF}^{x_n,\cdots , x_1},s_0s_{x_n}^\top\models \xi_{n-1}'$. Since every right extension of $s_0s_{x_n^\top}$ models $x_n$, $\Ku_{QBF}^{x_n,\cdots , x_1},s_0s_{x_n}^\top\models \xi_{n-1}$, and thus $\Ku_{QBF}^{x_n,\cdots , x_1},s_0\models \hsBt(\tilde{x_n}\wedge\xi_{n-1})(=\xi_n)$. To conclude, $\Ku_{QBF}^{x_n,\cdots , x_1}\models t\rightarrow\xi_n(=\xi)$. The only other possible case is that $Q_{n-1} x_{n-1} \cdots Q_1 x_1 \phi'(x_{n-1},\ldots , x_1)$ is true, with $\phi'(x_{n-1},\ldots , x_1)=\phi(x_n,x_{n-1},\ldots ,\allowbreak x_1)\{x_n/ \bot\}$. As before, it follows that $\Ku_{QBF}^{x_n,\cdots , x_1},s_0s_{x_n}^\bot\models \xi_{n-1}\{x_n/\bot\}$ and thus $\Ku_{QBF}^{x_n,\cdots , x_1},s_0\models \hsBt(\tilde{x_n}\wedge\xi_{n-1})$.

%If $\psi=Q_n x_n Q_{n-1} x_{n-1} \cdots Q_1 x_1 \phi(x_n,x_{n-1},\cdots , x_1)$ is true and 
Now, let 
%us consider the case where
$Q_n=\forall$. Both formulas $Q_{n-1} x_{n-1} \cdots$ $Q_1 x_1 \phi(x_n,x_{n-1},\ldots , x_1)\{x_n/ \top\}$ and  $Q_{n-1} x_{n-1} \cdots Q_1 x_1 \phi(x_n,x_{n-1},\ldots , x_1)\{x_n/ \bot\}$ are true. By reasoning as in the existential case, we have $\Ku_{QBF}^{x_n,\cdots , x_1},s_0s_{x_n}^\top\models \xi_{n-1}$ and $\Ku_{QBF}^{x_n,\cdots , x_1},s_0s_{x_n}^\bot\models \xi_{n-1}$. Thus, $\Ku_{QBF}^{x_n,\cdots , x_1},s_0\models \hsBtu(\tilde{x_n}\rightarrow \xi_{n-1})$ and $\Ku_{QBF}^{x_n,\cdots , x_1}\models s\rightarrow \hsBtu(\tilde{x_n}\rightarrow \xi_{n-1})$.

%\medskip
We now prove that if $\Ku_{QBF}^{x_n,\cdots , x_1}\models\xi$, then $\psi$
%=Q_n x_n Q_{n-1} x_{n-1} \cdots Q_1 x_1 \phi(x_n,x_{n-1},\cdots , x_1)$ 
is true.

If $Q_n=\exists$, then $\Ku_{QBF}^{x_n,\cdots , x_1},s_0\models \hsBt(\tilde{x_n}\wedge \xi_{n-1})$. Thus  $\Ku_{QBF}^{x_n,\cdots , x_1},s_0s_{x_n}^\top\models\xi_{n-1}$ or $\Ku_{QBF}^{x_n,\cdots , x_1},s_0s_{x_n}^\bot\models\xi_{n-1}$. In the former case, $\Ku_{QBF}^{x_n,\cdots , x_1},s_0s_{x_n}^\top\models \xi_{n-1}\{x_n/\top\}$ (since every right extension of $s_0s_{x_n}^\top$ models $x_n$). 
By Lemma~\ref{lemmaBbar}, $\Ku_{QBF}^{x_{n-1},\cdots , x_1},s_0'\models\xi_{n-1}\{x_n/\top\}$, and $\Ku_{QBF}^{x_{n-1},\cdots , x_1}\models t\rightarrow \xi_{n-1}\{x_n/\top\}$. By the inductive hypothesis, $Q_{n-1} x_{n-1} \cdots Q_1 x_1 \phi(x_n,x_{n-1},\ldots , x_1)\{x_n/\top\}$ is true, thus the formula $\psi=\exists x_n Q_{n-1} x_{n-1} \cdots Q_1 x_1 \phi(x_n,x_{n-1},\ldots , x_1)$ is true. In the latter case, we symmetrically have that $Q_{n-1} x_{n-1} .. Q_1 x_1 \phi(x_n,x_{n-1},\ldots , x_1)\{x_n/\bot\}$ is true, thus the formula $\psi=\exists x_n Q_{n-1} x_{n-1} \cdots Q_1 x_1 \phi(x_n,x_{n-1},\ldots , x_1)$ is true.

If $Q_n=\forall$, then $\Ku_{QBF}^{x_n,\cdots , x_1}, s_0 \models \hsBtu(\tilde{x_n}\rightarrow \xi_{n-1})$. So both $\Ku_{QBF}^{x_n,\cdots , x_1},$ $s_0s_{x_n}^\top\models  \xi_{n-1}$ and $\Ku_{QBF}^{x_n,\cdots , x_1},s_0s_{x_n}^\bot\models  \xi_{n-1}$.
By reasoning as in the existential case, both  
$Q_{n-1} x_{n-1} \cdots Q_1 x_1 \phi(x_n,x_{n-1},\ldots , x_1)\{x_n/\top\}$ and
$Q_{n-1} x_{n-1} \!\cdots Q_1 x_1 \phi(x_n,x_{n-1},\ldots ,\allowbreak x_1)\{x_n/\bot\}$ are true, thus $\psi=\forall x_n Q_{n-1} x_{n-1} \cdots  Q_1 x_1 \phi(x_n,x_{n-1},\ldots ,$ $x_1)$ is true.
\end{proof}


\section{Proof of Lemma~\ref{lemma:prefixSamplingOne}}\label{proof:lemma:prefixSamplingOne}

\begin{lemma*}[\ref{lemma:prefixSamplingOne}] Let $h\geq 1$, $\rho$ be a trace of $\Ku$, and let $i,j$ be two consecutive $\rho$-positions in the $h$-prefix sampling of $\rho$.
Then, for all $\rho$-positions $n,n'\in [i+1,j]$ such that $\rho(n)=\rho(n')$, it holds that $\rho(1,n)$ and $\rho(1,n')$ are $(h-1)$-prefix bisimilar.
\end{lemma*}

\begin{proof} The proof is by induction on $h\geq 1$.

Base case: $h=1$. The $1$-prefix sampling of $\rho$ is the prefix-skeleton sampling of $\rho$ in $[1,|\rho|]$. Hence, being $i$ and $j$ consecutive positions in this sampling, for each position $k\in [i,j-1]$, there is $\ell\leq i$ such that $\rho(\ell)=\rho(k)$. Since $\rho(n)=\rho(n')$, it holds that $\states(\rho(1,n))=\states(\rho(1,n'))$, and thus $\rho(1,n)$ and $\rho(1,n')$ are $0$-prefix bisimilar.

Inductive step: $h>1$. By definition of $h$-prefix sampling, there are two consecutive positions $i',j'$ in the $(h-1)$-prefix
  sampling of $\rho$ such that    $i,j$ are consecutive positions of the prefix-skeleton sampling of $\rho(i',j')$.

  If $i=i'$, then $j=i+1$, and thus, being $n,n'\in [i+1,j]$, we get that $n=n'$, and the result trivially holds.

  Let $i\neq i'$, thus $i>i'$.
     As in the base case, we easily deduce that $\rho(1,n)$ and $\rho(1,n')$ are $0$-prefix bisimilar. It remains to show that
for each proper prefix $\nu$ of $\rho(1,n)$ (resp., $\nu'$ of $\rho(1,n')$), there is a proper prefix $\nu'$ of $\rho(1,n')$ (resp., $\nu$ of $\rho(1,n)$) such that $\nu$ and $\nu'$ are $(h-2)$-prefix bisimilar. Let us consider a proper prefix $\nu$ of $\rho(1,n)$ (the proof for the other direction is symmetric). It holds that $\nu = \rho(1,m)$, for some $m<n$. We distinguish two cases:
\begin{itemize}
  \item $m\leq i$. Hence, $\rho(1,m)$ is a proper prefix of $\rho(1,n')$ and the result follows.
  \item $m>i$. Since $i$ and $j$ are consecutive positions of the prefix-skeleton sampling of $\rho(i',j')$, $i>i'$,
   and $m\in [i+1,j-1]$ (hence $m<j'$), there is $m'\in [i'+1,i]$ such that $\rho(m')=\rho(m)$ and $m'$ is in the prefix-skeleton sampling of $\rho(i',j')$. Let $\nu'=\rho(1,m')$. Clearly, $\nu'$ is a proper prefix of $\rho(1,n')$ (as $n'\geq i+1$). Moreover, since
       $m,m'\in [i'+1,j']$ and $i',j'$ are consecutive positions in the $(h-1)$-prefix sampling of $\rho$, by the inductive   hypothesis, $\nu=\rho(1,m)$ and $\nu'=\rho(1,m')$ are $(h-2)$-prefix bisimilar.
 \end{itemize}
%
This concludes the proof of Lemma~\ref{lemma:prefixSamplingOne}.
\end{proof}