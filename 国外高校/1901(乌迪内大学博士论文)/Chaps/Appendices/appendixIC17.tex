\chapter{Proofs of Chapter~\ref{chap:IC17}}
\minitoc\mtcskip

\section{Proof of Lemma~\ref{lemmaOracle}}\label{proof:lemmaOracle}

\begin{lemma*}[\ref{lemmaOracle}]
Let $\Ku=\KuDef$ be a finite Kripke structure, $\psi$ be an $\AAbarB$ formula, and $V_{\A}(\cdot,\cdot)$, $V_{\Abar}(\cdot,\cdot)$ be two Boolean arrays. We assume that 
\begin{enumerate}
	\item for each $\hsA \phi\in\mods(\psi)$ and $s'\in \States$, $V_{\A}(\phi,s')=\top$ if and only if there exists $\rho\in\Trk_{\Ku}$ such that $\fst(\rho)=s'$ and $\Ku,\rho\models \phi$, and
	\item for each $\hsAt \phi\in\mods(\psi)$ and $s'\in \States$, $V_{\Abar}(\phi,s')=\top$ if and only if there exists $\rho\in\Trk_{\Ku}$ such that $\lst(\rho)=s'$ and $\Ku,\rho\models \phi$.
\end{enumerate}
Then \texttt{Oracle}$(\Ku,\psi,s,\textsc{direction},V_{\A}\cup V_{\Abar})$ features a successful computation (returning $\top$) if and only if:
\begin{itemize}
	\item there exists $\rho\in\Trk_{\Ku}$ such that $\fst(\rho)=s$ and $\Ku,\rho\models \psi$, when \textsc{direction} is \textsc{forward};
	\item there exists $\rho\in\Trk_{\Ku}$ such that $\lst(\rho)=s$ and $\Ku,\rho\models \psi$, when \textsc{direction} is \textsc{backward}.
\end{itemize}
\end{lemma*}

\begin{proof}
 It is easy to check that if $\tilde{\rho}$ is the trace non-deterministically generated by \texttt{A\_trace} at line 1  then, for $i=1,\ldots, |\tilde{\rho}|$, it holds that $\Ku,\tilde{\rho}(1,i)\models\phi$ if and only if $T[\phi,i]=\top$, either by hypothesis, when  $\phi$ occurs in  $\mods(\psi)$ (lines 2--7), or by construction, when $\phi$ does not occur in $\mods(\psi)$ (lines 8--22).

Let us now assume that the value of the parameter \textsc{direction} is \textsc{forward} (the proof for the other direction is analogous).

\begin{itemize}
	\item[$(\Rightarrow)$] If \texttt{Oracle}$(\Ku,\psi,s,\textsc{forward},V_{\A}\cup V_{\Abar})$ features a successful computation, it means that there exists a trace $\tilde{\rho} \in\Trk_{\Ku}$ (generated at line 1) such that $\fst(\tilde{\rho})=s$ and $T[\psi,|\tilde{\rho}|]=\top$ implying that $\Ku,\tilde{\rho}\models\psi$.

	\item[$(\Leftarrow)$] If there exists $\rho\in\Trk_{\Ku}$ such that $\fst(\rho)=s$ and $\Ku,\rho\models \psi$, by Theorem~\ref{theorem:polynomialSizeModelProperty} there exists $\tilde{\rho}\in\Trk_{\Ku}$ such that $\Ku,\tilde{\rho}\models \psi$, $\fst(\tilde{\rho})=\fst(\rho)$, and $|\tilde{\rho}|\leq
	%|\States|\cdot (|\NNF(\psi)|+1)^2\leq 
	|\States|\cdot (2|\psi|+1)^2$.
	%where $\NNF(\psi)$ is the negation normal form of $\psi$. 
	It follows that in some non-deterministic instance of \texttt{Oracle}$(\Ku,\psi,s,\textsc{forward},V_{\A}\cup V_{\Abar})$, $\texttt{A\_trace}(\Ku,s,|\States|\cdot(2|\psi|+1)^2,\textsc{forward})$ returns such $\tilde{\rho}$ (at line 1). Finally, we have that $T[\psi,|\tilde{\rho}|]=\top$ as $\Ku,\tilde{\rho}\models \psi$, and hence the considered instance of \texttt{Oracle}$(\Ku,\psi,s,\allowbreak \textsc{forward},V_{\A}\cup V_{\Abar})$ is successful.\qedhere
\end{itemize}
\end{proof}


\section{Proof of Theorem~\ref{th:cx}}\label{proof:th:cx}

\begin{theorem*}[\ref{th:cx}]
Let $\psi$  be an $\AAbar$ formula and  $\Ku=(\Prop,\States,\Edges,\Lab,s_1)$ be a finite Kripke structure.
For every block $B$ of $T_{\Ku,\neg\psi}$,
if $B$ is associated with an $\AAbar$ formula $\varphi$, then
\begin{itemize}
	\item if $B$ is a \forw{} block, for all $i\in\{1,\ldots,|\States|\}$, $B(z_i)=\top$ if and only if there exists a trace $\rho\in\Trk_\Ku$ such that $\fst(\rho)=s_i$ and $\Ku,\rho\models\varphi$;
	\item if $B$ is a \back{} block, for all $i\in\{1,\ldots,|\States|\}$, $B(z_i)=\top$ if and only if there exists a trace $\rho\in\Trk_\Ku$ such that $\lst(\rho)=s_i$ and $\Ku,\rho\models\varphi$.
\end{itemize}
\end{theorem*}

\begin{proof}
The proof is by induction on the level $L \geq 1$ of the block $B$. The proof of the base case, i.e., for $L=1$, is just a simpler version of the inductive step and it is therefore omitted.

Assume that $B$ is a \forw{} block  at level $L\geq 2$ associated with a formula $\varphi$ (the \back{} case is symmetric).

We first prove the implication $(\Leftarrow$). We have to show that if there exists a trace $\rho\in\Trk_\mathpzc{K}$ such that $\fst(\rho)=s_i$ (for some $i\in\{1,\ldots,|\States|\}$) and $\mathpzc{K},\rho\models\varphi$, then $B(z_i)=\top$ that is, there exists a truth assignment $\omega$ to the variables in $V$ satisfying the formula $F_i(Y,V)$ of
$G_i$. In \cite{MMP15}, it is proved that if $\varphi$ is an $\AAbar$ formula and $\mathpzc{K},\rho\models\varphi$ (as in this case), there exists a trace $\rho'\in\Trk_\mathpzc{K}$, with $|\rho'|\leq |\States|^2+2$, such that $\fst(\rho)=\fst(\rho')=s_i$, $\lst(\rho)=\lst(\rho')$, and $\mathpzc{K},\rho'\models\varphi$. 
Thus, by Proposition~\ref{remk}, there exists a truth assignment $\omega$ to the variables in $V$, that satisfies $trace(V_{trace},V_{last},V_{\mathpzc{AP}})$, such that for all $1 \leq r \leq |\rho'|$ and $1 \leq j \leq |\States|$, $\rho(r)=s_j\iff\omega(v_j^r)=\top$ and $\omega(v_j^{|\rho|})=\omega(v_j)$, and for all $p\in\mathpzc{AP}$, $\omega(v_p)=\top\iff\mathpzc{K},\rho'\models p$ $(\star)$. 

Since $L\geq 2$, it holds that $\mods(\varphi)\neq \emptyset$. Let us consider a \forw{} child $B'$ of $B$ (if any), at a level lower than $L$, associated with some formula $\xi$ such that $\hsA\xi\in\mods(\varphi)$. By the inductive hypothesis, for all $j$, $B'(z_j)=\top$ if and only if there exists a trace $\overline{\rho}\in\Trk_\mathpzc{K}$ such that $\fst(\overline{\rho})=s_j$ and $\mathpzc{K},\overline{\rho}\models\xi$. Thus, $\mathpzc{K},\rho'\models\hsA\xi$ if and only if there exists $\tilde{\rho}\in\Trk_\mathpzc{K}$, with $\fst(\tilde{\rho})=\lst(\rho')=s_j$, for some $j$, and $\mathpzc{K},\tilde{\rho}\models\xi$ if and only if $B'(z_j)(=y_j^\xi)=\top$. Thus if $\mathpzc{K},\rho'\models\hsA\xi$, then $y_j^\xi=\top$, and $\omega(v_j)\wedge y_j^\xi=\top$. Now, to satisfy $F_i(Y,V)$, the truth assignment $\omega$ has to be such that $\omega(v_{\hsA\xi})=\top$. If $\mathpzc{K},\rho'\not\models\hsA\xi$, then $y_j^\xi=\bot$, thus $\bigvee_{u=1}^{|\States|} (\omega(v_u) \wedge y_u^{\xi})$ is false, and $\omega$ must be such that $\omega(v_{\hsA\xi})=\bot$. To conclude, $\mathpzc{K},\rho'\models\hsA\xi$ if and only if $\omega(v_{\hsA\xi})=\top$ $(\star\star)$. The symmetric reasoning can be applied to \back{} children of $B$.
%
Since $\mathpzc{K},\rho'\models\varphi$, by $(\star)$ and $(\star\star)$, we have $\omega(\overline{\varphi}(V_{\mathpzc{AP}},V_{modSubf}))=\top$.

We prove now the implication $(\Rightarrow$). If $B(z_i)=\top$, then there exists a truth assignment $\omega$ of $V$ satisfying $F_i(Y,\! V)$. In particular, $\omega$ satisfies $trace(V_{trace},\! V_{last},\! V_{\mathpzc{AP}})$ and $v_i^1$, thus, by Proposition~\ref{remk}, there exists a trace $\rho\in\Trk_\mathpzc{K}$ such that $\fst(\rho)=s_i$, $\lst(\rho)=s_j$, for some $j$, and $\mathpzc{K},\rho\models p\iff \omega(v_p)=\top$, for any $p\in\mathpzc{AP}$. 
By the inductive hypothesis, for all the formulas $\hsA\xi\in\mods(\varphi)$, $\mathpzc{K},\rho\models\hsA\xi$ if and only if $\omega(v_{\hsA\xi})=\top$, and symmetrically, for all $\hsAt\xi'\in\mods(\varphi)$, $\mathpzc{K},\rho\models\hsAt\xi'$ if and only if $\omega(v_{\hsAt\xi'})=\top$. 
%
Since $\omega(\overline{\varphi}(V_{\mathpzc{AP}},V_{modSubf}))=\top$, then we have $\mathpzc{K},\rho\models \varphi$.
\end{proof}


\section{Proof of Theorem~\ref{thcorr}}\label{proof:thcorr}

\begin{theorem*}[\ref{thcorr}]
Let $\mathcal{I}$ be an instance of SNSAT, with $|\mathcal{I}|=n$, and let $\Ku_\mathcal{I}$ and $\mathcal{F}_\mathcal{I}$
%$=\{\psi_k \mid 0\leq k\leq n+1\}$ 
be defined as above. For all $0\leq k\leq n+1$ and all $r=1,\ldots , n$, it holds that:
	\begin{enumerate}
		\item if $k\geq r$, then $v_\mathcal{I}(x_r)=\top \iff \Ku_\mathcal{I},w_{x_r}\models \psi_k;$
		%\[v_\mathcal{I}(x_r)=\top \iff \Ku_\mathcal{I},w_{x_r}\models \psi_k;\]
		\item if $k\geq r+1$, then $v_\mathcal{I}(x_r)=\bot \iff \Ku_\mathcal{I},\overline{w_{x_r}}\models \psi_k.$
		%\[v_\mathcal{I}(x_r)=\bot \iff \Ku_\mathcal{I},\overline{w_{x_r}}\models \psi_k.\]
	\end{enumerate}
\end{theorem*}

\begin{proof}
The proof is by induction on $k\geq 0$. If $k=0$, the thesis trivially holds.
Therefore, let us assume that $k\geq 1$. We first prove the $(\Leftarrow)$ implication for both (1.) and (2.).
\begin{itemize}
\item[(1.)] Assume that $k\geq r$ and $\Ku_\mathcal{I},w_{x_r}\models \psi_k$. Thus, there exists $\rho\in\Trk_{\Ku_\mathcal{I}}$ such that $\rho=w_{x_r}\cdots s_0$ does not pass through any $\overline{s_m}$, for $1\leq m\leq r$, and $\Ku_\mathcal{I},\rho\models \varphi_k$. We show by induction on $1\leq m\leq r$ that $\omega_\rho(x_m)=v_\mathcal{I}(x_m)$. 
%
	\begin{itemize}
		\item Let us consider first the case where $\rho$ passes through $w_{x_m}$, implying that $\omega_\rho(x_m)=\top$; thus $\Ku_\mathcal{I},\rho\!\models\! x_m\wedge \neg r_m$ and $\Ku_\mathcal{I},\rho\!\models\! F_m(x_1,\ldots ,x_{m-1},\allowbreak Z_m)$. If $m=1$ (base case), since $F_1$ is satisfiable, then $v_\mathcal{I}(x_1)=\top$. If $m\geq 2$ (inductive step), by the inductive hypothesis $\omega_\rho(x_1)=v_\mathcal{I}(x_1)$, \dots , $\omega_\rho(x_{m-1})=v_\mathcal{I}(x_{m-1})$. Since $\Ku_\mathcal{I},\rho\models F_m(x_1,\ldots ,x_{m-1},Z_m)$ or, equivalently, $F_m(\omega_\rho(x_{1}),\ldots ,\allowbreak \omega_\rho(x_{m-1}), \omega_\rho(Z_m))=\top$, it holds that $F_m(v_\mathcal{I}(x_{1}),\ldots , v_\mathcal{I}(x_{m-1}), \allowbreak  \omega_\rho(Z_m))=\top$ and, by definition of $v_\mathcal{I}$, we have $v_\mathcal{I}(x_m)=\top$.
		
		\item Conversely, let us consider the case where $\rho$ passes through $\overline{w_{x_m}}$, implying that $\omega_\rho(x_m)=\bot$ and $m<r$, as we are assuming $\fst(\rho)=w_{x_r}$. In this case, the prefix $w_{x_r}\cdots \overline{w_{x_m}}$ of $\rho$ satisfies both $\bigvee_{i=1}^n \hsA p_{\overline{x_i}}$ and $\hsA\big(\neg s \wedge \Length_2\wedge \hsA (\Length_2\wedge \neg\psi_{k-1})\big)$. Therefore, $\Ku_\mathcal{I},\overline{w_{x_m}}\cdot \overline{s_m}\models \hsA (\Length_2\wedge \neg\psi_{k-1})$ and $\Ku_\mathcal{I}, \overline{s_m}\cdot w_{x_m}\not\models \psi_{k-1}$, with $\psi_{k-1}=\hsA\varphi_{k-1}$. Hence $\Ku_\mathcal{I}, w_{x_m}\not\models \psi_{k-1}$. Since $1\leq m<r$, we have $1\leq m<r\leq k$, thus $k'=k-1\geq m\geq 1$. By the inductive hypothesis (on $k'=k-1$), we get that $v_\mathcal{I}(x_m)=\bot$.
	\end{itemize}
Therefore $v_\mathcal{I}(x_r)=\omega_\rho(x_r)$ and, since $w_{x_r}\in\states(\rho)$, we have $\omega_\rho(x_r)=\top$ and then $v_\mathcal{I}(x_r)=\top$ proving the thesis.
	
\item[(2.)] Assume that $k\geq r+1$ and $\Ku_\mathcal{I},\overline{w_{x_r}}\models \psi_k$. The proof follows the same steps as the previous case and it is thus only sketched: there exists $\rho\in\Trk_{\Ku_\mathcal{I}}$ such that $\rho=\overline{w_{x_r}}\cdots s_0$ does not pass through any $\overline{s_m}$, for $1\leq m\leq r$, and $\Ku_\mathcal{I},\rho\models \varphi_k$. The only difference is that the prefix $\overline{w_{x_r}}$ satisfies $\bigvee_{i=1}^n \hsA p_{\overline{x_i}}$, thus, as before, we get $\Ku_\mathcal{I}, w_{x_r}\not\models \psi_{k-1}$. Now, $k'=k-1\geq r\geq 1$ and, by the inductive hypothesis (on $k'=k-1$),  $v_\mathcal{I}(x_r)=\bot$.
\end{itemize}

%For $k\geq 1$, 
We prove now the converse implication $(\Rightarrow)$ for both (1.) and (2.).
\begin{itemize}
\item[(1.)] Assume that $k\geq r$ and $v_\mathcal{I}(x_r)=\top$. Let us consider the trace $\rho\in\Trk_{\Ku_\mathcal{I}}$, $\rho=w_{x_r}\cdots s_0$ never passing through any $\overline{s_m}$, for $1\leq m\leq r$, such that $w_{x_m}\in\states(\rho)$ if $v_\mathcal{I}(x_m)=\top$, and $\overline{w_{x_m}}\in\states(\rho)$ if $v_\mathcal{I}(x_m)=\bot$, for $1\leq m\leq r$. Such a choice of 
$\rho$ ensures that $v_\mathcal{I}(x_m)=\omega_\rho(x_m)$. In addition, the choice of $\rho$ has to induce also the proper truth-assignment of private variables,
that is, if $v_\mathcal{I}(x_m)=\top$, then for $1\leq u_m\leq j_m$, $w_{z_m^{u_m}}\in\states(\rho)$ if $F_m(v_\mathcal{I}(x_1),\ldots , v_\mathcal{I}(x_{m-1}),Z_m)$ is satisfied for $z_m^{u_m}= \top$, and $\overline{w_{z_m^{u_m}}}\in\states(\rho)$ otherwise. Note that such a choice of $\rho$ is always possible.
We have to show that $\Ku_\mathcal{I},\rho\models \varphi_k$, hence $\Ku_\mathcal{I},w_{x_r}\models \psi_k$.
\begin{itemize}
	\item For all $1\leq m\leq r$ such that $v_\mathcal{I}(x_m)=\top$, it holds that $F_m(v_\mathcal{I}(x_1),\ldots ,\allowbreak v_\mathcal{I}(x_{m-1}), Z_m)$ is satisfiable. Hence, by our choice of $\rho$, $F_m(\omega_\rho(x_1),\ldots ,\allowbreak \omega_\rho(x_{m-1}),\omega_\rho(Z_m))\!=\!\top$, or, equivalently, $\Ku_\mathcal{I},\rho\!\models\! F_m(x_1, \ldots, x_{m-1}, Z_m)$. Thus, $\Ku_\mathcal{I},\rho\models \bigwedge_{i=1}^n \Big((x_i\wedge \neg r_i)\rightarrow F_i(x_1, \ldots, x_{i-1}, Z_i)\Big)$. 

	\item Conversely, for all $1\leq m< r$ such that $v_\mathcal{I}(x_m)=\bot$ ($m\neq r$ as, by hypothesis, $v_\mathcal{I}(x_r)=\top$), it holds that $\overline{w_{x_m}}\in\states(\rho)$. Since $m<r$, we have $k\geq r>m$ and $k-1\geq m\geq 1$. By the inductive hypothesis, $\Ku_\mathcal{I},w_{x_m}\not\models \psi_{k-1}$. It follows that $\Ku_\mathcal{I},\overline{s_m}\cdot w_{x_m}\models \neg\psi_{k-1}\wedge \Length_2$, $\Ku_\mathcal{I},\overline{w_{x_m}}\cdot\overline{s_m}\models \neg s\wedge \Length_2\wedge\hsA(\neg\psi_{k-1}\wedge \Length_2)$ and $\Ku_\mathcal{I},\overline{w_{x_m}}\models\hsA( \neg s\wedge\Length_2\wedge\hsA(\neg\psi_{k-1}\wedge \Length_2))$. Hence, $\Ku_\mathcal{I},\rho\models \hsBu((\bigvee_{i=1}^n \hsA p_{\overline{x_i}})\rightarrow\hsA( \neg s\wedge\Length_2\wedge\hsA(\neg\psi_{k-1}\wedge \Length_2)))$.
\end{itemize} Combining the two cases, we can conclude that $\Ku_\mathcal{I},\rho\models \varphi_k$.
\item[(2.)] Assume that $k\geq r+1$ and $v_\mathcal{I}(x_r)=\bot$. The proof is as before and it is only sketched. In this case, we choose a trace $\rho=\overline{w_{x_r}}\cdots s_0$. Since $k'=k-1\geq r$, by the inductive hypothesis, $\Ku_\mathcal{I},w_{x_r}\not\models \psi_{k-1}$, and we can prove that $\Ku_\mathcal{I},\overline{w_{x_r}}\models\hsA( \neg s\wedge\Length_2\wedge\hsA(\neg\psi_{k-1}\wedge \Length_2))$.\qedhere
\end{itemize}
\end{proof}