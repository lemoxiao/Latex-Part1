\section{Preliminaries}\label{sec:preliminaries}

To start with, we introduce some preliminary notions.
Let $\mathbb{S} = (\mathpzc{S}, <)$ be a  linear order. An
\emph{interval} over $\mathbb{S}$ is an ordered pair $[x, y]$,
where $x,y\in \mathpzc{S},\, x \leq y$, representing the set $\{z\in \mathpzc{S}\mid x\leq z\leq y\}$. We denote the set of all intervals over
$\mathbb{S}$ by $\mathbb{I(S)}$.
%
We consider three possible \emph{sub-interval relations}: 
\begin{enumerate}
    \item the \emph{reflexive} sub-interval relation (denoted as
    $\sqsubseteq$), defined by $[x, y] \sqsubseteq [x', y']$ if and only if
    $x' \leq x$ and $y\leq y'$,
    % 
    \item the \emph{proper (or
    irreflexive)} sub-interval relation (denoted as $\subint$), defined
    by $[x, y]\subint  [x', y']$ if and only if $[x, y] \sqsubseteq
    [x', y']$ and $[x, y] \neq [x', y']$, and 
    %
    \item the \emph{strict}
    sub-interval relation (denoted as $\ssubint$), defined by 
    $[x, y] \ssubint [x', y'] $ if and only if $x' < x$ and $y<y'$.
\end{enumerate}

The three modal logics $\Dref$, $\Dirr$, and $\Dstr$ feature the same
language, consisting of a set $\AP$ of proposition
letters, the logical connectives $\neg$ and $\vee$, and the
modal operator $\hsD$. Formally, formulas are
defined by the grammar: \[\varphi ::= p \ |\ \neg\varphi\ |\ \varphi \vee \varphi\ |\ \hsD \varphi,\]
%\[
%\varphi ::= p \ |\ \neg\varphi\ |\ \varphi \vee \varphi\ |\ \D \varphi ,
%\]
with $p\in\AP$.
The other connectives, as well as
the logical constants $\top$ (true) and $\bot$ (false), are defined as usual; moreover, the
dual universal modal operator $\hsDu\varphi$ is defined as $\neg\hsD\neg\varphi$. The length of a formula
$\varphi$, denoted as $|\varphi|$, is the number of subformulas of $\varphi$.

The semantics of~\Dstr, \Dirr, and \Dref only differ in the
interpretation of the $\hsD$ operator. For the sake of brevity, we use
$\circ \in \{\ssubint,\subint,\subinteq \}$ as a shorthand for
any of the three sub-interval relations. The semantics of a
sub-interval logic \Dsim is defined in terms of \emph{interval models}\footnote{Not to be confused with the previously introduced \emph{abstract interval models}.}
$\bM=(\mathbb{I(S)},\circ,\cV)$. The \emph{valuation
function} $\cV : \AP \to 2^{\mathbb{I(S)}}$ assigns to
every proposition letter $p$ the set of intervals $\cV(p)$ over
which $p$ holds. The \emph{satisfiability relation} $\models$ is
defined as:
%
\begin{itemize}
  \item for every proposition letter  $p \in \AP$, $\bM, [x,y]
        \models p$ if and only if $[x,y] \in \cV(p)$;

  \item $\bM, [x,y] \!\models\! \neg \psi$ if and only if $\bM, [x,y]
        \!\not\models\! \psi$ (i.e. it is not true that $\bM, [x,y]
        \!\models\! \psi$);

  \item $\bM, [x,y] \models \psi_1 \vee \psi_2$ if and only if $\bM, [x,y] \models \psi_1$ or $\bM, [x,y] \models \psi_2$;

  \item $\bM, [x,y] \models \hsD \psi$ if and only if there exists an interval $[x',y'] \in
        \bbI(\mathbb{S})$ such that $[x',y'] \circ [x,y]$ and
        $\bM, [x',y'] \models \psi$.
\end{itemize}
%
A \Dsim formula is \emph{satisfiable} if it holds over some interval of an interval model, and \emph{valid} if it holds over every interval of every interval model.

As we mentioned earlier on, it can be shown that the logic $\Dref$ turns out to be equivalent to the standard modal logic S4 \cite{DBLP:journals/jphil/Benthem91}. 
%The decidability of the satisfiability problem for the considered logics 
%\Dsim has been widely studied by restricting the class of linear orders to the dense ones 
%\cite{DBLP:journals/corr/MontanariPS15} and to the finite ones 
%\cite{DBLP:journals/fuin/MarcinkowskiM14}. 
%
Here we restrict our attention to the finite SAT problem, that is, satisfiability over the class of finite linear orders. The problem has been shown to be \emph{undecidable} for $\Dirr$ and $\Dstr$~\cite{DBLP:journals/fuin/MarcinkowskiM14} and \emph{decidable} for $\Dref$~\cite{DBLP:conf/time/MontanariPS10}. In the following, we prove that decidability can be recovered for $\Dirr$ and $\Dstr$ by restricting to the class of \emph{homogeneous} interval models. We fully work out the case of $\Dirr$ (for the sake of simplicity, we will write $\D$\ for $\Dirr$), and then we briefly explain how to adapt the proofs to $\Dstr$. 

\begin{definition}\label{def:homogeneous_models}
An interval model
$\bM=(\mathbb{I(S)},\circ,\cV)$ is \emph{homogeneous} if,
for every interval $[x,y]  \in \mathbb{I(S)}$ and every proposition letter $p \in \AP$,
it holds that $[x,y] \in \cV(p)$ if and only if $[x',x'] \in \cV(p)$
for every $x\leq x'\leq y$.
\end{definition}
% 
Hereafter we will assume the logic $\D$ to be interpreted over homogeneous interval models.
