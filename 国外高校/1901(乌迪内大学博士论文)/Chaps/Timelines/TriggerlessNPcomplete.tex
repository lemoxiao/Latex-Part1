\section{TP with trigger-less rules only is \NP-complete}\label{sec:NPtriggerless}

In this section we describe a TP algorithm,
for planning domains where only trigger-less rules are allowed,
which requires a polynomial number of (non-deterministic) computation steps.
We recall that trigger-less rules are useful, for instance, to express initial, intermediate conditions and reachability goals.

We want to start with the following example, with which we highlight that there is no \emph{polynomial-size} plan for some problem instances/domains. Thus, an explicit enumeration of all tokens of a multi-timeline \emph{does not} represent a suitable polynomial-size certificate.

\begin{example}
Let us consider the following planning domain.
We denote by $p(i)$ the $i$-th prime number, assuming $p(1)=1$, $p(2)=2$, $p(3)=3$, $p(4)=5$,\dots .
We define, for $i=1,\ldots,n$, the state variables $x_i=(\{v_i\},\{(v_i,v_i)\},D_{x_i})$ with $D_{x_i}(v_i)=\mathopen[p(i),p(i)\mathclose]$.
The following rule
\[
\true\to \exists o_1[x_1=v_1] \cdots \exists o_n[x_n=v_n].\bigwedge_{i=1}^{n-1} o_i\leq_{[0,0]}^{\mathsf{e},\mathsf{e}} o_{i+1}
\]
is asking for the existence of a \lq\lq synchronization point\rq\rq , where $n$ tokens (one for each variable) have their ends aligned.
Due to the allowed token durations, the first such time point is $\prod_{i=1}^{n} p(i)\geq 2^{n-1}$.
Hence, in any plan, the timeline for $x_1$ features at least $2^{n-1}$ tokens: no explicit polynomial-time enumeration of such tokens is possible.
\end{example}
%
As a consequence, there exists no trivial guess-and-check \NP\ algorithm.
Conversely, one can easily prove the following result.
\begin{theorem}
The TP problem with trigger-less rules only is \NP-hard, even with one state variable (under polynomial-time reductions).
\end{theorem}
\begin{proof}
There is a trivial reduction from the problem of the existence of a Hamiltonian path in a directed graph.

Given a directed graph $G=(V,E)$, with $|V|=n$, 
we define the state variable $x=(V,E,D_x)$, where $D_x(v)=[1,1]$ for each $v\in V$.
We add the following trigger-less rules, one for each $v\in V$:
\[
\true \to \exists o[x=v]. o\geq ^{\mathsf s}_{[0,n-1]} 0 .
\]
The rule for $v\in V$ requires that there is a token $(x,v,1)$ along the timeline for $x$, which starts no later than $n-1$.
It is easy to check that $G$ contains a Hamiltonian path if and only if there exists a plan for the defined planning domain.
\end{proof}

We now present the aforementioned non-deterministic polynomial-time algorithm, proving that timeline-based planning with trigger-less rules is in \NP.

We preliminarily have to derive a \emph{finite horizon} (namely, the end time of the last token) for the plans of a (any) instance of TP with trigger-less rules. That is, 
if an instance $P=(SV,R)$ admits a plan, then $P$ also has a plan whose horizon is no greater than a given bound. Analogously, we have to calculate a \emph{bound to the maximum number of tokens} in a plan.
Both can be obtained from 
the constructions of the \TA s described in the proof of 
Theorem~\ref{theorem:UpperBounds}:
since only trigger-less rules are now allowed,
we disregard the construction of the \MTL\ formula $\varphi_\forall$,
and restrict our attention to the \TA\  
$\Au_P$ (i.e., the intersection between $\Au_{SV}$ for the state variables in $SV$ from Proposition~\ref{prop:AtutomataForMultiTimeline} and $\Au_{_\exists}$ for the trigger-less rules in $R$ from Proposition~\ref{prop:TATriggerLessRules}), which has $\alpha_s=2^{O(N_q+\sum_{x\in SV} |V_x|)}$ states, $\alpha_c=O(N_q+|SV|)$ clocks and maximum constant $\alpha_K=O(K_p)$, where
$N_q$  is the overall number of quantifiers   in the trigger-less  rules of $R$, 
and accepts all and only the encodings $w_\Pi$ of multi-timelines $\Pi$ of $SV$ satisfying all the trigger-less rules in $R$.

The language emptiness checking algorithm for \TA s executed over $\Au_P$ visits the (untimed) region automaton for $\Au_P$~\cite{ALUR1994183},
which features $\alpha=\alpha_s\cdot O(\alpha_c!\cdot 2^{\alpha_c}\cdot 2^{2 N_q^2}\cdot (2\alpha_K +2)^{\alpha_c})$
states%
\footnote{The factor $2^{2 N_q^2}$ is present due to diagonal clock constraints in $\Au_P$.}%
, trying to find a path, from the initial state to a final state, whose length can clearly be bounded by the number of states.
%
We observe that each edge/transition of the region automaton in such a path corresponds, in the worst case, to the start point of a token for each timeline for the variables in $SV$ (i.e., assuming that all these tokens start simultaneously).
This yields a bound on the number of tokens, which is $\alpha \cdot |SV|$.
We can also derive a bound on the horizon of the plan, which is $\alpha \cdot |SV| \cdot (\alpha_K+1)$, as every transition taken in $\Au_P$ may let at most $\alpha_K+1$ time units pass, as $\alpha_K$ accounts in particular for the maximum constant to which a (any) clock is compared.\footnote{Clearly, and unbounded quantity of time units may pass, but after $\alpha_K+1$ the last region of the region automaton will certainly have been reached.}

Having this pair of bounds, 
we are now ready to describe the two main phases of the algorithm, corresponding to the following pair of observations.
On the one hand, $(i)$~each trigger-less rule requires, as we said,
the existence of an \emph{a priori bounded number} of temporal events satisfying mutual temporal relations (namely, in the worst case, the start time and end time of all tokens associated with the quantifiers of one of its existential statements).
On the other hand, $(ii)$~timelines for different state variables evolve independently of each other.
In order to deal with $(i)$, we non-deterministically position such temporal events along timelines; as for $(ii)$, we enforce a correct evolution of each timeline between pairs of \lq\lq positioned\rq\rq\ events, completely independently of the other timelines.

% \paragraph{Preprocessing}
% As a preliminary preprocessing phase, we consider all rational values occurring in the input planning problem $P=(SV,S)$---be either 
% upper/lower bounds of an interval of a token duration, a time constant in an atom, or 
% upper/lower bounds ($u$ or $\ell$) at the subscript of an atom---and make them integers by multiplying them
% by the least common multiple $\gamma$ of all denominators. This involves a quadratic blowup in the input size, being all constants encoded in binary. 

% It is routine to check that, having a plan for $P'$---where all values are integers---we can obtain one for the original $P$, by dividing the start/end times of all tokens in each timeline by $\gamma$.

\paragraph{Non-deterministic token positioning.}
The algorithm starts by non-determin\-istically selecting, for every trigger-less rule in $R$, a disjunct---and deleting all the others. Then, for every (left) quantifier $o_i[x_i=v_i]$, it generates the integer part of both the start and the end time of the token for $x_i$ to which $o_i$ is mapped. We call such time instants, respectively, $\mathsf{s}_{int}(o_i)$ and $\mathsf{e}_{int}(o_i)$.\footnote{We can assume w.l.o.g.\ that all quantifiers refer to distinct tokens. As a matter of fact, the algorithm can non-deterministically choose to make two (or more) quantifiers $o_i[x_i=v_i]$ and $o_j[x_i=v_i]$ over the same variable and value \lq\lq collapse\rq\rq\ to the same token just by rewriting all occurrences of $o_j$ as $o_i$ in the atoms of the rules.} We observe that all start/end time $\mathsf{s}_{int}(o_i)$ and $\mathsf{e}_{int}(o_i)$, being less or equal to $\alpha \cdot |SV| \cdot (\alpha_K+1)$ (the finite horizon bound), have an integer part that can be encoded with polynomially many bits (and thus can be generated in polynomial time). 
%Let $\xi$ denote the number of quantifiers of all rules. 

Let us now consider the fractional parts of the start/end time of the tokens associated with quantifiers. We denote them by $\mathsf{s}_{frac}(o_i)$ and $\mathsf{e}_{frac}(o_i)$. The algorithm non-deterministically generates an \emph{order} of all such fractional parts. In particular we have to specify, for every token start/end time, whether it is integer ($\mathsf{s}_{frac}(o_i)=0$, $\mathsf{e}_{frac}(o_i)=0$) or not ($\mathsf{s}_{frac}(o_i)>0$, $\mathsf{e}_{frac}(o_i)>0$).
%There are $2^{2\xi}\cdot 2^{2\xi}\cdot (2\xi)!$ such possibilities, and each one can be encoded in binary.
Every such possibility can be generated in polynomial time.

Some trivial tests should now be performed, namely that, for all $o_i$, $\mathsf{s}_{int}(o_i)\leq \mathsf{e}_{int}(o_i)$, each token is assigned an end time equal or greater than its start time, and no two tokens for the same variable are overlapping.

It is routine to check that, if we change the start/end time of (some of the) tokens associated with quantifiers, 
but we leave unchanged $(i)$~all the integer parts, $(ii)$~zeroness/non-zeroness of fractional parts, and $(iii)$~the fractional parts' order,
then the satisfaction of the (atoms in the) trigger-less rules does not change. This is due to all the constants being integers.%
%, as a result of the preprocessing step.
\footnote{We may observe that, by leaving unchanged all the integer parts and the fractional parts' order, the region of the region graph of the timed automaton does not change.} Therefore we can now check whether all rules are satisfied.

\paragraph{Enforcing legal token durations and timeline evolutions.}

We now continue by checking that: $(i)$ all tokens associated with a quantifier have a legal duration, and that $(ii)$ there exists a legal timeline evolution between pairs of adjacent such tokens over the same variable (here \emph{adjacent} means that there is no other token associated with a quantifier in between). 
We will enforce all these requirements as constraints of a \emph{linear problem}, which can be solved in deterministic polynomial time (e.g., using the ellipsoid algorithm).
When needed, we use \emph{strict inequalities}, which are not allowed in linear programs. We shall show later how to convert these into non-strict ones.

We start by associating non-negative variables $\alpha_{o_i,s}, \alpha_{o_i,e}$ with the fractional parts of the start/end times $\mathsf{s}_{frac}(o_i)$, $\mathsf{e}_{frac}(o_i)$ of every token for a quantifier $o_i[x_i=v_i]$.
First, we add the linear constraints
\begin{equation*}
    0\leq \alpha_{o_i,s}<1,\quad 0\leq \alpha_{o_i,e}<1.
\end{equation*}
Then, we also need to enforce that the values of $\alpha_{o_i,s}, \alpha_{o_i,e}$ respect the decided order of the fractional parts: for example,
\begin{equation*}
    0=\alpha_{o_i,s}=\alpha_{o_j,s}<\alpha_{o_k,s}<\ldots <\alpha_{o_j,e}<\alpha_{o_i,e}=\alpha_{o_k,e}<\ldots
\end{equation*}
%
To enforce requirement $(i)$, we set, for all $o_i[x_i=v_i]$,
\begin{equation*}
    a\leq (\mathsf{e}_{int}(o_i)+\alpha_{o_i,e})-(\mathsf{s}_{int}(o_i)+\alpha_{o_i,s})\leq b
\end{equation*}
where $D_{x_i}(v_i)=\mathopen[a,b\mathclose]$. Clearly, strict ($<$) inequalities must be used for a left/right open interval.

To enforce requirement $(ii)$, namely that there exists a legal timeline evolution between each pair of adjacent tokens for the same state variable, say $o_i[x_i=v_i]$ and $o_j[x_i=v_j]$, we proceed as follows (for a correct evolution between $t=0$ and the first token, analogous considerations can be made).

Let us consider each state variable $x_i=(V_i,T_i,D_i)$ as a directed graph $G=(V_i,T_i)$ where 
$D_i$ is a function associating with each vertex $v\in V_i$ a duration range.
We have to decide whether or not there exist
\begin{itemize}
    \item a path in $G$, possibly with repeated vertices and edges, $v_0 \cdot v_1 \cdots v_{n-1}\cdot v_n$, where $v_0\in T_i(v_i)$ and $v_n$ with $v_j\in T_i(v_n)$ are non-deterministically generated, and
    \item a list of non-negative real values $d_0,\ldots,d_n$,
such that 
\[\sum_{t=0}^n d_t = (\mathsf{s}_{int}(o_j)+\alpha_{o_j,s}) - (\mathsf{e}_{int}(o_i)+\alpha_{o_i,e}),\]
and for all $s=0,\ldots, n$, $d_s\in D_i(v_s)$.
\end{itemize}
%  -  
%  - 

% Intuitively, we have to find a path in $G$, possibly where we get to the same vertices/edges also more than once, and where we remain in each vertex a non-negative rational amount of time allowed by the minimum/maximum duration function, such that the overall time of the path equals $h$.

We guess 
a set of integers $\{\alpha'_{u,v}\mid (u,v)\in T_i\}$.
Intuitively, $\alpha'_{u,v}$ is the number of times the solution path
traverses $(u,v)$. Since every time an edge is traversed a new token starts, each $\alpha'_{u,v}$ is bounded by the number of tokens, i.e., by $\alpha \cdot |SV|$. Hence the binary encoding of $\alpha'_{u,v}$ can be generated in polynomial time.

We then perform the following deterministic steps.
\begin{enumerate}
\item We consider the subset $E'$ of edges of $G$, $E'=\{(u,v)\in T_i\mid \alpha'_{u,v}>0\}$. We check whether $E'$ induces a strongly (undirected) connected subgraph of $G$.
\item We check whether 
    \begin{itemize}
        \item $\sum_{(u,v)\in E'} \alpha'_{u,v}=\sum_{(v,w)\in E'} \alpha'_{v,w}$, for all $v \in V_i\setminus\{v_0,v_n\}$;
        \item $\sum_{(u,v_0)\in E'} \alpha'_{u,v_0}=\sum_{(v_0,w)\in E'} \alpha'_{v_0,w}-1$;
        \item $\sum_{(u,v_n)\in E'} \alpha'_{u,v_n}=\sum_{(v_n,w)\in E'} \alpha'_{v_n,w}+1$.
    \end{itemize}

\item For all $v \in V_i\setminus\{v_0\}$, we define $y_v=\sum_{(u,v)\in E'} \alpha'_{u,v}$ ($y_v$ is the number of times the solution path gets into $v$). Moreover, 
$y_{v_0} = \sum_{(v_0,u)\in E'} \alpha'_{v_0,u}$.
\item We define the real non-negative variables $z_v$, for every $v \in V_i$ ($z_v$ is the total waiting time of the path on the node $v$), subject to the following constraints:
\[a\cdot y_v \leq z_v \leq b\cdot y_v,\]
where $D_i(v)=[a,b]$ (an analogous constraint should be written for open intervals). Finally we set:
\[\sum_{v \in V_i} z_v = (\mathsf{s}_{int}(o_j)+\alpha_{o_j,s}) - (\mathsf{e}_{int}(o_i)+\alpha_{o_i,e}).\]
\end{enumerate}

Steps (1.) and (2.) together check that the values $\alpha'_{u,v}$ for the arcs specify
a directed Eulerian path from $v_0$ to $v_n$ in a multigraph. Indeed,
the following theorem holds.
\begin{theorem}\cite{Jung}
Let $G'=(V',E')$ be a directed multigraph ($E'$ is a multiset).
$G'$ has a (directed) Eulerian path from $v_0$ to $v_n$ if and only if:
\begin{itemize}
    \item the undirected version of $G'$ is connected, and
    \item $|\{(u,v)\in E'\}| =| \{(v,w)\in E'\}|$, for all $v \in V'\setminus\{v_0,v_n\}$;
    \item $|\{(u,v_0)\in E'\}|=|\{(v_0,w)\in E'\}|-1$;
    \item $|\{(u,v_n)\in E'\}|=|\{(v_n,w)\in E'\}|+1$.
\end{itemize}
\end{theorem}

Steps (3.) and (4.) evaluate the waiting times of the path in some
vertex $v$ with duration interval $\mathopen[a,b\mathclose]$.
If the solution path visits the vertex $y_v$ times, then every single
visit must take at least $a$ and at most $b$ units of time.
Hence the overall visitation time is in between $a\cdot y_v$ and
$b\cdot y_v$.
Vice versa, if the total visitation time is in between $a\cdot y_v$ and
 $b\cdot y_v$, then it can be slit into $y_v$ intervals, each one falling into $\mathopen[a,b\mathclose]$. 

The algorithm concludes by solving the linear program given by the variables $\alpha_{o_i,s}$ and $\alpha_{o_i,e}$ for each quantifier $o_i[x_i=v_i]$, and for each pair of adjacent tokens in the same timeline for $x_i$, for each $v\in V_i$, the variables $z_v$ subject to their constraints.

Finally, in order to conform to linear programming, we have to replace all strict inequalities with non-strict ones.
It is straightforward to observe that all constraints involving strict inequalities we have written so far are of
(or can easily be converted into) the following forms: 
$\xi s<\eta q+k$ or $\xi s>\eta q+k$, where $s$ and $q$ are variables, and $\xi$, $\eta$, $k$ are constants.
We replace them, respectively, by $\xi s-\eta q-k+\beta_t\leq 0$ and $\xi s-\eta q-k-\beta_t\geq 0$, where $\beta_t$ is an additional fresh non-negative variable, which is \emph{local} to a single constraint. 
We observe that the original inequality and the new one are equivalent if and only if $\beta_t$ is a small enough \emph{positive} number.
Moreover, we add another non-negative variable, say $r$, which is subject to a constraint $r\leq \beta_t$, for each of the introduced variables $\beta_t$ (i.e., $r$ is less than or equal to the minimum of all $\beta_t$'s). Finally, we maximize the value of $r$ when solving the linear program. We have that $\max r>0$ if and only if there is an admissible solution where the values of all $\beta_t$'s are positive (and thus the original strict inequalities hold true). 

This ends the description of the planning algorithm. We can thus conclude the section with the main result.
\begin{theorem}
The TP problem with trigger-less rules only is \NP-complete.
\end{theorem}

In the next section, using the results on the variants of the TP problem, we shall study MC over timelines.