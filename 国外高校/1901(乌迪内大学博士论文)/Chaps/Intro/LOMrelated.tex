\section{Related work}\label{sec:LOMrelated}
We would like to conclude the chapter by summarizing the content of other research papers that deal with $\HS$ MC.
As already mentioned, the MC problem for interval temporal logics has not been extensively studied in literature. Indeed the only three papers (apart from the ones by these authors) that study $\HS$ MC are all by A.\ Lomuscio and J.\ Michaliszyn~\cite{LM13,LM14,lm16}.

In~\cite{LM13,LM14,lm16}, they address the MC problem for some fragments of $\HS$ extended with epistemic modalities. Their semantic assumptions are different from the ones we make here, thus 
a systematic comparison of the two research lines is quite difficult. In both cases, formulas of $\HS$ are evaluated over traces/intervals of a Kripke structure; however, in~\cite{LM13,LM14} truth of proposition letters over an interval depends only on its endpoints. 

In~\cite{LM13}, the authors focus on the $\HS$ fragment $\B\E\D$ of Allen's relations \emph{started-by}, \emph{finished-by}, and \emph{contains} (since modality $\hsD$ is definable in terms of modalities $\hsB$ and $\hsE$, $\B\E\D$ is actually as expressive as $\BE$), extended with epistemic modalities. They consider a \emph{restricted} form of MC (\lq\lq local\rq\rq\ MC), which verifies a given specification against a single (finite) initial computation interval. Their goal is indeed to reason about a given computation of a multi-agent system, rather than on all its admissible computations.
They prove that the considered MC problem is $\PSPACE$-complete; furthermore, they show that the same problem restricted to the pure temporal fragment $\B\E\D$, that is, the one obtained by removing epistemic modalities, is in $\PTIME$ as, basically, modalities $\hsB$ and $\hsE$ allow one to access only sub-intervals of the initial one, whose number is quadratic in the length (number of states) of the initial interval.

In~\cite{LM14}, they show that the picture drastically changes with other $\HS$ fragments that allow one to access infinitely many intervals. In particular, they prove that the MC problem for the fragment $\ABbar\L$ of Allen's relations \emph{meets}, \emph{starts}, and \emph{before} (since modality $\hsL$ is definable in terms of modality $\hsA$, $\ABbar\L$ is actually as expressive as $\ABbar$) extended with epistemic modalities, is decidable in nonelementary time. Note that, thanks to modalities $\hsA$ and $\hsBt$, formulas of $\ABbar\L$ can possibly refer to infinitely many (future) intervals.

Finally, in~\cite{lm16}, Lomuscio and Michaliszyn show how to use regular expressions in order to specify the way in which intervals of a Kripke structure get labelled. Such an extension leads to a significant increase in 
expressiveness, as the labelling of an interval is no more determined by that of its endpoints only, 
but it depends on the ordered sequence of states the interval consists of. They also prove that there is no corresponding increase in computational complexity, as the bounds given in~\cite{LM13,LM14} still hold with the new semantic variant: MC for $\B\E\D$ is in $\PSPACE$, and it is nonelementarily decidable for $\ABbar\L$.
We will come back to this idea of using regular expressions to define interval labelling in Chapter~\ref{chap:Gand17}.
