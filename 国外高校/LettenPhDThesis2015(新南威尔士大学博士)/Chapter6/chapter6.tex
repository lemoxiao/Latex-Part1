%*******************************************************************************
%*********************************** Last Chapter *****************************
%*******************************************************************************

\graphicspath{{Chapter6/Figs}}

\chapter{Conclusions}  %Title of the last chapater

In common with many modern doctoral theses/dissertations, the studies presented in each chapter of this thesis were conducted, and taken through to completion (i.e. publication), in a chronological order, rather than the traditional model where manuscript writing is left till last. An interesting by-product of this approach is that it provides a vantage point to critically examine one's own work from an appreciable distance. For instance, a final draft of Chapter 2 was written-up by late 2012 and accepted for publication in April 2013 \citep{Letten2013}. In the two years that have since passed, my knowledge of the field and general skill-set have, I like to think, expanded. This raises the question, if I had the time again, would I do it differently? In this final chapter, I briefly summarise the contributions of each of the preceding chapters, whilst also considering their limitations, and where relevant the scope for future enquiry.

In Chapter 2, I provided evidence that spatial heterogeneity in temporal climate variability may in some systems be a better predictor of local-scale plant species richness patterns than more commonly used absolute climate measures. Modern coexistence theory \citep[\textit{sensu}][]{Chesson2000} provides an intuitively appealing explanation for this result, and yet perhaps more than any other chapter in this thesis, this study illustrates the tension between the mechanistic insight afforded by experimental manipulation, and the broad scale, but strictly correlative, insight derived from observational studies. Is climate variability a better predictor than absolute climate because it stabilizes coexistence, or is climate variability merely a better proxy for some other factor regulating richness patterns? Unfortunately, no amount of statistical duress will yield a conclusive answer to this question from the dataset at hand. Species richness patterns are a product of a myriad factors, some deterministic and some stochastic, and thus the extent to which causation may be inferred from correlation will always be less than satisfactory.  

In juxtaposition with the low-resolution but broad-scale potential of richness data, data-intensive single-site studies afford opportunities for deeper interrogation of underlying assembly processes. Long-term demographic datasets have proven particularly effective in demonstrating the role of temporal climate variability in annual plant coexistence \citep{Adler2006, Adler2009, Angert2009}. An exciting middle ground between these approaches, would be to investigate inter- and intra- specific demographic variability across sites characterised by different levels of temporal climate variability. The analysis of demographic data along environmental gradients has recently been shown to be a powerful method for exploring spatially varying niche relationships \citep{Diez2014}. The logistical challenge in this instance will be locating study sites that exhibit sufficient spatial variation in temporal climate variability but that are close enough together to facilitate regular monitoring. Given these constraints, the topographically heterogeneous terrain typical of alpine grasslands may represent one potential system for such a study. However, it also remains unclear to what extent relatively low intra-seasonal and intra-annual climate variability (as exhibited in sheltered gorges compared to exposed ridges) translates into more buffered inter-annual variability. Given that inter-annual climate variability is likely to play a considerably more important role in species coexistence than intra-annual variability, this assumption warrants further investigation.  

The exploration of several largely unexplored assumptions specific to phylogenetic approaches in community ecology formed the basis of both Chapters 3 and 4 of this thesis. More specifically, Chapter 3 highlighted the absence of any theoretical basis to the \textit{"widespread assumption in the literature that phylogenetic and functional distance scale linearly"} \citep{Letten2014}. In fairness, this is probably less an explicit assumption and more an oversight. Nonetheless, it has consequences for assessing phylogenetic community structure. The simple, but theoretically robust, solution to avoid the over-weighting of early diverged clades is a square root transformation of the phylogenetic distance matrix. As briefly mentioned in the discussion to that chapter, while there are almost certainly more sophisticated approaches available, a square-root transformation is for now the most parsimonious approach to the scaling of evolutionary relatedness and functional distance. 

Notwithstanding the importance of appropriate phylogenetic and functional scaling, it is important to keep in mind that functional traits are themselves proxies for the underlying processes structuring communities. For instance, plant height is commonly used to predict competitive dominance in competition for light \citep{Westoby2002, McGill2006}. However, it is less clear how these two factors scale. Does a doubling in height confer a doubling in competitive fitness? There has been some compelling recent work examining the relationship between functional traits and demographic vital rates \citep{Adler2013} as well as fitness/niche differences \citep{Kraft2014}, but whether any generalisations can be made as to the precise scaling of these relationships remains an unknown. \citet{Kraft2015} only tested for linear relationships owing to the apparent absence of strong non-linear relationships in their data but acknowledged that more complex relationships may occur. This deserves to be a prominent focus of future research in trait-based approaches in community ecology. 

In Chapter 4, I used standard phylogenetic and functional-trait metrics to evaluate the temporal stability of community structure through succession. Contrary to paradigmatic assumptions relating to both the changing importance of competition through succession and the community-level signature of competitive interactions, community structure did not become increasingly functionally and phylogenetically over-dispersed with time. Again, from the results alone it is unclear to what extent this reflects a diminished role for competition in this system, or rather simply a misconception in how competition structures communities \citep[\textit{sensu}][]{Mayfield2010}. Certainly, the circumstantial evidence for the latter has grown rapidly in recent years, with numerous studies re-emphasizing the role of equalizing mechanisms (and not just stabilizing mechanisms) in mediating coexistence. Nevertheless, the observed temporal increase in positive patterns of co-occurrence presented in Chapter 5 (Fig. 5.6) does lend credibility to the notion that the importance of competition may indeed also tail-off late in succession. 

Despite being central to two chapters (or perhaps as a consequence of it), my overall enthusiasm for inferring community assembly processes from phylogenetic patterns has waned over the timeline of this thesis. With the possible exception of the most diverse, hard to study, systems, it is difficult not to conclude that the fragility and contingency of the underlying assumptions outweighs the attractive simplicity of the approach. As a brief scan of the recent literature will attest, this is by no means an original perspective. Nevertheless, I would argue that the widespread adoption of community phylogenetic approaches over the last 10-15 years has still been a net gain for the field. Generally, it is only once the uptake of a given method reaches a critical mass that the assumptions come into sharp focus. In the case of community phylogenetics, it seems this critical process has fostered much broader exposure to fundamental theory amongst empirically minded ecologists, whilst also motivating a timely mini-synthesis of current eco-evolutionary theory. At the same time, it is also important not to lose sight of other motivations for combining phylogenetic and community-level data, e.g. to identify how coexistence and community assembly control macro-evolutionary processes \citep{Gerhold2015}  

While still strictly an observational study, of all the chapters in this thesis, Chapter 5 arguably provided the tightest link between pattern and process. Whereas Chapter 4 considered temporal patterns in community structure independent of environmental factors other than time since fire, the aim of Chapter 5 was to explicitly model the environmental basis for observed patterns of co-occurrence. To this end, a strength of Chapter 5 was the adoption of a latent variable model approach. As illustrated with R code in Appendix D, latent variable models provide a robust means of decomposing non-random co-occurrence patterns into that which is attributable to measured and unmeasured factors. Nevertheless, even the most sophisticated models can only go so far. Niche dimensions modelled using observed abundances reflect species' realised niches, but say little about their potential breadth in the absence of competition. Ultimately it is not clear from these results to what extent spatial heterogeneity in soil moisture (or other factors) stabilizes coexistence or merely `filters' species into their respective fundamental niches. 

Documenting significant patterns of co-occurrence is undoubtedly an important first step, but as of yet I am unaware of any strict tests of species coexistence under soil moisture heterogeneity. To this end, it is necessary to not only demonstrate species-specific environmental responses, but also that the effect of the environment and competition positively covary such that intraspecific competition is concentrated relative to interspecific competition \citep{Chesson2000c, Chesson2000, Sears2007a, Adler2007}. In an effort to draw an explicit link between differences in functional traits, relevant physiological processes and the underlying coexistence mechanism, an ideal future study would investigate the extent to which environment-competition covariance is positively related to differences in traits related to plant water usage and stress tolerance \citep{Adler2013}.
 
A feature of soil moisture that makes it such a worthy focal point of plant coexistence research is its dual role as both a depletable resource factor (cf. nutrients or light) subject to density-dependent feedbacks and as a non-resource factor (cf. temperature, soil type or topography) mediating density-independent responses. As such, depending on supply rates, spatial heterogeneity in soil moisture will likely produce complex interactions between niche and fitness differences amongst interacting species. While there is some evidence that opportunities for coexistence may be greater for heterogeneity in non-resource factors \citep{schoolmaster2013}, it might be expected that the breadth of plant responses to soil moisture will allow for higher-dimensional trade-offs relative to more narrowly defined environmental factors. Isolating the underlying physiological processes and trade-offs should be a high priority. As recently argued by \citet{Silvertown2015}, the time is ripe for an integrated research programme into hydrological niches drawing on the combined expertise of community ecologists, plant-physiologists, and hydrologists. 

\subsection*{Shifting scales and pass\'{e} paradigms: concluding remarks} 

My overall goal in this thesis was to explore alternative spatio-temporal perspectives on the relationship between heterogeneity and plant community assembly. While the individual chapters provide unique insights, the thesis as a whole admittedly only scratched the surface of this broad overarching theme. Nevertheless, I hope the reader will indulge me in extrapolating two concluding points that emerge from the chapters examined in unison. 

\begin{enumerate}
  
\item Subtle shifts in the scales at which we examine spatial and temporal heterogeneity can yield new insights into patterns of diversity and their organizing processes.

\item Reliable mapping of processes from patterns requires that paradigmatic assumptions are subject to continuous theoretical and empirical scrutiny, and are updated accordingly.  
 
\end{enumerate}
 

%At a recent social gathering, I was asked by an acquaintance, aware of my impending intention to submit my thesis, 'what had I found?'. Until that point, I thought the worst question you could ask a doctoral student was what their thesis was about. However, to be confronted with the expectation of tangible bite-sized contributions to existing knowledge took me off-guard. At the time I muttered some  
 
%Chapter 5 accusations of statistical machismo, but in multivariate data the linear algebra and eigen vectors employed in distance based measures are arguably far more abstract then the  
%
%
%
%The lost experimental chapter...

%\section{Physics envy}
%
%% Frustrated by the difficulties involved in inferring processes from patterns...
%%I have a yearning for experimental work..but I imagine that once immersed in it for a while I'll get a strong to return to 'big picture' observational studies...
%
%Limitations
%
%Chapter 2: better statistical methods, probably a range of other factors that covary with climate variability and may be the primary driver of species richness patterns...
%- not an information theoretic approach, at the time (fresh from a break from academia) I thought IT was synonymous with AIC but no..
%
%Chapter 3\&4 Long been recognised in parts of the literature that ecologically similar species may be expected to coexist...but took a while for anyone to spell it out directly...
%
%Filtering of fitness differences is the same as environmental filtering! This doesn't seem to be very clear in the literature...
%
%Although well understood in parts of the theoretical literature \citep{Abrams1986, Chesson2000, Leibold2006}, the notion that competition may also foster ecological similarity amongst coexisting species was for a long-time poorly recognised. \citet{Mayfield2010} were not the first to make this observation, but in addressing the issue head-on, their paper has a had a big impact on the field. My only criticism in the way Chessonian principles have been brought to bear on phylogenetic and functional-trait based approaches, is the juxtaposition of fitness difference with respect to environmental filtering. Fitness differences and environmental filtering are essentially the same thing! The only difference is we don't speak about fitness differences in the context of species that never coexist i.e. that don't have overlapping fundamental niches. 
%
%Fitness differences are more akin to the Hutchinsonian niche...whereas as niche differences are more akin to the Eltonian niche...
%
%%
%\section{Physics envy}
%
%Anecdote about Jeremy Midgeley re ecology being much closer to the edge than say Physics...but I think this is a misconception and it is the disorganisation of ecology that presents a semblance of being close to the edge. Too much reinventing of the wheel...not consciously, but of a product of too much literature and the lack of a coherent literature that establishes a baseline knowledge...
%
%It is tempting to infer that this is a latter-day phenomena but even the authors of one of ecology's most seminal texts \citep{macarthur1967theory} seemingly overlooked the analogues between their own theory and one developed a quarter of a century earlier (\citealp{wright1940}; see discussion in \citealp{vellend2009island}).
%
%
%an overhal of the curricula is needed - perhaps organised around the framework of Vellend...perhaps the US/South African systems are better with their longer PhD times and early emphasis on coursework...  
%
%I feel that I am now considerably closer to the edge, but it's taken a lot of dead ends... 
%
%cite Egler article ESA bulletin 1986
