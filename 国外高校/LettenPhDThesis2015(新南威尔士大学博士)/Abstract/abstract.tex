% ************************** Thesis Abstract *****************************
% Use `abstract' as an option in the document class to print only the titlepage and the abstract.
\begin{abstract}

Although environmental heterogeneity has been a focal point of ecological research for decades, we have tended to stick to comparatively formulaic interpretations of its dimensionality. In this thesis I explore alternative perspectives on the relationship between heterogeneity and plant community assembly in both spatial and temporal dimensions. Beginning with perhaps the most basic property of an ecological community, its taxonomic diversity, in Chapter 2 I investigate the explanatory power of fine-scale spatial heterogeneity in temporal climate variability. Consistent with coexistence theory, the results of this study indicate that climate variability can be a better predictor of richness than more commonly used climatic averages, and highlight the need for ecologists to expand their purview beyond absolutes and averages. Turning to multivariate distance-based descriptors of community structure, in Chapters 3 and 4, I consider the under-explored role of temporal scale and variability on inference in community phylogenetic and trait-dispersion studies. Given a classic Brownian model of trait evolution, in Chapter 3, I show that the expected functional displacement of any two taxa is most parsimoniously represented as a linear function of time's square root. On this basis I argue that existing methods overweight deep time relative to recent time. Taking into account this methodological adjustment, in Chapter 4, I use standard phylogenetic and functional-trait metrics to evaluate the temporal stability of community structure through succession in a fire-prone heathland. Contrary to widely-held assumptions, community structure did not become increasingly functionally and phylogenetically dispersed with time. This contributes to an emerging body of evidence indicating that limits to the similarity of coexisting species are rarely observed at fine-scales. In Chapter 5, I adopt a model-based approach to the analysis of species co-occurrence patterns in response to fine-scale spatial environmental heterogeneity. This study confirms the vital role of hydrological niches for the maintenance of within-community plant diversity. Two primary conclusions emerge from the thematically broad investigations within the thesis: subtle shifts in the scales at which spatial and temporal heterogeneity are examined can yield new insights into community assembly; and reliable mapping of processes from patterns requires the continuous scrutiny of paradigmatic assumptions.

\end{abstract}

