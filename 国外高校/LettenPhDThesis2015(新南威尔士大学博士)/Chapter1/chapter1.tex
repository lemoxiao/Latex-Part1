%*******************************************************************************
%*********************************** First Chapter *****************************
%*******************************************************************************

\graphicspath{{Chapter1/Figs}}


\chapter{Introduction}  %Title of the First Chapter

\section{Pattern, process and heterogeneity} 

Ecologists like to think about heterogeneity. Since beginning my doctoral studies in 2011, 350 articles containing the word `heterogeneity' in the title have been published in the peer-reviewed ecological literature\footnote{According to Web of Science.}. At a glance, such an impressive number could be taken as evidence of an emerging `hot-topic' or perhaps even a bandwagon \citep{bandwagons}, and yet the concept has pervaded ecological discourse in some shape or form for decades \citep[e.g.][]{hutchinson1961, wiens1977, connell1978, kotler1988, li1995}. Given such longevity, you may expect ecologists to have a pretty good handle on the topic. The reality is that we do and we don't. Whilst theoreticians have made significant progress, particularly with respect to understanding the role of spatial and temporal environmental heterogeneity on species coexistence \citep{Shmida1984, Chesson1985, Chesson2000, Chesson1997, Amarasekare2003}, empiricists arguably have had a harder time of it. Historical contingencies \citep{belyea1999, Chase2003a, Fukami2005}, system-specific properties \citep{lawton1999}, and inconsistencies across different spatial and temporal scales \citep{levin1992, Swenson2006} are just some of the factors that can obfuscate clear lines of inference and generalisation. It is the last of these `issues' that has been my preoccupation for the last four years, and that provides the overarching, albeit heterogenous, focus of this thesis.

The concept of environmental heterogeneity is multi-dimensional in both scope and scale. Probably the first distinction that needs to be made is between spatial and temporal heterogeneity. Both are considered important for diversity and are to some degree analogous in their effect on species coexistence. Too little of it and one or two species out-compete the rest; too much of it, and no species are favoured over a sufficient spatial extent, or length of time, to maintain a stable population \citep{Levine2004, Adler2008, Shurin2010}. Dig a little deeper into the heterogeneity hierarchy and things get a little murkier. The perceived amount of heterogeneity in any given community, landscape or region is entirely scale dependent. It will vary depending on the extent of the area under study, the length of time it is under observation and the spatial and temporal grain of the analysis. It gets particularly interesting when \textit{spatio-temporal} processes are added to the mix. For instance, to what extent does temporal variability exhibit spatial heterogeneity, or conversely to what extent is spatial heterogeneity temporally stable? Measuring even a small subset of these diverse properties at community and/or ecosystem scales, and relating them to assembly processes, has long been both a logistical and analytical challenge. Fortunately, with the recent availability of inexpensive instrumentation, the development of sophisticated statistical models, and the accumulation of spatially broad, long-term datasets, ecologists are currently in a position to make significant progress in our understanding of heterogeneity across multiple scales and dimensions.

A common theme running through each chapter of this thesis is how we perceive heterogeneity/variability/dissimilarity, and how this bears on how we interpret community assembly and species diversity. With the exception of Chapter 4, which is taxonomically neutral, a further unifying thread is the usage of observational data derived from a range of terrestrial plant communities in southeast Australia. There has long been an antagonism in community ecology research between the mechanistic insight afforded by manipulative experimentation versus the potential for broader-scale inference derived from less resource-intensive observational studies. The case against the latter typically invokes the inherent uncertainty in inferring processes from observed patterns \citep{weiher2001, Gotelli2006, Vellend2014}. In recent years, several authors have put forward cogent arguments outlining the potential for misattributing different ecological processes to observed patterns in community ecology, particularly in relation to the increasingly popular trait-based and phylogenetic approaches described in greater detail below \citep{Cavender-Bares2009, Mayfield2010, Fox2012a, Adler2013, Kraft2014}. At the same time, there has been a string of exemplary manipulative experiments and long-term demographic studies that have helped bridge the gap between (coexistence) theory and data \citep{Adler2006, Angert2009, Levine2009, Adler2010, Narwani2013, Fritschie2013, Godoy2014, Alexandrou2015, Kraft2015}. Together these studies have undoubtedly fostered a more nuanced understanding of community assembly and coexistence, and yet they almost exclusively derive from a rather narrow spectrum of ecological systems (e.g. annual plant and phytoplankton communities) which are particularly amenable to manipulative experiments. In contrast, in many perennial systems dominated by comparatively long-lived species, experimental approaches are less tenable and so we must often rely on observational data. It is also not clear to what degree the experimental findings from annual plant and phytoplankton communities scale-up to more complex systems comprising a much greater variety of life-history strategies. To this end, observational studies remain enormously valuable, but necessitate careful evaluation of the potential processes generating emergent patterns in community structure.

\section{Diversity under spatio-temporal heterogeneity}

Perhaps the most basic property of a community is its richness i.e. the number of species (or other taxonomic unit) observed per unit area. When measured in concert with other attributes of the environment (e.g. climate, disturbance, topographical heterogeneity etc.), it can provide insight into community structure and the processes driving it. If local diversity is high, we may infer some (multi-variate) property of the environment acts to stabilize the coexistence of a large number of species. In contrast, if diversity is low we may infer either that the environment falls beyond the fundamental niche of most species, or that it favours a minority of species to such a degree that they competitively exclude other would-be inhabitants. Given the comparative ease with which richness data can be obtained, a vast literature has built up exploring the role of heterogeneity as a driver of species richness across a variety of scales \citep[reviewed in][]{Stein2014}. In particular, there has been considerable interest in how spatial variation in climatic conditions, such as temperature and rainfall, regulates species richness \citep[e.g.][]{O'Brien1998, Francis2003, Currie2004, Kozak2012}. While these studies have tended to focus primarily on mean climatic conditions, an emerging body of theory and empirical evidence suggests that local scale temporal climate variability may also play an important role in stabilizing species coexistence and thus fostering diversity. Nevertheless, few studies have compared the predictive capacity of temporal climate variability with respect to spatial patterns in species richness. To this end, in Chapter 2, I explore the relationship between temporal climate variability and plant diversity using a large dataset (2,400 standardized floristic plots) from temperate forests in Southeast Australia, together with fine-resolution climate grids derived from data collected over two years by near surface climate loggers. The main objective of the study was to compare the relative strength of local-scale temporal temperature variability and absolute temperature, as predictors of regional scale spatial patterns in plant diversity. As part of the analysis, I also consider the shape of the relationship between temperature variability and species diversity in light of various predictions drawing on coexistence theory and population viability analysis.

\section{Community composition, heterogeneity and temporal scale}

Whilst species richness provides an effective first approximation of community structure, the distillation of community structure into a single metric comes at the cost of a tremendous amount of potentially useful information. For instance, we may also want to know what types of species are in the community; to what extent do they share similar life-history strategies; or what are their relative abundances and patterns of dominance/rarity. This type of multivariate information can potentially be used to make more direct inferences on the kinds of processes structuring the community. In particular, over the last 15-20 years there has been a rapid proliferation in trait-dispersion analyses, where community-level information on functional traits is used as a means of differentiating the relative contribution of biotic (e.g. competition) vs. abiotic processes (e.g. environmental filtering) to community structure \citep[e.g.][]{Weiher1995, Weiher1998, Cavender-Bares2004a, Cornwell2009, Kraft2010}. This in turn has given rise to the closely related, and hugely popular, phylogenetic approaches to community structure analysis, where evolutionary relatedness is used as a proxy for the functional distance between taxa \citep{Webb2000, Tofts2000, Slingsby2006, Cavender-Bares2006, Purschke2013}. 

The classical paradigm, dating back to Darwin, holds that competition should inhibit species with high niche overlap from coexisting \citep[limiting similarity \textit{sensu}][]{macarthur1967}, while environmental filtering has the opposite effect of limiting the range of successful ecological strategies at any one location \citep{Weiher1995,Diaz1998, Stubbs2004}. If niche overlap can be approximated by the distance between functional traits, which in turn can be approximated by phylogenetic relatedness, then we might infer that communities comprising of functionally similar or closely-related species are a product of strong environmental filters, whilst highly functionally and/or phylogenetically dispersed communities are indicative of strong competitive interactions \citep{Webb2000, Webb2002}. This intuitively appealing line of reasoning has been adopted in numerous studies conducted in recent years, but as increasingly recognised, such straight-forward inference is not always justified. 

Even before functional and phylogenetic based approaches gained momentum in the early 2000's, several authors had noted that the concept of limiting similarity was over-simplistic. Specifically, it ignores the concurrent role of competition in 'filtering' out all but the best competitors in the community, thus also limiting the \textit{dissimilarity} of coexisting species \citep[e.g.][]{Abrams1986, Leibold1998}. More recently, \citet{Mayfield2010} provided a clear articulation of how this duality pertains to community phylogenetic approaches. Drawing on the coexistence framework laid out by \citet{Chesson2000}, \citet{Mayfield2010} argued that because the outcome of competition depends on both niche and fitness differences (with the former favouring coexistence and the latter driving exclusion), competition may favour the coexistence of both distantly and closely related species, contingent on the relative strength of stabilizing mechanisms (those that enhance niche differences) or equalizing mechanisms (those that reduce fitness differences). A further complication was recently proffered by \citet{Godoy2014} who showed that phylogenetic signal in niche and fitness difference may not always be correlated.

Notwithstanding these shortcomings, the widespread adoption of phylogenetic and functional based approaches has undoubtedly fostered a timely revival of research at the interface of ecology and evolutionary biology \citep{Emerson2008, Cavender-Bares2009, Vamosi2009}. Despite their intrinsic links, as researchers have become increasingly specialised in their work, ecology and evolutionary biology have to some extent grown artificially apart. Indeed, that it took so long for some of the weaker assumptions in community phylogenetics to be widely acknowledged likely points to both the paucity of deep phylogenetic understanding amongst contemporary ecologists, and the lack of exposure to modern coexistence theory amongst evolutionary biologists. As such, rather than abandon these approaches altogether, there is considerable merit in working towards the development of more robust forms of analysis and inference. This is particularly true given that functional traits and phylogenetic information may remain for some time the only logistically accessible sources of data on community assembly in complex and highly diverse systems, e.g. the tropical forests on which \citet{Webb2000} originally pioneered the community phylogenetic approach.   

In the interest of working towards a more nuanced understanding of the interaction between ecological and evolutionary processes, Chapters 3 \& 4 of this thesis consider the comparatively weakly explored role of temporal scale and variability on inference in community phylogenetic and trait-dispersion studies. More specifically, Chapter 3 questions the adequacy of using divergence time as a linear approximation of functional distance between pairs of taxa, and via simulations of community assembly offers a simple solution to the overweighting of early-diverged clades. In contrast, Chapter 4 evaluates the stability of phylogenetic and functional community structure over ecological time-scales. To date, most studies of phylogenetic and functional community structure have been limited to a single snapshot in time, where by necessity observed patterns are assumed to be indicative of dominant assembly processes through time. While the robustness of this assumption seems plausible in long-lived late successional systems (e.g. tropical forests), questions remain as to how stable measures of community structure are through time in more frequently disturbed communities, and what this means for inference. To explore this problem, I investigated within- and between-community measures of phylogenetic and functional community structure in a fire-prone heathland along a 21-year time-series. The results of this study are interpreted in light of the conflicting predictions forwarded by `classical' and more recent perspectives on non-random compositional patterns.    

\section{Modelling fine-scale responses to heterogeneity}

While the analysis of community-level phylogenetic and functional trait data provides a deeper level of inference than is possible from investigations of species richness alone, there is still a significant element of potential information loss that comes with collapsing multivariate data into summary metrics (e.g. the distance-based approaches used in Chapters 3 \& 4). Indeed, different species, or even different individuals within species, may be expected to respond in complex ways to the same environmental gradient. In recent years,  the importance of variability in inter- and intra-specific responses at fine spatial and temporal scales has come under greater scrutiny \citep[e.g.][]{Sears2007, Fridley2011, Dwyer2015, Lai2015}. One of the current challenges is the development of analytical methods that are able to delineate individualistic responses at fine-scales.

An emerging array of model-based approaches to community level analysis has the potential to provide more transparent and statistically robust lines of inference for multivariate data-sets \citep{Ovaskainen2010, Ives2011, Warton2012, Clark2013, Pollock2014, Hui2014, Harris2015}. In particular, there has been several recent studies advocating the use of models for disentangling patterns of co-occurrence and their abiotic and/or biotic drivers \citep{Ovaskainen2010, Pollock2014, Harris2015}. What distinguishes these multivariate co-occurrence models apart from conventional null-randomisation based approaches is that they provide a direct means of assessing shared environmental responses separately from other processes that may generate non-random patterns of co-occurrence. Moreover, they facilitate more transparent and accurate representations of the statistical properties of the data \citep[e.g. overdispersion of counts,][]{Warton2012}, as well as provide a means of accounting for complexity and uncertainty in a hierarchical framework \citep{Cressie2009}.

In Chapter 5, I adopt a novel model-based approach to co-occurrence analysis in order to evaluate the comparative importance of fine-scale hydrological niche segregation for plant species co-occurrence. Although theory suggests fine-scale heterogeneity can promote species coexistence \citep{Chesson2000, Amarasekare2003, Snyder2004}, owing to a dearth of studies at sufficiently high spatial resolution, the empirical evidence remains comparatively sparse. One fine-scale axis of differentiation, for which there is growing evidence of potentially wide taxonomic and geographic generality, is the partitioning of species along fine-scale hydrological gradients \citep{Silvertown2015}. This evidence, however, stems largely from studies evaluating plant species responses to soil moisture independently of other environmental factors.  In this final chapter, I assess the comparative importance of fine-scale hydrological niche differentiation for species co-occurrence using a high resolution study of soil hydrology and other edaphic variables, coupled with the same long-term study of heathland community dynamics analysed in Chapter 4.


\section{Concluding remarks}

In sum, the individual chapters of this thesis are thematically broad but all address in a different way the role of spatio-temporal scale in investigating ecological and environmental heterogeneity and their interaction. Chapter 2 explores the explanatory power of fine-scale spatial heterogeneity in temporal climate variability as a predictor of species richness; Chapter 3 offers a more parsimonious definition for the scaling of functional trait distance with divergence time in the context of contemporary ecological processes; Chapter 4 evaluates the temporal stability of phylogenetic and functional community structure; and Chapter 5 assesses the importance of fine scale spatial environmental heterogeneity in driving within-community patterns of co-occurrence. With the possible exception of Chapters 3 \& 4, there is no strict linear format to the ordering of the thesis. However, by presenting the chapters in chronological order\footnote{Refers to the order in which the studies were conducted and written-up (Chs. 3 \& 4 were written concurrently).}, the thesis follows a natural progression reflecting my own search for greater mechanistic insight into assembly processes. In the final chapter (Ch. 6), I briefly summarize the contributions of each of the preceding chapters, draw attention to their limitations and where relevant outline opportunities for future enquiry.   

\subsection*{A note on formatting}

Each data chapter is a standalone body of work that has already been published in the peer-reviewed literature \citep[Chapters 2-5;][]{Letten2013, Letten2014, Letten2014a, Letten2015}, with some small modifications specifically for this thesis. As such the formatting of each chapter is tailored to the journal in which it was published. The only departure from journal specific formatting is in the use of a consistent citation style, with references and appendices/supporting information provided at the end of the thesis, rather than at the end of each chapter.  The contribution of each author is stated at the start of each chapter.   

