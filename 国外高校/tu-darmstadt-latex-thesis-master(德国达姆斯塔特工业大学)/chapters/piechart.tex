Figure \ref{fig:co2_wind} shows a pie chart.

\newcommand{\slice}[4]{
  \pgfmathparse{0.5*#1+0.5*#2}
  \let\midangle\pgfmathresult
  % slice
  \draw[thick] (0,0) -- (#1:1) arc (#1:#2:1) -- cycle;
  % outer label
  \node[label=\midangle:#4] at (\midangle:1) {};
  % inner label
  \pgfmathparse{min((#2-#1-10)/110*(-0.3),0)}
  \let\temp\pgfmathresult
  \pgfmathparse{max(\temp,-0.5) + 0.8}
  \let\innerpos\pgfmathresult
  \node at (\midangle:\innerpos) {#3};
}

	\begin{figure}[htb]
		\centering
		\begin{tikzpicture}[scale=3]
		\newcounter{a}
		\newcounter{b}
		\foreach \p/\t in {5/Shot break, 19/Rotor, 7/Gear + Generator, 18/remaining Housing, 20/Tower, 6/Power connection, 8/Basement, 3/Misc, 14/Operation}
		 {
		    \setcounter{a}{\value{b}}
		    \addtocounter{b}{\p}
		    \slice{\thea/100*360}
		          {\theb/100*360}
		          {\p\%}{\t}
		  }
		\end{tikzpicture}
		\caption[Break down of the CO$_2$ emissions of a wind turbine]{Break down of the CO$_2$ emissions of a wind turbine \cite{kaltschmitt2006}}
		\label{fig:co2_wind}
	\end{figure}
