\documentclass[cs4size,a4paper]{ctexart}   
%==================== 数学符号公式 ============
\usepackage{amsmath}                 % AMS LaTeX宏包
\usepackage[style=1]{mdframed}
\usepackage{amsthm}
\usepackage{amsfonts}
\usepackage{mathrsfs}                % 英文花体字 体
\usepackage{bm}                      % 数学公式中的黑斜体
\usepackage{bbding,manfnt}           % 一些图标,如 \dbend
\usepackage{lettrine}                % 首字下沉,命令\lettrine
\def\attention{\lettrine[lines=2,lraise=0,nindent=0em]{\large\textdbend\hspace{1mm}}{}}
\usepackage{longtable}
\usepackage[toc,page]{appendix}
\usepackage{geometry}                % 页边距调整
\geometry{top=3.0cm,bottom=2.7cm,left=2.5cm,right=2.5cm}
%====================公式按章编号==========================
\numberwithin{equation}{section}
\numberwithin{table}{section}
\numberwithin{figure}{section}
%================= 基本格式预置 ===========================
\usepackage{fancyhdr}
\pagestyle{fancy}
\fancyhf{}  
\fancyhead[C]{\zihao{5}  \kaishu 北邮SDN大赛报告模板}
\fancyfoot[C]{~\zihao{5} \thepage~}
\renewcommand{\headrulewidth}{0.65pt} 
\CTEXsetup[format={\centering\bfseries\zihao{-2}},name={第, 章}]{section}
\CTEXsetup[nameformat={\bfseries\zihao{3}}]{subsection}
\CTEXsetup[nameformat={\bfseries\zihao{4}}]{subsubsection}
%================== 图形支持宏包 =========================
\usepackage{subfigure}
\usepackage{graphicx}                % 嵌入png图像
\usepackage{color,xcolor}            % 支持彩色文本、底色、文本框等
\usepackage{hyperref}                % 交叉引用
\usepackage{caption}
\captionsetup{figurewithin=section}
%==================== 源码和流程图 =====================
\usepackage{listings}                % 粘贴源代码
\usepackage{xcolor}
\usepackage{color}
\definecolor{dkgreen}{rgb}{0,0.6,0}
\definecolor{gray}{rgb}{0.5,0.5,0.5}
\definecolor{mauve}{rgb}{0.58,0,0.82}
 \usepackage{xcolor}
 \lstset{
  %行号
    numbers=left,
    %背景框
    framexleftmargin=8mm,
    frame=none,
     %背景色
    %backgroundcolor=\color[rgb]{1,1,0.76},
     backgroundcolor=\color[RGB]{245,245,244},
     %样式
   keywordstyle=\bf\color{blue},
   identifierstyle=\bf,
    numberstyle=\color[RGB]{0,192,192},
    commentstyle=\it\color[RGB]{0,96,96},
   stringstyle=\rmfamily\slshape\color[RGB]{128,0,0},
   %显示空格
    showstringspaces=false
 }


%--------------------
\hypersetup{hidelinks}
\usepackage{booktabs}  
\usepackage{shorttoc}
\usepackage{tabu,tikz}
\usepackage{float}

\usepackage{multirow}



\tabcolsep=1ex
\tabulinesep=\tabcolsep
\newlength\tikzboxwidth
\newlength\tikzboxheight
\newcommand\tikzbox[1]{%
        \settowidth\tikzboxwidth{#1}%
        \settoheight\tikzboxheight{#1}%
        \begin{tikzpicture}
        \path[use as bounding box]
                (-0.5\tikzboxwidth,-0.5\tikzboxheight)rectangle
                (0.5\tikzboxwidth,0.5\tikzboxheight);
        \node[inner sep=\tabcolsep+0.5\arrayrulewidth,line width=0.5mm,draw=black]
                at(0,0){#1};
        \end{tikzpicture}%
        }

\makeatletter
\def\hlinew#1{%
  \noalign{\ifnum0=`}\fi\hrule \@height #1 \futurelet
   \reserved@a\@xhline}
   
\newcommand{\tabincell}[2]{\begin{tabular}{@{}#1@{}}#2\end{tabular}}%

\usepackage{subfigure}

\usepackage{CJK}
\usepackage{ifthen}


\usepackage{graphicx} 
\newcommand{\HRule}{\rule{\linewidth}{0.5mm}}

\newtheorem{Theorem}{定理}
\newtheorem{Lemma}{引理} 
%%使得公式随章节自动编号
\makeatletter
\@addtoreset{equation}{section}
\makeatother
\renewcommand{\theequation}{\arabic{section}.\arabic{equation}}

%-------------------------
	
\usepackage{pythonhighlight}
\usepackage{tikz}                    
\usepackage{tikz-3dplot}
\usetikzlibrary{shapes,arrows,positioning}
%===================   正文开始    ===================
\begin{document}
\bibliographystyle{gbt7714-2005}     %论文引用格式
%===================  定理类环境定义 ===================
\newtheorem{example}{例}              % 整体编号
\newtheorem{algorithm}{算法}
\newtheorem{theorem}{定理}            % 按 section 编号
\newtheorem{definition}{定义}
\newtheorem{axiom}{公理}
\newtheorem{property}{性质}
\newtheorem{proposition}{命题}
\newtheorem{lemma}{引理}
\newtheorem{corollary}{推论}
\newtheorem{remark}{注解}
\newtheorem{condition}{条件}
\newtheorem{conclusion}{结论}
\newtheorem{assumption}{假设}
%==================重定义 ===================
\renewcommand{\contentsname}{目录}     
\renewcommand{\abstractname}{摘要} 
\renewcommand{\refname}{参考文献}     
\renewcommand{\indexname}{索引}
\renewcommand{\figurename}{图}
\renewcommand{\tablename}{表}
\renewcommand{\appendixname}{附录}
\renewcommand{\proofname}{证明}
\renewcommand{\algorithm}{算法} 
%============== 封皮和前言 =================
\thispagestyle{empty}
\begin{center}
    \includegraphics[scale=0.7]{figures/cdut.png}
\end{center}
\vskip1.5cm
\begin{center}
    \makebox[109mm][s]{\heiti\zihao{-0}\bf 本科生实验报告}
\end{center}
\vskip2cm
\begin{center}
    \makebox[20mm][s]{\heiti\zihao{4} 实验课程}\underline{\makebox[130mm][c]{\heiti\zihao{3}\LaTeX 书写行为规范}}\\
    \vskip1cm
    \makebox[20mm][s]{\heiti\zihao{4} 实验名称}\underline{\makebox[130mm][c]{\heiti \zihao{3} 利用\LaTeX 书写成都理工大学实验报告}}\\
    \vskip1cm
    \makebox[20mm][s]{\heiti\zihao{4} 专业名称}\underline{\makebox[130mm][c]{\heiti\zihao{3} 专业全称(有专业方向的用小括号标明)}}\\
    \vskip1cm
    \makebox[20mm][s]{\heiti\zihao{4} 学生姓名}\underline{\makebox[130mm][c]{\heiti\zihao{3} 您的姓名}}\\
    \vskip1cm
    \makebox[20mm][s]{\heiti\zihao{4} 学生学号}\underline{\makebox[130mm][c]{\heiti\zihao{3} 您的学号}}\\
    \vskip1cm
    \makebox[20mm][s]{\heiti\zihao{4} 指导教师}\underline{\makebox[130mm][c]{\heiti\zihao{3} 您的授课或者指导老师}}\\
    \vskip1cm
    \makebox[20mm][s]{\heiti\zihao{4} 实验地点}\underline{\makebox[130mm][c]{\heiti\zihao{3} 授课地点(如:6C403)}}
    \vskip1cm
    \makebox[20mm][s]{\heiti\zihao{4} 实验成绩}\underline{\makebox[130mm][c]{\heiti\zihao{3} 由指导老师书写}}\\
\end{center}
\vskip1.85cm
\vfill\begin{center}
    {\songti\zihao{3}二〇一八年三月二十一日}
\end{center}
\newpage
\thispagestyle{empty}
\tableofcontents
\newpage
\setcounter{page}{1}
\pagestyle{plain}
\pagenumbering{Roman}
%# -*- coding: utf-8-unix -*-
%%==================================================

\begin{abstract}
本项目为年产50万吨MTO工厂的初步设计。通过分析当前国内外MTO生产和研究现状,对生产工艺进行了选择论证。然后运用Aspen软件模拟初步的工艺流程,并通过对一系列工艺参数,如精馏塔的塔板数—产品纯度、进料塔板数—产品纯度、产品纯度—回流比、再沸器负荷—回流比等进行灵敏度分析,优化设备操作条件,提高工艺的合理性和经济性。本设计还针对工艺流程进行换热网络设计和对全局换热网络进行了优化和评估,通过内部流股之间相互换热以减少公用工程的消耗,最终优化后节约$79.4\%$的热公用工程资源和$73.7\%$的冷公用工程资源。本设计还运用水夹点技术优化了用水网络,根据水硬度分类处理水操作单元,并合理再生利用,使得本项目新鲜水用量和废水排放量达到最小,优化后的用水网络节约用水$53.59\%$。本设计对于MTO工厂的生产和设计建造具有一定的现实指导意义。\\

\keywords{\zihao{-4} 工厂\quad 设计\quad MTO \quad 工艺 \quad 水夹点  \quad 网络 \quad 控制}
\end{abstract}

\begin{englishabstract}

This project is the preliminary design of a MTO plant with an annual output of 500,000 tons of light olefins. Based on the current production and research situation all through the world, the production method was selected and demonstrated. Aspen software was used to simulate the preliminary process. Heat integration method was applied to optimize the heat exchange network. Rational heat exchange between process streams were suggested which resulted in the decreasing of utilities consumption and exchanger number. The heat integration leaded to energy saving of $79.4\%$ of heat utilities and $73.7\%$ of the cold utilities. In addition, the water pinch technology was also implemented to optimize the water network. The water operating unit was classified according to water hardness, with a reasonable recycling. The amount of fresh water consumption and wastewater emission was minimized. The optimized water network achieved $53.59\%$ water saving. Finally, a preliminary economic analysis to the entire project was estimated in order to get the project construction cost and profitability. In summary, this design is of some practical significance for the production and design of the MTO industry.

\englishkeywords{\zihao{-4} Plant design\;Sensitivity analysis  \; Energy balance\; calculation \; Water pinch  Dynamic control}
\end{englishabstract}


\pagestyle{empty}
\tableofcontents 
\thispagestyle{empty}
%============== 论文正文   =================
\pagestyle{fancy}
\bodychapter{INTRODUCTION: A REALLY LONG CHAPTER TITLE THAT SPANS MULTIPLE LINES SHOULD BE SINGLE-SPACED IN THE TEXT}
\label{Intro}

There is a substantial body of work in HCI that guides the evaluation of productivity support tools. Shneiderman compared the growing community of researchers developing and studying creativity support tools to the earlier rise of researchers working on productivity support tools~\cite{Shneiderman:2007wp}. He said that researchers in CSTs are ``moving from the comparatively safe territory of productivity support tools to the more risky frontier of creativity support tools.'' Shneiderman noted that one of the challenges that makes CST research `risky' is that there are no obvious measures of success~\cite{Shneiderman:2007wp}. 

\begin{table}[t]
\centering
\scriptsize
\caption[Overview of Creativity Support Tools]{A summary of creativity support tools, including examples from research and industry.}
\begin{tabular}{|l|l|}
\hline
\textbf{Category} & \textbf{Example} \\
\hline
Visualization \& Simulation  & Tableau, D3, netLogo \\
Concept Mapping \& Information Collage & combinFormation, Visio, Omnigraffle \\
Architectural \& Design & AutoCAD, Rhino3D \\ 
Mathematics & SPSS, MatLab, WolframAlpha \\
Software development environments & Eclipse, Visual Studio \\
Video Editing & Final Cut Pro, iMovie \\
Drawing/Painting &  Illustrator, InkScape, CorelDraw \\
Animation & Flash, Maya, SoftImage, Houdini \\
Music & GarageBand, Zya, Sequel, NodeBeat \\
Photography & Photoshop, Lightroom \\
Wikis, Blogs, \& Online Presence  & MediaWiki, WordPress, DreamWeaver \\
Writing \& Presentation & Google Docs, MS Word, Prezi \\
\hline
\end{tabular}
\label{CSTSummary}
\end{table}%

\begin{figure}[t]
\centering
\includegraphics[width=4in]{images/spectrum.png}
\caption[Novelty-Impact Space of Creativity (Along with some extra text that makes multiple lines)]{The creativity literature contains classifications of creative contributions across two dimensions: the Novelty-Impact space. Highly novel contributions are more rare, contributions with minimal novelty are more frequent.}
\label{NIspace}
\end{figure}

\begin{figure}[t]
\centering
\includegraphics[width=4in]{images/spectrum.png}
\caption{The creativity literature contains classifications of creative contributions across two dimensions: the Novelty-Impact space. Highly novel contributions are more rare, contributions with minimal novelty are more frequent.}
\end{figure}

\bodysection{I have a super super super super super super super super super super super long title}
\bodysubsection{Another super super super super super super super super super super super super super super long title}

\bodysubsection{Evaluation of Creativity Support Tools}
\label{CSTEvaluation}
While there is an extensive history of evaluating creativity, the evaluation of tools to support creativity is a much newer field of study. As previously discussed, Shneiderman noted that the evaluation of creativity support tools is challenging because there are no obvious metrics for researchers to quantify~\cite{Shneiderman:2007wp}. 
      %
\chapter{The importance of temporal climate variability for spatial patterns in plant diversity}
\chaptermark{Temporal climate variability and plant diversity}

\graphicspath{{Chapter2/Figs/}}

\begin{center}

{\large Andrew D. Letten, Michael B. Ashcroft, David A. Keith, John R. Gollan and Daniel Ramp}

\small\textit{\textbf{Ecography}} \textbf{(2013), 36: 1341-1349}\\
\url{http://doi.wiley.com/10.1111/j.1600-0587.2013.00346.x}

\vspace{1in}
\includegraphics[width=0.15\linewidth]{Chapter2/Figs/mountain}

\vfill
This study was conceived by ADL with input from MAB, DAK, JRL \& DR. MAB and JRL collected climate data and modelled surfaces. ADL compiled floristic data, conducted analyses and wrote the manuscript, with contributions from MAB, DAK, JRL \& DR. 

\end{center}

\newpage
\section{Abstract}

Spatial variation in absolute climatic conditions (means, maxima or minima) is widely acknowledged to play a fundamental role in controlling species diversity patterns. In contrast, while evidence is accumulating that variability around mean climatic conditions may also influence species coexistence and persistence, the importance of spatial variation in temporal climatic variability for species diversity is still largely unknown. We used a unique dataset capturing fine-scale spatial heterogeneity in temperature variability across 2,490 plots in Southeast Australia to examine the comparative strength of absolute temperature and temperature variability in explaining spatial variation in plant diversity. Across all plots combined and in three of five forest types, temperature variability emerged as the better predictor of diversity. In all but one forest type, diversity also exhibited either a significant unimodal or positive linear correlation with temperature variability. This relationship is consistent with theory that predicts diversity will initially increase along a climate variability gradient due to temporal niche partitioning, but at an intermediary point, may decline as the risk of stochastic extinction exceeds competitive stabilization. These findings provide critical empirical evidence of a linkage between spatial variation in temporal climate variability and plant species diversity, and in light of changing climate variability regimes, highlight the need for ecologists to expand their purview beyond absolutes and averages.

\newpage
\section{Introduction}

Ecologists have long been interested in the role of environmental gradients in driving patterns of biodiversity \citep{Fox2011}. The relationship between species richness and spatial variation in climatic conditions, such as temperature and rainfall, has in particular received considerable attention \citep[e.g.][]{O'Brien1998, Currie2004, Kozak2012}. While the overarching emphasis has been on richness responses to spatial variation in mean (including maxima or minima) climate variables, recent evidence indicates that deterministic and stochastic elements of variability around mean climatic conditions may play an underappreciated role in species coexistence and persistence, and therefore in the maintenance of species diversity \citep{Chesson2000, Levine2004, Adler2006}. At the same time, the spectre of altered climatic variability regimes under future climates, including more frequent extreme weather events \citep{Easterling2000a, ippc2007, Hansen2012}, is providing an urgent mandate for ecologists to expand their purview beyond mean climate measures \citep{Adler2008, Shurin2010, White2010}.

At broad biogeographic scales, gradients in climatic means undoubtedly have a fundamental role in controlling richness gradients \citep{Francis2003}. But at the local scale, where ecological processes such as competition take increasing precedence, contemporary coexistence theory makes a salient argument for giving greater consideration to fluctuations around mean climate. Specifically, coexistence theory stipulates that under certain conditions variability can stabilize competition, and thus promote diversity, by increasing the number of temporal niches available within a fixed space (temporal niche partitioning) \citep{Chesson1997, Chesson2000}. Competitive stabilization occurs when competing species prosper under different conditions, and have the ability to `bank' fitness gains made during good times via what are termed `storage effects', such as long-lived adults or seed banks \citep{Chesson2000}. At the same time, it is widely recognised by practitioners of population viability analysis that too much environmental variability can be detrimental to population persistence because it reduces long-term population growth rates and increases vulnerability to stochastic extinction \citep{Alvarez2001, Boyce2006}. These differing perspectives make opposing predictions on the long-run trajectory of species diversity under climatic variability \citep{Levine2004}, but in some instances may be expected to generate a richness peak at intermediate levels of climate variability \citep{Adler2008}.

Unlike the rich body of literature providing evidence for broad scale richness-responses along mean climatic gradients \citep[as reviewed in][]{Pausas2001}, evidence relating temporal climate fluctuations to species coexistence and diversity has only recently begun to accumulate \citep{White2010}. The favoured approach has been to analyse historical population dynamics amongst coexisting plant species in temporally variable environments \citep{Levine2004, Adler2006, Adler2009, Angert2009}, but a small number of studies have employed alternative approaches including lab-based microcosm experiments \citep{Descamps-Julien2005, Jiang2007, Tuck2012} and comparative analyses of species richness across sites characterized by different levels of climate variability \citep{Shurin2010}. Notably, in their study of lake zooplankton communities, \citet{Shurin2010} demonstrated that richness increased with increasing temporal variability in water temperature and was more strongly correlated with variability than mean temperature. To our knowledge theirs is the only study to date to relate spatial variation in within-community species diversity to temporal climate variability.

In a recent review, \citet{White2010} highlighted the need for more comparative spatial studies on diversity-variability relationships to complement the traditional emphasis on population dynamics. A novel approach to testing diversity-variability relationships that reduces the confounding effects of large-scale climate patterns is to compare diversity at sites that share the same regional climate but exhibit heterogeneous temporal climate variability at the landscape-scale. For instance, organisms occupying a sheltered gully will typically experience less climatic variability, over various time-scales, than those inhabiting a nearby ridgeline. Capturing this fine-scale heterogeneity in climate variability across landscapes was until recently constrained by the limited availability of accurate climate data at fine spatial scales. Indeed, the coarse-scale grids routinely utilized by ecologists for modelling species distributions are inappropriate for this purpose because they are typically derived from data collected by standardized weather stations (i.e. sparsely distributed Stevenson screens positioned at ${\sim}$1.5--2 m on flat, cleared terrain) which are specifically designed to reduce the effects of landscape features, such as topographic shelter, that generate fine-scale heterogeneity \citep{Ashcroft2012b}. Fortunately, the availability of small, low-cost microclimatic sensors has made it more viable for researchers to produce fine-resolution climate surfaces that consider the effects of a much wider variety of climate forcing factors, including cold air drainage, topographic exposure and canopy cover \citep{Ashcroft2011, Ashcroft2012b}. With these sensors and methods it is possible to produce climate surfaces that are more tightly coupled to the conditions experienced by most organisms. 


Here we provide a unique analysis of the influence of temporal climate variability on species diversity in the terrestrial plant realm. To explore the relationship between temporal climate variability and plant diversity we assembled a dataset comprising more than 2,400 standardized floristic plots from temperate forests in Southeast Australia, together with fine-resolution climate grids derived from data collected over two years by near surface climate loggers \citep[see][]{Ashcroft2012b}. Our two main objectives were: (1) to compare the relative strength of temperature variability and absolute temperature, alongside other climate variables, as predictors of plant species diversity; and (2) to investigate the shape of the relationship between temperature variability and species diversity; i.e. does it exhibit consistency with either the predictions of coexistence theory (positive monotonic relationship), population viability theory (negative monotonic relationship) or a unified unimodal model \citep[\textit{sensu}][]{Adler2006}.

\section{Materials and Methods}

\subsection{Study area}

The study area comprises approximately 60,000 km2 in the Hunter Valley region of New South Wales, Australia (Fig. \ref{fig:studyarea}). Although parts of the region have been cleared or modified for agriculture and mining, there remain large patches of native vegetation within protected areas, including parts of two world heritage sites. These relatively undisturbed areas are topographically complex with an altitudinal range of more than 1,400m. The vegetation is dominated by dry and wet sclerophyll eucalypt forest, with smaller areas of heath, upland swamp, and temperate rainforest \citep{Keith2004}.

\begin{figure}[H]
\centering
\includegraphics[width=1.0\linewidth]{./LettenetalEcographyRevision-img001}
\caption{Map of greater Hunter Valley study area (31.4--33.4{\textdegree}S, 149.4--152.6{\textdegree}E) showing locations of survey plots (white triangles) and climate loggers (black circles) overlaid on topoclimatic grid (25 x 25 m) of near-surface average variability in maximum temperature. Light to dark shading indicates transition from variable to stable localities. }
\label{fig:studyarea}
\end{figure}


\subsection{Floristic data}

Floristic data was assembled from the New South Wales Office and Environment and Heritage's YETI 3.2 vegetation plot database (\url{www.environment.nsw.gov.au/research/VISplot.htm}). Before performing analyses, the metadata for all surveys conducted in the study area were intensively scrutinized to ensure included records were from plots of a standard size (0.04 ha) and sampling effort; comprised a complete list of vascular plants; were accurately geo-referenced; and coincided geographically with regions mapped as native vegetation types on digitized maps. After implementing the evaluation criteria, a total of 2490 standardized 0.04 ha full-floristic plot records conducted between 1998 and 2010 were included in the analysis. 

Interspecific competition can occur between plants differing in size by several orders of magnitude. For instance, herbaceous species are known to limit the establishment of tree seedlings or even compete directly with adult trees \citep[e.g.][]{Riginos2009}. All vascular plants were therefore included in the assessment of species richness, which was quantified as the total number of species in each quadrat and ranged from 3 to 108 in any given plot, with a sum total of 2889 species representing 189 families. To evaluate whether observed response patterns varied across different growth-forms, supplementary analyses were performed on richness quantified for two separate groups: trees or arborescent shrubs; and all other growth-forms (hereafter trees and non-trees). Allotment of species into these groups was based on descriptions given in published floras \citep{Harden1993}, and may not reflect the actual growth form at a particular site. We performed all analyses on both the entire dataset and on subsets of the data grouped by forest type, which enabled us to explore whether the relationship between richness and climate variability varies between habitats, as has been shown for richness-disturbance relationships in wet vs. dry forests \citep{Bongers2009}.The exclusion of all forest types (derived from digitized maps) with less than 50 plots allowed for the partitioning of the data into five distinct forest types: grassy woodlands (GW; n = 225), dry sclerophyll forest (DSF; n = 1600), wet sclerophyll forest (WSF; n = 349), rainforest (RF; n = 101), and forested wetlands (FW; n = 215).

\subsection{Environmental data}

For the purposes of the study we refer to absolute temperature as any variable that provides an indication of the extreme or mean temperature at a site, but does not provide any indication of variability around extreme/mean temperature. For instance, two sites might have a mean annual temperature of 20 {\textdegree}C, but while one might experience relatively constant temperatures throughout the year, the other might fluctuate seasonally from a low of 5 {\textdegree}C to a high of 35 {\textdegree}C. Although there are a number of approaches to quantifying absolute temperature, we specifically focused on extreme temperature, rather than mean temperature, as extreme temperature is more tightly coupled with fine-scale topographical heterogeneity \citep{Suggitt2011}; is more likely to affect individual fitness than mean temperature \citep{Stenseth2002, Reyer2012}; and is predicted to fluctuate with greater frequency under climate change \citep{Smith2011}. As such, we defined absolute (extreme) temperature as the 95th percentile of daily maximum temperature and 5th percentile of daily minimum temperature over a one-year period.

For each quadrat location, temperature variables were extracted from a series of fine-resolution (25 m) topoclimatic grids interpolated from 113 climate loggers deployed within the study area for a total of 666 days from June 2009 to May 2011 (Fig. \ref{fig:studyarea}). To produce the grids of absolute temperature and temperature variability, the topoclimatic data was interpolated using a regional regression approach (Daly 2006), which involves fitting a multiple linear regression of temperature variables against climate-forcing factors including elevation, distance to coast, canopy cover, latitude, cold-air drainage, and topographic exposure \citep[see][for full details]{Ashcroft2011}. Variability in maximum temperature was initially partitioned into three time-scales: (i) intra-seasonal variation in maximum temperatures, calculated as the 95th percentile of summer (December--February) maximums minus the 5th percentile of summer maximums; (ii) intra-annual variation in maximum temperatures, calculated as the 95th percentile of summer (December--February) maximum temperatures minus the 95th percentile of winter (June--August) maximum temperatures; and (iii) inter-annual variation in maximum temperatures, calculated as the difference in the 95th percentile of maximum temperatures between the two years. In order to obtain a measure of overall variability in maximum temperature, we then calculated the average of the three variability grids at each locality \citep{Ashcroft2011, Ashcroft2012b}. To enable comparative analyses of temperature variability against other absolute climate variables, we also extracted the raw maximum and minimum temperature and humidity data (95th and 5th percentile of maximum and minimum temperatures) from the topoclimate surfaces, as well as five measures of rainfall derived from BioClim \citep{HoulderDHutchinsonMNixH2003}. Given that variability in minimum temperature was strongly correlated with absolute climate factors, we constrained our measure of variability to variability in maximum temperatures, while also controlling for potential direct effects of canopy cover on species richness by including remotely sensed canopy cover estimates \citep{DECC2008} as covariables in the multiple predictor models. The complete list of variables is provided in (Table \ref{tab:vars}). All explanatory variables were Z-standardized. 

\begin{flushleft}
\begin{table}
\renewcommand{\arraystretch}{1.2}
\caption{\footnotesize Factors available for selection as correlates of species richness in multi-variable models.}
\scriptsize
\begin{tabular}{m{3.4in}m{0.7in}m{1.3in}}
\hline
Variable &
Abbreviation &
Source\\\hline
Average variability in 95th percentile of maximum temperature &
VarMT &
Ashcroft et al., 2012\\
95th percentile of maximum temperature &
MaxT95 &
Ashcroft \& Gollan, 2011\\
5th percentile of maximum temperature &
MaxT5 &
Ashcroft \& Gollan, 2011\\
95th percentile of minimum temperature &
MinT95 &
Ashcroft \& Gollan, 2011\\
5th percentile of minimum temperature &
MinT5 &
Ashcroft \& Gollan, 2011\\
95th percentile of maximum humidity &
MaxH95 &
Ashcroft \& Gollan, 2011\\
5th percentile of maximum humidity &
MaxH5 &
Ashcroft \& Gollan, 2011\\
95th percentile of minimum humidity &
MinH95 &
Ashcroft \& Gollan, 2011\\
5th percentile of minimum humidity &
MinH5 &
Ashcroft \& Gollan, 2011\\
Mean annual precipitation &
AP &
Bioclim\\
Precipitation of warmest quarter &
PWaQ &
Bioclim\\
Precipitation of coldest quarter &
PCQ &
Bioclim\\
Precipitation of driest quarter &
PDQ &
Bioclim\\
Precipitation of wettest quarter &
PWeQ &
Bioclim\\
Canopy cover &
CC &
DECC, 2008\\\hline
\end{tabular}
\label{tab:vars}
\end{table}
\end{flushleft}

\subsection{Analysis}

In order to provide a direct comparison between temperature variability and absolute temperature as predictors of plant species richness (Objective 1), we first fitted single-predictor generalized linear models (GLMs) for species richness as a quadratic and a linear function of each of average variability in maximum temperatures and the 95th percentile of maximum temperatures. This simultaneously enabled us to investigate the shape of the relationship between species richness and climate variability in each forest type (Objective 2). The models were initially fitted with Poisson error-distributions to account for the strictly non-normal distribution of count data, but due to overdispersion (variance {\textgreater} mean) we subsequently corrected the standard errors using a quasi-Poisson GLM model. For all significant quadratic and linear terms, we calculated the quasi-AIC (QAIC) value, a modification of AIC based on quasi-likelihood appropriate for overdispersed response variables \citep{Burn}, and the percent deviance explained by the model as: [1 -- (observed deviance - residual deviance)] x 100. If both linear and quadratic models showed statistical significance, we chose the model with the lowest QAIC value (using the dispersion parameter derived from the global quadratic model). 

To determine how important temperature variability was, compared to other climate variables in predicting species richness patterns across the different forest types, we employed a multi-model information theoretic approach \citep{Burn}. The explanatory variables available for inclusion in the master model were first selected from the pool of climate variables (plus canopy cover) in Table \ref{tab:vars}. Prior to model selection, independent relationships (linear and quadratic) between species richness and each potential explanatory variable were evaluated for significance. To mitigate the problematic effects of collinearity in explanatory variables \citep{Dormann2012}, we evaluated correlations between all significant variables. Wherever two variables were strongly correlated (Pearson's {\textbar}r{\textbar} {\textgreater} 0.7), we excluded the variable exhibiting the weakest (greater QAIC) independent relationship with species richness. 

Model selection was performed using the dredge function in the R package `MuMIN' \citep{Barton2012}, whereby GLMs with quasi-Poisson error distributions were run for all variable combinations in each forest type and were evaluated and ranked according to their QAIC. We intentionally employed such an indiscriminate method in order to make any inference on the relationship between temperature variability (1 factor), versus absolute climate variables (14 factors) (Table \ref{tab:vars}), and species richness as conservative as possible. That is to say our approach ensured we would only conclude there was a significant relationship between richness and temperature variability if no other factors could explain this trend.

To evaluate the predictive power of each explanatory variable, we first constructed a 95\% confidence set of models for each forest type by summing the cumulative Akaike weights, $w_{i}$, of the highest ranked models until the sum exceeded 0.95. The relative importance of each variable was then calculated as the sum of the Akaike weights `$w_{+}(j)$' for all the models in which the variable of interest occurred in the 95\% confidence set. Model averaging was finally applied to determine model averaged parameter coefficients for the 95\% confidence set of models in each forest type \citep{Burn}.


Given the spatially clustered nature of some of the quadrats, we tested for spatial autocorrelation using Moran's I tests. Moran's I tests revealed significant autocorrelation in the residuals of the global models for each habitat type. While there a number of methods for dealing with spatial auto-correlation in normally distributed data, the available methods for dealing with non-normal data are more limited, and either suffer from a lack of precision in their ability to accurately estimate model coefficients \citep{Dormann2007}, or have only been tested on a limited number of datasets \citep{Murphy2010}. However, because the coefficients obtained in our models are almost identical to those obtained assuming normally distributed data (Table S2.1), we were satisfied that running simultaneous autoregressive (SAR) models, an effective approach to removing spatial autocorrelation from residuals assuming normal error distributions \citep{Kissling2007}, would provide a valid test of the effect of spatial autocorrelation on the model results. SAR models were generated using the R package `spdep' (Bivand 2012).

\section{Results}

The percentage variation in species richness explained by temperature variability in the single-predictor models followed an increasing trend from the driest to the wettest (based on annual precipitation) forest types (Table \ref{tab:temp}, Appendix A, Fig. S2.1). Variability in maximum temperature performed better (greater explanatory power) than absolute maximum temperature as a univariate predictor of species richness in all plots combined and in the three wetter forest types but performed poorer in the two driest forest types: grassy woodland and dry sclerophyll forest (Table \ref{tab:vars}). Species richness exhibited a significant (P {\textless} 0.05) unimodal relationship with temperature variability across all plots combined and across all individual forest types, with the exception of forested wetlands and grassy woodlands, which showed a significant positive linear relationship and no relationship respectively (Fig. \ref{fig:humpplots}). Relative performance of the two temperature metrics in predicting tree species richness were similar to those described for all vascular plants in each of the five different forest types, but in all plots combined absolute temperature performed better than temperature variability (12.8 vs. 5.1 percentage deviance explained) (Table S2.4). For all non-tree species, patterns were again similar to those observed for all vascular plants, with the exception of dry sclerophyll forest and rainforest, where there was very little difference in performance between absolute temperature and temperature variability (Table S2.5).

\begin{flushleft}
\begin{table}[H]
\renewcommand{\arraystretch}{1.2}
\caption{\footnotesize Percentage deviance explained by variability in maximum temperature (VarMT) and absolute maximum temperature (MaxT95) as independent predictors of species richness, and the range of percentage deviance explained in the 95\% confidence set of multiple predictor models for each forest type and all plots combined (ALL = all plots combined, DSF = dry sclerophyll forest, WSF = wet sclerophyll forest, GW = grassy woodland, RF = rainforest \ and FW = forested wetlands). \ Non-significant results are shown in brackets. The ratio VarMT/MaxT95 provides a measure of the relative explanatory power of temperature variability and absolute temperature.}
\scriptsize
\begin{tabular}{m{0.4in}m{0.7in}m{0.7in}m{1.5in}m{1.6in}}
\hline
~
 &
\multicolumn{2}{m{1.6851599in}}{\centering Single-predictor models} &
\centering Relative explanatory power &
\centering\arraybackslash Multiple-predictor models\\
~ &
\centering VarMT &
\centering MaxT95 &
\centering (VarMT/MaxT95) &
\centering\arraybackslash 95\% confidence set *\\\hline
ALL &
\centering 4.85 &
\centering 1.00 &
\centering 4.9 &
\centering\arraybackslash 16.19 - 16.49\\
GW &
\centering (0.64) &
\centering 6.56 &
\centering 0.1 &
\centering\arraybackslash 11.11 - 12.03\\
DSF &
\centering 1.94 &
\centering 3.32 &
\centering 0.6 &
\centering\arraybackslash 17.35 - 17.61\\
WSF &
\centering 10.32 &
\centering 3.63 &
\centering 2.8 &
\centering\arraybackslash 13.50 - 16.82\\
RF &
\centering 13.03 &
\centering (4.65) &
\centering 2.9 &
\centering\arraybackslash 29.35 - 38.15\\
FW &
\centering 19.28 &
\centering 10.90 &
\centering 1.8 &
\centering\arraybackslash 26.27 \ {}- 31.10\\\hline
\multicolumn{5}{l}{* see Table S2.3 for models comprising the 95\% confidence set.}
\end{tabular}
\label{tab:temp}
\end{table}
\end{flushleft}

As is to be expected, a number of the climate variables available for model selection were highly correlated ({\textbar}r{\textbar} {\textgreater} 0.7), necessitating the exclusion of the weaker ({\textgreater}QAIC) of each collinear pair (Table S2.2). Variability in maximum temperature generally exhibited weak to moderate collinearity with the other variables, but in dry sclerophyll forest it was strongly negatively correlated with the 95th percentile of minimum temperature (r = -0.704), while in forested wetlands it was strongly negatively correlated with both precipitation of the driest quarter (r = -0.746) and precipitation of the coldest quarter (r = -0.734). In each instance, variability in maximum temperature exhibited the lowest QAIC and thus was retained.

\begin{figure}[H]
\centering
\includegraphics[width=0.85\linewidth]{./thesisversionFig2}
\caption{Relationship between species richness and average variability in maximum temperature (VarMT) in: all plots combined (ALL, n = 2490) (a), grassy woodland (GW, n = 255) (b), dry sclerophyll forest (DSF, n = 1600) (c), wet sclerophyll forest (WSF, n = 349) (d), rainforest (RF, n = 101) (e), and forested wetlands (FW, n = 215) (f). Solid grey lines are the lines of best fit for significant single-predictor GLMs (linear model in FW exhibits slight upward curvature due to log-link function of Poisson regression). Percentage deviance explained by each model is given in Table 2.2.}
\label{fig:humpplots}
\end{figure}


In the multiple-predictor models, variability in maximum temperature emerged as an important predictor of species richness relative to the absolute climate variables across all plots combined and in all forest types with the exception of grassy woodlands (Table \ref{tab:coefs}, Table S2.3). In particular, variability in maximum temperatures was the single most important climate variable explaining species richness in wet sclerophyll forests, and the joint most important variable (together with the 5th percentile of maximum humidity) in forested wetlands. In rainforest, variability in maximum temperature was the joint third most important variable, while in dry sclerophyll forest and all plots combined, almost all the variables, including variability in maximum temperature, occurred in the 95\% confidence set of best models. 

\begin{flushleft}
\begin{table}[t]
\renewcommand{\arraystretch}{1.2}
\caption{\footnotesize Coefficient estimates, standard errors and associated P{}-values (*P {\textless} 0.05, **P {\textless} 0.01, ***P {\textless} 0.001) of model averaged parameters in the 95\% confidence set for each forest type. Summed Akaike weights [$w_{+}(j)$] provide a measure of the importance of each covariate. (ALL = all plots combined, DSF = dry sclerophyll forest, WSF = wet sclerophyll forest, GW = grassy woodland, RF = rainforest \ and FW = forested wetlands.}
\scriptsize
\begin{tabular}{m{0.4in}m{0.3in}m{0.15in}m{0.05in}m{0.1in}m{0.01in}m{0.4in}m{0.3in}m{0.15in}m{0.05in}m{0.1in}m{0.01in}m{0.4in}m{0.3in}m{0.15in}m{0.05in}m{0.1in}}
\hline
ALL &
~ &
~ &
~ &
~ &
~ &
GW &
~ &
~ &
~ &
~ &
~ &
DSF &
~ &
~ &
~ &
~\\
Variable &
\centering Estimate &
\centering SE &
\centering P &
\centering w+(j) &
\centering ~ &
Variable &
\centering Estimate &
\centering SE &
\centering P &
\centering w+(j) &
~ &
Variable &
\centering Estimate &
\centering SE &
\centering P &
\centering\arraybackslash w+(j)\\\hline
\textbf{VarMT} &
\centering \textbf{1.45} &
\centering \textbf{0.33} &
\centering *** &
\centering ~ &
\centering ~ &
PWaQ &
\centering {}-0.79 &
\centering 0.22 &
\centering *** &
\centering ~ &
~ &
\textbf{VarMT }&
\centering \textbf{0.02} &
\centering \textbf{0.32} &
\centering ~ &
\centering\arraybackslash ~\\
\textbf{VarMT$^{2}$} &
\centering \textbf{{}-2.61} &
\centering \textbf{0.22} &
\centering *** &
\centering \textbf{1.00} &
\centering ~ &
PWaQ$^{2}$ &
\centering {}-1.61 &
\centering 0.20 &
\centering *** &
\centering 1.00 &
~ &
\textbf{VarMT$^{2}$} &
\centering \textbf{{}-1.51} &
\centering \textbf{0.20} &
\centering *** &
\centering\arraybackslash \textbf{1.00}\\
MinT5 &
\centering {}-2.84 &
\centering 0.45 &
\centering *** &
\centering ~ &
\centering ~ &
MinT5 &
\centering 0.45 &
\centering 0.18 &
\centering * &
\centering ~ &
~ &
MinT5 &
\centering 0.16 &
\centering 0.33 &
\centering ~ &
\centering\arraybackslash ~\\
MinT5$^{2}$ &
\centering 2.21 &
\centering 0.18 &
\centering *** &
\centering 1.00 &
\centering ~ &
MinT5$^{2}$ &
\centering {}-0.14 &
\centering 0.18 &
\centering ~ &
\centering 0.20 &
~ &
MinT5$^{2}$ &
\centering 1.55 &
\centering 0.17 &
\centering *** &
\centering\arraybackslash 1.00\\
MinT95 &
\centering 3.61 &
\centering 0.51 &
\centering *** &
\centering ~ &
\centering ~ &
MaxH5 &
\centering {}-0.29 &
\centering 0.19 &
\centering ~ &
\centering ~ &
~ &
MaxH5 &
\centering {}-0.40 &
\centering 0.30 &
\centering ~ &
\centering\arraybackslash ~\\
MinT95$^{2}$ &
\centering {}-0.01 &
\centering 0.37 &
\centering ~ &
\centering 1.00 &
\centering ~ &
MaxH5$^{2}$ &
\centering 0.09 &
\centering 0.18 &
\centering ~ &
\centering 0.16 &
~ &
MaxH5$^{2}$ &
\centering {}-1.74 &
\centering 0.20 &
\centering *** &
\centering\arraybackslash 1.00\\
MaxH5 &
\centering {}-3.47 &
\centering 0.47 &
\centering *** &
\centering ~ &
\centering ~ &
MaxH95 &
\centering {}-0.41 &
\centering 0.22 &
\centering . &
\centering ~ &
~ &
MaxH95 &
\centering {}-0.72 &
\centering 0.29 &
\centering * &
\centering\arraybackslash ~\\
MaxH5$^{2}$ &
\centering {}-2.77 &
\centering 0.21 &
\centering *** &
\centering 1.00 &
\centering ~ &
MaxH95$^{2}$ &
\centering {}-0.17 &
\centering 0.23 &
\centering ~ &
\centering 0.11 &
~ &
MaxH95$^{2}$ &
\centering {}-1.59 &
\centering 0.19 &
\centering *** &
\centering\arraybackslash 1.00\\
MaxH95 &
\centering 0.76 &
\centering 0.39 &
\centering . &
\centering ~ &
\centering ~ &
\centering ~ &
\centering ~ &
\centering ~ &
\centering ~ &
\centering ~ &
~ &
MinH5 &
\centering 1.88 &
\centering 0.44 &
\centering *** &
\centering\arraybackslash ~\\
MaxH95$^{2}$ &
\centering {}-0.66 &
\centering 0.26 &
\centering * &
\centering 0.43 &
\centering ~ &
\centering ~ &
\centering ~ &
\centering ~ &
\centering ~ &
\centering ~ &
~ &
MinH5$^{2}$ &
\centering {}-1.19 &
\centering 0.22 &
\centering *** &
\centering\arraybackslash 1.00\\
MinH5 &
\centering 2.91 &
\centering 0.43 &
\centering *** &
\centering ~ &
\centering ~ &
\centering ~ &
\centering ~ &
\centering ~ &
\centering ~ &
\centering ~ &
~ &
PCQ &
\centering {}-0.91 &
\centering 0.36 &
\centering * &
\centering\arraybackslash ~\\
MinH5$^{2}$ &
\centering {}-1.80 &
\centering 0.24 &
\centering *** &
\centering 1.00 &
\centering ~ &
\centering ~ &
\centering ~ &
\centering ~ &
\centering ~ &
\centering ~ &
~ &
PCQ$^{2}$ &
\centering {}-2.73 &
\centering 0.24 &
\centering *** &
\centering\arraybackslash 1.00\\
PCQ &
\centering {}-0.52 &
\centering 0.38 &
\centering ~ &
\centering ~ &
\centering ~ &
\centering ~ &
\centering ~ &
\centering ~ &
\centering ~ &
\centering ~ &
~ &
CC &
\centering 1.02 &
\centering 0.23 &
\centering *** &
\centering\arraybackslash 0.83\\
PCQ$^{2}$ &
\centering {}-4.00 &
\centering 0.23 &
\centering *** &
\centering 1.00 &
\centering ~ &
\centering ~ &
\centering ~ &
\centering ~ &
\centering ~ &
\centering ~ &
~ &
~ &
~ &
~ &
~ &
~\\
CC &
\centering 1.07 &
\centering 0.23 &
\centering *** &
\centering 0.86 &
\centering ~ &
\centering ~ &
\centering ~ &
\centering ~ &
\centering ~ &
\centering ~ &
~ &
~ &
~ &
~ &
~ &
~\\
~ &
~ &
~ &
~ &
~ &
~ &
~ &
~ &
~ &
~ &
~ &
~ &
~ &
~ &
~ &
~ &
~\\
WSF \  &
~ &
~ &
~ &
~ &
~ &
RF &
~ &
~ &
~ &
~ &
~ &
FW &
~ &
~ &
~ &
~\\
Variable &
\centering Estimate &
\centering SE &
\centering P &
\centering w+(j) &
\centering ~ &
Variable &
\centering Estimate &
\centering SE &
\centering P &
\centering w+(j) &
\centering ~ &
Variable &
\centering Estimate &
\centering SE &
\centering P &
\centering\arraybackslash w+(j)\\\hline
\textbf{VarMT} &
\centering \textbf{0.25} &
\centering \textbf{0.24} &
\centering ~ &
\centering ~ &
~ &
MaxH95 &
\centering 0.84 &
\centering 0.46 &
\centering . &
\centering ~ &
~ &
\textbf{VarMT} &
\centering \textbf{2.01} &
\centering \textbf{0.31} &
\centering *** &
\centering\arraybackslash \textbf{1.00}\\
\textbf{VarMT$^{2}$} &
\centering \textbf{{}-1.60} &
\centering \textbf{0.20} &
\centering *** &
\centering \textbf{1.00} &
~ &
MaxH95$^{2}$ &
\centering {}-1.51 &
\centering 0.65 &
\centering * &
\centering 0.82 &
~ &
MaxH5 &
\centering {}-1.98 &
\centering 0.27 &
\centering *** &
\centering\arraybackslash ~\\
PDQ &
\centering {}-0.64 &
\centering 0.23 &
\centering ** &
\centering ~ &
~ &
MinT95 &
\centering 1.64 &
\centering 0.39 &
\centering *** &
\centering ~ &
~ &
MaxH5$^{2}$ &
\centering {}-1.44 &
\centering 0.25 &
\centering *** &
\centering\arraybackslash 1.00\\
PDQ2 &
\centering {}-0.74 &
\centering 0.19 &
\centering *** &
\centering 0.98 &
~ &
MinT95$^{2}$ &
\centering {}-1.48 &
\centering 0.36 &
\centering *** &
\centering 0.76 &
~ &
MinT95 &
\centering 1.32 &
\centering 0.35 &
\centering *** &
\centering\arraybackslash ~\\
MaxT5 &
\centering 0.14 &
\centering 0.28 &
\centering ~ &
\centering ~ &
~ &
\textbf{VarMT} &
\centering \textbf{1.04} &
\centering \textbf{0.52} &
\centering * &
\centering ~ &
~ &
MinT95$^{2}$ &
\centering {}-0.28 &
\centering 0.19 &
\centering ~ &
\centering\arraybackslash 0.87\\
MaxT5$^{2}$ &
\centering {}-0.67 &
\centering 0.20 &
\centering ** &
\centering 0.56 &
~ &
\textbf{VarMT$^{2}$} &
\centering \textbf{{}-0.55} &
\centering \textbf{0.23} &
\centering * &
\centering \textbf{0.45} &
~ &
MaxH95 &
\centering 1.34 &
\centering 0.33 &
\centering *** &
\centering\arraybackslash 0.81\\
MaxT95 &
\centering 0.08 &
\centering 0.29 &
\centering ~ &
\centering ~ &
~ &
PDQ &
\centering {}-0.01 &
\centering 0.42 &
\centering ~ &
\centering ~ &
~ &
AP &
\centering 0.21 &
\centering 0.38 &
\centering ~ &
\centering\arraybackslash 0.28\\
MaxT95$^{2}$ &
\centering {}-0.56 &
\centering 0.21 &
\centering ** &
\centering 0.40 &
~ &
PDQ$^{2}$ &
\centering {}-1.01 &
\centering 0.31 &
\centering ** &
\centering 0.45 &
~ &
MinH5 &
\centering 0.79 &
\centering 0.55 &
\centering ~ &
\centering\arraybackslash ~\\
MinH95 &
\centering 0.17 &
\centering 0.18 &
\centering ~ &
\centering ~ &
~ &
MinH5 &
\centering 0.70 &
\centering 0.41 &
\centering . &
\centering ~ &
~ &
MinH5$^{2}$ &
\centering {}-0.15 &
\centering 0.22 &
\centering ~ &
\centering\arraybackslash 0.23\\
MinH95$^{2}$ &
\centering {}-0.40 &
\centering 0.18 &
\centering * &
\centering 0.31 &
~ &
MinH5$^{2}$ &
\centering {}-0.01 &
\centering 0.23 &
\centering ~ &
\centering 0.20 &
~ &
MaxT95 &
\centering 0.83 &
\centering 0.67 &
\centering ~ &
\centering\arraybackslash ~\\
MinT5 &
\centering 0.10 &
\centering 0.22 &
\centering ~ &
\centering ~ &
~ &
~ &
~ &
~ &
~ &
~ &
~ &
MaxT95$^{2}$ &
\centering 0.14 &
\centering 0.21 &
\centering ~ &
\centering\arraybackslash 0.22\\
MinT5$^{2}$ &
\centering {}-0.07 &
\centering 0.18 &
\centering ~ &
\centering 0.11 &
~ &
~ &
~ &
~ &
~ &
~ &
~ &
~ &
\centering\arraybackslash ~\\\hline
\end{tabular}
\label{tab:coefs}
\end{table}
\end{flushleft}
In support of the single-predictor models, coefficient estimates derived from model averaging confirmed that species richness was consistently best modelled as a significant negative quadratic function of variability in maximum temperature in three of five forest types, as well as the combined data. Exceptions were forested wetlands, for which variability in maximum temperature was consistently positive and linear, and grassy woodlands where non-significant univariate models precluded the inclusion of variability in maximum temperature in the model selection process.

The global models for all forest types and all plots combined exhibited significant spatial correlation in the residuals. However, parameter estimates obtained with spatial regression models (SAR error models) were almost identical in sign and significance to those obtained with non-spatial GLMs (Table S2.1). In particular, the sign and significance of the linear and quadratic terms for variability in maximum temperature were consistent across almost all global models, regardless of whether the models were regressed using a non-spatial quasi-Poisson model, a non-spatial Gaussian model or a spatially sensitive SAR error model. The only exception was for dry sclerophyll forest where the quadratic term for variability in maximum temperature was no longer significant when modelled using SAR models. We therefore concluded that the results were largely insensitive to spatial autocorrelation in the data.


\section{Discussion}

The results of this study provide novel evidence that at local spatial scales temperature variability may be a better predictor of plant species diversity than absolute measures of temperature such as means, maxima, or minima. Specifically, our most important finding was that across all plots combined and in three of five individual forest types, the explanatory power of temperature variability was in the order of around two to five times that of absolute temperature (Table \ref{tab:temp}). This relationship supports the contention that spatial variation in temporal temperature variability plays a more important role in regulating species coexistence and persistence than comparative spatial variation in absolute temperature. In a previous study that compared diversity patterns across localities differing in their temporal climate variability profile, \citet{Shurin2010} similarly found that variation in freshwater zooplankton richness was better explained by temperature variability than mean temperature. To our knowledge these results provide the first empirical evidence of a linkage between spatial variation in temporal climate variability and plant species diversity at local scales.


It is instructive that in both the single- and multiple-predictor models the performance of temperature variability as a predictor of plant diversity increased from the driest to the wettest forest types, which would suggest that temperature variability has a more crucial role in structuring plant communities where water is non-limiting. However, given that absolute temperature also performed poorly as a predictor of species richness in the drier forest types, we cannot rule out the possibility of higher level water-energy dynamics precluding richness responses to temperature in dry habitats \citep{Francis2003}. It is also likely that the greater frequency of fires in dry forest types \citep{CLARKE2005} supersedes the influence of climatic factors, such as temperature, on species richness patterns. Indeed, the best multiple-predictor models for grassy woodlands, the driest vegetation type in our study, were poorer than all other forest types in explaining variation in species richness, suggesting that climate factors in general may have a less important role in structuring plant communities in drier habitats. Notably, \citet{Bongers2009} found that a disturbance index integrating primarily non-climatic variables such as logging and fire was better at explaining variation in tree species diversity in dry forest types than in wet forest types. This is consistent with our findings that climatic factors appear to be less important drivers of richness patterns in dry habitats than mesic habitats. 


The statistically strong independent relationship between temperature variability and species richness for all forest types combined is counterbalanced by weak explanatory power (4.85\%), albeit within the range typical of many ecological studies \citep{Jennions2002} and almost five times that explained by absolute temperature. One plausible explanation is that factors other than temperature variability exert a stronger influence on species richness in our study area. A number of \ variables that were not included in the models may be expected to exert a substantial influence on fine-scale richness patterns, including soil and nutrient profiles, water availability, fire frequency, logging history and other forms of anthropogenic disturbance \citep{Pausas2001}. It is also possible that inconsistencies in the shape of the richness response across the individual forest types will serve to cancel each other out, precipitating a weak relationship across all plots combined. For example, percentage variability in species richness explained univariately by temperature variability was comparatively high in the wetter forest types (10.32\% in wet sclerophyll forest, 13.03\% in rainforest and 19.28\% in forested wetlands), yet the shape differed between them (two unimodal and one positive linear response). As such, these results suggest climate variability may play a more significant role in determining species richness at the within-community level rather than across communities.

One potential source of uncertainty in our analyses arises from the necessary interpolation of the climate data to obtain climate measures at each of the quadrat locations. However, because the identical interpolation method was used for both temperature variability and absolute measures of temperature, this should not have introduced any bias into the analyses. As such, while uncertainty derived from the interpolation may have weakened the overall explanatory power of all variables, the relative performance of temperature variability versus absolute temperature should have been unaffected. It is also notable that our estimate of temperature variability is likely subject to some measurement error on account of the climate model being derived from only two years of data, which may dilute its explanatory power. In particular the estimates of inter-annual variability would presumably benefit most significantly from a longer time-series as more climatically extreme years are sampled. Nevertheless, we are confident that the locations we identify as variable/stable are relatively static in their positioning along the variability gradients given their distinct topographic, geographic and environmental features \citep{Ashcroft2012b}. Furthermore, it is instructive that even with just two years of data we were able to show that temperature variability is, in many cases, a better predictor of plant diversity.

The separate univariate analyses conducted for each of the two sub-groups (trees and non-trees) provides some evidence that the relative performance of temperature variability and absolute temperature as predictors of diversity may vary depending on the life-history of the taxa in question. Although temperature variability exhibited better (or equal) performance relative to absolute temperature in the three wettest forest types for both trees and non-trees, the performance of absolute temperature exceeded that of temperature variability for trees when all plots were combined (Table S2.4). While it is tempting to speculate that variability at the temporal scales considered in this study may be less important for species with longer life histories, such an inference is weakened by the unavailability of data on the site-specific growth-habit of each species within each quadrat. Further work would benefit from targeted investigations of richness-variability relationships across different functional groups.

The finding that the relationship between plant diversity and variability in maximum temperature exhibited significant unimodality across multiple forest types (Fig. \ref{fig:humpplots}, Table \ref{tab:coefs}) corroborates recent theory \citep{Adler2008} reconciling the opposing effects of climate variability on coexistence in ecological communities. Whilst strictly correlational, the inference is that along a gradient of temporal climate variability, richness will at first increase in response to temporal niche partitioning, but at some intermediary point, will decline as the risk of stochastic extinction exceeds competitive stabilization. Aside from grassy woodlands, where temperature variability was not significantly correlated with species richness, forested wetlands was the only forest type to exhibit a significant non-unimodal (positive linear response) relationship between temperature variability and species richness. It is noteworthy however, that maximum temperature variability tended to be lower in forested wetlands compared to the other forest types (Fig. S2.2). As such, it may be that temperature fluctuations rarely reach sufficient amplitude in the forested wetlands in our study area to elicit negative effects on species persistence. 

The notion that temperature variability can be an important regulator of plant diversity may seem intuitive, but in the past biogeographers and ecologists have tended to give greater precedent to spatial heterogeneity in mean temperature \citep{Currie2004}. Where temporal variability is acknowledged as a potential driver of richness patterns, conventional thinking holds that climate stability supports high diversity by facilitating adaptation to narrower niches, thus increasing the number of species per unit area \citep{Stevens1989}, or by fostering persistence and greater length of time for speciation \citep{McGlone1996, Hopper2009}. To this end, relative climate stability in the tropics compared to temperate regions has been invoked to explain the latitudinal gradients in species richness (\citealp{Stevens1989}; but see \citealp{Gaston1998a}). However, together with a number of previous studies, we provide further evidence that climate stability may, in some environments and over shorter time-scales, actually limit the maintenance of species diversity. This contention that the relative influence of climate variability on diversity may vary between ecological and evolutionary time scales, and across spatial scales, is particularly pertinent to the identification of refugia under future climates. Both short and long period environmental stochasticity is thought to be detrimental to the persistence of relict populations under climate change \citep{Hampe2011}, yet our results suggest sites that maintain high rates of occupancy under future climate regimes may not necessarily be those that are the most climatically stable at fine spatial scales. 

\subsection{Conclusion}

In light of changes to climate variability regimes under climate change, empirical studies of this kind highlight the need for ecologists to broaden their focus beyond mean climate shifts. As evidenced in our system, it is likely that the comparative influence of variability and absolute climate on community structuring is conditional on multiple limiting factors. Our ability to predict where, when and at what spatio-temporal scales climatic variability or absolutes take precedence will benefit from further empirical research in different taxa and communities. To complement the array of experimental and population demographic approaches, we hope that the observational evidence presented here will provide motivation for others to investigate the strength and shape of richness-variability relationships across space in other systems.

\section*{Acknowledgements}

This research was part of Australian Research Council Linkage Project LP100200080 in collaboration with the NSW Office of Water, Australian Museum, Central West Catchment Management Authority, the Australian Wetlands, Rivers and Landscapes Centre at the University of New South Wales and the University of Technology Sydney. We are grateful to Renee Woodward for assistance with acquisition of the YETI vegetation dataset, Cam Webb for valuable feedback on the original research proposal, and Habacuc Flores-Moreno for valuable comments on an earlier draft of the manuscript.


\newpage

\chapter{Trees, branches and (square) roots: why evolutionary relatedness is not linearly related to functional distance}
\chaptermark{Trees, branches and (square) roots}

\graphicspath{{Chapter3/Figs/}}

\begin{center}

{\large Andrew D. Letten and William K. Cornwell}

\small\textit{\textbf{Methods in Ecology and Evolution}} \textbf{(2014)}\\
\url{http://doi.wiley.com/10.1111/2041-210X.12237}

\vspace{1in}
\includegraphics[width=0.15\linewidth]{Chapter3/Figs/Einstein_whitejpg}

\vfill
This study was conceived by ADL and WCK. ADL and WCK contributed equally to scripting code for simulations and writing the manuscript. 

\end{center}

\newpage
\section{Summary}

\begin{enumerate}

\item{ 
An increasingly popular practice in community ecology is to use the 
evolutionary distance amongst interacting species as a proxy for their 
overall functional similarity.
}

\item{
At the core of this approach is the  implicit, yet poorly recognized, 
assumption that trait dissimilarity increases linearly with divergence time, 
i.e. all evolutionary time is considered equal. However, given a classic 
Brownian model of trait evolution, we show that the expected functional 
displacement of any two taxa is more appropriately represented as a linear 
function of time's square root. 
}
\item{
In light of this mismatch between theory and methodology, we argue that
current methods at the interface of ecology and evolutionary biology often greatly 
overweight deep time relative to recent time. 
}
\item{
An easy solution to this weighting problem is a square--root transformation of 
the phylogenetic distance matrix. Using simulated models of trait evolution 
and community assembly, we show that this transformation yields considerably 
higher statistical power, with improvements in 92\% of trials. This methodological update is likely to improve our understanding of the connection between evolutionary relatedness and contemporary ecological processes.
}
\end{enumerate}

\newpage

\section{Introduction}

With increasingly precise estimates of the most common ancestor amongst interacting species, modern phylogenetics offers the 
promise of a synthesis of contemporary ecology and evolutionary history \citep{Webb2002, Johnson2007, Cavender-Bares2009, cadotte2013, 
Swenson2013}. Following on this, the last 10-15 years have seen a precipitous rise in the number of studies examining ecological 
patterns and processes through the lens of evolutionary relatedness. Whilst this integrative approach has undoubtedly yielded new 
insights, much of the foregoing research has proved inconclusive; contemporary ecological interactions often appear, using 
conventional methods, to be unrelated to evolutionary history \citep{Cahill2008, Bennett2013, Narwani2013, Fritschie2013}. In this 
comment we offer one explanation for the poor performance of the phylogenetic metrics used in contemporary ecology, as well as a 
partial solution.

From the beginning of the phylogeny--ecology synthesis, evolutionary 
relatedness has often been used a proxy for the traits mediating species' 
interactions with each other and their environment. It is impossible to 
measure all the relevant traits for complex ecological interactions, but 
because evolution is a conservative branching process and traits are on 
average more conserved than random, phylogenies have the potential to provide 
an integrative measure across all traits \citep{Webb2000}. It follows that the 
strength of this inference is contingent on an accurate model of how 
phylogenetic and functional distance covary. There is a widespread assumption in the literature \citep[see Figure 1b of][] 
{cadotte2013} that phylogenetic and functional distance scale linearly.  This assumption is 
implicit in many of the conventional metrics for assessing phylogenetic community structure and 
phylogenetic diversity \citep{vellend2010}. Below we consider this assumption 
critically, using current theory on trait evolution.  

The classic, ``default'' model for trait evolution is Brownian motion 
\citep[Figure \ref{sq_root_fig}b and][]{Felsenstein1985}.
The diffusion equation for Brownian motion \citep{Einstein1905} has the form:
\begin{equation} 
\frac{\partial\rho}{\partial t}=\sigma^2\frac{\partial^2\rho}{\partial x^2}
\end{equation} 

where $\sigma^2$ is the diffusion constant, t is time, $\rho$ is density and $x$ is position in space. That equation has the solution:
\begin{equation} 
\rho(x,t)=\frac{\rho_0}{\sqrt{4\pi \sigma^2t}}e^{-\frac{x^2}{4\sigma^2t}}
\end{equation} 
which implies that the second moment of the distribution is:
\begin{equation} 
\overline{x^2}=2\sigma^2t.
\end{equation} 
In other words, the variance goes up linearly with time and the standard deviation rises with time's square root, or as 
Einstein put it: ``the mean displacement is therefore proportional to the square root of the time''. Applied in a phylogenetic context 
this means that while among species variance in trait values goes up linearly 
with time \citep{Felsenstein1985}, the expected displacement of any two taxa in trait space
does not increase linearly with time, but rather with time's square root (Figure \ref{sq_root_fig}). This non-linearity is true both for one trait
and for Euclidean distance in \textit{n}-dimensional trait space. Indeed, there is no plausible 
model for expected trait dissimilarity to be linearly related to evolutionary 
time. For there to be a linear relationship, after a speciation event, the functional distance between two lineages would increase constantly and continuously as their traits evolve away from each other. It is difficult to imagine a scenario where that would be the case.

\begin{figure}[H]
\centering
\includegraphics[width=1.00\textwidth]{sq_root_resub.png}
\caption{Panel (a) shows a Yule phylogeny. Panel (b) shows a ``traitgram'' with one Brownian motion 
simulation with the trait value on the y--axis.  Panel (c) shows the effect of time on the standard deviation of trait values at the tips for 2000 simulations with the same Brownian motion rate parameter; each point represents the standard deviation of the trait values of extant species within a separate simulation.  Panel (d) shows the pairwise trait differences for 50000 simulations plotted against phylogenetic distance.  Panel (e) simply takes the data from panel (d) and places them in bins, to show the statistical expectation at a given relatedness.  Panel (e) shows the effect of taking the square root of the phylogenetic distance matrix on the relationship between the phylogenetic distance and the expected trait difference.  Note that the expectation for the trait standard deviation and the pairwise difference is linear with respect to the square root of time as shown analytically in the text.  All simulations use code from FitzJohn (2012) and Revell (2012).}
\label{sq_root_fig}
\end{figure}

While this technical point is well understood in parts of the comparative methods literature 
\citep{Hardy2012}, the implications for many ecological applications has gone largely unnoticed. 
If evolutionary relatedness is used as a proxy for functional distance, the non-linearity of their 
scaling relationship means that more recent evolution should have a disproportional 
influence on contemporary ecology. Current methods used in community phylogenetics \citep[see 
review of methods 
in][]{vellend2010} typically treat evolutionary relatedness linearly; 1 and 6 million 
year relatedness difference is treated as having the same expected effect as a 
101 and 106 million year relatedness difference. All evolutionary time is 
considered equal, whether that time occurred over the last 5 million years or 
more than 100 million years ago. Combined with imbalanced trees, this creates 
a problem that is well known in empirical investigations: the statistical 
over-weighting of early-diverged, low diversity clades \citep{Kembel2006}. 
When those clades are included in the sample (or in a randomization), they 
have a disproportionate weight on the test statistic, a weight that is highly 
disproportionate to the expected trait difference under a Brownian model.  

The root of this problem is in the distribution of pairwise phylogenetic distances.  In the 
basic simulated birth-death trees, there is an equal probability at every time 
step for every branch of either a speciation or an extinction event.  This creates what are known as balanced trees \citep[][also Figure 
\ref{tree_fig}]{heard1997}. Balanced trees are rare in empirical studies, as heterogeneity in net diversification \citep{alfaro2009} creates trees that are imbalanced \citep{mooers1995} and have peculiar distributions of pairwise distances (e.g. the vascular plant tree from \citet{zanne2014} in Figure \ref{phylo_dist_hist}).  

\begin{figure}[H]
\centering
\includegraphics[width=1.00\textwidth]{trees_resub.pdf}
\caption{Example of a real tree (top), obtained by randomly selecting taxa from the Zanne et al. (2014) tree, compared with a 
homogeneous birth-death simulation tree (bottom).}
\label{tree_fig}
\end{figure}

Drawing from theory on Brownian motion reveals a simple solution to the weighting 
problem: a square--root transform of the phylogenetic distance matrix. For 
example, the mean pairwise distance \citep[sensu][]{Webb2000}:
\begin{equation} 
MPD=\frac{2\sum_{i=1}^{n-1} \sum^n_{j=i+1} d_{i,j}}{(n)(n-1)}
\end{equation} 
can be redefined as the mean of the square--root transformed pairwise distances:
\begin{equation} 
MPD^*=\frac{2\sum_{i=1}^{n-1} \sum^n_{j=i+1} \sqrt{d_{i,j}}}{(n)(n-1)}
\end{equation} 
where $n$ is the number of taxa and $d_{i,j}$ is the pairwise phylogenetic 
distance between species $i$ and species $j$. This quantity $MPD^*$ is 
proportional to the mean of the expected pairwise differences for traits 
evolving under a Brownian model, for both one trait in one dimension and for 
$m$ traits in $m$ dimensional space. This equation is provided as an example: 
very similar adjustments are possible to most of the common community 
phylogenetics statistics \citep[see definitions within][]{vellend2010}, simply 
via a square-root transform of the distance matrix.  

With this transformation, long--ago time is down weighted compared to recent 
time. As such, the effect of the presence or absence of a species on MPD
is weighted in proportion to the mean pairwise expected trait difference to all other species under a Brownian model. In practical terms this re-scaling can be accomplished exceedingly 
easily with one additional line of code in combination with the tools in 
widely available statistical packages
\citep{Kembel2010}. 

\begin{figure}[H]
\centering
\includegraphics[width=1.00\textwidth]{phylo_dist_hist_corrected.pdf}
\caption{The pairwise phylogenetic distances from a recent phylogeny of vascular plants by Zanne et al. (2014) and a simulated phylogeny 
of the same age and number of
extant species.}
\label{phylo_dist_hist}
\end{figure} 

\section{Simulations}

To explore the empirical implications of this idea, we conducted simulations applying a similar framework to that described by \citet[][in this case repeating the `filtering-derived' and `neutral assembly' algorithms]{kraft2007}. We simulated trait evolution by Brownian motion on both a real tree and a homogenous birth-death simulated tree (Figure \ref{tree_fig}). To keep the pool size (number of tree tips) consistent across the real and simulated trees, the real tree was randomly pruned down to 200 taxa. In each run, we 'evolved' a trait across the phylogeny and then applied one of two community assembly filters to obtain a final community of 40 taxa. Under the 'filtering derived' assembly filter, the most derived (extreme) trait-value was treated as the optimum, with the remaining 39 places in the community selected from taxa having the nearest trait values to that optimum. This process simulated  community assembly via habitat filtering \citep{Diaz1998}, whereby the abiotic environment sets some threshold on the range of strategies (and thereby trait values) that are able to sustain a positive population growth rate (e.g. tolerance of inundation along a hydrological gradient). In contrast with the deterministic nature of the `filtering-derived' algorithm, under the `neutral assembly' algorithm the community was obtained by randomly selecting 40 species independent of their trait values. One--thousand runs were conducted for each community assembly algorithm on each of the real and simulated phylogenies. Finally, we quantified the effect of the filter using conventional community phylogenetics methods (MPD and MNTD - mean nearest taxon distance), and compared the standard approach with that of a square--root transform of the phylogenetic distance matrix (all code to perform replicate simulations is provided in Appendix B). 

Simulation results indicate the transformed test has considerably higher statistical power (Figure 
\ref{stat_power_fig}) for detecting the signal of community assembly. Using 
the square--root transform improved 92\% of trials for both MPD and MNTD. Using the standardized effect size metric developed by 
\cite{Webb2000}, whereby:

$$SES_{METRIC} = \frac{METRIC_{observed} - mean(METRIC_{null})}{sd(METRIC_{null})}$$

the median improvement in standardized effect size was 0.64 
for MPD and 0.43 for MNTD (simulated and real trees combined). 
This is a comparatively large increase in effect size from a simple statistical 
adjustment.  The improvement was similar for real and simulated 
trees, but may be more crucial in the real case because the general power of 
community phylogenetics is lower for the real tree \citep{Kembel2006}. While a comprehensive exploration of the effect of the transformation on type 1 error rates is beyond the scope of this paper  \citep[see][]{kraft2007}, simulations indicated that the expected reduction in type 2 error rate is in the order of 5-25\% (depending on the community to pool size ratio and other factors).

\begin{figure}[H]
\centering
\includegraphics[width=1.00\textwidth]{bm_diff_p220_c40.pdf}
\caption{Improved statistical power from down-weighting long-ago evolution: change in standardized effect size for mean pairwise 
distance (MPD) and mean nearest taxa distance (MNTD) using a square-root transformed phylogenetic distance matrix versus the 
conventional approach. Box plots with grey-fill group communities assembled under a neutral (random sample) model; box plots with 
red-fill group communities assembled under a 'habitat filter' model.}
\label{stat_power_fig}
\end{figure}

 
\section{Other models of trait evolution}

While a highly useful ``default'' model, we do not expect that a Brownian 
motion model will prove to be a fully adequate model for trait 
evolution at large scales. In general, the current evidence suggests that 
actually the square--root transform does not go far enough toward 
down--weighting long--ago evolution compared to recent evolution in many cases 
\citep{butler2004, harmon2010, smith2010evolution}. In the event that there 
are bounding or mean-reverting processes \citep[e.g. Ornstein--Uhlenbeck (OU)][]{butler2004}, phylogenetic 
signal will be less strong than under a Brownian model.  Under these alternative models the effect 
of evolutionary 
relatedness
decays more rapidly, a phenomena defined as ``phylogenetic half-life'' by \citet{hansen2008comparative}.  
If trait evolution typically includes this type of process, the problem we describe here will be even more extreme. 
In this case \citep[see][]{kelly2014phylogenetic}, the square--root transformation will not 
go far enough. For the alternatives to Brownian motion, such as OU and
heterogeneous models where rates of evolution vary among clades \citep{beaulieu2012modeling}, there are analogous tree scaling approaches \citep{Pearse2013a}, but these, unlike the square--root transformation, 
require \emph{a priori} information about trait evolution in the relevant 
clade.  

\section{Conclusion}

We recommend a square--root transform of the phylogenetic distance matrix for 
all uses where phylogenetic relatedness is used as a proxy for current--day 
functional disparity. There are some cases where the number of years of 
evolutionary history in a place may be an interesting quantity in and of 
itself \citep{purvis2000}. In those cases linear relatedness may still be of 
interest; however, many ecological studies use evolutionary relatedness as a 
proxy for trait dissimilarity and in these cases using relatedness linearly 
will decrease the power of the investigation. 

While we have made a statistical argument above, in conclusion we stress that 
this is actually a conceptual point. We argue that conventional approaches over-weight long-ago 
evolutionary time and under-weight recent evolution both conceptually and statistically, and in 
doing so inadvertently limit the statistical power and success of efforts to leverage phylogenetic 
information in ecological contexts. There are many other reasons why the mapping of 
ecological process to phylogenetic community patterns may be inconclusive \citep{Mayfield2010, Godoy2014}; many of these issues are hard to address.  Here, we have identified one problem---
the weighting of evolutionary history---where a simple 
adjustment may help.

Of course, by using the square-root transformation we make the assumption that trait differences scale linearly with ecological fitness. While disentangling the ecological/evolutionary processes is somewhat intractable in this instance, this is at least a parsimonious assumption. Instead, justification for using more complex measures of functional distance (e.g. squared distance) in the context of ecological selection/assembly should be contingent on supporting theory. To our knowledge none exists. The magnitude of fitness-trait relationships has received some attention \citep{Kimball2011, Adler2013a} but explicitly exploring their scaling properties could well prove an invaluable area of future research.

In general, there needs to be a more nuanced use of evolutionary relatedness within community 
ecology.  By improving the connection between metrics within community phylogenetics and trait 
evolution, we can increase the power and utility of using evolutionary relatedness to ask ecological questions. 
This methodological update will not be a one-off: as our understanding of the processes and patterns in trait macro-evolution at large 
scales grows \citep{omeara2012, pennell2013integrative}, the phylogenetic metrics used in contemporary ecology will need to be 
continually updated.

\subsection*{Acknowledgements}

Thanks to Nathan Kraft, Rich FitzJohn, Rob Kooyman, Sandrine Pavoine and an anonymous reviewer for insightful comments.  

\subsection*{Data accessibility}

R code to reproduce simulations provided in Appendix B.

\nocite{zanne2014, FitzJohn2012, Revell2012}
%\bibliographystyle{mee}
%\bibliography{bm_scaling}





\chapter{Phylogenetic and functional dissimilarity does not increase during temporal heathland succession} 
\chaptermark{Phylogenetic and functional dissimilarity during succession}

\graphicspath{{Chapter4/Figs}}

\begin{center}

{\large Andrew D. Letten, David A. Keith and Mark G. Tozer}

\small{\textit{\textbf{Proceedings of the Royal Society B}} \textbf{(2014), 281, 20142102}}
\url{http://dx.doi.org/10.1098/rspb.2014.2102}

\vspace{1in}
\includegraphics[width=0.15\linewidth]{Chapter4/Figs/flame2}

\vfill
This study was conceived by ADL with input from DAK. DAK and MGT provided floristic plot data. ADL compiled plant trait data, assembled phylogeny, conducted analyses and wrote the manuscript, with contributions from DAK \& MGT. 

\end{center}

\newpage
\section{Summary}

Succession has been a focal point of ecological research for over a century, but thus far has been poorly explored through the lens of modern phylogenetic and trait-based approaches to community assembly. The vast majority of studies conducted to date have comprised static analyses where communities are observed at a single snapshot in time. Long-term datasets present a vantage point to compare established and emerging theoretical predictions on the phylogenetic and functional trajectory of communities through succession. We investigated within, and between, community measures of phylogenetic and functional diversity in a fire-prone heathland along a 21-year time-series. Contrary to widely-held expectations that increased competition through succession should inhibit the coexistence of species with high niche overlap, plots became more phylogenetically and functionally clustered with time since fire. There were significant directional shifts in individual traits through time indicating deterministic successional processes associated with changing abiotic and/or biotic conditions. However, relative to the observed temporal rate of taxonomic turnover, both phylogenetic and functional turnover were comparatively low, suggesting a degree of functional redundancy amongst close relatives. These results contribute to an emerging body of evidence indicating that limits to the similarity of coexisting species are rarely observed at fine spatial scales.

\newpage
\section{Introduction}

Given limited scope for experimental manipulation in natural systems, a common approach in community ecology is to infer the mechanisms structuring communities from the distribution of their component species and traits. Inferring processes from patterns is of course non-trivial, relying as it necessarily does on a raft of assumptions about how the components of communities (i.e. species) respond to each other and their environment. This \textit{modus operandi} is nowhere more apparent than in the phylogenetic and trait-based analyses of community assembly that have proliferated over the last 10-15 years \citep[e.g.][]{Webb2000, Cornwell2009, Kraft2010}. To date, the vast majority of phylogenetic and trait-based studies of community assembly have comprised `static' analyses where assembly processes are inferred from patterns observed at a single snapshot in time \citep[as reviewed in][]{Swenson2013, Gotzenberger2012}. By necessity, static studies of this kind either ignore the dynamic properties of communities, treat community assembly as a one-off event, or at best assume that observed patterns are representative of prevailing processes. While this assumption may hold in some late successional systems, in dynamic or frequently disturbed systems, the processes that govern community structure may fluctuate considerably over time.

Disentangling sequential assembly processes from observed temporal patterns is complicated by competing and/or unresolved theoretical predictions. One oft-repeated axiom of community ecology holds that competition inhibits species with high niche overlap from coexisting, while environmental filtering has the opposite effect of limiting the range of successful ecological strategies at any one location \citep{Weiher1995, Stubbs2004, Purschke2013}. It follows logically that if ecological niches are phylogenetically conserved, these two apparently opposing processes will leave different signatures on the phylogenetic structure of communities; competition will drive phylogenetic divergence, whilst strong environmental filters will lead to communities consisting largely of close relatives \citep{Webb2000, Webb2002}. Coupling this framework with classical successional theory \citep{Clements1916, Connell1977, Walker1987, Wilson1999}, we might anticipate communities will transition from exhibiting functional and phylogenetic convergence early in succession to becoming increasingly functionally and phylogenetically dispersed as competition increases in relative importance. However, even when niches are phylogenetically conserved, it has recently been argued that this dichotomous framework makes untenable assumptions about the relative importance of niche differences and fitness differences in determining the outcome of community assembly \citep{Chesson2000, Mayfield2010}. As recognised by \citet{Mayfield2010}, when differences in competitive ability exceed niche differences for a large proportion of the species pool, competition may exclude all but the most effective resource competitors. From this alternative perspective, we might predict phylogenetic and functional convergence, rather than divergence, if competition increases through succession. Evidence that competition may indeed drive phylogenetic convergence has recently begun to emerge from a variety of systems and taxa \citep{Kunstler2012, Bennett2013, Price2013, Narwani2013}.

Given the paucity of suitable long-term datasets, most phylogenetic and trait-based research on the effects of disturbance and/or succession on community structure has been limited to static comparisons of relatedness and functional similarity in disturbed versus non-disturbed communities \citep{VERDU2007, Knapp2008, Dinnage2009, Helmus2010}, or across different successional states in a chronosequence (i.e. a space-for-time substitution) \citep{Verdu2009, Letcher2010, Kunstler2012, Purschke2013}. With a few notable exceptions \citep{Verdu2009,Kunstler2012}, most studies have reported greater functional and/or phylogenetic dispersion in undisturbed or late successional communities, including along a rare time series \citep{Norden2012}. Nevertheless, given the known dangers of space-for-time substitutions in ecological research \citep{Johnson2008}, additional temporal successional studies are needed to more robustly explore the generality of this pattern.

An important advantage of phylogenetic and functional analyses of temporal datasets is that it enables the compilation of species pools that are a truer representation of potential colonisers. In the past, static studies have been criticised for deriving species pools from regional species-lists which may include numerous taxa that may never colonise a site even in the absence of competitors \citep{Grime2006, DeBello2012}. This coarse approach may potentially bias analyses towards finding phylogenetic or functional convergence resulting from broad-scale environmental filtering \citep{Gotzenberger2012}. An alternative approach is to try to eliminate the effects of large scale filters \textit{a priori} by constraining the species pool to known colonisers of a site \citep{DeBello2012}. This of course is limited by the availability of data collected over a sufficiently long period of time to record not only those species present at any given time but also the aptly-named dark diversity \citep{Partel2011} i.e. species that may only be competitive (and therefore more likely to be detectable) for a brief period during community assembly/succession. Long-term studies, where the presence of species is monitored at intervals at permanent sites, make this achievable.

In this study we investigated temporal trends in the phylogenetic and functional community structure of understorey plants in fire-prone heathland in southeast Australia. Previous work has provided evidence of strong competitive hierarchies related to vertical stature in this system, with overstorey shrubs typically eliminating understorey species through succession post-fire \citep{Keith1994, Tozer2003, KEITH2007}. However, unlike much of the existing literature on phylogenetic and functional community structure through succession in plant communities \citep[e.g.][]{Letcher2010,Norden2012, Kunstler2012, Bhaskar2014}, here we explicitly focus on understorey communities. With access to compositional data collected over more than 20 years through multiple fire events, this study represents one of the most comprehensive assessments of temporal dynamics in both phylogenetic and functional community structure to date. 

Whilst the phylogenetic and functional structure of plant communities may arise through a complex interplay of various evolutionary (e.g. trait evolution and niche conservatism) and ecological processes (e.g. competition, environmental filtering, herbivory etc.), we concentrated on a subset of hypotheses that reflect the competing theories which have received the most attention in the recent literature. Firstly, assuming fire acts as a filter on the species pool we hypothesised that plots would exhibit functional clustering immediately following fire, and correspondingly also exhibit phylogenetic clustering if the traits mediating early dominance are conserved. Alternatively, if key assembly traits are not conserved, or if measured functional traits have little bearing on community assembly, then we would expect functional and phylogenetic patterns to be uncorrelated. In addition, we made alternative predictions about the trajectories of communities in the period following fire. If increased competition through succession enforces a limit on the similarity of coexisting species, communities should become increasingly functionally dispersed though time, and therefore also phylogeneticaly dispersed if species function is conserved. Alternatively, if increased competition results in the exclusion of all but the most dominant resource competitors \citep[\textit{sensu}][]{Mayfield2010}, or if assembly is primarily driven by fluctuating environmental processes, functional and phylogenetic clustering should remain static or increase through time.  

Our secondary aim was to infer what processes are likely to be driving any observed trends. To this end, we not only considered within community structure but also trends in community-weighted mean trait values and rates of temporal phylogenetic and functional turnover relative to taxonomic turnover. This provided a means to evaluate the extent to which community assembly through succession is structured by deterministic or stochastic processes \citep[e.g.][]{Swenson2012}. For instance, a deterministic model of community assembly assumes that species turnover through time is non-random with respect to species function, and by inference phylogeny. If biotic and abiotic conditions are relatively constant through time, we would predict less phylogenetic and/or functional turnover relative to the observed rate of taxonomic turnover. Conversely, if conditions fluctuate though time, we would predict greater than expected phylogenetic/functional turnover. Finally, if function and phylogeny have no bearing on community assembly \citep[i.e. a stochastic or neutral model \textit{sensu}][]{Hubbell2001}, we should expect taxonomic turnover to be random with respect to phylogeny and function.

\section{Materials and methods}

\subsection{Study site, sampling and fire history}

The study was conducted in an area of fire-prone coastal heathland in Royal National Park, New South Wales, Australia (centred on 34$^{o}$05'46.00" S 151$^{o}$09'02.73" E). The vegetation in the area is characterised by sclerophyllous plants, with a herbaceous ground layer dominated by species within the Restionaceae, Cyperacece and Poaceae families, and woody overstorey layers dominated by shrubs within the Proteaeceae, Myrtaceae, Ericaceae and Fabaceae.  The vegetation is fire-prone, with the herbaceous component regenerating rapidly and gradually becoming overtopped by shrubs within 5-6 years post fire \citep{KEITH2007}. The soils, which derive from sandstone, tend to be highly infertile, acidic and silaceous. The topography is relatively flat with elevations ranging from 68 to 72 m above sea level.

In 1990, fifty-six permanent 0.25 m$^{2}$ plots were established along eight transects arranged in pairs, with each transect comprising seven plots. Plots in each transect pair are spaced an average of 18 metres apart (min = 5 m, max = 45 m), while plots in different transect pairs are separated by an average distance of 211 metres (min = 104 m; max = 323 m). Henceforth we refer to these two discrete spatial scales as the `plot' scale (\textit{n} = 56) and the `site' scale (\textit{n} = 4). Since 1990, the total abundance (number of stems) of all herbaceous species within each plot has been censused on nine separate occasions (1990-1994, 1999, 2002, 2007 and 2011) \citep{Keith2012}.  A fire in 1988 prior to the first census burnt the entire site, with subsequent fires in 1994 (entire site burnt) and 2001 (14 plots along one transect-pair burnt). The 1994 fire occurred prior to the census of that year. 

\subsection{Species pool and community phylogeny}

A species pool was defined comprising all herbaceous angiosperm species (49 unique taxa) recorded across all 56 plots since monitoring began. A species accumulation curves generated from random permutation of samples indicated adequate sampling of the species pool (Appendix C).Non-angiosperms were excluded because they were at low abundance and are known, due to their early divergence from all other species in the pool, to have a disproportionately large effect on metrics of phylogenetic community structure \citep{Kembel2006}. A community phylogeny (figure.~\ref{phylog}), derived from DNA sequence data for two commonly used plastid gene regions (rbcL and matK), was generated using the programs phyloGenerator \citep{Pearse2013} and BEAST \citep{Drummond2007}. Full phylogeny construction details are provided in Appendix C.

 \begin{figure}[H]
 \centering
 \includegraphics[width=0.9\linewidth]{Chapter4/Figs/phylog_long.pdf}
 \caption{\footnotesize Community phylogeny for all species recorded in plots across the full monitoring period. Vertical dashed/dotted lines denote species belonging to two nested taxonomic groups (monocots and Poales) on which separate analsyes were performed. Node labels denote posterior support values, with unlabelled nodes indicating points where taxa were manually added to the phylogeny post processing. Timescale is in millions of years before present.}
 \label{phylog}
 \end{figure}


\subsection{Functional traits}

All recorded species were scored for seven functional traits related to competitive ability and/or tolerance of disturbance (seed mass, plant height, Raunki\ae r life-form, fire response, fecundity, longevity and seedbank persistence) \citep{Westoby2002, Adler2013a}. All trait data were obtained from existing databases, the primary literature and expert knowledge, with the exception of seed mass which was supplemented with measurements made from herbarium specimens. Seed mass (mg) and plant height (cm) were scored on a continuous scale, while the five remaining traits were coded on an ordinal scale. This included two strictly categorical traits (fire response and life-form) and three implicitly continuous traits (fecundity, longevity and seedbank persistence) that owing to the absence of sufficiently high resolution quantitative data were classified in bins. Fire response was coded with three levels (killed = 1, facultative resprouter = 2, obligate resprouter = 3); life-form with three levels (geophyte = 1, hemicryptophyte = 2, epiphyte = 3); fecundity with four levels (low = 1, low-moderate = 2, moderate = 3, high = 4); longevity with five levels ($<$5 years = 1, 5-10 years = 2,  10-25 years = 3, 25-50 years = 4, $>$50 years = 5); and seedbank persistence with three levels (transient = 1, moderate persistence = 2, persistent = 3). Several commonly measured leaf traits recognised to represent important axes of niche differentiation (e.g. specific leaf area and leaf dry matter content) were not included in the study due to a large proportion of the dominant species lacking true leaves.

In order to assess the correlation between phylogenetic relatedness and functional similarity we tested for significant phylogenetic signal in continuous traits using Blomberg's \textit{K} statistic \citep{Blomberg2003} and in ordinally coded traits using the `fixed tree, character randomly reshuffled' model of \citet{Maddison1991} with ordered costs for character state transitions.

\subsection{Temporal change in phylogenetic and functional community structure}

Phylogenetic community structure within individual plots at each census was evaluated using two commonly used metrics: mean nearest taxon distance (MNTD, mean distance separating each species in each community from it's closest relative), and mean pairwise phylogenetic distance (MPD, mean pairwise distance between all species in each community) \citep{Webb2000, Webb2002}. While these two metrics are typically correlated they provide complementary information, with MNTD being more sensitive to clustering or dispersion near the tips of the phylogeny, and MPD being more sensitive to tree-wide patterns of clustering or dispersion \citep{Kembel2010}. For both MNTD and MPD we used a square-root transformation of phylogenetic distance in order to account for non-linear scaling of phylogenetic relatedness and functional distance \citep{Letten2014}. Standardized effects sizes (SES) for MNTD and MPD were obtained by comparing observed values to those expected under a null model of community assembly. Both observed and null values were quantified using abundance-weighted data. A null model was used that randomly shuffled the names of individuals across the tips of the phylogeny 999 times. This is considered to be the most appropriate null model for temporal analyses \citep{Swenson2012,Norden2012}.    

An identical framework was used to evaluate functional community structure at each census, where the functional analogue of MNTD (F-MNTD) represents the mean distance to each species' nearest neighbour in multi-trait space, and the functional analogue of MPD (F-MPD) represents the mean pairwise distance in multi-trait space between all species in the community. A Gower distance (which allows for range-standardised quantitative and qualitative data) was used to generate the functional distance matrix representing the functional similarity of species in multivariate trait space. 

In order to account for potential sensitivity of community-wide patterns to spatial scale \citep{Swenson2006}, all analyses of individual plots were replicated at a larger spatial scale by summing plot composition within each site (transect-pair, \textit{n} = 4). In addition, whilst we were primarily interested in the trajectory of the full understorey community through time, we replicated all analyses at three nested phylogenetic scales: i) all species; ii) monocots; and iii) Poales (figure.~\ref{phylog}). 

To examine trends in phylogenetic and functional community structure, linear models and linear mixed-models were fit for phylogenetic and functional community structure as a function of time. To account for spatial and temporal non-independence, random effects were fit for transect-pair (random intercept; transect-pairs that didn't burn in 2001 only) and plots (random slope and intercept) at the plot scale, and for transect-pair (random intercept; transect-pairs that didn't burn in 2001 only) at the site scale. Models were fit for either side of the 1994 fire i.e. 1990-1993 and 1994-2011. For the period from 1994-2011, separate models were fit for those plots/transect-pairs that haven't burnt since 1994 (\textit{n} = 42/3) and those that burnt in the 2001 fire (\textit{n} = 14/1). Given that coefficients may be biased for random effects with fewer than five levels, we checked our results against those obtained when using transects (n = 8) as a random effect at the plot scale, and when treating transect as the site grouping at the site scale. All analyses of phylogenetic and functional community structure were performed with the R-package 'picante' \citep{Kembel2010}. An R$^{2}$ summarising the variance in phylogenetic and functional community structure explained by time was calculated using the approach for mixed-effects models provided by \citet{Nakagawa2013} and extended by \citet{Johnson2014}. Finally Welch's t-test was used to determine whether observed differences in the temporal trajectory between sites with different burn frequencies may be attributable to different initial values.

\subsection{Temporal trends in community-weighted mean trait values}

To complement the core analyses, we also investigated temporal trends in community-weighted mean trait values at the plot scale. Community-weighted mean trait values were calculated for each trait in each plot at each time-step by weighting species' trait values by their proportional abundance. For the purposes of obtaining a single value for each trait in each plot, ordinal traits were treated quantitatively. As for measures of phylogenetic and functional community structure, to examine temporal trends, linear models and linear mixed effect models were fit for each community-weighted mean trait value as a function of time. 

\subsection{Temporal phylogenetic and functional beta turnover}

To quantify temporal phylogenetic and functional beta turnover we used between community equivalents of MNTD and MPD which provide a measure of the phylogenetic and functional dissimilarity between pairs of plots sampled over different years \citep{Ricotta2009, Swenson2012}. The formulas for calculating phylogenetic and functional nearest neighbour dissimilarity (\textit{D}$_{nn}$, the beta diversity analogue of MNTD) and pairwise dissimilarity (\textit{D}pw, the beta diversity analogue of MPD) are provided in Appendix C. To determine whether temporal phylogenetic and functional beta diversity was different from that expected given the rate of taxonomic turnover, we compared observed values to those expected under null models. As for MNTD and MPD, null models for \textit{D}$_{nn}$ and \textit{D}$_{pw}$ were generated by randomly shuffling individuals across the tips of the phylogeny or the columns of the functional distance matrix 999 times. Given the large number of possible temporal pairwise comparisons, we only considered the rate of phylogenetic and functional beta turnover of all census points relative to the first census in 1990 and relative to the immediately preceding census point. As for within-community measures, the above analyses were replicated at the two additional phylogenetic scales (monocots and Poales).  

\section{Results}

\subsection{Temporal change in phylogenetic and functional community structure}

Throughout the 20 year survey period, species composition at both spatial scales was consistently phylogenetically clustered relative to the species pool (figure.~\ref{phy_alphadiv}). Clustering was particularly pronounced for MPD with 53\% of plots and 100\% of sites exhibiting signifcant clustering (\textit{SES$_{MPD}$} $<$ -1.96) over the full sampling period. Between 1990 and 1993 there was a weak increase (towards zero from negative) in MNTD at both the plot scale ($R^{2}$ = 0.05, p $<$ 0.001) and the site scale ($R^{2}$ = 0.10, p $<$ 0.05) (figure.~\ref{phy_alphadiv} b \& d), but no significant trend in MPD (figure.~\ref{phy_alphadiv} a \& c). In contrast, following the 1994 fire, both MNTD and MPD exhibited a consistent decreasing trend in those plots (MNTD: $R^{2}$ = 0.10, p $<$ 0.001; MPD:  $R^{2}$ = 0.02, p $<$ 0.05) and sites (MNTD: $R^{2}$ = 0.29, p $<$ 0.005; MPD:  $R^{2}$ = 0.16, p $=$ 0.06) that did not burn again in 2001. The plots and single site that burnt again in 2001 had flatter, non-significant slopes for both MNTD and MPD over the same time interval. 


 \begin{FPfigure}
 %\floatbox[{\capbeside\thisfloatsetup{capbesideposition={right,top},capbesidewidth=7cm}}]{figure}
 {\caption{\footnotesize Phylogenetic and functional community structure through time based on the standardized effect size of: mean pairwise phylogenetic distance (MPD) between all species in each a) plot and c) site; mean phylogenetic distance between nearest taxonomic neighbours (MNTD) in each b) plot and d) site; mean pairwise functional distance (F-MPD) between all species in each e) plot and g) site; mean functional distance between nearest taxonomic neighbours (F-MNTD) in each f) plot and h) site. Trendlines correspond to models of MNTD/MPD vs. time for all plots/sites through the first four years of sampling (solid line); plots/sites that only burnt in 1994 (dashed line) and plots that burnt in 1994 and 2001 (dotted lines). Trend lines shaded black indicate significant slope coefficients at p $<$ 0.05; grey lines indicate insignificant slopes. In a, b, e \& f, boxplot midline corresponds to median value for all 56 plots at each census point; upper and lower hinges give the first and third quartiles; whiskers extend from the hinge to the highest value that is within 1.5 x interquartile range; and data points beyond whiskers are shown as outliers. In c, d, g \& h, open circles correspond to sites that burnt in the 2001 fire; grey-filled circles correspond with sites that did not burn in 2001.}\label{phy_alphadiv}}
 \centering
 {\includegraphics[width=0.65\linewidth]{Chapter4/Figs/eightpanel_resizefont.pdf}}
 \end{FPfigure}

When the species pool was constrained to just monocots or Poales, results were similar to those described for the full species pool with all spatial-phylogenetic scale combinations exhibiting a decreasing trend in MPD and MNTD following the 1994 fire (Appendix C, figure S4.1-S4.2). The only notable difference between the main results and those obtained with reduced species pools was less negative effect sizes in the latter case. This was particularly true for species pools limited to Poales, for which MPD and MNTD was mostly positive, but still low with only MNTD at the plot scale having any ($<$ 1\%) significantly phylogenetically over-dispersed plots over the entire monitoring period. 

All seven measured traits exhibited significant (p $<$ 0.05) phylogenetic signal (Appendix C, table S4.1). In addition, patterns in functional community structure broadly mirrored those observed for phylogenetic community structure. Almost all plots and sites exhibited functional clustering throughout the monitoring period (figure.~\ref{phy_alphadiv} e--h), although notably within the -1.96 threshold for statistical significance. In addition, temporal trends were similar to the phylogenetic analyses, with plots/sites that last burnt in 1994 becoming increasingly functionally clustered through time (plots [F-MNTD: $R^{2}$ = 0.07, p $<$ 0.001; F-MPD:  $R^{2}$ = 0.06, p $<$ 0.001]; sites [F-MNTD: $R^{2}$ = 0.45, p $<$ 0.001; F-MPD:  $R^{2}$ = 0.27, p $<$ 0.05]). As with the phylogenetic analyses, the plots and single site that burnt again in 2001 had flatter, non-significant slopes, although it is important to note that at the site scale statistical power was low given the small number (\textit{n} = 5) of data points. Prior to the 1994 fire, the only significant trend was a very weak increase (towards zero from negative) in F-MNTD at the plot scale ($R^{2}$ = 0.01, p $<$ 0.01]). When the species pool was constrained to just monocots or Poales results were qualitatively identical to the main results (Appendix C, figure S4.3-S4.4), but as for the phylogenetic analyses, effect sizes were reduced.

The observed differences in temporal trajectory cannot be attributed to different initial values given that there was no significant differences at the start of the sampling period between sites with different subsequent burn frequencies (MNTD: \textit{t} = -0.8406, p = 0.4312; MPD: \textit{t} = 0.7769, p = 0.4431; F-MNTD: \textit{t} = 0.7769, p = 0.4431; F-MPD: \textit{t} = 1.3654, p = 0.1812). In addition, coefficients and standard errors were nearly identical when transects was treated as a random effect at the plot scale, or when the site scale was formed by summing abundance across transects rather than transect pairs. Finally, it is worth noting that effect sizes were weaker but remained predominately negative when null randomisations were restricted to those species observed within any given year. 

\subsection{Temporal trends in community-weighted mean trait values}

In common with observed patterns of phylogenetic and functional community structure, temporal trends in mean-trait values tended to differ between plots with different burn frequencies (figure.~\ref{CWM}). Over the four years prior to the 1994 fire, mean-trait values in all plots were comparatively static, with time explaining no more than 3\% of the variation in community weighted mean trait values for any given trait. In contrast, following the 1994 fire, plots within those transects that went unburnt for the remainder of the sampling period exhibited a significant increase in mean seed-weight ($R^{2}$ = 0.07, p $<$ 0.005), longevity ($R^{2}$ = 0.09, p $<$ 0.001) and the proportion of obligate resprouters ($R^{2}$ = 0.16, p $<$ 0.001), and a decrease in fecundity ($R^{2}$ = 0.18, p $<$ 0.001) and the proportion of hemicryptophytes ($R^{2}$ = 0.10, p $<$ 0.001). In contrast, plots that burnt again in 2001 exhibited mostly static mean-trait values, with only a small decrease in fecundity ($R^{2}$ = 0.02, p $<$ 0.001).

\begin{figure}[H]
%\floatbox[{\capbeside\thisfloatsetup{capbesideposition={right,top},capbesidewidth=5cm}}]{figure}
{\caption{\footnotesize Community-weighted mean trait-values through time. Trendlines correspond to models of individual traits vs. time for all plots/sites through the first four years of sampling (solid line); plots/sites that only burnt in 1994 (dashed line) and plots that burnt in 1994 and 2001 (dotted lines). Trend lines shaded black indicate significant slope coefficients at p $<$ 0.05; grey lines indicate insignificant slopes. Boxplot midline corresponds to median value for all 56 plots at each census point; upper and lower hinges give the first and third quartiles; whiskers extend from the hinge to the highest value that is within 1.5 x interquartile range; and data points beyond whiskers are shown as outliers.}\label{CWM}}
{\includegraphics[width=0.8\linewidth]{Chapter4/Figs/CWM.pdf}}
\end{figure}


\subsection{Temporal phylogenetic and functional beta turnover}

Patterns of phylogenetic and functional beta turnover relative to the first census in 1990 were similar to those relative to the immediately preceding census point (for the latter see Appendix C, figure S4.5). Observed temporal phylogenetic turnover was strongly dependent on the metric used and the phylogenetic resolution of the analysis (figure.~\ref{phylo_betadiv}). When evaluated under a nearest-neighbour dissimilarity metric for all species (\textit{D}$_{nn}$), phylogenetic turnover tended to be relatively random with respect to taxonomic turnover, except during early census-point comparisons when some plots showed greater than expected phylogenetic turnover. In contrast, under a pairwise dissimilarity metric (\textit{D}$_{pw}$), temporal phylogenetic turnover was mostly less than that expected given observed taxonomic turnover for all species, but tended to exhibit more random patterns of turnover when the species pool was limited to monocots and in particular to Poales (Appendix C, figure S4.6). Taken together, these results indicate that plot composition through time tended to be constrained from a phylogeny-wide perspective, i.e limited to a small subset of clades (mainly Poales) relative to the community phylogeny. However, within those well represented clades, turnover tended to be either largely random as indicated by (\textit{D}$_{pw}$) for Poales, or occasionally directional as indicated by high values for (\textit{D}$_{nn}$) for the whole phylogeny. This latter inference is based on (\textit{D}$_{nn}$) being more sensitive to turnover near the tips of the phylogeny.   



\begin{figure}[H]
\centering
\includegraphics[width=1\linewidth]{Chapter4/Figs/beta_2rows_resize.pdf}
\caption{\footnotesize Temporal phylogenetic beta turnover for all species quantified by a) \textit{D}$_{nn}$, and b) \textit{D}$_{pw}$. Low quantile scores (white) indicate low turnover in phylogenetic composition relative to the observed rate of taxonomic turnover; high quantile scores (black) indicate high turnover in phylogenetic composition relative to the observed rate of taxonomic turnover. Values $<$2.5 or $>$97.5 are significant at the 0.05 level.}
\label{phylo_betadiv}
\end{figure}

For both metrics and all three higher phylogenetic scales, observed temporal functional turnover was less than expected given taxonomic turnover (Appendix C, figure S4.7-S4.9) suggesting that the overall functional composition, as derived from the seven measured traits, was relatively fixed through time. However, as indicated by trends in community-weighted mean trait values and observations of temporal turnover in individual traits, this appeared to be driven by low turnover in several traits, in particular plant height and seedbank persistence. 

\section{Discussion}

A classical axiom of community ecology holds that abiotic processes (e.g. environmental filtering) dominate early in succession, while the relative importance of biotic processes (e.g competition) increases as communities mature \citep{Clements1916, Connell1977, Walker1987, Wilson1999}. Assuming that closely related and functionally similar species compete most intensely, it follows that during succession communities may be expected to transition from those that are dominated by closely related and/or functionally similar taxa to those that comprise more phylogenetically and functionally dispersed assemblages \citep{Purschke2013, Bhaskar2014}. In the present study we found no evidence for an increase in phylogenetic and functional dispersion over 20 years of succession post-fire in a coastal heathland community. Instead, species composition at both the plot scale and the site scale tended on average to become more phylogenetically and functionally clustered over time, except notably in those plots where succession was interrupted by fire. 

These findings contrast with a number of previous studies, where an opposite pattern of increased phylogenetic and/or functional dispersion has been reported in non-disturbed versus disturbed communities \citep{Dinnage2009,Helmus2010}, in late successional stages along chronosequences \citep{Letcher2010, Purschke2013}, and along a rare time series of succession \citep{Norden2012}. However, in several rare exceptions to the rule that echo our own findings, Verdu \textit{et al.} \citep{Verdu2009} found that the competitive exclusion of pioneer species appeared to reduce phylogenetic diversity in the very latter stages of a chronosequence of post-fire succession, while Kunstler \textit{et al.} \citep{Kunstler2012} attributed greater functional and phylogenetic convergence with increasing forest plot age to competition sorting species along a competitive hierarchy in plant height. Most recently, Bhaskar \textit{et al.} \citep{Bhaskar2014} found little evidence for an increase in functional dispersion in secondary tropical forests at various stages post agricultural abandonment. Together with these earlier studies, our results reinforce growing awareness of the complex, and often unpredictable, interaction between community assembly processes and patterns of phylogenetic and functional community structure \citep{Mayfield2010, Kunstler2012, Price2013, Bennett2013}.

While it remains difficult to infer underlying processes from static snapshots of community structure, the observed directional shifts in several community-weighted traits through time are indicative of deterministic turnover in this system. These trends tended to be consistent with predictions based on known relationships between functional traits and life-history strategies \citep{Westoby2002, Adler2013a} i.e. the transition from species with relatively fast life-histories (small seeds, high fecundity and short life-spans) in early succession, to those with slower life histories (large seeds, low fecundity and long life-spans) in late succession (figure.~\ref{CWM}). In particular, the observed increase in community weighted mean seed size, a trait often correlated with shade tolerance \citep{Coomes2003}, is consistent with the replacement of good dispersers and fast germinators/resprouters in the immediate post-fire environment by understorey species with greater shade tolerance as the shrub layer becomes more established. It is also telling that the same traits that exhibited directional shifts in the unburnt plots post 1994, were mostly temporally static in those plots that burnt again in 2001, suggesting that fire prevented the competitive exclusion of early successional species by late successional species. 
 
Mediated by the shrub overstorey, light deprivation to the ground layer may act more like an abiotic stressor external to the system. As such, we might attribute phylogenetic and functional clustering later in succession to a filtering process that is biotic but largely external to the ground layer component of the community. However, even if we treat the effect of shading as an external process, competition for resources in the light deprived understorey is still likely to play a significant role in the assembly process. Indeed, at the local scale of interacting species, relative shade tolerance will translate into fitness differences, whereby the more shade tolerant species are at a competitive advantage. If there is a hierarchy of shade tolerance within the ground layer, competition amongst ground layer species will be expected to drive functional clustering. The notion that competition may drive functional and phylogenetic clustering via hierarchical differences in species' competitive ability was  highlighted by Mayfield and Levine \citep{Mayfield2010}, and has received empirical backing by several recent studies \citep{Kunstler2012, Bennett2013, Narwani2013}. However, to our knowledge this is the first study to present patterns of phylogenetic and functional community structure consistent with this process emerging through a successional time-series.

In spite of the observed directional shifts in community weighted mean-trait values, overall functional and pairwise phylogenetic beta-diversity was comparatively constrained through time. This however was counter-balanced by a relatively high rate of temporal phylogenetic beta-diversity amongst closely related taxa early in succession (\textit{D}$_{nn}$, figure.~\ref{phylo_betadiv}). Together with largely stochastic patterns of turnover amongst the highly represented Poales, a picture emerges of a system in which the dominant species are consistently drawn from a small number of clades within the overall phylogeny. However, within those well represented clades, turnover appears largely stochastic or neutral, with functionally similar close relatives replacing each other through time. This disconnect between results observed at the individual trait level and summary statistics of overall phylogenetic and functional turnover highlights the importance of combined approaches. One explanation is that despite directional shifts in individual traits indicating that the relative fitness of different traits is changing through time, there is still substantial functional redundancy near the tips of the phylogeny, with the fate of individual taxa being relatively stochastic \citep{Rosenfeld2002}.      

Our approach focused specifically on the ground layer within a vertically stratified community, whereas most previous studies of phylogenetic and functional community structure through succession in plant communities have focused on the canopy layer \citep[e.g.][]{Letcher2010,Norden2012} or have been conducted in less vertically stratified communities such as grasslands \citep[e.g.][]{Purschke2013}. Where data are available for both canopy and ground layer components of a community, separate analyses for each strata may be preferable to avoid conflating the relative effects of different assembly processes in each strata. Phylogenetic and functional convergence through succession may be a more general feature of understorey communities than it is of less stratified or canopy `communities' \citep[but see][for an example of convergence amongst canopy species in temperate forests]{Kunstler2012}. Ultimately, more phylogenetic and functional based studies of community assembly through succession in the understorey of vertically stratified communities are needed to verify this hypothesis.

Disentangling the relative contribution of abiotic and biotic processes in driving community assembly through succession remains a major challenge for community ecologists. Contrary to expectations, here we have shown that phylogenetic and functional dispersion is not the only, or necessarily the most likely, outcome of succession. This finding contributes to an emerging body of research re-evaluating the role of limiting functional similarity in determining the outcome of community assembly \citep{Mayfield2010,Kunstler2012, Price2013, Bennett2013}. Efforts to partition out the relative importance of the underlying processes driving functional and phylogenetic convergence through succession in this, and other systems, will be a valuable focus of future work.

\section*{Acknowledgements}

Thanks to Will Pearse for assistance with phyloGenerator, Nate Swenson for providing code for the analysis of phylogenetic and functional beta diversity, and Jodi Price, Will Cornwell and two anonymous reviewers for valuable comments on an earlier draft of this paper.


% this file is called up by thesis.tex
% content in this file will be fed into the main document

%: ----------------------- introduction file header -----------------------

\graphicspath{{5/figures/}} % specifies where the figures are stored

% ----------------------------------------------------------------------
%: ----------------------- content ----------------------- 
% ----------------------------------------------------------------------

\chapter{\label{ch5}Name of chapter 5} % top level followed by section, subsection

\section{\label{5:1}Name of section}

\subsection{\label{5:1:1}Name of subsection}



% ---------------------------------------------------------------------------
%: ----------------------- end of thesis sub-document ------------------------
% ---------------------------------------------------------------------------


% this file is called up by thesis.tex
% content in this file will be fed into the main document

%: ----------------------- introduction file header -----------------------

\graphicspath{{6/figures/}} % specifies where the figures are stored

% ----------------------------------------------------------------------
%: ----------------------- content ----------------------- 
% ----------------------------------------------------------------------

\chapter{\label{ch6}Name of chapter 6} % top level followed by section, subsection

\section{\label{6:1}Name of section}

\subsection{\label{6:1:1}Name of subsection}



% ---------------------------------------------------------------------------
%: ----------------------- end of thesis sub-document ------------------------
% ---------------------------------------------------------------------------


%============= 参考文献 =====================
\addcontentsline{toc}{section}{参考文献}
\bibliography{bibfile}
\clearpage
%=============  致谢  ======================
\include{body/acknowledge}
\newpage
\appendix

%%附录第一个章节
\section{第一附录}


%%变量列举

\begin{table}[H]
\caption{Symbol Table-Constants}
\centering
\begin{tabular}{lll}
\toprule
Symbol & Definition  & Units\\
\midrule[2pt]
\multicolumn{3}{c}{\textbf{Constants} }\\
$DL$&Expectancy of poisson-distribution &  unitless \\
$NCL$ &Never- Change-Lane& unitless\\
$CCL$&Cooperative-Change-Lane& unitless\\
$ACL$&Aggressive-Change-Lane& unitless\\
$FCL$&Friendly-Change-Lane& unitless\\
$SCC$&Self-driving-Cooperative-Car& unitless\\
$NSC$&None-Self-drive-Car& unitless\\
\bottomrule
\end{tabular}
\end{table}


\section{第二附录}
\textcolor[rgb]{0.98,0.00,0.00}{\textbf{Simulation Code}}
\begin{python}
import java.util.*;  
public class test {  
    public static void main (String[]args){   
        int day=0;  
        int month=0;  
        int year=0;  
        int sum=0;  
        int leap;     
        System.out.print("请输入年,月,日\n");     
        Scanner input = new Scanner(System.in);  
        year=input.nextInt();  
        month=input.nextInt();  
        day=input.nextInt();  
        switch(month) /*先计算某月以前月份的总天数*/    
        {     
        case 1:  
            sum=0;break;     
        case 2:  
            sum=31;break;     
        case 3:  
            sum=59;break;     
        case 4:  
            sum=90;break;     
        case 5:  
            sum=120;break;     
        case 6:  
            sum=151;break;     
        case 7:  
            sum=181;break;     
        case 8:  
            sum=212;break;     
        case 9:  
            sum=243;break;     
        case 10:  
            sum=273;break;     
        case 11:  
            sum=304;break;     
        case 12:  
            sum=334;break;     
        default:  
            System.out.println("data error");break;  
        }     
        sum=sum+day; /*再加上某天的天数*/    
        if(year%400==0||(year%4==0&&year%100!=0))/*判断是不是闰年*/    
            leap=1;     
        else    
            leap=0;     
        if(leap==1 && month>2)/*如果是闰年且月份大于2,总天数应该加一天*/    
            sum++;     
        System.out.println("It is the the day:"+sum);  
        }  
} 
\end{python}




\end{document}
%%%%%%%%%% 结束 %%%%%%%%%%